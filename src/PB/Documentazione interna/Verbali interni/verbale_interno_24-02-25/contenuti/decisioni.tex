% Intestazione
\fancyhead[L]{4 \hspace{0.2cm} Decisioni} % Testo a sinistra

\section{Decisioni}

Durante la riunione sono state prese le seguenti decisioni:

\vspace{0.5cm}

\begin{table}[htbp]
    \centering
    \rowcolors{2}{lightgray}{white}
    \begin{tabular}{|c|p{0.8\textwidth}|}
        \hline
        \rowcolor[gray]{0.75}
        \textbf{Codice} & \textbf{Descrizione}\\
        \hline
        VI 30.1 & È stato deciso che l’intensità di lavoro deve rimanere alta, con un focus sulla chiusura anticipata delle issues per favorire un flusso di lavoro più efficiente.\\
        \hline
        VI 30.2 & È stato deciso che le modifiche correttive urgenti saranno gestite da Riccardo Stefani, garantendo una risposta tempestiva. \\
        \hline
        VI 30.3 & È stato deciso che i \emph{test} di \emph{Angular} rimarranno nella cartella dedicata, mentre quelli \emph{Python} saranno collocati nella cartella "tests", adottando un approccio basato sul \emph{TDD}. \\ 
        \hline
        VI 30.4 & È stato deciso di suddividere le responsabilità per lo sviluppo e il testing, tenendo conto delle dipendenze tra i vari componenti. \\ 
        \hline
        VI 30.5 & È stato deciso di rimuovere la guida per l'avvio della variante da terminale dell'applicazione dalla guida per l'istallazione del \emph{Manuale Utente}. \\ 
        \hline
        VI 30.6 & È stato deciso di valutare la possibilità di creare in \emph{Angular} un unico file di supporto per la comunicazione con il backend. \\ 
        \hline
        VI 30.7 & È stato deciso di confermare l’integrazione della funzione di copia dei messaggi e di copia degli \emph{snippet} di codice. \\        
        \hline
        VI 30.8 & È stato stabilito che Federica Bolognini redigerà il verbale della riunione interna del 24/02/2025. \\      
        \hline
    \end{tabular}
\end{table}