% Intestazione
\fancyhead[L]{3 \hspace{0.2cm} Diario della riunione} % Testo a sinistra

\section{Diario della riunione}

\begin{itemize}
    \item Si è discusso dell’intensità di lavoro e della gestione delle \emph{issues}, ribadendo l’importanza di chiudere i task in anticipo e di adottare un approccio collaborativo;  
    \item Si è parlato dell’importanza di gestire le modifiche correttive "urgenti" per garantire un miglioramento tempestivo;  
    \item Si è analizzato il feedback del professore sulla progettazione;  
    \item Si è definita la strategia \emph{TDD}\textsubscript{\textit{\textbf{G}}}, stabilendo che i \emph{test}\textsubscript{\textit{\textbf{G}}} di \emph{Angular}\textsubscript{\textit{\textbf{G}}} rimarranno nella cartella dedicata e quelli \emph{Python}\textsubscript{\textit{\textbf{G}}} saranno collocati nella cartella “tests”;
    \item Si è concordata la suddivisione dei rami di sviluppo in base ai diagrammi;  
    \item Si è assegnata la suddivisione delle responsabilità tra i membri del team per lo sviluppo e il testing, tenendo conto delle dipendenze tra i componenti;  
    \item Si è deciso di rimuovere la guida per l'avvio della variante da terminale dell'applicazione dalla guida per l'istallazione del \emph{Manuale Utente}\textsubscript{\textit{\textbf{G}}}, poichè infatti quella variante non fa parte del capitolato ed è stata utile solamente in fase di sviluppo del \emph{PoC}\textsubscript{\textit{\textbf{G}}};
    \item Si è analizzata la possibilità di creare in \emph{Angular}\textsubscript{\textit{\textbf{G}}} un unico file di supporto per la comunicazione con il backend invece che tanti, ed è stato deciso di analizzare la questione in seguito;
    \item Si è confermata l’integrazione della funzione di copia dei messaggi e e di copia degli \emph{snippet}\textsubscript{\textit{\textbf{G}}} di codice.
\end{itemize}


