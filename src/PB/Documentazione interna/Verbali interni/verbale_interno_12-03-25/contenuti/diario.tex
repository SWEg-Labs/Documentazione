% Intestazione
\fancyhead[L]{3 \hspace{0.2cm} Diario della riunione} % Testo a sinistra

\section{Diario della riunione}

\begin{itemize}
    \item È stato detto che ogni membro del team deve controllare le \emph{issue} contrassegnate come “Sotto verifica” e chiuderle se già verificate;
    \item Per il miglioramento dell'organizzazione del lavoro, sono state apportate delle modifiche a \emph{Jira}\textsubscript{\textit{\textbf{G}}} per semplificare l’inserimento delle date di inizio e scadenza delle \emph{issue};
    \item Si è parlato della suddivisione del lavoro per la scrittura del \emph{Manuale Utente}\textsubscript{\textit{\textbf{G}}}, assegnando le varie sezioni ai membri del team;
    \item Si è parlato della possibilità di chiedere al proponente se includere la guida per la visualizzazione dei file di log nel \emph{Manuale Utente};
    \item Si è parlato della suddivisione del lavoro per la scrittura della \emph{Specifica Tecnica}\textsubscript{\textit{\textbf{G}}};
    \item Si è parlato del fatto che, secondo le \emph{Norme di Progetto}\textsubscript{\textit{\textbf{G}}}, le \emph{docstrings}\textsubscript{\textit{\textbf{G}}} sono richieste solo per il \emph{backend}\textsubscript{\textit{\textbf{G}}}, quindi è preferibile non aggiungerle nel \emph{frontend}\textsubscript{\textit{\textbf{G}}} per evitare incoerenze;
    \item Si è parlato della stabilizzazione dei requisiti, confermando che i 50 messaggi non causano problemi di inefficienza;
    \item Si è deciso di procedere con la trascrizione dei requisiti nel \emph{Piano di Qualifica}\textsubscript{\textit{\textbf{G}}} e nella \emph{Specifica Tecnica}\textsubscript{\textit{\textbf{G}}};
    \item Si è concordato che i test già completati possono essere scritti nel \emph{Piano di Qualifica};
    \item Si è deciso chi ha il compito di finire gli ultimi test;
    \item Si è parlato del limite della funzione query di \emph{Chroma}\textsubscript{\textit{\textbf{G}}}, che non permette di recuperare più di 514 documenti, e si è deciso di impostare n\_results = 500 per evitare errori;
    \item Si è discusso della necessità di unire i branch newFrontend e feature/UnitTests, con l’obiettivo di integrarli e successivamente fare il \emph{merge}\textsubscript{\textit{\textbf{G}}} nel \emph{branch}\textsubscript{\textit{\textbf{G}}} develop.
\end{itemize}


