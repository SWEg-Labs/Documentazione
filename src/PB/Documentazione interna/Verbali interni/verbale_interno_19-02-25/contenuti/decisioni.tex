% Intestazione
\fancyhead[L]{4 \hspace{0.2cm} Decisioni} % Testo a sinistra

\section{Decisioni}

Durante la riunione sono state prese le seguenti decisioni:

\vspace{0.5cm}

\begin{table}[htbp]
    \centering
    \rowcolors{2}{lightgray}{white}
    \begin{tabular}{|c|p{0.8\textwidth}|}
        \hline
        \rowcolor[gray]{0.75}
        \textbf{Codice} & \textbf{Descrizione}\\
        \hline
        VI 28.1 & E' stato deciso di introdurre nel documento di \emph{Specifica Tecnica} una sezione "Logica del prodotto",
        in cui verrà descritto il funzionamento ad alto livello del sistema.\\
        \hline
        VI 28.2 & E' stato deciso di introdurre nel documento di \emph{Specifica Tecnica} una sezione "Architettura logica",
        in cui verrà descritta l'architettura logica del sistema, cioè l'\emph{Architettura esagonale}.\\
        \hline
        VI 28.3 & E' stato deciso di introdurre nel documento di \emph{Specifica Tecnica} una sezione "Architettura di deployment",
        in cui verrà descritta l'architettura di deployment del sistema, cioè l'\emph{Architettura monolitica}\textsubscript{\textit{\textbf{G}}}.\\
        \hline
        VI 28.4 & E' stato deciso che i requisiti saranno trascritti nelle tabelle di \emph{checklist} del \emph{Piano di Qualifica}
        solamente quando i requisiti saranno terminati.\\
        \hline
        VI 28.5 & E' stato deciso che la questione del pattern \emph{Singleton} sarà discussa nel prossimo incontro con
        \emph{AzzurroDigitale} e nel prossimo incontro con il professor Cardin.\\
        \hline
        VI 28.6 & E' stato deciso che il numero di sessioni da recuperare inizialmente e il numero di sessioni da recuperare
        quando l'utente scorre la lista dei messaggi sarà deciso in fase di programmazione.\\
        \hline
        VI 28.7 & E' stato deciso di sistemare il \emph{Diagramma di sequenza} relativo ad \emph{Angular} includendo anche 
        l'aggiornamento del badge di segnalazione dell'esito dell'aggiornamento automatico e separando il diagramma in più diagrammi
        per migliorare la leggibilità.\\
        \hline
        VI 28.8 & E' stato deciso di sistemare l'errore di progettazione relativo alla comunicazione tra un \emph{Service} ed un
        \emph{Controller} per rispettare l'\emph{Architettura esagonale}, e quindi è stato deciso di separare il Controller di 
        salvataggio di un messaggio dal Service di recupero della risposta all'interrogazione.\\
        \hline
        VI 28.9 & E' stato deciso che tutto il gruppo deve essere aggiornato sui diagrammi di progettazione, così da poterci allineare
        nel futuro sviluppo.\\
        \hline
        VI 28.10 & E' stato deciso che la migliore struttura di cartelle per i \emph{Test} prevede di suddividere i test per tipologia
        e poi per tematica/correlazione, tale per cui chi svilupperà i test avrà la possibilità di creare una cartella in cui scrivere
        dei file di test correlati fra loro che scriverà assieme.\\
        \hline
        VI 28.11 & E' stato deciso di provare ad implementare una strategia di sviluppo basata sull'approccio \emph{TDD}.\\
        \hline
        VI 28.12 & E' stato deciso di seguire l'\emph{Architettura esagonale} per lo sviluppo dei \emph{Test di integrazione}.\\
        \hline
        VI 28.13 & E' stato deciso di suddividere tra di noi lo sviluppo dei \emph{Test di integrazione} nel corso della prossima
        \emph{Sprint}.\\
        \hline
        VI 28.14 & E' stato deciso di progettare i \emph{test di unità}\textsubscript{\textit{\textbf{G}}} solamenente durante la
        \emph{progettazione di dettaglio}, quindi più tardi.\\
        \hline
        VI 28.15 & E' stata decisa la struttura di cartelle del repository dell'\emph{MVP}\textsubscript{\textit{\textbf{G}}} seguendo
        l'\emph{Architettura esagonale}, e mantenendo una cartella denominata "Utils" per classi e metodi di utilità.\\
        \hline
        VI 28.16 & E' stato deciso di esercitarsi per provare a migliorare l’aggiornamento automatico del
        \emph{database vettoriale}\textsubscript{\textit{\textbf{G}}} \emph{Chroma}\textsubscript{\textit{\textbf{G}}}.\\
        \hline
        VI 28.17 & E' stato deciso di partire a esercitarsi su come sviluppare le componenti dell'interfaccia grafica perchè 
        rispettino quanto previsto dai \emph{casi d'uso}\textsubscript{\textit{\textbf{G}}}.\\
        \hline
    \end{tabular}
\end{table}

\newpage

\begin{table}[htbp]
    \centering
    \rowcolors{2}{lightgray}{white}
    \begin{tabular}{|c|p{0.8\textwidth}|}
        \hline
        \rowcolor[gray]{0.75}
        \textbf{Codice} & \textbf{Descrizione}\\
        \hline
        VI 28.18 & E' stato deciso di provare a rendere più \emph{efficiente}\textsubscript{\textit{\textbf{G}}} il recupero dei file
        da \emph{Chroma} per la risposta ad una interrogazione.\\
        \hline
        VI 28.19 & Abbiamo deciso che domani \emph{20 febbraio} chiederemo dei chiarimenti ad \emph{AzzurroDigitale} per quanto riguarda
        il requisito di recupero dei file dai quali è stata ricavata la risposta.\\
        \hline
        VI 28.20 & Sono stati decisi i ruoli per la presentazione di domani con AzzurroDigitale.\\
        \hline
        VI 28.21 & E' stato deciso di provare a chiedere al professor Cardin un incontro venerdì 21 febbraio, eventualmente
        ripiegando su lunedì 24.\\
        \hline
        VI 28.22 & E' stato deciso di riportare nel \emph{Piano di Progetto} e nel \emph{Piano di Qualifica} il ritardo dei lavori
        causato da impegni accademici, segnalando che non è stato un problema di organizzazione, ma bensì una causa di forza maggiore.\\
        \hline
        VI 28.23 & E' stato deciso che il \emph{Consuntivo} della sprint passata ed il \emph{Preventivo} della sprint futura è
        possibile redigerli in forma asincrona.\\
        \hline
        VI 28.24 & E' stato deciso che Riccardo Stefani redigerà il verbale della riunione interna corrente.\\
        \hline
    \end{tabular}
\end{table}
