% Intestazione
\fancyhead[L]{4 \hspace{0.2cm} Decisioni} % Testo a sinistra

\section{Decisioni}

Durante la riunione sono state prese le seguenti decisioni:

\vspace{0.5cm}

\begin{table}[htbp]
    \centering
    \rowcolors{2}{lightgray}{white}
    \begin{tabular}{|c|p{0.8\textwidth}|}
        \hline
        \rowcolor[gray]{0.75}
        \textbf{Codice} & \textbf{Descrizione}\\
        \hline
        VI 27.1 & È stato approvato il nuovo sistema di gestione delle \emph{issue}\textsubscript{\textit{\textbf{G}}}, che prevede la loro chiusura con aggiornamento della timeline e l'inserimento di Autore e Verificatore nel \emph{Product Backlog}\textsubscript{\textit{\textbf{G}}}. Inoltre, si è deciso che sarà il verificatore a chiudere le \emph{issue}, in quanto ultimo a intervenire su di esse. \\
        \hline
        VI 27.2 & È stato deciso di eliminare le \emph{issue} senza assegnatario, sostituendole con \emph{issue} assegnate a singoli individui per garantirne una gestione più efficace. \\
        \hline
        VI 27.3 & È stato deciso di procedere alla correzione degli errori notificati dal professor Vardanega. \\
        \hline
        VI 27.4 & È stato deciso di scrivere una mail al professor Vardanega per dei chiarimenti sulle sue correzioni. \\
        \hline
        VI 27.5 & È stato deciso di inviare una mail al professor Cardin per organizzare un incontro in cui discutere alcuni chiarimenti sulle correzioni e ottenere il suo parere sulla prova di Architettura esagonale. \\
        \hline
        VI 27.6 & Sono stati decisi gli scrittori e i verificatori ufficiali dei documenti. \\
        \hline
        VI 27.7 & Sono stati decisi gli incarichi per la stesura della documentazione e assegnati ai rispettivi responsabili. \\
        \hline
        VI 27.8 & È stato deciso di dare priorità alla sistemazione di quanto già realizzato per adattarlo alla progettazione prima di iniziare a implementare nuove funzionalità. \\
        \hline
        VI 27.9 & È stato deciso di iniziare a lavorare sui diagrammi e di presentare le varie proposte durante il prossimo incontro con il proponente. \\
        \hline
        VI 27.10 & È stato deciso di iniziare a progettare i test dalla prossima sprint. \\
        \hline
        VI 27.11 & È stato deciso di fissare la prossima riunione interna al 19 Febbraio. \\
        \hline
        VI 27.12 & È stato deciso che Filippo avrà il compito di scrivere il verbale del 10/02/25. \\
        \hline
    \end{tabular}
\end{table}
