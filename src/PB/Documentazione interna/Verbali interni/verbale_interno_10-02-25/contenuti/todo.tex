% Intestazione
\fancyhead[L]{5 \hspace{0.2cm} Todo} % Testo a sinistra

\section{Todo}

Durante la riunione sono emersi i seguenti task da svolgere:

\vspace{0.5cm}

\begin{table}[htbp]
\centering
\rowcolors{2}{lightgray}{white}
\begin{tabular}{|c|c|p{0.25\textwidth}|p{0.4\textwidth}|}
    \hline
    \rowcolor[gray]{0.75}
    \textbf{Codice} & \textbf{Dalla decisione} & \textbf{Assegnatario} & \textbf{Task Todo} \\
    \hline
    BUD-239 & VI 27.3 & Riccardo Stefani & Spostare glossario nella documentazione interna. \\
    \hline
    BUD-240 & VI 27.3 & Davide Verzotto & Sistemare i riferimenti (Normativi Informativi). \\
    \hline
    BUD-241 & VI 27.3 & Filippo Righetto & Sistemare il riferimento a riferimenti. Renderli più specifici. \\
    \hline
    BUD-242 & VI 27.3 & Michael Fantinato & Sistemare i riferimenti a risorse web soggette a variazione. \\
    \hline
    BUD-243 & VI 27.3 & Giacomo Loat & Sistemare i riferimenti a documenti soggetti a ciclo di vita specificando la versione del documento di interesse. \\
    \hline
    BUD-244 & VI 27.3 & Riccardo Stefani & Rendere i link espliciti e non nascosti. \\
    \hline
    BUD-245 & VI 27.3 & Federica Bolognini & Rinominare "piano di qualità" con "metriche di qualità". \\
    \hline
    BUD-246 & VI 27.3 & Federica Bolognini & Sistemare PdQ e Ndp. \\
    \hline
    BUD-247 & VI 27.3 & Giacomo Loat & Sistemare le figure 11-14 del PdQ. \\
    \hline
    BUD-248 & VI 27.3 & Riccardo Stefani & Fare checklist per inspection. \\
    \hline
    BUD-249 & VI 27.3 & Davide Verzotto & Scrivere sezione "Fasi della fornitura" nell'NdP. \\
    \hline
    BUD-250 & VI 27.3 & Giacomo Loato & Togliere valutazioni per il miglioramento. \\
    \hline
    BUD-251 & VI 27.7 & Davide Verzotto &  Sistemare tracciamento Fonte-Requisiti in AdR.\\
    \hline
    BUD-252 & VI 27.7 & Riccardo Stefani & Aggiungere alle Tecnologie coinvolte della specifica tecnica: Scrivere PyTest per Pyton e Jasmine per Angular. \\
    \hline
    BUD-253 & VI 27.7 & Giacomo Loat & Aggiungere alle Tecnologie coinvolte della specifica tecnica: Scrivere Postgres per il salvataggio dello storico. \\
    \hline
    BUD-254 & VI 27.7 & Riccardo Stefani & Scrivere Requisiti del manuale utente. \\
    \hline
    BUD-255 & VI 27.7 & Michael Fantinato & Scrivere l'installazione del programma nel manuale utente. \\
    \hline
    BUD-256 & VI 27.7 & Riccardo Stefani & Scrivere guida all'utilizzo del manuale utente.\\
    \hline
    BUD-257 & VI 27.7 & Federica Bolognini & Scrivere introduzione Product Baseline. \\
    \hline
    BUD-258 & VI 27.7 & Riccardo Stefani & Iniziare con la metrica "Code Coverage" ne PdQ e nel NdP. \\
    \hline
    BUD-259 & VI 27.9 & Riccardo Stefani & Modellare Architettura della generazione di una risposta. \\
    \hline
    BUD-260 & VI 27.9 & Riccardo Stefani & Modellare Architettura dell'aggiornamento automatico del database vettoriale. \\
    \hline
    BUD-261 & VI 27.9 & Michael Fantinato & Modellare Architettura dell'aggiornamento dell'interfaccia grafica durante la chat. \\
    \hline
\end{tabular}
\end{table}

\clearpage
\begin{table}[htbp]
    \centering
    \rowcolors{2}{lightgray}{white}
    \begin{tabular}{|c|c|p{0.25\textwidth}|p{0.4\textwidth}|}
        \hline
        \rowcolor[gray]{0.75}
        \textbf{Codice} & \textbf{Dalla decisione} & \textbf{Assegnatario} & \textbf{Task Todo} \\
        \hline
        BUD-262 & VI 27.9 & Michael Fantinato & Modellare Architettura dell'aggiornamento del badge di segnalazione dell'esito dell'ultimo aggiornamento automatico. \\
        \hline
        BUD-263 & VI 27.9 & Giacomo Loat & Modellare Architettura del salvataggio e recupero dei messaggi dallo storico. \\
        \hline
        BUD-264 & VI 27.9 & Giacomo Loat & Modellare Architettura della visualizzazione di domande per iniziare la conversazione. \\
        \hline
        BUD-265 & VI 27.9 & Riccardo Stefani & Modellare Architettura della generazione di domande per proseguire la conversazione. \\
        \hline
        BUD-266 & VI 27.12 & Filippo Righetto & Scrivere il verbale del 10/02/25. \\
        \hline
    \end{tabular}
    \end{table}