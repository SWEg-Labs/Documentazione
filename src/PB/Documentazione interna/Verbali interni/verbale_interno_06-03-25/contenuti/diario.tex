% Intestazione
\fancyhead[L]{3 \hspace{0.2cm} Diario della riunione} % Testo a sinistra

\section{Diario della riunione}

\begin{itemize}
    \item Ci siamo confrontati sui vari punti emersi durante la riunione con il \emph{proponente}, tra cui:
    \begin{itemize}
        \item È stato accettato di non porre domande iniziali per avviare la conversazione del chatbot;
        \item La modifica dei file in \emph{Chroma}\textsubscript{\textit{\textbf{G}}} deve avvenire tenendo conto della data di ultima modifica;
        \item Per l’eliminazione dei file in \emph{Chroma} si procederà prelevando solamente gli ID e non i documenti interi, per confrontarli con gli ID dei documenti "in arrivo";
        \item Non è necessaria la gestione del riavvio della connessione a \emph{Postgres}\textsubscript{\textit{\textbf{G}}} né del caso di errore nel salvataggio ma riuscita nel recupero dei log;
        \item Per il meccanismo di aggiornamento del badge, si potrà optare tra due soluzioni: aggiornamento in tempo reale tramite \emph{Polling REST}\textsubscript{\textit{\textbf{G}}} oppure un richiamo ad ogni refresh della pagina;
        \item È stata approvata anche la gestione e visualizzazione dei file di log;
        \item È stato stabilito che il non funzionamento del chatbot durante gli aggiornamenti deve essere segnalato nel \emph{Manuale Utente}\textsubscript{\textit{\textbf{G}}};
        \item Non sono emerse richieste grafiche aggiuntive per il \emph{frontend};
        \item I link verranno mostrati all’interno di un riquadro posizionato in fondo al messaggio;
        \item È stata accettata la visualizzazione di data e ora nei messaggi;
        \item È stata confermata la necessità di impostare una coverage delle righe adeguata per i test, che sarà comunicata successivamente.
    \end{itemize}
    \item Abbiamo discusso dei contenuti da inserire nella documentazione durante la prossima sprint e concordato che, prima di aggiungere i requisiti da segnare come soddisfatti al \emph{Piano di Qualifica}\textsubscript{\textit{\textbf{G}}} e alla \emph{Specifica Tecnica}\textsubscript{\textit{\textbf{G}}}, è necessario testare la gestione dello storico;
    \item Abbiamo discusso a riguardo di alcune attività di sviluppo da svolgere che non sono molto urgenti, tra cui: 
    \begin{itemize}
        \item Migliorare la gestione delle variabili d'ambiente in \emph{Docker}\textsubscript{\textit{\textbf{G}}} usando "environment".
        \item Esplorare l'uso dei volumi \emph{Docker} per mettere in cache l'installazione dei pacchetti e velocizzare la creazione dell'immagine di BuddyBot.
    \end{itemize}
    \item Gestione dei formalismi di fine \emph{Sprint}, che includono l'aggiornamento del \emph{Piano di Progetto}\textsubscript{\textit{\textbf{G}}} e la verifica generale di quanto è stato scritto finora nei documenti.
\end{itemize}


