% Intestazione
\fancyhead[L]{4 \hspace{0.2cm} Decisioni} % Testo a sinistra

\section{Decisioni}

Durante la riunione sono state prese le seguenti decisioni:

\vspace{0.5cm}

\begin{table}[htbp]
    \centering
    \rowcolors{2}{lightgray}{white}
    \begin{tabular}{|c|p{0.8\textwidth}|}
        \hline
        \rowcolor[gray]{0.75}
        \textbf{Codice} & \textbf{Descrizione}\\
        \hline
        VI 32.1 & È stato deciso che la modifica dei file in \emph{Chroma} deve avvenire tenendo conto della data di ultima modifica.\\
        \hline
        VI 32.2 & È stato deciso che per l'eliminazione dei file in \emph{Chroma} si dovrà procedere prelevando solamente gli ID e non i documenti interi, per confrontarli con gli ID dei documenti "in arrivo".\\
        \hline
        VI 32.3 & È stato deciso che non è necessaria la gestione del riavvio della connessione a \emph{Postgres} né del caso di errore nel salvataggio ma riuscita nel recupero dei log.\\
        \hline 
        VI 32.4 & È stato deciso che per il meccanismo di aggiornamento del badge, si potrà optare tra due soluzioni: aggiornamento in tempo reale tramite \emph{Polling REST} oppure un richiamo ad ogni refresh della pagina;\\
        \hline
        VI 32.5 & È stato deciso che il non funzionamento del chatbot durante gli aggiornamenti deve essere segnalato nel \emph{Manuale Utente}.\\
        \hline
        VI 32.6 & È stato decisa la visualizzazione di data e ora nei messaggi.\\
        \hline
        VI 32.7 & È stato decisa di impostare una coverage delle righe adeguata per i test, che sarà comunicata successivamente.\\
        \hline
        VI 32.8 & È stato deciso che i link verranno mostrati all'interno di un riquadro posizionato in fondo al messaggio. \\
        \hline
        VI 32.9 & È stato deciso che, prima di aggiungere i requisiti da segnare come soddisfatti al \emph{Piano di Qualifica} e alla \emph{Specifica Tecnica}, è necessario testare la gestione dello storico.\\
        \hline
        VI 32.10 & È stato deciso di migliorare la gestione delle variabili d’ambiente in \emph{Docker} utilizzando "environment". \\
        \hline
        VI 32.11 & È stato deciso di esplorare l’uso dei volumi \emph{Docker} per mettere in cache l’installazione dei pacchetti e velocizzare la creazione dell’immagine di BuddyBot.\\
        \hline
        VI 32.12 & Si è deciso di continuare con le solite pratiche di chiusura \emph{Sprint} anche per questa. \\
        \hline
        VI 32.13 & È stato deciso che il verbale della riunione sarà scritto da Filippo Righetto. \\
        \hline
        VI 32.14 & È stato deciso che Davide Verzotto dovrà scrivere il diario di bordo per l'esposizione del 10/03/25. \\
        \hline
    \end{tabular}
\end{table}