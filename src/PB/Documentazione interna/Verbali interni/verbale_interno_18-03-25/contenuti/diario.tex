% Intestazione
\fancyhead[L]{3 \hspace{0.2cm} Diario della riunione} % Testo a sinistra

\section{Diario della riunione}

\begin{itemize}
    \item È stato deciso di chiedere ad \emph{AzzurroDigitale} di effettuare una \emph{demo} domani 19 Marzo, con una seconda data possibile per venerdì, se necessario. I ruoli per la gestione dell'evento sono stati assegnati;
    \item Si è discusso della necessità di verificare urgentemente i documenti di \emph{Specifica Tecnica} e \emph{Analisi dei Requisiti}, in quanto oggetto di valutazione del professore Cardin, con cui viene svolto il primo incontro della \emph{Product Baseline}\textsubscript{\textbf{\textit{G}}};
    \item Si è parlato della pianificazione delle verifiche della restante documentazione, includendo il \emph{Piano di Qualifica}, il \emph{Piano di Progetto}, le \emph{Norme di Progetto}, \emph{Manuale Utente} e \emph{Manuale Sviluppatore} con l'assegnazione dei verificatori;
    \item È stato deciso che l'ultima \emph{sprint} dovrà essere gestita con attenzione, in quanto è necessario scrivere il suo resoconto prima del suo termine per poter verificare il \emph{Piano di Progetto} e il \emph{Piano di Qualifica}, ed allora bisognerà impegnarsi a rispettare quanto sarà stato scritto in anticipo in questi documenti;
    \item Si è discusso dei formalismi di fine \emph{sprint}, cioè la redazione del \emph{preventivo} e \emph{consuntivo} per le sprint 7 e 8, l'analisi delle metriche del \emph{Piano di Qualifica} e la creazione della pull request per il \emph{repository} della documentazione.
\end{itemize}
