% Intestazione
\fancyhead[L]{M} % Testo a sinistra

\section{}

\hypertarget{sec:manuale_utente}{}
\subsection*{Manuale Utente}
Uno dei documenti che è richiesto che vengano redatti per la Product Baseline, con lo scopo di fornire istruzioni dettagliate su 
come utilizzare il sistema sviluppato. Nell'ambito dell'ingegneria del software, il manuale utente è progettato per aiutare gli 
utenti finali a comprendere le funzionalità del software, eseguire operazioni specifiche e risolvere problemi comuni. Include 
descrizioni delle interfacce utente, spiegazioni delle funzionalità principali, guide passo-passo, e soluzioni ai problemi più 
frequenti. Il manuale utente è essenziale per garantire che gli utenti possano utilizzare il software in modo efficace e autonomo. 

\hypertarget{sec:manuale_sviluppatore}{}
\subsection*{Manuale Sviluppatore}
Un manuale sviluppatore è un documento tecnico che fornisce istruzioni dettagliate, 
linee guida e riferimenti necessari per progettare, sviluppare e mantenere software o hardware. 
Contiene le funzionalità, gli strumenti, le best practice e gli esempi di codice destinati 
specificamente ai programmatori che lavorano con una particolare tecnologia, linguaggio di programmazione o piattaforma.

\hypertarget{sec:manutenibilità}{}
\subsection*{Manutenibilità}
Caratteristica di un software che ne facilita la manutenzione, ovvero la capacità di apportare modifiche, correzioni e miglioramenti
al codice in modo efficiente, sicuro e affidabile.

\hypertarget{sec:markup}{}
\subsection*{Markup}
Un sistema di codifica che utilizza tag o annotazioni per strutturare e formattare documenti digitali (es. HTML o XML).

\hypertarget{sec:merge}{}
\subsection*{Merge}
Operazione fondamentale nei sistemi di controllo delle versioni. Essa è utilizzata per combinare le modifiche apportate in due branch separati in un 
singolo branch.

\hypertarget{sec:metrica}{}
\subsection*{Metrica}
Misura quantitativa utilizzata per valutare, quantificare e analizzare diversi aspetti del processo di sviluppo del software, del prodotto software stesso o della gestione del
progetto. Le metriche forniscono dati numerici che consentono di valutare l’andamento del
progetto, la qualit`a del software, l’efficacia dei processi e altri aspetti rilevanti.

\subsection*{Milestone}
In ingegneria del software e nella gestione dei progetti, punto di riferimento o traguardo significativo che sancisce il termine di un periodo nel ciclo 
di vita di un progetto. Le milestone rappresentano generalmente eventi chiave, compimenti o obiettivi importanti che indicano il progresso del progetto. 
L’obiettivo che ci si pone durante una milestone è realizzare una baseline.

\hypertarget{sec:mock}{}
\subsection*{Mock}
Un oggetto simulato che riproduce il comportamento di un oggetto reale in modo controllato. I mock sono spesso utilizzati nei test di unità per sostituire
le dipendenze esterne e garantire che il codice venga testato in isolamento.

\hypertarget{sec:model_mvvm}{}
\subsection*{Model (MVVM)}
Nel pattern architetturale MVVM (Model-View-ViewModel), il Model rappresenta i dati e la logica di business dell'applicazione. Il Model è responsabile della
gestione dei dati, della logica di business e delle operazioni di accesso ai dati, senza dipendere dalla View o dal ViewModel. Il Model fornisce i dati
necessari alla View attraverso il ViewModel, che li prepara e li esporta in un formato adatto alla visualizzazione.

\hypertarget{sec:modello_di_embedding}{}
\subsection*{Modello di embedding}
Vedi \bulhyperlink{sec:embedding}{Embedding (modello di)}.

\hypertarget{sec:modello_di_question_answering}{}
\subsection*{Modello di question answering}
Vedi \bulhyperlink{sec:question_answering}{Question answering (Modello di)}.

\hypertarget{sec:MVC}{}
\subsection*{MVC}
MVC (Model-View-Controller) è un pattern architetturale che suddivide un'applicazione in tre componenti principali.
Il Model gestisce i dati e la logica dell'applicazione, mentre la View si occupa della presentazione e dell'interfaccia utente. 
Il Controller, infine, funge da intermediario tra il Model e la View, gestendo le interazioni dell'utente e aggiornando i dati o la visualizzazione di conseguenza.

\hypertarget{sec:MVP}{}
\subsection*{Minimum Viable Product (MVP)}
Versione ridotta di un prodotto, che incorpora solo le funzioni essenziali per soddisfare le esigenze di base. Viene utilizzato per rilasciare un prodotto 
come test e ricevere feedback dall’utenza per migliorare poi il prodotto finito con tutte le funzionalità.

\hypertarget{sec:MVVM}{}
\subsection*{MVVM}
MVVM (Model-View-ViewModel) è un pattern architetturale che separa l'applicazione in tre componenti. 
Il Model si occupa dei dati e della logica, mentre la View gestisce la visualizzazione. 
Il ViewModel, a differenza del Controller di MVC, prepara i dati in un formato che la View può legare facilmente, spesso tramite data binding, e non gestisce la logica di navigazione. 
Questo approccio è particolarmente utile nelle applicazioni che supportano il data binding, come nelle app mobili.
\newpage


% Intestazione
\fancyhead[L]{N} % Testo a sinistra

\section{}

\hypertarget{sec:nestjs}{}
\subsection*{NestJS}
Framework Node.js per lo sviluppo di applicazioni lato server, basato su TypeScript. Utilizza un'architettura modulare e concetti come iniezione delle 
dipendenze e decoratori, ispirandosi a framework come Angular, per creare applicazioni scalabili e facilmente mantenibili.

\hypertarget{sec:nextjs}{}
\subsection*{Next.js}
Un framework open-source full stack basato su React, progettato per lo sviluppo di applicazioni web moderne. Consente la gestione sia del frontend sia 
del backend, grazie al supporto integrato per API e funzioni serverless.

\hypertarget{sec:nodejs}{}
\subsection*{Node.js}
Runtime JavaScript open-source e multipiattaforma, progettato per l'esecuzione di codice JavaScript lato server. Basato sul motore V8 di Google Chrome, 
permette di creare applicazioni veloci, scalabili e basate su eventi, sfruttando un modello non bloccante per la gestione delle operazioni I/O.

\subsection*{Norme di Progetto}
Insieme di linee guida, procedure e regole stabilite per regolare e standardizzare l’approccio, il processo e l’output del lavoro all’interno del progetto. 
Queste norme possono riguardare diversi aspetti del progetto, come la gestione del codice, la documentazione, la comunicazione e la gestione dei rischi. 
L’obiettivo delle norme di progetto è promuovere la coerenza, la qualità e l’efficienza nel processo di sviluppo del software, consentendo al team di 
lavorare in modo più efficace e collaborativo.

\newpage


% Intestazione
\fancyhead[L]{O} % Testo a sinistra

\section{}

\hypertarget{sec:onboarding}{}
\subsection*{Onboarding}
Il processo di integrazione e formazione di un nuovo dipendente, collaboratore o utente in un'organizzazione o piattaforma. L'onboarding ha l'obiettivo di 
fornire tutte le informazioni e le risorse necessarie per adattarsi al nuovo ambiente di lavoro, comprendere la cultura aziendale, acquisire competenze 
specifiche e diventare operativi nel più breve tempo possibile. 

\hypertarget{sec:openai}{}
\subsection*{OpenAI}
OpenAI è un'organizzazione di ricerca e sviluppo nel campo dell'intelligenza artificiale, fondata nel 2015. Si dedica alla creazione di AI avanzate e 
sicure, con l'obiettivo di garantire che i benefici dell'intelligenza artificiale siano condivisi da tutta l'umanità. OpenAI è nota per lo sviluppo di 
modelli come GPT (Generative Pre-trained Transformer), ChatGPT e DALL·E, utilizzati in applicazioni come chatbot, creazione di contenuti, e generazione 
di immagini. Combina ricerca all'avanguardia con implementazioni pratiche per promuovere l'accessibilità e la sostenibilità dell'AI.


\newpage