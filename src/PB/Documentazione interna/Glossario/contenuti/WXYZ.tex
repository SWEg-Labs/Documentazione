% Intestazione
\fancyhead[L]{W} % Testo a sinistra

\section{}

\hypertarget{sec:way_of_working}{}
\subsection*{Way of working}
Il termine “Way of working” (modo di lavorare) si riferisce al modo in cui un individuo, un team o un’organizzazione svolge le proprie attività lavorative. 
Questo concetto può includere processi, metodologie, abitudini, strumenti e culture aziendali che influenzano la gestione del lavoro e la collaborazione. 
Un modo di lavorare efficace ed efficiente può migliorare la produttività, la qualità del lavoro e la soddisfazione dei membri del team, promuovendo un 
ambiente di lavoro positivo e collaborativo. \\
Il way of working deve essere:
\begin{itemize}
    \item \textbf{Sistematico}: non deve dipendere dalle singole persone, bensì deve essere adattabile anche a futuri membri del team;
    \item \textbf{Disciplinato}: deve far seguire norme di comportamento che blocchino l'istinto e favoriscano il ragionamento;
    \item \textbf{Quantificabile}: si deve poter verificare concretamente la sua implementazione.
\end{itemize}

\hypertarget{sec:web_server}{}
\subsection*{Web server}
Un web server è un'applicazione software che, in esecuzione su un computer dedicato o virtuale, riceve richieste HTTP da client 
(solitamente browser web) e restituisce le risorse richieste, come pagine HTML, file CSS, JavaScript, immagini e altri asset digitali. 
Opera come un endpoint di rete, ascoltando su una specifica porta TCP (tipicamente la 80 o la 443 per HTTPS) per le connessioni in 
ingresso.

\hypertarget{sec:websocket}{}
\subsection*{WebSocket}
WebSocket è una tecnologia che fornisce un canale di comunicazione bidirezionale e full-duplex su una singola connessione TCP.
È progettato per essere implementato sia dai browser che dai server web, consentendo una comunicazione in tempo reale tra client e server.
WebSocket è spesso utilizzato per applicazioni web che richiedono aggiornamenti in tempo reale o comunicazione bidirezionale, come chatbot, giochi online,
applicazioni di collaborazione e dashboard di monitoraggio.

\hypertarget{sec:whisper.ai}{}
\subsection*{Whisper.ai}
Sistema di riconoscimento automatico del parlato (ASR) sviluppato da OpenAI.
È progettato per trascrivere il linguaggio parlato in testo con alta precisione.

\newpage



% Lettere inutilizzate
\begin{comment}

% Intestazione
\fancyhead[L]{X} % Testo a sinistra

\section{}

\dots

\newpage


% Intestazione
\fancyhead[L]{Y} % Testo a sinistra

\section{}

\dots

\newpage


% Intestazione
\fancyhead[L]{Z} % Testo a sinistra

\section{}

\dots

\end{comment}
