% Intestazione
\fancyhead[L]{A \hspace{0.2cm} Metriche per la qualità} % Testo a sinistra
\pagenumbering{roman} % Fa ripartire la numerazione romana delle pagine da I

\appendix % Cambia la numerazione delle sezioni da numeri a lettere

\section{Metriche per la qualità}
\label{sec:metriche_qualita}

\subsection{Metriche per la qualità di processo}
Le metriche per la qualità di processo valutano la qualità dei processi adottati durante 
lo sviluppo del software. Esse includono metriche per miglioramento, fornitura, 
\emph{codifica}\textsubscript{\textit{\textbf{G}}} e documentazione, fornendo indicazioni su come i processi 
possono essere ottimizzati per migliorare la qualità del prodotto finale:
\begin{itemize}
    \item \textbf{Miglioramento}: queste metriche valutano l’efficacia dei processi di miglioramento continuo implementati
    nel ciclo di vita dello sviluppo del software.
    \item \textbf{Fornitura}: le metriche di fornitura misurano la qualità del processo di consegna del software, compreso 
    il rispetto dei tempi, la gestione delle risorse e la conformità agli standard.
    \item \textbf{Codifica}: queste metriche valutano la qualità del processo di scrittura del codice, inclusa la correttezza 
    sintattica, la chiarezza e la conformità agli standard di codifica.
    \item \textbf{Documentazione}: misurano la qualità della documentazione associata al software, fornendo indicazioni sulla
    chiarezza e completezza della documentazione.
\end{itemize}

\subsubsection{Metriche utilizzate}
\subsubsubsection{Varianza di Budget}
\textbf{Codice:} MPC-1 \\
\textbf{Processo:} Fornitura. \\
\textbf{Formula:}
\[
100 \cdot \frac{\text{Budget Consuntivato} - \text{Budget Preventivato}}{\text{Budget Preventivato}}
\]
\textbf{Descrizione:} Questa metrica valuta la percentuale di variazione del budget tra preventivo e consuntivo in uno sprint. Il valore è positivo quando viene preventivato un budget inferiore a quello effettivamente utilizzato, mentre è negativo quando viene preventivato un budget maggiore a quello effettivamente utilizzato.

\subsubsubsection{Varianza dell’impegno orario}
\textbf{Codice:} MPC-2 \\
\textbf{Processo:} Fornitura. \\
\textbf{Formula:}
\[
100 \cdot \frac{\text{Ore Consuntivate} - \text{Ore Preventivate}}{\text{Ore Preventivate}}
\]
\textbf{Descrizione:} Questa metrica valuta la percentuale di variazione dell’impegno orario complessivo tra preventivo e consuntivo in uno sprint. Il valore è positivo quando viene preventivato un impegno orario inferiore a quello effettivamente svolto, mentre è negativo quando viene preventivato un impegno orario maggiore a quello effettivamente svolto.

\subsubsubsection{Earned Value}
\textbf{Codice:} MPC-3 \\
\textbf{Processo:} Fornitura. \\
\textbf{Formula:}
\[
\text{Budget Preventivato} \cdot \% \text{Completamento di Attività Sprint}
\]
\textbf{Descrizione:} Questa metrica rappresenta il valore effettivo del lavoro realizzato alla fine di uno sprint. Se l’Earned Value è maggiore dell’Actual Cost, significa che è stato speso meno del previsto. Al contrario, se è minore, significa che è stato speso di più del previsto.

\subsubsubsection{Actual Cost}
\textbf{Codice:} MPC-4 \\
\textbf{Processo:} Fornitura. \\
\textbf{Formula:}
\[
\sum_{i=1}^{n\_Sprint} \text{Budget Consuntivato}_i
\]
\textbf{Descrizione:} Questa metrica rappresenta il costo totale effettivamente sostenuto in base al lavoro eseguito nello sprint.

\subsubsubsection{Planned Value}
\textbf{Codice:} MPC-5 \\
\textbf{Processo:} Fornitura. \\
\textbf{Formula:}
\[
\text{Actual Cost}_{\text{sprint-1}} + \text{Budget Preventivato}_{\text{sprint}}
\]
\textbf{Descrizione:} Rappresenta il totale dei costi pianificati allo sprint e viene calcolata prima che esso inizi.

\subsubsubsection{Cost Variance}
\textbf{Codice:} MPC-6 \\
\textbf{Processo:} Fornitura. \\
\textbf{Formula:}
\[
\text{Earned Value} - \text{Actual Cost}
\]
\textbf{Descrizione:} Rappresenta lo scostamento dai costi pianificati. Un valore positivo indica che il lavoro effettivamente prodotto è costato meno di quanto preventivato, mentre un valore negativo indica il contrario.

\subsubsubsection{Schedule Variance}
\textbf{Codice:} MPC-7 \\
\textbf{Processo:} Fornitura. \\
\textbf{Formula:}
\[
\text{Earned Value} - \text{Planned Value}
\]
\textbf{Descrizione:} Indica lo scostamento dai tempi pianificati.

\subsubsubsection{Cost Performance Index}
\textbf{Codice:} MPC-8 \\
\textbf{Processo:} Fornitura. \\
\textbf{Formula:}
\[
\frac{\text{Earned Value}}{\text{Actual Cost}}
\]
\textbf{Descrizione:} Rappresenta l’efficienza economica del progetto.

\subsubsubsection{Schedule Performance Index}
\textbf{Codice:} MPC-9 \\
\textbf{Processo:} Fornitura. \\
\textbf{Formula:}
\[
\frac{\text{Earned Value}}{\text{Planned Value}}
\]
\textbf{Descrizione:} Rappresenta l’efficienza temporale del progetto.

\subsubsubsection{Estimate to Complete}
\textbf{Codice:} MPC-10 \\
\textbf{Processo:} Fornitura. \\
\textbf{Formula:}
\[
\frac{\text{Budget at Completion} - \text{Earned Value}}{\text{Cost Performance Index}}
\]
\textbf{Descrizione:} Indica il costo totale ancora da sostenere per il completamento del progetto.

\subsubsubsection{Estimate at Completion}
\textbf{Codice:} MPC-11 \\
\textbf{Processo:} Fornitura. \\
\textbf{Formula:}
\[
\text{Actual Cost} + \text{Estimate to Complete}
\]
\textbf{Descrizione:} Indica il costo totale alla fine del progetto in base all’andamento attuale.

\subsubsubsection{Budget at Completion}
\textbf{Codice:} MPC-12 \\
\textbf{Processo:} Fornitura. \\
\textbf{Descrizione:} Indica il budget totale del progetto.

\subsubsubsection{Code Coverage}
\textbf{Codice:} MPC-13 \\
\textbf{Processo:} Sviluppo. \\
\textbf{Descrizione:} Percentuale di codice attraversato dai test rispetto al totale della codebase.

\subsubsubsection{Misure di mitigazione insufficienti}
\textbf{Codice:} MPC-14 \\
\textbf{Processo:} Risoluzione dei problemi. \\
\textbf{Descrizione:} Indica il numero totale di misure di mitigazione previste che si sono rivelate insufficienti.

\subsubsubsection{Rischi inattesi}
\textbf{Codice:} MPC-15 \\
\textbf{Processo:} Risoluzione dei problemi. \\
\textbf{Descrizione:} Indica il numero totale di rischi inattesi (non analizzati) che si sono verificati.


\newpage
\subsection{Metriche di qualità di prodotto}
Le metriche per la qualità di prodotto valutano le caratteristiche del software come funzionalità, 
usabilità, manutenibilità e altre. Queste metriche forniscono indicazioni specifiche
sulla qualità del prodotto software in termini di conformità agli standard e soddisfazione
degli utenti.
\begin{itemize}
    \item \textbf{Funzionalità}: queste metriche valutano la completezza e la correttezza delle funzionalità del software,
    assicurando che risponda adeguatamente ai requisiti
    \item \textbf{Usabilità}: misurano la facilità con cui gli utenti possono interagire con il software, considerando aspetti
    come la comprensibilità, l’apprendibilità e l’operabilità.
\end{itemize}

\subsubsection{Metriche utilizzate}
\subsubsubsection{Indice di Gulpease}
\textbf{Codice:} MPD-1 \\
\textbf{Processo:} Documentazione. \\
\textbf{Formula:}
\[
89 + \frac{300 \cdot (\text{Numero Frasi}) - 10 \cdot (\text{Numero Lettere})}{\text{Numero Parole}}
\]
\textbf{Descrizione:} Misura il grado di leggibilità di un testo su una scala da 1 a 100. Un valore minimo accettabile è 40, mentre un valore preferibile è 60 o più.

\subsubsubsection{Errori ortografici}
\textbf{Codice:} MPD-2 \\
\textbf{Processo:} Documentazione. \\
\textbf{Descrizione:} Indica il numero di errori grammaticali presenti nella documentazione. Il valore accettabile è 5, mentre il valore preferibile è 0.


\subsubsubsection{Requisiti obbligatori soddisfatti}
\textbf{Codice:} MPD-3 \\
\textbf{Processo:} Sviluppo. \\
\textbf{Formula:}
\[
100 \cdot \frac{\text{Numero Requisiti Obbligatori Soddisfatti}}{\text{Numero Requisiti Obbligatori}}
\]
\textbf{Descrizione:} Indica la percentuale di requisiti obbligatori soddisfatti. Deve raggiungere il 100\%.

\subsubsubsection{Requisiti desiderabili soddisfatti}
\textbf{Codice:} MPD-4 \\
\textbf{Processo:} Sviluppo. \\
\textbf{Formula:}
\[
100 \cdot \frac{\text{Numero Requisiti Desiderabili Soddisfatti}}{\text{Numero Requisiti Desiderabili}}
\]
\textbf{Descrizione:} Indica la percentuale di requisiti desiderabili soddisfatti. Il valore accettabile è 0\%, mentre il valore preferibile è 100\%.

\subsubsubsection{Requisiti opzionali soddisfatti}
\textbf{Codice:} MPD-5 \\
\textbf{Processo:} Sviluppo. \\
\textbf{Formula:}
\[
100 \cdot \frac{\text{Numero Requisiti Opzionali Soddisfatti}}{\text{Numero Requisiti Opzionali}}
\]
\textbf{Descrizione:} Indica la percentuale di requisiti opzionali soddisfatti. Il valore accettabile è 0\%, mentre il valore preferibile è 100\%.