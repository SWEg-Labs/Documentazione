% Intestazione
\fancyhead[L]{2 \hspace{0.2cm} Processi primari} % Testo a sinistra


\section{Processi primari}
\label{sec:processi_primari}
\subsection{Fornitura}
\subsubsection{Descrizione}
Questa sezione riporta tutte le norme, gli strumenti e i metodi che ogni membro del gruppo \emph{SWEg Labs} si impegna a rispettare al fine di preservare al meglio i rapporti con il proponente \emph{AzzurroDigitale}\textsubscript{\textit{\textbf{G}}}.
\subsubsection{Scopo}
Il \emph{processo}\textsubscript{\textit{\textbf{G}}} di fornitura intende occuparsi della gestione dei rapporti con il proponente \emph{AzzurroDigitale} con l’obbiettivo di evitare qualsiasi tipo di ostacolo alla comunicazione ed avere un a buona qualità nella stessa.
\subsubsection{Aspettative}
Durante il rapporto il nostro gruppo desidera mantenere una comunicazione disponibile e con \emph{AzzurroDigitale}, in particolare con i referenti Martina Daniele, Camilla Picello, Nicola Boscaro e Mattia Gottardello, così da poter:
\begin{itemize}
    \item Discutere \emph{requisiti}\textsubscript{\textit{\textbf{G}}} chiave necessari da soddisfare nel prodotto finale;
    \item Stabilire tempistiche di lavoro;
    \item Ricevere \emph{feedback}\textsubscript{\textit{\textbf{G}}} sul lavoro in corso;
    \item Ottenere chiarimenti relativi a dubbi e incomprensioni;
    \item Stabilire i \emph{vincoli}\textsubscript{\textit{\textbf{G}}} riguardanti i processi intermedi.
\end{itemize}
\subsubsection{Fasi della Fornitura}
\label{sec:Fasi della Fornitura}
Il processo di fornitura si articola nelle seguenti fasi:
\begin{itemize}
    \item Avvio;
    \item Preparazione della proposta;
    \item Contrattazione;
    \item Pianificazione;
    \item Realizzazione e controllo;
    \item Revisione e Valutazione;
    \item Consegna e Completamento.
\end{itemize}
\subsubsubsection{Avvio}L'attività consiste nelL'attività di avvio consiste nel primo contatto tra il fornitore e il proponente. Durante questa fase, vengono stabiliti i primi accordi e vengono raccolte le informazioni preliminari necessarie per comprendere le esigenze del proponente.
\subsubsubsection{Preparazione della proposta}L'attività si articola nel compito che prevede la formulazione, da parte del fornitore, di una proposta di fornitura che risponda ai requisiti del proponente. La proposta deve essere chiara e dettagliata, in modo da permettere al proponente di valutarla in modo corretto. La preparazione della proposta avviene tramite la redazione di una lettera di presentazione rivolta ai committenti, corredata da un documento indicante il preventivo dei costi e dei tempi necessari alla realizzazione del prodotto, e da un documento che riporta l'analisi dettagliata dell'opportunità.
\subsubsubsection{Contrattazione}L'attività si articola nel compito che prevede la definizione di un contratto tra fornitore e proponente. Il contratto deve essere chiaro e dettagliato, in modo da permettere ad entrambe le parti di avere un quadro chiaro delle attività da svolgere e dei tempi di realizzazione. La contrattazione avviene tramite l'invio della documentazione redatta nella fase precedente al proponente, e la stipula consiste nell'accettazione della proposta da parte del proponente. Da quel momento varranno i vincoli espressi nel documento di preventivo.
\subsubsubsection{Pianificazione}
L'attività di pianificazione consiste nella definizione delle attività da svolgere per la realizzazione del prodotto. Ciò avviene tramite la redazione della documentazione necessaria, e si articola nei seguenti compiti che il fornitore deve svolgere:
\begin{enumerate}
    \item Vengono definiti i requisiti del progetto, i quali vanno poi raccolti in un documento apposito denominato analisi dei requisiti. Andranno a determinare i test necessari alla verifica e validazione del prodotto finale del progetto;
    \item Viene stabilito e adottato un modello di ciclo di vita del software adeguato alla grandezza, alla complessità e alla portata del progetto;
    \item Devono essere stabiliti, da parte del fornitore, i requisiti necessari alla gestione e verifica del progetto e del prodotto, oltre che alla verifica della loro qualità;
    \item Deve essere sviluppato e documentato il piano di gestione del progetto, basandosi sui requisiti individuati al punto precedente, e deve essere redatta una documentazione per descrivere tale piano. Questa documentazionie si baserà sui seguenti punti:
    \begin{itemize}
        \item Struttura organizzativa del progetto e delle autorità. Sono identificati dei ruoli, con compiti, costi e responsabilità differenti, che i membri del team fornitore dovranno ricoprire lungo tutto il progetto, variando tra essi;
        \item Ambiente tecnico. Devono essere riportati gli strumenti a sostegno della realizzazione dei compiti e delle relative attività come parte dei processi in atto nel progetto;
        \item Scomposizione dei processi e delle attività in task eseguibili, in considerazione del budget, delle risorse e del personale disponibile. Nel piano di progetto, la pianificazione delle risorse deve dar luogo ad unpreventivo dei costi, in termini di tempo e risorse, necessari al completamento del progetto. Le task eseguibili saranno assegnate tramite ticket a singoli membri del gruppo, coordinando il lavoro del team tramite unopportuno framework di project management;
        \item Gestione delle caratteristiche qualitative dei prodotti dei processi. In tal senso, deve essere redatto un documento contenente tali metriche, denominato piano di qualifica;
        \item Accertamento della qualità, con il relativo processo di supporto;
        \item Verifica e validazione, con i relativi processi di supporto;
        \item Revisioni congiunte con il cliente, con il relativo processo di supporto mirato al coinvolgimento costante del proponente nella realizzazione del progetto. Un’opportuna verbalizzazione di tali incontri andrà a documentare il loro effettivo avvenimento;
        \item Gestione dei rischi. Un omonimo capitolo deve essere redatto all’interno del piano di progetto, andando ad individuare tutti i potenziali rischi riscontrabili, stimando occorrenza e pericolosità, e identificando un’azione mitigativa da intraprendere in caso di occorrenza effettiva della problematica;
        \item Mezzi di sostegno alla pianificazione e all’analisi dell’avanzamento effettivo. Per facilitare le operazioni di pianificazione, devono essere prodotti dei diagrammi di Gantt , con cui rappresentare graficamente la dislocazione temporale delle attività, rappresentandone la durata, la sequenzialità e il parallelismo;
\end{itemize}
\subsubsubsection{Realizzazione e controllo}
L'attività di realizzazione e controllo consiste nello sviluppo del prodotto software secondo quanto pianificato e nella verifica continua del lavoro svolto. Questa fase si articola nei seguenti compiti:
\begin{itemize}
    \item Osservazione oculata di ciò che prevede il piano di gestione del progetto sviluppato nell’attività precedente;
    \item Sviluppo del codice sorgente, implementazione delle funzionalità e integrazione dei vari componenti del sistema;
    \item Monitoraggio dello stato di avanzamento in relazione all’utilizzo di risorse, sia in termini di budget che di personale, tramite la realizzazione di consuntivi di periodo, analizzando il discostamento di questi ultimi rispetto le stime preventivate nella fase di pianificazione. A sostegno di tale compito, opportune metriche saranno adottate per costituire i grafici del cruscotto valutativo, o dashboard, presente nel documento denominato piano di qualifica;
    \item Identificazione di problemi con opportune fasi retrospettive, documentando ed analizzando questi momenti, con l’intento di giungere ad una soluzione e ad un miglioramento continuo.
\end{itemize}
\subsubsubsection{Revisione e Valutazione}
L'attività consiste L'attività di revisione e valutazione consiste nel verificare che il prodotto sviluppato sia conforme ai requisiti e alle aspettative del proponente. Questa fase deve rispettare i seguenti criteri:
\begin{itemize}
    \item È nell’interesse del team cercare quanto più il confronto conl’azienda proponente, così da ricevere costantemente feedback sull’operato del gruppo. Il fornitore deve essere parte attiva nella coordinazione delle comunicazioni con l’acquirente;
    \item Il dialogo con il proponente deve essere supportato dal relativo processo di supporto Revisioni congiunte con il proponente;
    \item Deve essere garantito il rispetto di quanto indicato nei processi di supporto relativi a Verifica e Validazione, e in particolare nel documento piano di qualifica, in modo da dimostrare che prodotti e processi rispettano pienamenti loro relativi requisiti;
    \item Devono essere messe a disposizione dell’acquirente le documentazioni relative all’analisi dei requisiti e alla qualità dei prodotti del progetto;
    \item Deve essere garantito il corretto svolgimento di quanto indicato nel processo di supporto Accertamento della qualità.
\end{itemize}
\subsubsubsection{Consegna e Completamento}
L'attività consiste L'attività di consegna e completamento consiste nella formalizzazione della consegna del prodotto finale al proponente.

\subsubsection{Rapporti col proponente}
Il proponente mette a disposizione un canale Discord al fine di rispondere a domande/dubbi occasionali, e una mail di riferimento per comunicazioni più formali.

Inoltre il proponente si rende disponibile a degli incontri da remoto, la cui frequenza è di una volta ogni 2 settimane.
E in aggiunta due incontri in presenza, di cui il primo ad inizio progetto e il secondo nella parte finale.
Ad ognuno di questi incontri la discussione verterà su 2 punti:
\begin{itemize}
    \item \emph{Revisione}\textsubscript{\textit{\textbf{G}}} sul lavoro svolto nel precedente sprint;
    \item Raccolta delle nuove \emph{specifiche}\textsubscript{\textit{\textbf{G}}} e richieste da soddisfare per lo sprint successivo.
\end{itemize}
Durante tali incontri, l’azienda non richiede della specifica documentazione, ma gradisce strumenti per visualizzare, anche graficamente, lo stato di avanzamento del lavoro appena svolto.
Per ciascun incontro verrà compilato dal nostro gruppo un verbale esterno, contenente gli argomenti di discussione e le decisioni prese, e sarà firmato dalla rappresentanza del proponente.
Tutti i verbali saranno visibili nella \emph{repository}\textsubscript{\textit{\textbf{G}}} dedicata alla documentazione.

\subsubsection{Documentazione fornita}
\label{sec:documentazione_fornita}
Di seguito viene riportato l'elenco completo dei documenti che il gruppo \emph{SWEg Labs} si impegna a fornire al proponente \emph{AzzurroDigitale}\textsubscript{\textit{\textbf{G}}}
e ai \emph{committenti}\textsubscript{\textit{\textbf{G}}} Prof. Tullio Vardanega e Prof. Riccardo Cardin.

\subsubsubsection{Analisi dei Requisiti}
L'\emph{Analisi dei Requisiti}\textsubscript{\textit{\textbf{G}}} è un documento che descrive in modo dettagliato i requisiti del progetto, i \emph{casi d'uso}\textsubscript{\textit{\textbf{G}}} e le funzionalità che il prodotto dovrà avere. 
Questo documento ha lo scopo di chiarire eventuali dubbi e ambiguità che possono presentarsi dopo la lettura del \emph{capitolato}\textsubscript{\textit{\textbf{G}}}.
Nella sezione \S\bulref{sec:analisi_dei_requisiti} è possibile trovare una descrizione più dettagliata dell'analisi dei requisiti.
Il documento conterrà:
\begin{itemize}
    \item \textbf{Descrizione del prodotto};
    \item \textbf{Lista dei casi d'uso}: elenca tutti i possibili scenari di utilizzo del sistema software
    da parte degli utenti finali, ovvero le azioni o attività che essi possono svolgere con il
    sistema. Tutti i casi d’uso sono dotati di una descrizione dettagliata delle azioni che
    l’utente compie, in modo da far emergere tutti i requisiti che non erano ovvi dopo la
    sola lettura del capitolato;
    \item \textbf{Requisiti}: elenco dei vincoli richiesti dal \emph{proponente}\textsubscript{\textit{\textbf{G}}} o dedotti in seguito all'individuazione dei casi d'uso ad essi collegati;
\end{itemize}

\subsubsubsection{Piano di Progetto}
Il \emph{Piano di Progetto}\textsubscript{\textit{\textbf{G}}} è un documento che tratta i seguenti punti:
\begin{itemize}
    \item \textbf{Analisi dei rischi}: vengono analizzati eventuali rischi che potrebbero emergere durante lo sviluppo del progetto. 
    Ad ogni rischio individuato viene associata una strategia di mitigazione, così da risolvere o ridurre l’entità del problema 
    in caso di occorrenza del rischio in questione;
    \item \textbf{Modello di sviluppo}: viene delineato l’approccio metodologico impiegato durante il
    \emph{processo}\textsubscript{\textit{\textbf{G}}} di sviluppo del \emph{prodotto software}\textsubscript{\textit{\textbf{G}}};
    \item \textbf{Pianificazione}: viene pianificato il periodo temporale relativo a ciascuna attività da svolgere, con relativa descrizione;
    \item \textbf{Preventivo}: dalla durata dei periodi per poter completare tutte le attività si ricava il preventivo. 
    Al termine di ogni periodo verrà redatto il consuntivo che metterà a confronto il preventivo con quanto realmente realizzato, 
    così da visualizzare lo stato di avanzamento del progetto;
    \item \textbf{Consuntivo}: in questo capitolo sono indicate le spese effettive nelle diverse fasi del progetto.
\end{itemize}

\subsubsubsection{Piano di Qualifica}
Il \emph{Piano di Qualifica}\textsubscript{\textit{\textbf{G}}} è un documento che descrive le attività e le strategie adottate per
garantire la qualità del \emph{prodotto software}\textsubscript{\textit{\textbf{G}}}. In esso vengono descritte le metodologie, le
tecniche e gli strumenti che verranno utilizzati per far sì che il prodotto software sia in linea
con le aspettative del \emph{proponente}\textsubscript{\textit{\textbf{G}}}. Tale documento è utile per la gestione del \emph{processo}\textsubscript{\textit{\textbf{G}}}
di sviluppo: in particolare, permette di monitorare lo stato di avanzamento del progetto in relazione agli obiettivi di qualità prefissati.
Il Piano di Qualifica tratta i seguenti punti:

\begin{itemize}
    \item \textbf{Qualità di processo}: definisce i parametri e le \emph{metriche}\textsubscript{\textit{\textbf{G}}} che ciascun membro del
    team è tenuto a rispettare al fine di garantire processi di alta qualità;
    \item \textbf{Qualità di prodotto}: definisce i parametri e le metriche che ciascun membro del team
    è tenuto a rispettare al fine di garantire un prodotto finale di alta qualità;
    \item \textbf{Strategie di testing}: descrive il piano di testing, con l’obiettivo di garantire la correttezza del prodotto finale;    
    \item \textbf{Obiettivi di qualità}: definisce i valori  che dovranno assumere le metriche per essere ritenute accettabili o pienamente soddisfatte;
    \item \textbf{Cruscotto delle metriche}: presenta il cruscotto delle metriche utilizzato durante il periodo dello sviluppo del progetto;
    \item \textbf{Valutazioni per il miglioramento}: analizza le criticità rilevate durante il processo di sviluppo e le azioni intraprese per agevolare tale processo.
\end{itemize}

\subsubsubsection{Lettera di Presentazione}
Ad ogni revisione di avanzamento del progetto è associata una \emph{Lettera di Presentazione}.
È un documento con il quale si formalizza la consegna della revisione di avanzamento in questione. 
In essa è presente l’elenco della documentazione che verà messa a disposizione del proponente e dei committenti. 
Il gruppo si impegna a rispettare i requisiti minimi e a consegnare il prodotto software entro la data prestabilita.

\subsubsubsection{Glossario}
Il \emph{Glossario}\textsubscript{\textit{\textbf{G}}} consiste in un elenco di termini che compaiono nei documenti, e relative definizioni, 
il cui significato può non essere immediato. È utile per evitare potenziali ambiguità ed agevolare la comunicazione tra i membri del gruppo.

\subsubsubsection{Manuale Utente}
Il \emph{Manuale Utente}\textsubscript{\textit{\textbf{G}}} è un documento essenziale progettato per fornire istruzioni dettagliate
sull’uso di un determinato prodotto. La sua funzione principale è quella di guidare gli utenti attraverso le varie funzionalità e operazioni disponibili, 
offrendo istruzioni passo-passo su come utilizzare efficacemente il software in questione. Questo documento fornisce informazioni chiare e concise 
sulle caratteristiche del prodotto, sui requisiti di sistema, sulle procedure di installazione e configurazione, nonchè sulle modalità 
di risoluzione dei problemi comuni.

\subsubsubsection{Specifica Tecnica}
La \emph{Specifica Tecnica}\textsubscript{\textit{\textbf{G}}} ha lo scopo di elencare e motivare le scelte architetturali prese per
la realizzazione dell’infrastruttura informatica di un prodotto, oltre a riportare e descrivere tutte le tecnologie e i linguaggi utilizzati.


\subsubsection{Strumenti}

Di seguito sono riportati gli strumenti utilizzati per realizzare il processo di fornitura:
\begin{itemize}
    \item \textbf{\emph{Git}}\textsubscript{\textit{\textbf{G}}}: software per il \emph{controllo di versione}\textsubscript{\textit{\textbf{G}}};
    \item \textbf{\emph{GitHub}}\textsubscript{\textit{\textbf{G}}}: servizio di \emph{hosting}\textsubscript{\textit{\textbf{G}}} per progetti software;
    \item \textbf{\emph{Jira}}\textsubscript{\textit{\textbf{G}}}: è un sistema software utilizzato per la gesitone delle attività, l’assegnazione delle
    risorse, la verifica dei tempi del progetto e l’analisi del lavoro svolto e da svolgere.
    Questo strumento è utile anche per generare i \emph{diagrammi di Gantt}\textsubscript{\textit{\textbf{G}}} presenti nel Piano
    di Progetto;
    \item \textbf{\emph{Discord}}\textsubscript{\textit{\textbf{G}}}: \emph{piattaforma}\textsubscript{\textit{\textbf{G}}} che mette a disposizione dei canali vocali con la possibilità di condivisione dello schermo;
    Utilizzata non solo dai membri del gruppo per comunicazioni interne ma anche per le comunicazioni rapide con il proponente;
    \item \textbf{\emph{Google Meet}}\textsubscript{\textit{\textbf{G}}}: piattaforma che permette di effettuare videoconferenze online. Utilizzata per organizzare incontri con il proponente;
    \item \textbf{\emph{\LaTeX}}\textsubscript{\textit{\textbf{G}}}: linguaggio di \emph{markup}\textsubscript{\textit{\textbf{G}}} scelto dal gruppo per la produzione della documentazione.
\end{itemize}

\subsection{Sviluppo}
\subsubsection{Scopo}
La fase di sviluppo si occupa di definire le attività che il team compie per soddisfare i requisiti delineati con il proponente.

\subsubsection{Descrizione}
Il processo di sviluppo consiste nello strutturare, suddividere ai membri del team e completare le attività relative alla \emph{codifica}\textsubscript{\textit{\textbf{G}}}. L’obbiettivo è che il software soddisfi le \emph{aspettative}\textsubscript{\textit{\textbf{G}}} del proponente.
Nel processo di sviluppo saranno effettuate le seguenti attività.
\begin{itemize}
    \item Analisi dei requisiti;
    \item Progettazione;
    \item Codifica.
\end{itemize}

\subsubsection{Aspettative}
Il gruppo \emph{SWEg Labs} intende ottenere tramite il processo di sviluppo un prodotto software in grado di superare i test e soddisfare i requisiti del proponente \emph{AzzurroDigitale}.

\subsubsection{Analisi dei requisiti}
\label{sec:analisi_dei_requisiti}

\subsubsubsection{Descrizione}
L’analisi dei requisiti è un’attività svolta dall’analista, e produce il documento denominato “Analisi dei requisiti”. 
Tale documento descrive:
\begin{itemize}
    \item Lo scopo del prodotto;
    \item Le sue \emph{funzionalità}\textsubscript{\textit{\textbf{G}}};
    \item Gli \emph{attori}\textsubscript{\textit{\textbf{G}}} e le loro caratteristiche;
    \item I \emph{casi d'uso}\textsubscript{\textit{\textbf{G}}};
    \item I requisiti;
    \item La stima del lavoro necessario.
\end{itemize}

\subsubsubsection{Scopo}
L’analisi dei requisiti intende raccogliere, chiarire e precisare tutti i requisiti necessari da completare per soddisfare il cliente. Per poter fare ciò è necessario aver letto e compreso al meglio le specifiche del progetto, e poter comunicare al meglio con il proponente.

\subsubsubsection{Casi d'uso}
Un \emph{caso d'uso}\textsubscript{\textit{\textbf{G}}} è un insieme di \emph{scenari}\textsubscript{\textit{\textbf{G}}} che hanno in comune uno scopo finale per un \emph{attore}\textsubscript{\textit{\textbf{G}}}.\\
Gli elementi che lo compongono sono i seguenti:
\begin{itemize}
    \item l'attore;
    \item il sistema;
    \item precondizioni;
    \item postcondizioni;
    \item \emph{Scenario principale}\textsubscript{\textit{\textbf{G}}};
    \item \emph{Scenario alternativo}\textsubscript{\textit{\textbf{G}}};
    \item inclusione/i;
    \item estensione/i;
    \item specializzazione/i;
    \item commento/i;
    \item descrizione.  
\end{itemize}
I casi d’uso sono identificati nel seguente modo:\\
\begin{center}
    \textbf{UC[Numero]+(UC[Numero sottocaso])-[Titolo]}
\end{center}
dove:
\begin{itemize}
    \item UC: acronimo di "use case";
    \item Numero: numero identificativo del caso d’uso;
    \item Numero sottocaso: numero identificativo del sottocaso (se presente);
    \item Titolo: titolo assegnato al caso d’uso.
\end{itemize}

\subsubsubsection{Struttura dei requisiti}
I requisiti sono identificati da un codice univoco strutturato nel seguente modo:\\
\begin{center}
    \textbf{R[Importanza][Tipologia] [Codice](+[Codice figlio])}
\end{center}
dove:
\begin{itemize}
    \item R: acronimo di "Requisito";
    \item Importanza: indica l’importanza del requisito e può assumere i seguenti valori:
    \begin{itemize}
        \item O: requisito obbligatorio, cioè deve essere soddisfatto necessariamente per garantire la realizzazione del prodotto corrispondente agli accordi col \emph{proponente}\textsubscript{\textit{\textbf{G}}};
        \item D: requisito desiderabile, cioè porterebbe al prodotto ulteriori funzionalità e completezza qualora fosse soddisfatto;
        \item Z: requisito opzionale, cioè che potrebbe essere implementato solo se ci sono risorse, tempo e budget sufficienti, senza che la sua mancanza impatti negativamente il prodotto.
    \end{itemize}
    \item Tipologia:
        \begin{itemize} 
            \item F: requisito funzionale che delinea gli obiettivi e le azioni chiave che l’utente deve essere in grado di compiere;
            \item Q: requisito qualitativo che delinea le specifiche qualitative che devono essere rispettate per garantire la qualità del sistema;
            \item V: requisito di vincolo che rappresenta le restrizioni e le condizioni che devono essere soddisfatte durante lo sviluppo e l’implementazione del sistema;
            \item I: requisito implementativo che delinea le specifiche tecniche e operative che indicano come un sistema o una funzionalità deve essere sviluppato per soddisfare i requisiti del progetto;  
            \item P: requisito prestazionale che definisce le metriche di performance che il sistema deve soddisfare. 
        \end{itemize}
    \item Codice: identificatore numerico univoco per quella tipologia di vincolo;
    \item Codice figlio: numero identificativo del sottorequisito (se presente).
\end{itemize}

\subsubsection{Progettazione}

\subsubsubsection{Scopo}
L'attività di progettazione ha l'obiettivo di definire l'architettura del prodotto in modo da soddisfare le esigenze di tutti gli \emph{stakeholder}\textsubscript{\textit{\textbf{G}}}, identificate dall’analisi dei requisiti. Identificando e documentando le soluzioni che rispondono ai requisiti evidenziati si garantisce al contempo una chiara suddivisione delle responsabilità di sviluppo e manutenzione. Questa attività fornisce una documentazione completa e dettagliata sulla struttura del prodotto, incluse le specifiche tecniche e le scelte tecnologiche adottate. 

\subsubsubsection{Descrizione}
La progettazione si articola in più livelli per assicurare una completa copertura delle funzionalità e della struttura del prodotto:
\begin{itemize}
    \item \textbf{Progettazione logica}: Questa fase definisce le tecnologie, i \emph{framework}\textsubscript{\textit{\textbf{G}}} e le librerie scelti per la realizzazione del prodotto, motivando l'adeguatezza delle scelte e dimostrando la fattibilità tecnica attraverso un \emph{Proof of Concept}\textsubscript{\textit{\textbf{G}}}(PoC);
    \item \textbf{Progettazione di dettaglio}: In questa fase si definisce l'\emph{architettura}\textsubscript{\textit{\textbf{G}}} di dettaglio, seguendo quanto stabilito nella progettazione logica e sviluppando una rappresentazione completa delle componenti software.
\end{itemize}

\subsubsection{Codifica}

\subsubsubsection{Scopo}
L’attività di \emph{codifica}\textsubscript{\textit{\textbf{G}}} è mirata allo sviluppo del prodotto software da parte dei programmatori, il quale deve soddisfare le esigenze concordate con il \emph{proponente}.

\subsubsubsection{Descrizione}

Durante questa attività, lo sviluppatore si impegna a soddisfare i requisiti implementando il codice nel linguaggio di programmazione scelto.
Il codice deve rispettare le linee guida definite nella documentazione del progetto. Contestualmente, tutte le nuove unità software sviluppate e le modifiche apportate devono essere adeguatamente documentate.

\subsubsubsection{Stile della codifica}
Per garantire lo sviluppo del codice di qualità useremo i seguenti criteri:

\begin{itemize}
    \item Backend:
    \begin{itemize}
        \item \textbf{Variabili, attributi, funzioni e metodi:} \emph{Snake Case}\textsubscript{\textit{\textbf{G}}}.
        \item \textbf{Classi:} \emph{Pascal Case}\textsubscript{\textit{\textbf{G}}}.
        \item \textbf{Nomi di file:}
        \begin{itemize}
            \item Camel Case se il file contiene una classe;
            \item Snake Case se non contiene una classe.
        \end{itemize}
        \item \textbf{Docstring}: Ogni metodo o funzione deve avere un commento descrittivo sotto alla firma, scritto in inglese.
    \end{itemize}
    
    \item Frontend:
    \begin{itemize}
        \item \textbf{Variabili, attributi, funzioni e metodi:} \emph{Camel Case}\textsubscript{\textit{\textbf{G}}}.
        \item \textbf{Classi:} Pascal Case.
        \item \textbf{Nomi dei file:} \emph{Kebab Case}\textsubscript{\textit{\textbf{G}}}.
    \end{itemize}
    
    \item \textbf{Lunghezza delle righe di codice}:
    \begin{itemize}
        \item La lunghezza massima di una riga di codice non deve superare i 100 caratteri.
    \end{itemize}
    
    \item \textbf{Indentazioni}:
    \begin{itemize}
        \item I blocchi annidati del codice devono seguire un'indentazione con un carattere di tabulazione equivalente a 4 spazi.
    \end{itemize}
\end{itemize}

\subsubsubsection{Tecnologie Utilizzate}
\label{sec:tecnologie_utilizzate}

\begin{itemize}
    \item \textbf{\emph{Python}\textsubscript{\textit{\textbf{G}}}}: un linguaggio di programmazione ad alto livello orientato ad oggetti. Si
    utilizza per realizzare la logica dell’applicazione.
    \begin{center}
        \textbf{\url{https://www.python.org/}} \\
        \emph{(Ultimo accesso: 03/04/2025)}
    \end{center}

    \item \textbf{\emph{Angular}\textsubscript{\textit{\textbf{G}}}}: un framework usato per lo sviluppo dell’interfaccia grafica dell’applicazione.
    \begin{center}
        \textbf{\url{https://angular.dev/}} \\
        \emph{(Ultimo accesso: 03/04/2025)}
    \end{center}
    
    \item \textbf{\emph{Chroma}\textsubscript{\textit{\textbf{G}}}}: un database vettoriale che memorizza e ricerca dati basandosi sulla loro somiglianza semantica.
    \begin{center}
        \textbf{\url{https://www.trychroma.com/}} \\
        \emph{(Ultimo accesso: 03/04/2025)}
    \end{center}

    \item \textbf{\emph{GPT-4o}\textsubscript{\textit{\textbf{G}}}}: un \emph{LLM}\textsubscript{\textit{\textbf{G}}}, sviluppato da OpenAI, capace di comprendere e generare testo in modo contestuale.
    \begin{center}
        \textbf{\url{https://openai.com/index/hello-gpt-4o/}} \\
        \emph{(Ultimo accesso: 03/04/2025)}
    \end{center}

    \item \textbf{\emph{Docker}\textsubscript{\textit{\textbf{G}}}}: una piattaforma per eseguire applicazioni in container, ambienti isolati che includono tutto il necessario per funzionare ovunque.
    \begin{center}
        \textbf{\url{https://www.docker.com}} \\
        \emph{(Ultimo accesso: 03/04/2025)}
    \end{center}
    
    \item \textbf{\emph{Python-Crontab}\textsubscript{\textit{\textbf{G}}}}: una libreria Python che permette di leggere, scrivere e gestire {\emph{cron job}\textsubscript{\textit{\textbf{G}}}} direttamente da codice Python.
    \begin{center}
        \textbf{\url{https://pypi.org/project/python-crontab/}} \\
        \emph{(Ultimo accesso: 03/04/2025)}
    \end{center}
   
\end{itemize}


