% Intestazione
\fancyhead[L]{A \hspace{0.2cm} Standard ISO/IEC 12207} % Testo a sinistra

\pagenumbering{roman} % Numerazione romana per l'appendice

\appendix % Cambia la numerazione delle sezioni da numeri a lettere


\section{Standard ISO/IEC 12207}
\label{sec:standard_iso-iec_12207}

L'\emph{ISO/IEC/IEEE 12207}\textsubscript{\textit{\textbf{G}}} definisce un framework completo per il ciclo di vita del software, identificando processi, attività e compiti necessari per sviluppare, gestire e mantenere software in modo sistematico. Implementare questo standard permette di mantenere un alto livello di qualità, tracciabilità e coerenza nei processi di sviluppo software. Fornisce un quadro di riferimento robusto per gestire i progetti, favorendo il miglioramento continuo e il rispetto delle tempistiche e dei requisiti iniziali.\\
Lo standard si articola in tre principali categorie di processi:
\begin{itemize}
    \item Processi primari
    \item Processi di supporto
    \item Processi organizzativi
\end{itemize}

\subsection{I processi primari}
I processi primari comprendono quelli necessari alla gestione e allo sviluppo del software, includendo sia la gestione del progetto sia il supporto tecnico.
\begin{itemize}
    \item \textbf{Processo di acquisizione}: Coinvolge tutte le attività di acquisizione del software o di servizi legati al software, come la definizione dei requisiti dell'acquirente, la selezione del fornitore, la gestione del contratto e il monitoraggio del progresso.
    \item \textbf{Processo di fornitura}: Descrive le attività del fornitore per sviluppare, modificare e consegnare il software in base alle specifiche di contratto. Include la pianificazione, lo sviluppo, la consegna e la gestione del software.
    \item \textbf{Processo di sviluppo}: Definisce le attività per creare o modificare il software. Le fasi principali sono:
    \begin{itemize}
        \item \textbf{Analisi dei requisiti}: Identificazione delle specifiche funzionali e non funzionali.
        \item \textbf{Progettazione del sistema e del software}: Progettazione dell'architettura e della struttura del software.
        \item \textbf{Implementazione}: Programmazione del codice sorgente.
        \item \textbf{Integrazione}: Assemblaggio delle diverse parti del software.
        \item \textbf{Testing}: Verifica del software per assicurare che rispetti i requisiti definiti.
        \item \textbf{Installazione e accettazione}: Rilascio del software e conferma della sua aderenza ai requisiti.
    \end{itemize}
    \item \textbf{Processo di gestione operativa}: Si occupa delle attività di gestione del software nel suo ambiente di produzione, incluse la manutenzione, il supporto utente, l'operatività continua e il monitoraggio delle performance.
    \item \textbf{Processo di manutenzione}: Comprende tutte le attività per correggere, migliorare o adattare il software dopo il rilascio, al fine di mantenerne o aumentarne l'efficienza e la rilevanza.
\end{itemize}

\subsection{I processi di supporto}
Questi processi assistono i processi primari e garantiscono che il software sia conforme agli standard di qualità.
\begin{itemize}
    \item \textbf{Processo di documentazione}: Definisce la creazione, la gestione e la manutenzione della documentazione di progetto.
    \item \textbf{Processo di configurazione}: Gestisce le versioni del software, tenendo traccia delle modifiche e assicurando che ogni versione sia stabile e rintracciabile.
    \item \textbf{Processo di verifica}: Garantisce che ogni fase dello sviluppo rispetti i requisiti iniziali attraverso attività di verifica e review, che coinvolgono sia il codice che la documentazione.
    \item \textbf{Processo di validazione}: Assicura che il software finale soddisfi le esigenze dell'utente, attraverso attività di testing e collaudo in condizioni reali.
    \item \textbf{Processo di qualità}: Definisce le attività di controllo qualità per monitorare la conformità agli standard e migliorare continuamente i processi di sviluppo.
    \item \textbf{Processo di revisione e audit}: Prevede revisioni periodiche e audit di progetto per identificare eventuali problemi o non conformità rispetto agli standard definiti.
\end{itemize}

\subsection{I processi organizzativi}
I processi organizzativi mirano a sostenere e migliorare i processi aziendali nel loro insieme, creando un ambiente di supporto che faciliti l'attività dei team.
\begin{itemize}
    \item \textbf{Processo di gestione}: Include tutte le attività di pianificazione, coordinamento e monitoraggio del progetto, come l'assegnazione delle risorse, la definizione del budget e la gestione del rischio.
    \item \textbf{Processo di miglioramento}: Riguarda le attività di analisi e ottimizzazione dei processi, come la raccolta di feedback, l'identificazione delle aree di miglioramento e la messa in atto di strategie di ottimizzazione.
    \item \textbf{Processo di formazione}: Definisce la formazione continua per il personale, migliorando le competenze tecniche e manageriali necessarie a svolgere i vari processi.
    \item \textbf{Processo di gestione delle risorse umane}: Include le attività di selezione, valutazione e gestione delle risorse umane impiegate nei vari processi.
\end{itemize}
