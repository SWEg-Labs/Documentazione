\thispagestyle{plain} % Niente intestazione e piè di pagina


\begin{tikzpicture}[remember picture, overlay]
    % Punto di partenza al centro orizzontale nella metà superiore
    \coordinate (top_center) at ($(current page.north)!0.3!(current page.south)$);

    % UniPD: Logo e descrizione
    \node at (top_center) [anchor=north, xshift=-3cm, yshift=4.85cm] 
        {\includegraphics[width=0.15\textwidth]{Logo Universita di Padova.png}};
    \node at (top_center) [anchor=north, xshift=1.7cm, yshift=4.5cm]
        {\textcolor{red}{\textbf{Università degli Studi di Padova}}};
    \node at (top_center) [anchor=north, xshift=1.7cm, yshift=4cm]
        {\textcolor{red}{Laurea: Informatica}};
    \node at (top_center) [anchor=north, xshift=1.7cm, yshift=3.5cm]
        {\textcolor{red}{Corso: Ingegneria del Software}};
    \node at (top_center) [anchor=north, xshift=1.7cm, yshift=3cm]
        {\textcolor{red}{Anno Accademico: 2024/2025}};

    % SWEg Labs: Logo e descrizione
    \node at (top_center) [anchor=north, xshift=-2.85cm, yshift=1.5cm] 
        {\includegraphics[width=0.16\textwidth]{Logo SWEg.png}};
    \node at (top_center) [anchor=north, xshift=1.7cm, yshift=0.5cm]
        {\textbf{Gruppo: SWEg Labs}};
    \node at (top_center) [anchor=north, xshift=1.7cm, yshift=0cm]
        {Email: \textsf{gruppo.sweg@gmail.com}};
\end{tikzpicture}


\vspace{10cm}

{
\centering
\Huge\bfseries Norme di Progetto\par
\vspace{0.5cm}
\Large Versione 2.0.0\par
}

\vspace{2cm}