% Intestazione
\fancyhead[L]{1 \hspace{0.2cm} Introduzione} % Testo a sinistra

\pagenumbering{arabic} % Numerazione araba per il contenuto



\section{Introduzione}
\label{sec:introduzione}

\subsection{Scopo del documento}
Per i progetti di sviluppo software che puntano a soddisfare gli elevati 
standard qualitativi delineati nei principi dell'ingegneria del software, 
un documento dettagliato del piano di qualità è indispensabile.
Comprendere e valutare la \emph{qualità}\textsubscript{\textit{\textbf{G}}} 
del prodotto sono concetti fondamentali 
per effettuare confronti e determinare quanto un prodotto aderisce alle aspettative.
Questo documento mira a fornire una descrizione completa e precisa delle 
\emph{metriche}\textsubscript{\textit{\textbf{G}}} e delle 
metodologie utilizzate per il controllo e la misurazione 
della qualità nelle diverse componenti del software.
Definiremo gli obiettivi di qualità, i processi e le risorse necessarie per 
raggiungerli, oltre a specificare i test previsti con la relativa documentazione, 
comprese metodologie ed esiti.
Il documento sarà una risorsa preziosa per chi sviluppa il prodotto, per gli 
utenti finali e per coloro che ne valuteranno la qualità.

\subsection{Glossario}
Al fine di evitare eventuali equivoci o incomprensioni riguardo la terminologia 
utilizzata all’interno di questo documento, 
abbiamo valutato di adottare un Glossario, con file apposito, in cui
vengono riportate tutte le definizioni rigogliose delle parole ambigue utilizzate 
in ambito di
questo progetto.\\
Nel documento appena descritto verranno riportati tutti i termini definiti nel
loro ambiente di utilizzo con annessa descrizione del loro significato.\\
La presenza di un termine all'interno del Glossario sarà indicata con una "G" 
posizionata al pedice della parola.

\subsection{Maturità e miglioramenti}
Questo documento è stato redatto seguendo un approccio incrementale, 
con l'obiettivo di facilitare l'adattamento alle esigenze mutevoli, stabilite 
di comune accordo tra i membri del gruppo di progetto e l'azienda proponente.\\
Pertanto, il documento non può essere considerato definitivo o esaustivo, ma 
piuttosto un punto di partenza per un continuo aggiornamento e affinamento.

\subsection{Riferimenti}
\label{sec:riferimenti}

\subsubsection{Riferimenti normativi}
\begin{itemize}
    \item \bulhref{https://sweg-labs.github.io/Documentazione/output/PB/Documentazione\%20interna/norme_progetto_v2.0.0.pdf}{Norme di Progetto v.2.0.0}; \\
    \item \textbf{Capitolato d'appalto C9 - BuddyBot (slide 3-18)}: \\
    \bulhref{https://www.math.unipd.it/~tullio/IS-1/2024/Progetto/C9.pdf}{https://www.math.unipd.it/~tullio/IS-1/2024/Progetto/C9.pdf}
    \emph{(Ultimo accesso: 03/04/2025)};\\
    \item \textbf{PD1 - Regolamento del Progetto Didattico (slide 2-25)}:\\
    \bulhref{https://www.math.unipd.it/~tullio/IS-1/2024/Dispense/PD1.pdf}{https://www.math.unipd.it/~tullio/IS-1/2024/Dispense/PD1.pdf}
    \emph{(Ultimo accesso: 03/04/2025)};\\
    \item \textbf{Standard ISO/IEC 31000:2018}: \\
    \bulhref{https://www.iso.org/obp/ui\#iso:std:iso:31000:ed-2:v1:en}{https://www.iso.org/obp/ui\#iso:std:iso:31000:ed-2:v1:en}
    \emph{(Ultimo accesso: 03/04/2025)}.\\
\end{itemize}

\subsubsection{Riferimenti informativi}
\begin{itemize}
    \item \bulhref{https://sweg-labs.github.io/Documentazione/output/PB/Documentazione\%20interna/glossario_v2.0.0.pdf}{Glossario v.2.0.0}; \\
    \item \textbf{T7 - Qualità del software (slide 2-30)}: \\
    \bulhref{https://www.math.unipd.it/~tullio/IS-1/2024/Dispense/T07.pdf}{https://www.math.unipd.it/~tullio/IS-1/2024/Dispense/T07.pdf}
    \emph{(Ultimo accesso: 03/04/2025)}; \\
    \item \textbf{T8 - Qualità di processo (slide 2-22)}: \\
    \bulhref{https://www.math.unipd.it/~tullio/IS-1/2024/Dispense/T08.pdf}{https://www.math.unipd.it/~tullio/IS-1/2024/Dispense/T08.pdf}
    \emph{(Ultimo accesso: 03/04/2025)}; \\
    \item \textbf{T9 - Verifica e validazione: introduzione (slide 2-25)}: \\
    \bulhref{https://www.math.unipd.it/~tullio/IS-1/2024/Dispense/T09.pdf}{https://www.math.unipd.it/~tullio/IS-1/2024/Dispense/T09.pdf}
    \emph{(Ultimo accesso: 03/04/2025)}; \\
    \item \textbf{T10 - Verifica e validazione: analisti statica (slide 2-26)}: \\
    \bulhref{https://www.math.unipd.it/~tullio/IS-1/2024/Dispense/T10.pdf}{https://www.math.unipd.it/~tullio/IS-1/2024/Dispense/T10.pdf}
    \emph{(Ultimo accesso: 03/04/2025)}; \\
    \item \textbf{T11 - Verifica e validazione: analisti dinamica (slide 2-36)}: \\
    \bulhref{https://www.math.unipd.it/~tullio/IS-1/2024/Dispense/T11.pdf}{https://www.math.unipd.it/~tullio/IS-1/2024/Dispense/T11.pdf}
    \emph{(Ultimo accesso: 03/04/2025)}. \\
\end{itemize}