% Intestazione
\fancyhead[L]{1 \hspace{0.2cm} Strategie di testing} % Testo a sinistra



\section{Strategie di testing}
\label{sec:strategie di testing}

\subsection{Test di unità}
\label{sec:Test di unità}
I test di unità vengono utilizzati per controllare il funzionamento corretto di singoli componenti del codice in modo indipendente. Con "unità" si intendono parti del software, come funzioni, metodi o classi, che eseguono compiti ben definiti all'interno del programma.
\begin{table}[h!]
    \centering
    \renewcommand{\arraystretch}{1.5}
    \begin{tabularx}{\textwidth}{|p{0.15\textwidth}|X|p{0.15\textwidth}|p{0.15\textwidth}|}\hline
    \rowcolor[HTML]{FFD700}
    \textbf{Codice} & \textbf{Descrizione}  & \textbf{Stato} \\ \hline
    TU1 & Verifica che il metodo get\_answer di ChatController, in caso di messaggio vuoto, restituisca un messaggio di errore. &  N-I \\ \hline
    TU2 & Verifica che il metodo get\_answer di ChatController gestisca correttamente le eccezioni. &  N-I \\ \hline
    TU3 & Verifica che il metodo get\_answer di ChatService restituisca la risposta corretta. &  N-I \\ \hline
    TU4 & Verifica che il metodo get\_answer di ChatService gestisca correttamente le eccezioni. &  N-I \\ \hline
    TU5 & Verifica che il metodo similarity\_search di ChatService gestisca correttamente le eccezioni. &  N-I \\ \hline
    TU6 & Verifica che il metodo generate\_answer di ChatService gestisca correttamente le eccezioni. &  N-I \\ \hline
    TU7 & Verifica che il metodo load di ChromaVectorStoreAdapter gestisca correttamente le eccezioni. &  N-I \\ \hline
    TU8 & Verifica che il metodo split di ChromaVectorStoreAdapter sollevi un'eccezione ValueError se un documento non ha un campo 'id' nei metadati. &  N-I \\ \hline
    TU9 & Verifica che il metodo split di ChromaVectorStoreAdapter inserisca correttamente la data di inserimento nel database vettoriale. &  N-I \\ \hline
    TU10 & Verifica che il metodo split di ChromaVectorStoreAdapter gestisca correttamente i documenti con id duplicato. &  N-I \\ \hline
    TU11 & Verifica che il metodo split di ChromaVectorStoreAdapter gestisca correttamente la conversione delle liste di file in stringhe. &  N-I \\ \hline
    TU12 & Verifica che il metodo split di ChromaVectorStoreAdapter gestisca correttamente la formattazione delle date. &  N-I \\ \hline
    TU13 & Verifica che il metodo similarity\_search di ChromaVectorStoreAdapter gestisca correttamente le eccezioni. &  N-I \\ \hline
    TU14 & Verifica che il metodo similarity\_search di ChromaVectorStoreAdapter salti i documenti nulli. &  N-I \\ \hline
    TU15 & Verifica che il metodo load di ChromaVectorStoreRepository carichi correttamente i documenti nel database vettoriale. &  N-I \\ \hline
    TU16 & Verifica che il metodo load di ChromaVectorStoreRepository gestisca correttamente le eccezioni. &  N-I \\ \hline
    TU17 & Verifica che il metodo similarity\_search di ChromaVectorStoreRepository restituisca correttamente i risultati della ricerca di similarità. &  N-I \\ \hline
    TU18 & Verifica che il metodo similarity\_search di ChromaVectorStoreRepository gestisca correttamente le eccezioni. &  N-I \\ \hline
    \end{tabularx}
\end{table}

\newpage

\begin{table}[h!]
    \centering
    \renewcommand{\arraystretch}{1.5}
    \begin{tabularx}{\textwidth}{|p{0.15\textwidth}|X|p{0.15\textwidth}|p{0.15\textwidth}|}\hline
    \rowcolor[HTML]{FFD700}
    \textbf{Codice} & \textbf{Descrizione} & \textbf{Stato} \\ \hline
    TU19 & Verifica che il metodo load\_confluence\_pages di ConfluenceAdapter gestisca correttamente le eccezioni. &  N-I \\ \hline
    TU20 & Verifica che il metodo clean\_confluence\_pages di ConfluenceCleanerService pulisca correttamente le pagine Confluence. &  N-I \\ \hline
    TU21 & Verifica che il metodo clean\_confluence\_pages di ConfluenceCleanerService gestisca correttamente le eccezioni. &  N-I \\ \hline
    TU22 & Verifica che il metodo remove\_html\_tags di ConfluenceCleanerService rimuova correttamente i tag HTML. &  N-I \\ \hline
    TU23 & Verifica che il metodo remove\_html\_tags di ConfluenceCleanerService gestisca correttamente le eccezioni. &  N-I \\ \hline
    TU24 & Verifica che il metodo replace\_html\_entities di ConfluenceCleanerService sostituisca correttamente le entità HTML. &  N-I \\ \hline
    TU25 & Verifica che il metodo replace\_html\_entities di ConfluenceCleanerService gestisca correttamente le eccezioni. &  N-I \\ \hline
    TU26 & Verifica che il metodo get\_base\_url di ConfluenceRepository ritorni l'URL base corretto. &  N-I \\ \hline
    TU27 & Verifica che il metodo load\_confluence\_pages di ConfluenceRepository ritorni le pagine correttamente. &  N-I \\ \hline
    TU28 &  Verifica che il metodo load\_confluence\_pages di ConfluenceRepository gestisca correttamente le eccezioni di richiesta. &  N-I \\ \hline
    TU29 & Verifica che il metodo load\_confluence\_pages di ConfluenceRepository gestisca correttamente le eccezioni generiche. &  N-I \\ \hline
    TU30 & Verifica che il metodo generate\_answer di GenerateAnswerService gestisca correttamente le eccezioni. &  N-I \\ \hline
    TU31 & Verifica che il metodo get\_last\_load\_outcome di GetLastLoadOutcomeController gestisca correttamente le eccezioni. &  N-I \\ \hline
    TU32 & Verifica che il metodo get\_last\_load\_outcome di GetLastLoadOutcomeService gestisca correttamente le eccezioni. &  N-I \\ \hline
    TU33 & Verifica che il metodo get\_messages di GetMessagesController gestisca correttamente le eccezioni. &  N-I \\ \hline
    TU34 & Verifica che il metodo get\_messages di GetMessagesController restituisca una lista vuota se il metodo get\_messages di GetMessagesUseCase restituisce una lista vuota. &  N-I \\ \hline
    TU35 & Verifica che il metodo get\_messages di GetMessagesService gestisca correttamente le eccezioni. &  N-I \\ \hline
    TU36 & Verifica che il metodo get\_next\_possible\_questions di GetNextPossibleQuestionsController sollevi un'eccezione se il dizionario ricevuto in input non contiene le chiavi richieste. &  N-I \\ \hline
    TU37 & Verifica che il metodo get\_next\_possible\_questions di GetNextPossibleQuestionsController sollevi un'eccezione se la chiave 'question' del dizionario ricevuto in input non è una stringa. &  N-I \\ \hline
    TU38 & Verifica che il metodo get\_next\_possible\_questions di GetNextPossibleQuestionsController sollevi un'eccezione se la chiave 'answer' del dizionario ricevuto in input non è una stringa. &  N-I \\ \hline
    \end{tabularx}
\end{table}

\newpage

\begin{table}[h!]
    \centering
    \renewcommand{\arraystretch}{1.5}
    \begin{tabularx}{\textwidth}{|p{0.15\textwidth}|X|p{0.15\textwidth}|p{0.15\textwidth}|}\hline
    \rowcolor[HTML]{FFD700}
    \textbf{Codice} & \textbf{Descrizione} & \textbf{Stato} \\ \hline
    TU39 & Verifica che il metodo get\_next\_possible\_questions di GetNextPossibleQuestionsController sollevi un'eccezione se la chiave 'quantity' del dizionario ricevuto in input non è un intero. &  N-I \\ \hline
    TU40 & Verifica che il metodo get\_next\_possible\_questions di GetNextPossibleQuestionsController sollevi un'eccezione se, nell'oggetto ricevuto da GetNextPossibleQuestionsUseCase, l'attributo del numero di domande non corrisponde alla lunghezza dell'attributo lista delle domande possibili. &  N-I \\ \hline
    TU41 & Verifica che il costruttore di GetNextPossibleQuestionsService sollevi un'eccezione se l'header non contiene la stringa '***quantity***'. &  N-I \\ \hline
    TU42 & Verifica che il metodo get\_next\_possible\_questions di GetNextPossibleQuestionsService gestisca correttamente le eccezioni. &  N-I \\ \hline
    TU43 & Verifica che il metodo load\_github\_commits di GitHubAdapter gestisca correttamente le eccezioni. &  N-I \\ \hline
    TU44 & Verifica che il metodo load\_github\_files di GitHubAdapter gestisca correttamente le eccezioni. &  N-I \\ \hline
    TU45 & Verifica che il metodo load\_github\_files di GitHubAdapter gestisca correttamente le eccezioni di tipo UnicodeDecodeError. &  N-I \\ \hline
    TU46 & Verifica che il metodo load\_github\_commits di GitHubRepository carichi correttamente i commit di GitHub. &  N-I \\ \hline
    TU47 & Verifica che il metodo load\_github\_commits di GitHubRepository gestisca correttamente le eccezioni. &  N-I \\ \hline
    TU48 & Verifica che il metodo load\_github\_files di GitHubRepository carichi correttamente i file di GitHub. &  N-I \\ \hline
    TU49 & Verifica che il metodo load\_github\_files di GitHubRepository gestisca correttamente le eccezioni. &  N-I \\ \hline
    TU50 & Verifica che il metodo load\_github\_files di GitHubRepository gestisca correttamente le directory. &  N-I \\ \hline
    TU51 & Verifica che il metodo load\_jira\_issues di JiraAdapter gestisca correttamente le eccezioni. &  N-I \\ \hline
    TU52 & Verifica che il metodo get\_base\_url di JiraRepository restituisca l'URL base corretto. &  N-I \\ \hline
    TU53 & Verifica che il metodo load\_jira\_issues di JiraRepository restituisca correttamente le issues. &  N-I \\ \hline
    TU54 & Verifica che il metodo load\_jira\_issues di JiraRepository gestisca correttamente le eccezioni di richiesta. &  N-I \\ \hline
    TU55 & Verifica che il metodo load\_jira\_issues di JiraRepository gestisca correttamente le eccezioni generiche. &  N-I \\ \hline
    TU56 & Verifica che il metodo generate\_answer di LangChainAdapter gestisca il superamento del limite di token escludendo i documenti che lo fanno superare. &  N-I \\ \hline
    TU57 & Verifica che il metodo generate\_answer di LangChainAdapter gestisca correttamente le eccezioni. &  N-I \\ \hline
    \end{tabularx}
\end{table}

\newpage

\begin{table}[h!]
    \centering
    \renewcommand{\arraystretch}{1.5}
    \begin{tabularx}{\textwidth}{|p{0.15\textwidth}|X|p{0.15\textwidth}|p{0.15\textwidth}|}\hline
    \rowcolor[HTML]{FFD700}
    \textbf{Codice} & \textbf{Descrizione} & \textbf{Stato} \\ \hline
    TU58 & Verifica che il metodo get\_next\_possible\_questions di LangChainAdapter gestisca correttamente le eccezioni. &  N-I \\ \hline
    TU59 & Verifica che il metodo generate\_answer di LangChainRepository generi correttamente una risposta. &  N-I \\ \hline
    TU60 & Verifica che il metodo generate\_answer di LangChainRepository gestisca correttamente le eccezioni. &  N-I \\ \hline
    TU61 & Verifica che il metodo get\_next\_possible\_questions di LangChainRepository generi correttamente le prossime domande possibili. &  N-I \\ \hline
    TU62 & Verifica che il metodo get\_next\_possible\_questions di LangChainRepository gestisca correttamente le eccezioni. &  N-I \\ \hline
    TU63 & Verifica che il metodo load di LoadFilesController gestisca correttamente le eccezioni. &  N-I \\ \hline
    TU64 & Verifica che il metodo load di LoadFilesService carichi correttamente i dati dai vari servizi e salvi i log di caricamento. &  N-I \\ \hline
    TU65 & Verifica che il metodo load di LoadFilesService gestisca correttamente le eccezioni. &  N-I \\ \hline
    TU66 & Verifica che il metodo load di LoadFilesService sollevi un'eccezione se il salvataggio nel database fallisce. &  N-I \\ \hline
    TU67 & Verifica che il metodo load\_github\_commits di LoadFilesService gestisca correttamente le eccezioni. &  N-I \\ \hline
    TU68 & Verifica che il metodo load\_github\_files di LoadFilesService gestisca correttamente le eccezioni. &  N-I \\ \hline
    TU69 & Verifica che il metodo load\_jira\_issues di LoadFilesService gestisca correttamente le eccezioni. &  N-I \\ \hline
    TU70 & Verifica che il metodo load\_confluence\_pages di LoadFilesService gestisca correttamente le eccezioni. &  N-I \\ \hline
    TU71 & Verifica che il metodo clean\_confluence\_pages di LoadFilesService gestisca correttamente le eccezioni. &  N-I \\ \hline
    TU72 & Verifica che il metodo load\_in\_vector\_store di LoadFilesService gestisca correttamente le eccezioni. &  N-I \\ \hline
    TU73 & Verifica che il metodo save\_loading\_attempt\_in\_db di LoadFilesService gestisca correttamente le eccezioni. &  N-I \\ \hline
    TU74 & Verifica che il metodo save\_loading\_attempt\_in\_txt di LoadFilesService gestisca correttamente le eccezioni. &  N-I \\ \hline
    TU75 & Verifica che il metodo get\_messages di PostgresAdapter ritorni una lista vuota se non ci sono messaggi. &  N-I \\ \hline
    TU76 & Verifica che il metodo save\_message di PostgresAdapter gestisca correttamente le eccezioni. &  N-I \\ \hline
    TU77 & Verifica che il metodo get\_messages di PostgresAdapter gestisca correttamente le eccezioni. &  N-I \\ \hline
    TU78 & Verifica che il metodo save\_loading\_attempt di PostgresAdapter gestisca correttamente le eccezioni. &  N-I \\ \hline
    TU79 & Verifica che il metodo get\_last\_load\_outcome di PostgresAdapter gestisca correttamente le eccezioni. &  N-I \\ \hline

    \end{tabularx}
\end{table}

\newpage

\begin{table}[h!]
    \centering
    \renewcommand{\arraystretch}{1.5}
    \begin{tabularx}{\textwidth}{|p{0.15\textwidth}|X|p{0.15\textwidth}|p{0.15\textwidth}|}\hline
    \rowcolor[HTML]{FFD700}
    \textbf{Codice} & \textbf{Descrizione} & \textbf{Stato} \\ \hline
    TU80 & Verifica che il metodo dsor\_converter di PostgresAdapter sollevi un'eccezione ValueError se il messaggio è vuoto. &  N-I \\ \hline
    TU81 & Verifica che il metodo postgres\_message\_converter di PostgresAdapter sollevi un'eccezione ValueError se il contenuto è vuoto. &  N-I \\ \hline
    TU82 & Verifica che il metodo message\_converter di PostgresAdapter sollevi un'eccezione ValueError se il contenuto è vuoto. &  N-I \\ \hline
    TU83 & Verifica che il metodo postgres\_loading\_attempt\_converter di PostgresAdapter sollevi un'eccezione ValueError se i platform logs sono vuoti. &  N-I \\ \hline
    TU84 & Verifica che il metodo save\_message di PostgresRepository salvi correttamente un messaggio. &  N-I \\ \hline
    TU85 & Verifica che il metodo save\_message di PostgresRepository restituisca False in caso di errore del database. &  N-I \\ \hline
    TU86 & Verifica che il metodo save\_message di PostgresRepository gestisca correttamente un'eccezione generica. &  N-I \\ \hline
    TU87 & Verifica che il metodo get\_messages di PostgresRepository recuperi correttamente i messaggi dal database. &  N-I \\ \hline
    TU88 & Verifica che il metodo get\_messages di PostgresRepository restituisca una lista vuota nel caso in cui non ci siano messaggi nel database.&  N-I \\ \hline
    TU89 & Verifica che il metodo get\_messages di PostgresRepository gestisca correttamente un'eccezione generica. &  N-I \\ \hline
    TU90 & Verifica che il metodo save\_loading\_attempt di PostgresRepository salvi correttamente un tentativo di caricamento nel database. &  N-I \\ \hline
    TU91 & Verifica che il metodo save\_loading\_attempt di PostgresRepository restituisca False in caso di errore del database. &  N-I \\ \hline
    TU92 & Verifica che il metodo save\_loading\_attempt di PostgresRepository gestisca correttamente un'eccezione generica. &  N-I \\ \hline
    TU93 & Verifica che il metodo get\_last\_load\_outcome di PostgresRepository recuperi correttamente l'ultimo esito di caricamento dal database. &  N-I \\ \hline
    TU94 & Verifica che il metodo get\_last\_load\_outcome di PostgresRepository restituisca False nel caso in cui non ci siano tuple nella tabella degli esiti di caricamento perchè ancora non è stato effettuato alcun caricamento. &  N-I \\ \hline
    TU95 & Verifica che il metodo get\_last\_load\_outcome di PostgresRepository restituisca Error in caso di errore del database. &  N-I \\ \hline
    TU96 & Verifica che il metodo get\_last\_load\_outcome di PostgresRepository gestisca correttamente un'eccezione generica. &  N-I \\ \hline
    TU97 & Verifica che il metodo get\_last\_load\_outcome di PostgresRepository gestisca correttamente il caso in cui l'esito recuperato sia False. &  N-I \\ \hline
    TU98 & Verifica che il metodo execute\_query di PostgresRepository gestisca correttamente le eccezioni. &  N-I \\ \hline

    \end{tabularx}
\end{table}

\newpage

\begin{table}[h!]
    \centering
    \renewcommand{\arraystretch}{1.5}
    \begin{tabularx}{\textwidth}{|p{0.15\textwidth}|X|p{0.15\textwidth}|p{0.15\textwidth}|}\hline
    \rowcolor[HTML]{FFD700}
    \textbf{Codice} & \textbf{Descrizione} & \textbf{Stato} \\ \hline
    TU99 & Verifica che il metodo execute\_query di PostgresRepository recuperi correttamente un singolo risultato per una query di lettura. &  N-I \\ \hline
    TU100 & Verifica che il metodo execute\_query di PostgresRepository recuperi correttamente tutti i risultati per una query di lettura. &  N-I \\ \hline
    TU101 & Verifica che il metodo execute\_query di PostgresRepository esegua correttamente un commit per una query di scrittura. &  N-I \\ \hline
    TU102 & Verifica che il metodo save di SaveMessageController gestisca correttamente un messaggio con sender 'CHATBOT'. &  N-I \\ \hline
    TU103 & Verifica che il metodo save di SaveMessageController sollevi un'eccezione per un messaggio avente un sender non valido. &  N-I \\ \hline
    TU104 & Verifica che il metodo save di SaveMessageController sollevi un'eccezione per un messaggio avente timestamp non valido. &  N-I \\ \hline
    TU105 & Verifica che il metodo save di SaveMessageController gestisca correttamente le eccezioni generali. &  N-I \\ \hline
    TU106 & Verifica che il metodo save di SaveMessageService gestisca correttamente le eccezioni. &  N-I \\ \hline
    TU107 & Verifica che il metodo similarity\_search di SimilaritySearchService salti i documenti che superano la soglia di similarità. &  N-I \\ \hline
    TU108 & Verifica che il metodo similarity\_search di SimilaritySearchService termini e restituisca i documenti trovati finora se il distacco massimo è superato. &  N-I \\ \hline
    TU109 & Verifica che il metodo similarity\_search di SimilaritySearchService gestisca correttamente le eccezioni. &  N-I \\ \hline

    \end{tabularx}
    \caption{Insieme dei test di unità}
\end{table}

\newpage

\subsection{Test di integrazione}
\label{sec:Test di integrazione}
I test di integrazione servono a verificare che i vari moduli o unità di codice dell’applicazione collaborino correttamente tra loro, individuando eventuali problemi di interazione o incompatibilità. In questo modo si garantisce che l’intero sistema operi come previsto quando i singoli componenti vengono combinati.\\
\begin{table}[h!]
    \centering
    \renewcommand{\arraystretch}{1.5}
    \begin{tabularx}{\textwidth}{|p{0.15\textwidth}|X|p{0.15\textwidth}|p{0.15\textwidth}|}\hline
    \rowcolor[HTML]{FFD700}
    \textbf{Codice} & \textbf{Descrizione}  & \textbf{Stato} \\ \hline
    TI1 & Verifica che il metodo get\_answer di ChatController chiami correttamente il metodo get\_answer di ChatUseCase, in caso di messaggio non vuoto. &  N-I \\ \hline
    TI2 & Verifica che il metodo get\_answer di ChatController chiami correttamente il metodo get\_answer di ChatUseCase, in caso di messaggio vuoto. &  N-I \\ \hline
    TI3 & Verifica che il metodo generate\_answer di ChatService chiami il metodo generate\_answer di GenerateAnswerService. &  N-I \\ \hline
    TI4 & Verifica che il metodo similarity\_search di ChatService chiami il metodo similarity\_search di SimilaritySearchService. &  N-I \\ \hline
    TI5 & Verifica che il metodo similarity\_search di ChromaVectorStoreAdapter chiami il metodo similarity\_search di ChromaVectorStoreRepository. &  N-I \\ \hline
    TI6 & Verifica che il metodo load di ChromaVectorStoreAdapter chiami il metodo load di ChromaVectorStoreRepository. &  N-I \\ \hline
    TI7 & Verifica che il metodo load\_confluence\_pages di ConfluenceAdapter chiami il metodo load\_confluence\_pages di ConfluenceRepository. &  N-I \\ \hline
    TI8 & Verifica che il metodo generate\_answer di GenerateAnswerService chiami il metodo generate\_answer di GenerateAnswerPort. &  N-I \\ \hline
    TI9 & Verifica che il metodo get\_last\_load\_outcome di GetLastLoadOutcomeController chiami il metodo get\_last\_load\_outcome di GetLastLoadOutcomeUseCase. &  N-I \\ \hline
    TI10 & Verifica che il metodo get\_last\_load\_outcome di GetLastLoadOutcomeService chiami il metodo get\_last\_load\_outcome di GetLastLoadOutcomePort. &  N-I \\ \hline
    TI11 & Verifica che il metodo get\_messages di GetMessagesController chiami il metodo get\_messages di GetMessagesUseCase. &  N-I \\ \hline
    TI12 & Verifica che il metodo get\_messages di GetMessagesService chiami il metodo get\_messages di GetMessagesPort. &  N-I \\ \hline
    TI13 & Verifica che il metodo get\_next\_possible\_questions di GetNextPossibleQuestionsController chiami il metodo get\_next\_possible\_questions di GetNextPossibleQuestionsUseCase. &  N-I \\ \hline
    TI14 & Verifica che il metodo get\_next\_possible\_questions di GetNextPossibleQuestionsService chiami il metodo get\_next\_possible\_questions di GetNextPossibleQuestionsPort. &  N-I \\ \hline
    TI15 & Verifica che il metodo load\_github\_commits di GitHubAdapter chiami il metodo load\_github\_commits di GitHubRepository. &  N-I \\ \hline
    \end{tabularx}
\end{table}

\newpage

\begin{table}[h!]
    \centering
    \renewcommand{\arraystretch}{1.5}
    \begin{tabularx}{\textwidth}{|p{0.15\textwidth}|X|p{0.15\textwidth}|p{0.15\textwidth}|}\hline
    \rowcolor[HTML]{FFD700}
    \textbf{Codice} & \textbf{Descrizione} & \textbf{Stato} \\ \hline
    TI16 & Verifica che il metodo load\_github\_files di GitHubAdapter chiami il metodo load\_github\_files di GitHubRepository. &  N-I \\ \hline
    TI17 & Verifica che il metodo load\_jira\_issues di JiraAdapter chiami il metodo load\_jira\_issues di JiraRepository. &  N-I \\ \hline
    TI18 & Verifica che il metodo generate\_answer di LangChainAdapter chiami il metodo generate\_answer di LangChainRepository. &  N-I \\ \hline
    TI19 & Verifica che il metodo get\_next\_possible\_questions di LangChainAdapter chiami il metodo get\_next\_possible\_questions di LangChainRepository. &  N-I \\ \hline
    TI20 & Verifica che il metodo load di LoadFilesController chiami il metodo load di LoadFilesUseCase. &  N-I \\ \hline
    TI21 & Verifica che il metodo clean\_confluence\_pages di LoadFilesService chiami il metodo clean\_confluence\_pages di ConfluenceCleanerService. &  N-I \\ \hline
    TI22 & Verifica che il metodo load\_confluence\_pages di LoadFilesService chiami il metodo load\_confluence\_pages di ConfluencePort. &  N-I \\ \hline
    TI23 & Verifica che il metodo load\_github\_commits di LoadFilesService chiami il metodo load\_github\_commits di GitHubPort. &  N-I \\ \hline
    TI24 & Verifica che il metodo load\_github\_files di LoadFilesService chiami il metodo load\_github\_files di GitHubPort. &  N-I \\ \hline
    TI25 & Verifica che il metodo load\_jira\_issues di LoadFilesService chiami il metodo load\_jira\_issues di JiraPort. &  N-I \\ \hline
    TI26 & Verifica che il metodo load\_in\_vector\_store di LoadFilesService chiami il metodo load di LoadFilesInVectorStorePort. &  N-I \\ \hline
    TI27 & Verifica che il metodo save\_loading\_attempt\_in\_db di LoadFilesService chiami il metodo save\_loading\_attempt di SaveLoadingAttemptInDbPort. &  N-I \\ \hline
    TI28 & Verifica che il metodo save\_message di PostgresAdapter chiami il metodo save\_message di PostgresRepository. &  N-I \\ \hline
    TI29 & Verifica che il metodo get\_messages di PostgresAdapter chiami il metodo get\_messages di PostgresRepository. & N-I \\ \hline
    TI30 & Verifica che il metodo save\_loading\_attempt di PostgresAdapter chiami il metodo save\_loading\_attempt di PostgresRepository. &  N-I \\ \hline
    TI31 & Verifica che il metodo get\_last\_load\_outcome di PostgresAdapter chiami il metodo get\_last\_load\_outcome di PostgresRepository. &  N-I \\ \hline
    TI32 & Verifica che il metodo save di SaveMessageController chiami il metodo save di SaveMessageUseCase. &  N-I \\ \hline
    TI33 & Verifica che il metodo save di SaveMessageService chiami il metodo save\_message di SaveMessagePort. &  N-I \\ \hline
    TI34 & Verifica che il metodo similarity\_search di SimilaritySearchService chiami il metodo similarity\_search di SimilaritySearchPort. &  N-I \\ \hline
    \end{tabularx}
    \caption{Insieme dei test di integrazione}
\end{table}

\newpage

\subsection{Test di sistema}
\label{sec:Test di sistema}
I test di accettazione descritti di seguito rappresentano la base per la validazione del \emph{Minimum Viable Product}\textsubscript{\textit{\textbf{G}}}. Questi test sono stati progettati per verificare l’implementazione delle funzionalità previste nel prodotto, con l’obiettivo di garantire che risponda agli standard di un prodotto completo e pronto per il collaudo.

\begin{table}[h!]
    \centering
    \renewcommand{\arraystretch}{1.5}
    \begin{tabularx}{\textwidth}{|p{0.15\textwidth}|X|p{0.15\textwidth}|p{0.15\textwidth}|}\hline
    \rowcolor[HTML]{FFD700}
    \textbf{Codice} & \textbf{Descrizione} & \textbf{Fonte} & \textbf{Stato} \\ \hline
    TS1 & Verificare che l'utente possa inserire un'interrogazione in linguaggio naturale 
    nel sistema. & ROF1 & N-I \\ \hline
    TS2 & Verificare che all'invio dell'interrogazione al sistema, venga generata una risposta. & ROF2 & N-I \\ \hline
    TS3 & Verificare che se il sistema fallisce nel generare una risposta per via di un problema interno, venga visualizzato all'utente un messaggio di errore, chiedendo di riprovare più tardi. & ROF3 & N-I \\ \hline
    TS4 & Verificare che se l'utente inserisce un'interrogazione che non riguarda i contenuti del database associato, il sistema risponda all'utente che la domanda inserita è fuori contesto. & ROF4 & N-I \\ \hline
    TS5 & Verificare che se il sistema non riesce a trovare le informazioni richieste dall'utente nonostante siano correlate al contesto, venga spiegata la mancanza dell'informazione richiesta. & ROF5 & N-I \\ \hline
    TS6 & Verificare che l'utente possa visualizzare i file da cui il sistema ha preso i dati per la risposta. & ROF6 & N-I \\ \hline
    TS7 & Verificare che se l'utente desidera accedere ai file utilizzati per generare la risposta ma ciò non sia possibile, venga visualizzato un messaggio di errore. & RDF7 & N-I \\ \hline
    TS8 & Verificare che sia presente un pulsante al cui click la risposta del chatbot venga copiata nel dispositivo dell'utente. & RZF8 & N-I \\ \hline
    TS9 & Verificare che nel caso la risposta contenga uno snippet di codice, sia presente un pulsante che permetta di copiare il singolo snippet nel dispositivo dell'utente. & RDF9 & N-I \\ \hline
    TS10 & Verificare che sia presente un sistema di archiviazione delle domande e delle risposte in un database relazionale. & RDF10 & N-I \\ \hline
    TS11 & Verificare che l'utente possa visualizzare lo storico della chat, recuperato dal database relazionale. & RDF11 & N-I \\ \hline
    TS12 & Verificare che nel caso il sistema fallisca nel recuperare lo storico della chat, venga visualizzato un messaggio di errore all'utente spiegando che non è stato possibile recuperare lo storico. & RDF12 & N-I \\ \hline
   \end{tabularx}
\end{table}

\newpage


\begin{table}[h!]
    \centering
    \renewcommand{\arraystretch}{1.5}
    \begin{tabularx}{\textwidth}{|p{0.15\textwidth}|X|p{0.15\textwidth}|p{0.15\textwidth}|}\hline
    \rowcolor[HTML]{FFD700}
    \textbf{Codice} & \textbf{Descrizione} & \textbf{Fonte} & \textbf{Stato} \\ \hline
    TS13 & Verificare che uno scheduler si colleghi al sistema e periodicamente aggiorni il database vettoriale con i dati più recenti. & ROF13 & N-I \\ \hline
    TS14 & Verificare che la risposta venga generata prendendo in considerazione i dati di contesto provenienti da \emph{GitHub}\textsubscript{\textit{\textbf{G}}}, \emph{Jira}\textsubscript{\textit{\textbf{G}}} e \emph{Confluence}\textsubscript{\textit{\textbf{G}}}. & ROF14 & N-I \\ \hline
    TS15 & Verificare che il sistema possa convertire i dati ottenuti in formato vettoriale. & ROF15 & N-I \\ \hline
    TS16 & Verificare che il sistema possa aggiornare il database vettoriale con i nuovi dati ottenuti. & ROF16 & N-I \\ \hline
    TS17 & Verificare che se la conversazione non è ancora avviata, l'utente possa visualizzare e selezionare alcune domande di partenza proposte. & RZF17 & N-I \\ \hline
    TS18 & Verificare che dopo la visualizzazione di una risposta, all'utente vengano suggerite alcune interrogazioni che è possibile porre al sistema per continuare la conversazione. & RZF18 & N-I \\ \hline
    TS19 & Verificare che nel caso il sistema vada in errore nel tentativo di proporre altre domande per proseguire la conversazione, venga mostrato un messaggio che comunica l'errore all'utente e invita a fare altre domande. & RZF19 & N-I \\ \hline
    TS20 & Verificare che il sistema registri data e ora degli aggiornamenti del database vettoriale, in modo da poter scrivere un log di aggiornamento. & RZF20 & N-I \\ \hline
    TS21 & Verificare che il sistema comunichi se il database vettoriale a cui vengono poste le interrogazioni è aggiornato o meno. & RZF21 & N-I \\ \hline
    TS22 & Verificare che l'applicazione sia compatibile con la versione più recente di \emph{Google Chrome}\textsubscript{\textit{\textbf{G}}} al momento della \emph{demo}\textsubscript{\textit{\textbf{G}}}. & ROV1 & N-I \\ \hline
    TS23 & Verificare che il sistema garantisca la piena integrazione con le \emph{API}\textsubscript{\textit{\textbf{G}}} di \emph{Confluence}. & ROV2 & N-I \\ \hline
    TS24 & Verificare che il sistema garantisca la piena integrazione con le \emph{API} di \emph{Jira}. & ROV3 & N-I \\ \hline
    TS25 & Verificare che il sistema garantisca la piena integrazione con le \emph{API} di \emph{GitHub}. & ROV4 & N-I \\ \hline
    \end{tabularx}
    \caption{Insieme dei test di sistema}
\end{table}
\newpage


\subsection{Test di accettazione}
\label{sec:Test di accettazione}
I test di accettazione descritti di seguito rappresentano la base per la validazione del \emph{Minimum Viable Product}\textsubscript{\textit{\textbf{G}}}. Questi test sono stati progettati per verificare l’implementazione delle funzionalità previste nel prodotto, con l’obiettivo di garantire che risponda agli standard di un prodotto completo e pronto per il collaudo.

\begin{table}[h!]
    \centering
    \renewcommand{\arraystretch}{1.5} % Per aumentare l'altezza delle righe
    \begin{tabularx}{\textwidth}{|p{0.1\textwidth}|X|p{0.2\textwidth}|p{0.1\textwidth}|}\hline
    \rowcolor[HTML]{FFD700}
    \textbf{Codice} & \textbf{Descrizione} & \textbf{Fonte} & \textbf{Stato} \\ \hline
    TA1 & Verificare che l'utente possa inserire un interrogazione in linguaggio naturale e visualizzare la risposta generata. & UC1, UC2, UC3, UC4, UC5 & N-I \\ \hline
    TA2 & Verificare che l'utente possa visualizzare i file da cui il sistema ha preso i dati per la risposta alla domanda con relativi errori nel caso in cui non siano stati rilevati file.  & UC2.1, UC14 & N-I \\ \hline
    TA3 & Verificare che l'utente possa copiare il testo della risposta generata oppure lo snippet di codice presente nella risposta. & UC6, UC7 & N-I \\ \hline
    TA4 & Verificare che l'utente possa visualizzare lo storico della chat con relativo errore nel caso non fosse possibile.  & UC8, UC8.1, UC9 & N-I \\ \hline
    TA5 & Verificare che l'utente possa visualizzare una lista di domande ideali per poter iniziare una conversazione. & UC11, UC11.1 & N-I \\ \hline
    TA6 & Verificare che l'utente possa visualizzare una lista di domande ideali per poter proseguire una conversazione con relativo errore nel caso non fosse possibile. & UC12, UC12.1, UC13 & N-I \\ \hline
    TA7 & Verificare che l'utente possa visualizzare un badge che segnala lo stato di aggiornamento del database vettoriale.
    \begin{itemize}
        \item Badge colore verde se il database è aggiornato;
        \item Badge di colore rosso se il database non è aggiornato.
    \end{itemize} 
        & UC15, UC16, UC17 & N-I \\ \hline
    TA8 & Verificare che lo scheduler effettui l'aggiornamento automatico del database vettoriale. & UC10 & N-I \\ \hline

    \end{tabularx}
    \caption{Insieme dei test di accettazione}
\end{table}



\newpage
\subsection{Checklist}
\label{sec:checklist}
Le checklist vengono costantemente aggiornate dai Verificatori durante lo sviluppo del progetto. Gli errori rilevati vengono analizzati e inclusi nella lista, contribuendo a una maggiore efficacia nella correzione e garantendo un livello di qualità sempre più elevato.

\subsubsection{Struttura della documentazione}
\begin{table}[h!]
    \centering
    \renewcommand{\arraystretch}{1.5} % Per aumentare l'altezza delle righe
    \begin{tabularx}{\textwidth}{|p{0.3\textwidth}|X|}
    \hline
    \rowcolor[HTML]{FFD700}
    \textbf{Aspetto} & \textbf{Spiegazione} \\ \hline
    A capo & Per facilitare la lettura, le frasi devono essere mantenute su 
    una sola riga, evitando interruzioni non necessarie. \\ \hline
    Ordine non alfabetico & I nomi nei documenti devono essere elencati 
    in ordine alfabetico per una maggiore chiarezza. \\ \hline
    Caption Assente & Ogni tabella e immagine deve includere una didascalia. \\ \hline
    Sezioni Fantasma & Le sezioni vuote devono essere rimosse dal documento. \\ \hline
    Documento non spezzato & I documenti devono essere creati utilizzando più file \texttt{.tex} collegati 
    con il comando input nella pagina principale. \\ \hline
    \end{tabularx}
    \caption{Struttura documentazione}
\end{table}



\subsubsection{Errori ortografici}
\begin{table}[h!]
    \centering
    \renewcommand{\arraystretch}{1.5} % Per aumentare l'altezza delle righe
    \begin{tabularx}{\textwidth}{|p{0.3\textwidth}|X|}
    \hline
    \rowcolor[HTML]{FFD700}
    \textbf{Aspetto} & \textbf{Spiegazione} \\ \hline
    Accenti invertiti & Usare l'accento grave al posto dell'acuto e viceversa. \\ \hline
    “D” eufonica & La "d" eufonica va inserita solo quando si incontrano due vocali uguali di seguito. \\ \hline
    Discordanza soggetto-verbo & Il verbo non concorda correttamente con il soggetto utilizzato. \\ \hline
    Errori di battitura & La maggior parte degli errori sono dovuti a distrazione o digitazione errata. \\ \hline
    Forma dei verbi  & È preferibile l’utilizzo del presente indicativo, altre forme verbali andranno valutate opportunamente. \\ \hline
    Forme impersonali & Il soggetto dev’essere sempre esplicito nella frase. \\ \hline

    \end{tabularx}
    \caption{Errori ortografici}
\end{table}


\subsubsection{Non conformità con le Norme di Progetto}
\begin{table}[h!]
    \centering
    \renewcommand{\arraystretch}{1.5} % Per aumentare l'altezza delle righe
    \begin{tabularx}{\textwidth}{|p{0.35\textwidth}|X|}
    \hline
    \rowcolor[HTML]{FFD700}
    \textbf{Aspetto} & \textbf{Spiegazione} \\ \hline
    Utilizzo scorretto di “:” in grassetto & Evitare l'uso di “:” in grassetto all'interno degli elenchi puntati. \\ \hline
    Punteggiatura scorretta negli elenchi & Ogni voce dell'elenco deve terminare con “;”, tranne l'ultima che termina con “.”. \\ \hline
    Minuscolo nei ruoli & I ruoli devono essere scritti con la lettera iniziale maiuscola. \\ \hline
    Maiuscole nei titoli & La maiuscola deve essere utilizzata solo per la prima lettera del titolo. \\ \hline
\end{tabularx}
\end{table}
    

\begin{table}[h!]
    \centering
    \renewcommand{\arraystretch}{1.5} % Per aumentare l'altezza delle righe
    \begin{tabularx}{\textwidth}{|p{0.35\textwidth}|X|}
    \hline
    \rowcolor[HTML]{FFD700}
    \textbf{Aspetto} & \textbf{Spiegazione} \\ \hline
    Mancata segnalazione glossario & Quando viene introdotto un termine del glossario per la prima volta, deve essere segnalato. \\ \hline
    Non aggiornare il changelog & È obbligatorio aggiornare il registro delle modifiche dopo ogni verifica. \\ \hline
    Versione documento mancante & Quando un documento viene citato, è necessario indicare la versione attuale, se presente; in caso contrario, va specificata la versione corretta. \\ \hline
    \end{tabularx}
    \caption{Non conformità con Norme di Progetto}
\end{table}

\newpage
\subsubsection{Analisi dei Requisiti}
\begin{table}[h!]
    \centering
    \renewcommand{\arraystretch}{1.5} % Per aumentare l'altezza delle righe
    \begin{tabularx}{\textwidth}{|p{0.35\textwidth}|X|}
    \hline
    \rowcolor[HTML]{FFD700}
    \textbf{Aspetto} & \textbf{Spiegazione} \\ \hline
    Tracciamento UC - R & Ogni caso d'uso deve essere collegato a uno o più requisiti specifici. \\ \hline
    Numerazione UC & La numerazione dei Use Case di errore deve essere al medesimo livello del caso di successo corrispondente. \\ \hline
    Requisiti & I requisiti devono essere formulati nella forma “[soggetto] deve [verbo all'infinito]”. \\ \hline
    \emph{UML}\textsubscript{\textit{\textbf{G}}} degli UC & Estensioni, inclusioni e specializzazioni di un caso d'uso devono essere inclusi nello stesso diagramma UML del caso d'uso principale. \\ \hline
    \end{tabularx}
    \caption{Aspetti di qualità per i casi d'uso}
\end{table}
