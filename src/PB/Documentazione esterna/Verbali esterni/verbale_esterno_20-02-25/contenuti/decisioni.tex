% Intestazione
\fancyhead[L]{4 \hspace{0.2cm} Decisioni} % Testo a sinistra


\section{Decisioni}

Durante la riunione sono state prese le seguenti decisioni:

\vspace{0.5cm}

\begin{table}[htbp]
    \centering
    \rowcolors{2}{lightgray}{white}
    \begin{tabular}{|c|p{0.8\textwidth}|}
        \hline
        \rowcolor[gray]{0.75}
        \multicolumn{1}{|c|}{\textbf{Codice}} & \multicolumn{1}{|c|}{\textbf{Descrizione}}\\
        \hline
        VE 9.1 & Sulla funzionalità "Visualizzazione dei file da cui il bot ha preso la risposta" è stato scelto e deciso di
        presentare il link al file sulla piattaforma.\\
        \hline
        VE 9.2 & Sulla funzionalità \emph{API} verso \emph{Telegram} e \emph{Slack} è stato deciso che non verrà implementata.\\
        \hline
        VE 9.3 & E' stato deciso che non verrà suddiviso lo storico dei messaggi in sessioni, bensì i messaggi precedenti verranno
        presentati amalgamati, e non verranno tutti recuperati in una volta sola ma bensì verrà recuperato un numero fisso degli
        stessi.\\
        \hline
        VE 9.4 & È stato deciso che, prima di preparare l'ambiente di test a livello di struttura delle cartelle, è necessario
        correggere gli errori di progettazione.\\
        \hline
        VE 9.5 & È stato deciso che la scelta dei metodi più importanti da testare avverrà solo dopo il completamento della
        progettazione di dettaglio.\\
        \hline
        VE 9.6 & È stato deciso che la sezione "Cosa chiedere e come chiederlo" del \emph{Manuale Utente} sarà il punto cardine del
        documento e dovrà essere scritta con particolare cura. Non è necessario aggiungere istruzioni dettagliate su altri aspetti.\\
        \hline
        VE 9.7 & È stato deciso che condivideremo con il \emph{proponente} la versione attuale della sezione "Cosa chiedere e come
        chiederlo" del \emph{Manuale Utente} per eventuali revisioni.\\
        \hline
        VE 9.8 & È stato deciso che Filippo Righetto ha il compito di scrivere il verbale della riunione esterna del 20/02/2025\\
        \hline
    \end{tabular}
\end{table}