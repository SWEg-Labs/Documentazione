% Intestazione
\fancyhead[L]{4 \hspace{0.2cm} Decisioni} % Testo a sinistra


\section{Decisioni}

Durante la riunione sono state prese le seguenti decisioni:

\vspace{0.5cm}

\begin{table}[htbp]
    \centering
    \rowcolors{2}{lightgray}{white}
    \begin{tabular}{|c|p{0.8\textwidth}|}
        \hline
        \rowcolor[gray]{0.75}
        \multicolumn{1}{|c|}{\textbf{Codice}} & \multicolumn{1}{|c|}{\textbf{Descrizione}}\\
        \hline
        VE 10.1 & E' stato deciso che, per la modifica dei file in \emph{Chroma}, si preleverà la data di ultima modifica del documento dalla sua piattaforma di provenienza, e la si confronterà con la data di inserimento del documento nel database vettoriale per capire se procedere con una sostituzione o meno.\\
        \hline
        VE 10.2 & E' stato deciso che, per l'eliminazione dei documenti in \emph{Chroma}, si preleveranno solamente gli id dei documenti presenti nel \emph{database vettoriale} per confrontarli con gli id dei documenti prelevati dalla piattaforma, senza prelevare i documenti interi, per ragioni di efficienza.\\
        \hline
        VE 10.3 & E' stato deciso che non è necessario gestire manualmente la ricreazione della connessione verso i database.\\
        \hline
        VE 10.4 & E' stato deciso che la gestione dell'errore di salvataggio dei log nel database relazionale \emph{Postgres} da parte del \emph{cron} non è richiesto.\\
        \hline
        VE 10.5 & E' stato deciso che, per trasmettere i log dal \emph{cron} ad \emph{Angular}, viene lasciata libera scelta a noi tra \emph{Polling REST} e verifica dello stato dell'aggiornamento automatico solamente all'apertura o refresh della finestra del \emph{browser}.\\
        \hline
        VE 10.6 & E' stato deciso che la visualizzazione dei log è accettabile, che siamo liberi di implementare il salvataggio degli stessi come pila o come coda a nostra discrezione, e che le responsabilità del file txt e del file cron.log sono state scelte in modo oculato.\\
        \hline
        VE 10.7 & E' stato deciso che bisogna riportare nel documento \emph{Manuale Utente} il fatto che, mentre il database vettoriale sta venendo aggiornato, il chatbot non è in grado di rispondere.\\
        \hline
        VE 10.8 & E' stato deciso che la grafica del \emph{frontend} è accettabile, che la soluzione gradita per la visualizzazione dei link verso le fonti è inserire un riquadro dedicato in fondo al messaggio di risposta, e che è una buona aggiunta la visualizzazione della data e ora di invio nei messaggi.\\
        \hline
        VE 10.9 & E' stato deciso che possiamo continuare ad utilizzare la \emph{API Key} di \emph{OpenAI} in modo normale per le nostre necessità.\\
        \hline
        VE 10.10 & È stato deciso che Riccardo Stefani ha il compito di redigere il verbale della riunione esterna del 06/03/25 e di inviarlo per mail assieme alle slide della riunione convertite in PDF.\\
        \hline
    \end{tabular}
\end{table}