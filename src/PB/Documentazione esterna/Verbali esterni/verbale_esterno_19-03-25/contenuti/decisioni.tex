% Intestazione
\fancyhead[L]{4 \hspace{0.2cm} Decisioni} % Testo a sinistra


\section{Decisioni}

Durante la riunione sono state prese le seguenti decisioni:

\vspace{0.5cm}

\begin{table}[htbp]
    \centering
    \rowcolors{2}{lightgray}{white}
    \begin{tabular}{|c|p{0.8\textwidth}|}
        \hline
        \rowcolor[gray]{0.75}
        \multicolumn{1}{|c|}{\textbf{Codice}} & \multicolumn{1}{|c|}{\textbf{Descrizione}}\\
        \hline
        VE 11.1 & Il \emph{proponente} \emph{AzzurroDigitale} ha confermato che il prodotto presentato è un \emph{MVP} valido per il capitolato "BuddyBot", approvando dunque il lavoro svolto e consentendo al gruppo \emph{SWEg Labs} di avviare le pratiche per terminare il progetto didattico.\\ \hline
        VE 11.2 & E' stato deciso che il gruppo si informerà per rispondere al proponente sulla presenza di un limite massimo di richieste che la libreria \emph{requests} di \emph{Python} può effettuare.\\ \hline
        VE 11.3 & E' stato confermato che la coverage raggiunta con i test di unità e di integrazione è sufficiente rispetto alla soglia di accettazione del 75\% concordata con il proponente.\\ \hline
        VE 11.4 & Il proponente ha confermato che non è necessario implementare il requisito RDF33 per l'\emph{MVP}.\\ \hline
        VE 11.5 & Il proponente ha confermato che è possibile ottenere la firma del verbale esterno della riunione già giovedì 20 marzo, il giorno successivo alla \emph{demo}, per garantire che sia possibile per il gruppo chiedere il primo colloquio della \emph{PB} il prima possibile.\\ \hline
        VE 11.6 & E' stato confermato dal proponente che il repository del prodotto è sufficiente per la reperibilità dello stesso, e che sarà molto importante il README lì presentato per la fornitura delle istruzioni di installazione e avvio.\\ \hline
        VE 11.7 & Il proponente ha confermato che il materiale richiesto da consegnargli è:
        \begin{itemize}
            \item Documento di analisi funzionale e tecnica;
            \item Test;
            \item Verbale di collaudo;
            \item \emph{Manuale Utente};
            \item Link al repository dell'\emph{MVP};
            \item Link al sito \emph{GitHub Pages} della documentazione;
            \item \emph{Manuale Sviluppatore}.
        \end{itemize}\\ \hline
        VE 11-8 & Il proponente ha confermato che è sufficiente inviare tutto il materiale richiesto via email successivamente all'ultimo incontro della \emph{PB}, e che è già possibile fornire il link al repository dell'\emph{MVP} via \emph{Discord}.\\ \hline
        VE 11.9 & Il gruppo \emph{SWEg Labs} ha accettato la proposta del proponente di svolgere un incontro in presenza nella loro
        sede aziendale, assieme all'altro gruppo che sta svolgendo lo stesso capitolato, per presentare il progetto a tutti i dipendenti di \emph{AzzurroDigitale}.\\
        \hline
    \end{tabular}
\end{table}