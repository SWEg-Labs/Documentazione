% Intestazione
\fancyhead[L]{1 \hspace{0.2cm} Introduzione} % Testo a sinistra

\pagenumbering{arabic} % Numerazione araba per il contenuto 

\section{Introduzione}
\label{sec:introduzione}

\subsection{Scopo del documento}
Questo documento ha lo scopo di illustrare le istruzioni per l’utilizzo e le funzionalità fornite
dall’applicativo. L’utente sarà quindi a conoscenza dei requisiti minimi necessari per il
corretto funzionamento del chatbot BuddyBot, di come accederci e di come farne un utilizzo
consapevole.

\subsection{Scopo del prodotto}
Nell’ultimo anno vi è stato un cambiamento repentino nello sviluppo e nell’applicazione
dell’\emph{Intelligenza Artificiale}\textsubscript{\textbf{\textit{G}}} all’elaborazione e raccomandazione dei contenuti alla generazione
di essi, come immagini, testi e tracce audio.
Il \emph{capitolato}\textsubscript{\textbf{\textit{G}}} C9, "BuddyBot", pone come obbiettivo la realizzazione di un applicativo che
permetta di porre interrogazioni in
linguaggio naturale sullo stato attuale del prodotto software in lavorazione, ricevendo una risposta il quanto più precisa. Tale
risposta dovrà essere generata tramite un \emph{LLM}\textsubscript{\textbf{\textit{G}}} collegato. Tale software sarà fruibile
attraverso un'\emph{applicazione web}\textsubscript{\textbf{\textit{G}}}, dove l'utente potrà interrogare il chatbot sullo stato
attuale del codice e della documentazione del prodotto software nelle piattaforme
utilizzate per il suo sviluppo.

\subsection{Glossario}
Al fine di prevenire ed evitare possibili ambiguità nei termini e acronimi presenti all’interno della documentazione, è stato
realizzato un glossario nel file \emph{glossario\_v2.0.0.pdf} in grado di dare una definizione precisa per ogni vocabolo potenzialmente
ambiguo. All’interno di ogni documento i termini specifici, che quindi hanno una definizione all’interno del
\emph{Glossario}\textsubscript{\textbf{\textit{G}}}, saranno
contrassegnati con una \textsubscript{\textbf{\textit{G}}} aggiunta a pedice e trascritti in corsivo. Tale prassi sarà rispettata
solamente per la prima occorrenza del termine in una determinata sezione del documento.
