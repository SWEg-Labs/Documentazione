% Intestazione
\fancyhead[L]{4 \hspace{0.2cm} Guida all’utilizzo} % Testo a sinistra 

\section{Guida all’utilizzo}
\label{sec:guida_utilizzo}
% Screenshot di cosa appare all'avvio del chatbot. Descrivere gli elementi presenti nella schermata iniziale.


\subsection{Cosa chiedere, e come chiederlo}
BuddyBot è pensato per essere l'assistente digitale di un'azienda informatica che deve aiutare i neoassunti a muoversi nel codice
e nella documentazione dei vari progetti aziendali. Più precisamente, ad ogni interrogazione posta dall'utente vengono associati dei
documenti di contesto, che provengono esclusivamente dai profili \emph{GitHub}\textsubscript{\textbf{\textit{G}}},
\emph{Jira}\textsubscript{\textbf{\textit{G}}} e \emph{Confluence}\textsubscript{\textbf{\textit{G}}} dell'azienda, e in base alle
informazioni presenti su tali documenti di contesto il chatbot fornirà la risposta alla domanda fornita.
Se la domanda viene malposta, il chatbot potrebbe non essere in grado di fornire una risposta, e in tal caso potrebbe restituire
uno dei seguenti messaggi:
\begin{itemize}
    \item \emph{"Domanda fuori contesto"}: la domanda posta non è inerente al contesto informatico aziendale;
    \item \emph{"Informazione non trovata"}: non è stato possibile trovare nei documenti di contesto informazioni utili per rispondere alla domanda.
\end{itemize}

E' tuttavia possibile che il chatbot restituisca tali risposte anche in casi in cui non ce lo si aspetta.
Per evitare di ottenere risposte negative, forniamo qui di seguito delle istruzioni per capire come porre la domanda per
ottenere una risposta soddisfacente:
\begin{itemize}
    \item La domanda deve essere concisa e diretta, senza giri di parole;
    \item Si devono mettere bene in risalto le parole chiave;
    \item La domanda deve essere il più breve possibile;
    \item La domanda deve essere pensata per ottenere una risposta unitaria, e non per aprire una discussione;
    \item La domanda deve riguardare un contesto chiaro e definito, quindi non può toccare tematiche troppo generali;
    \item Il suddetto contesto deve essere uno, infatti non si possono porre due o più domande a tema differente in una stessa
    interrogazione;
    \item Non si deve chiedere informazioni generiche a riguardo di una risorsa fornendo solo il nome della stessa, ma bisogna
    bensì specificare più di preciso cosa si vuole sapere a riguardo di tale risorsa;
    \item La domanda deve essere posta in un linguaggio quanto più possibile formale e tecnico, cercando di rimanere aderenti
    allo stile di scrittura dei documenti aziendali. Un linguaggio troppo colloquiale quasi sicuramente condurrà a risposte negative;
    \item Non si devono porre domande che richiedono una risposta soggettiva. In caso ne venga posta una per sbaglio, non ci si deve
    fidare della risposta fornita, è bensì consigliabile chiedere ad un collega;
    \item Si devono evitare domande che richiedano ragionamento e/o collegamenti logici, in quanto il chatbot non è in grado di
    processare informazioni in tal senso;
    \item Non si devono porre domande per la cui risposta è necessaria una ricerca su internet, in quanto il chatbot fa
    riferimento esclusivamente ai documenti di contesto, e non è in grado di navigare in rete.
\end{itemize}


\subsection{Interrogazione del chatbot}

\subsubsection{Come inserire una domanda}
% Parlare dei due metodi di inserimento, o domanda libera o selezione di domande iniziali proposte.

\subsubsection{La risposta del chatbot}
% Parlare:
% - della risposta testuale (che può contenere uno snippet di codice);
% - della possibilità di copiare il testo, e soprattutto di copiare lo snippet di codice, entrambi mediante pulsante apposito;
% - della visualizzazione del nome dei file da cui è stata presa la risposta (se implementeremo il requisito);
% - delle domande proposte per proseguire, e che rimane sempre disponibile la domanda libera.


\subsection{Visualizzazione dello storico delle sessioni}
% Parlare della linea orizzontale di suddivisione delle sessioni, e dello scorrimento verso l'alto per visualizzare le
% sessioni precedenti fino all'inizio della conversazione.


\subsection{L'aggiornamento automatico dei documenti}
% Parlare del fatto che i documenti vengono aggiornati automaticamente ogni tot ore
% Descrivere il badge di segnalazione dell'esito dell'ultimo aggiornamento: verde se avvenuto con successo, rosso se
% fallito.


\subsection{Possibili errori}
% Fare un elenco di possibili errori che possono verificarsi durante l'utilizzo del chatbot. Ovviamente per ciascuno va
% riportato uno screenshot.
% - Errore nel recupero dello storico (UC9)
% - Errore nella generazione della risposta (UC3)
% - Errore nel recupero del nome dei file utili per la risposta (UC14)
% - Errore nella generazione delle domande per proseguire la conversazione (UC13)
% - Visualizzazione di un badge che segnala un errore nel recupero dell'esito dell'ultimo aggiornamento del database vettoriale (UC20)
% - Visualizzazione di un messaggio che spiega il badge che segnala un errore nel recupero dell'esito dell'ultimo aggiornamento del database vettoriale (UC24)
