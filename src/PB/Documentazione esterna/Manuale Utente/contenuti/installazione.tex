% Intestazione
\fancyhead[L]{1 \hspace{0.2cm} Introduzione} % Testo a sinistra

\pagenumbering{arabic} % Numerazione araba per il contenuto 

\section{installazzione}
\label{sec:installazzione}

\subsection{Verifica e Installazione di Docker}
Prima di procedere con l'installazione di \textit{BuddyBot} è importante verificare che Docker sia installato e pronto all'uso.  
Per farlo, digitare nel terminale:
\begin{verbatim}
docker --version
\end{verbatim}
Se la console fornisce un output (ad esempio un numero di versione), Docker è correttamente installato e funzionante.  
Nel caso in cui il terminale non rilasci alcun output, è possibile scaricare Docker direttamente 
\href{https://www.docker.com/products/docker-desktop}{cliccando qui}.

\subsection{Download dell'Applicazione}
Clonare la repository GitHub di \textit{BuddyBot} dal seguente indirizzo:
\begin{verbatim}
https://github.com/SWEg-Labs/BuddyBot.git
\end{verbatim}

\subsection{Configurazione del file \texttt{.env}}
Tutte le variabili di sistema di configurazione sono già incluse nel \texttt{Dockerfile}.  
Tuttavia, per le impostazioni sensibili e personalizzabili, occorre inserire nella directory \texttt{src/backend} un file \texttt{.env} contenente le seguenti voci (adattandole alle proprie esigenze):
\begin{verbatim}
OPENAI_API_KEY = la_tua_chiave_openai
OPENAI_MODEL_NAME = modello_llm_scelto

GITHUB_TOKEN = il_tuo_token_github
OWNER = owner_repository
REPO = nome_repository

ATLASSIAN_TOKEN = il_tuo_token_atlassian
ATLASSIAN_USER_EMAIL = la_tua_mail_atlassian

JIRA_BASE_URL = url_base_jira
JIRA_PROJECT_KEY = jira_project_key

CONFLUENCE_BASE_URL = confluence_base_key
CONFLUENCE_SPACE_KEY = confluence_space_key
\end{verbatim}

\subsection{Avvio del Container}
Una volta pronti, è possibile avviare il container Docker posizionandosi nella cartella del progetto ed eseguendo:
\begin{verbatim}
docker compose up --build
\end{verbatim}
Il comando provvederà a costruire ed avviare il container. Al termine delle operazioni è possibile fermarlo con:
\begin{verbatim}
docker compose down
\end{verbatim}

\subsection{Esecuzione dell'Applicazione da Terminale}
Per avviare l'applicazione \textit{BuddyBot} dal terminale del container, seguire questi passi:
\begin{enumerate}
    \item Aprire Docker Desktop e, nella sezione \textit{Containers}, selezionare \texttt{backend}.
    \item Fare clic su \texttt{Exec} per aprire il terminale del container.
    \item Digitare il comando:
    \begin{verbatim}
    python backend/main.py
    \end{verbatim}
    \item Attendere da 2 a 5 minuti dall'inizio di un minuto che finisce per 0 
    (es.: 10:40, 16:30, 14:10). Quindi, nel ChatBot, inserire il comando \texttt{v} 
    per verificare i documenti presenti nel database; si noterà che il database non è vuoto.
    \item Per visionare i log dell’aggiornamento, uscire dall’app con \texttt{exit} e digitare:
    \begin{verbatim}
    cat /var/log/cron.log
    \end{verbatim}
\end{enumerate}

In alternativa, dal terminale della propria macchina locale, è possibile accedere al container 
\texttt{buddybot-backend} eseguendo:
\begin{verbatim}
docker exec -it buddybot-backend /bin/bash
\end{verbatim}

\subsection{Esecuzione dell'Applicazione da Browser}
Per avviare la parte front-end, è sufficiente aprire un browser e digitare:
\begin{verbatim}
localhost:4200
\end{verbatim}
