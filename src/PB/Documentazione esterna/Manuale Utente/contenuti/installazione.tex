% Intestazione
\fancyhead[L]{3 \hspace{0.2cm} Installazione} % Testo a sinistra


\section{Installazione}
\label{sec:installazione}

\subsection{Installazione di Docker}
Prima di procedere con l'installazione di \textit{BuddyBot} è importante verificare che \emph{Docker}\textsubscript{\textbf{\textit{G}}} sia installato sulla propria macchina e pronto all'uso.  
Per farlo, digitare nel terminale:
\begin{verbatim}
    docker --version
\end{verbatim}
Se la console fornisce come output un numero di versione (ad esempio "Docker version 27.3.1, build ce12230"), allora Docker è
correttamente
installato e funzionante.\\
Nel caso in cui il terminale segnali un errore, è possibile scaricare Docker seguendo la guida presente al link
\bulhref{https://docs.docker.com/get-started/get-docker/}{https://docs.docker.com/get-started/get-docker/}
\emph{(Ultimo accesso: 03/04/2025)}.\\
Il gruppo \emph{SWEg labs} ha testato l'applicazione utilizzando Docker in versione 27.3.1, dunque si consiglia di utilizzare una
versione uguale o superiore per garantire il corretto funzionamento dell'applicazione.

\subsection{Download dell'Applicazione}
E' possibile clonare la repository \emph{GitHub}\textsubscript{\textbf{\textit{G}}} di BuddyBot eseguendo sul proprio terminale:
\begin{verbatim}
    git clone https://github.com/SWEg-Labs/BuddyBot.git
\end{verbatim}
Una volta scaricato il repository, posizionarsi nella cartella del progetto con il comando:
\begin{verbatim}
    cd BuddyBot
\end{verbatim}

\subsection{Creazione e configurazione del file \texttt{.env}}
Tutte le variabili di sistema di configurazione sono già incluse nel \emph{Dockerfile}\textsubscript{\textbf{\textit{G}}}.  
Tuttavia, per le impostazioni sensibili e personalizzabili, occorre creare nella directory \texttt{src/backend} un file
\texttt{.env} contenente le seguenti voci (adattandole alle proprie esigenze):
\begin{verbatim}
    OPENAI_API_KEY = la_tua_chiave_openai
    OPENAI_MODEL_NAME = modello_llm_scelto

    GITHUB_TOKEN = il_tuo_token_github
    OWNER = proprietario_repository
    REPO = nome_repository

    ATLASSIAN_TOKEN = il_tuo_token_atlassian
    ATLASSIAN_USER_EMAIL = la_tua_mail_atlassian

    JIRA_BASE_URL = url_base_jira
    JIRA_PROJECT_KEY = jira_project_key

    CONFLUENCE_BASE_URL = confluence_base_key
    CONFLUENCE_SPACE_KEY = confluence_space_key
\end{verbatim}

\subsection{Creazione dell'immagine e avvio del container}
Una volta pronti, è possibile creare l'\emph{immagine Docker}\textsubscript{\textbf{\textit{G}}} posizionandosi nella cartella del progetto ed eseguendo:
\begin{verbatim}
    docker compose up --build
\end{verbatim}
La creazione dell'immagine impiegherà poco più di 5 minuti.
Al termine della creazione di quest'ultima, verrà creato ed avviato il \emph{container}\textsubscript{\textbf{\textit{G}}} \texttt{buddybot}. Al termine dell'utilizzo, per
spegnere l'applicazione è possibile fermare il container impartendo la combinazione di tasti Ctrl+C nel terminale, oppure premendo il tasto Stop nell'applicazione \emph{Docker Desktop}\textsubscript{\textbf{\textit{G}}}.\\
Per i successivi accessi, aprire \emph{Docker Desktop} e premere il tasto Play sul container \texttt{buddybot} per avviare di nuovo il
container dell'applicazione. Per stopparlo, premere il tasto Stop dalla stessa interfaccia.\\
Se si vuole continuare ad interagire da terminale con il container, è possibile eseguire il comando:
\begin{verbatim}
    docker compose up
\end{verbatim}
per avviarlo, e poi, come sopra, Ctrl+C per stopparlo.

\subsection{Esecuzione dell'Applicazione}
Per avviare BuddyBot, è sufficiente aprire un \emph{browser}\textsubscript{\textbf{\textit{G}}} e digitare nella barra
degli indirizzi:
\begin{verbatim}
    localhost:4200
\end{verbatim}
Si aprirà dunque l'interfaccia grafica dell'\emph{applicazione web}\textsubscript{\textbf{\textit{G}}}, pronta per ricevere domande
dalla barra di input visibile nella parte inferiore dello schermo.\\
Le istruzioni per l'utilizzo dell'applicazione sono fornite nella sezione \S\bulref{sec:guida_utilizzo}.
