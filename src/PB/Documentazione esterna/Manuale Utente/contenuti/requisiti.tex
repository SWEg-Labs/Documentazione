% Intestazione
\fancyhead[L]{2 \hspace{0.2cm} Requisiti} % Testo a sinistra 

\section{Requisiti}
\label{sec:requisiti}

Per poter utilizzare l’applicazione è necessario soddisfare i seguenti requisiti minimi.


\subsection{Requisiti di sistema operativo}

Per far si che le operazioni di installazione e avvio del software avvengano correttamente e
che si possa aver accesso a tutte le funzionalità, è necessario avere nella propria macchina 
uno tra i seguenti sistemi operativi.

\begin{table}[h!]
    \centering
    \renewcommand{\arraystretch}{1.6} % Per aumentare l'altezza delle righe
    \begin{tabularx}{\textwidth}{|p{4cm}|X|} \hline
    \rowcolor[HTML]{FFD700} 
    \textbf{Sistema Operativo} & \textbf{Distribuzione} \\ 
    \hline
    Linux-based (Consigliato) & Kernel 3.10 o successivo. Distribuzioni Ubuntu, Debian, RHEL, Fedora, Arch (sperimentale). \\ 
    \hline
    Windows & 10 64-bit o successivo. \\ 
    \hline
    macOS & 10.15 ”Catalina” o successivo. \\ 
    \hline
    \end{tabularx}
    \caption{Sistemi operativi compatibili}
\end{table}


\subsection{Requisiti software}

Per l’utilizzo del software è necessario avere installato \emph{Docker}\textsubscript{\textbf{\textit{G}}}. Creando l’immagine
\emph{Docker} tutti i moduli sono già presenti e non si necessita di ulteriori installazioni.

\begin{table}[h!]
    \centering
    \renewcommand{\arraystretch}{1.6} % Per aumentare l'altezza delle righe
    \begin{tabularx}{\textwidth}{|p{1.5cm}|p{1.5cm}|X|} \hline
    \rowcolor[HTML]{FFD700}
    \textbf{Software} & \textbf{Versione} & \textbf{Download} \\ 
    \hline
    Docker & 27.3.1 & \bulhref{https://docs.docker.com/get-started/get-docker}{docs.docker.com/get-started/get-docker}\emph{(Ultimo accesso: 03/04/2025)} \\ 
    \hline
    \end{tabularx}
    \caption{Requisiti software}
\end{table}


\subsection{Requisiti hardware}

Perchè l'applicazione abbia delle prestazioni accettabili è preferibile disporre di componenti hardware con almeno le seguenti caratteristiche:

\begin{table}[h!]
    \centering
    \renewcommand{\arraystretch}{1.6} % Per aumentare l'altezza delle righe
    \begin{tabularx}{0.5\textwidth}{|p{4cm}|X|} \hline
    \rowcolor[HTML]{FFD700} 
    \textbf{Componente} & \textbf{Requisito} \\ 
    \hline
    CPU & Quad-Core 2,8 GHz \\ 
    \hline
    RAM & 8GB DDR4 \\ 
    \hline
    \end{tabularx}
    \caption{Requisiti hardware}
\end{table}


\subsection{Browser supportati}

L’applicazione è stata testata e quindi resa compatibile con i \emph{browser}\textsubscript{\textbf{\textit{G}}} riportati di seguito:

\newpage

\begin{table}[h!]
    \centering
    \renewcommand{\arraystretch}{1.6} % Per aumentare l'altezza delle righe
    \begin{tabularx}{0.5\textwidth}{|p{4cm}|X|} \hline
    \rowcolor[HTML]{FFD700} 
    \textbf{Browser} & \textbf{Versione} \\ 
    \hline
    Google Chrome & 131 \\ 
    \hline
    Microsoft Edge & 133 \\ 
    \hline
    Mozilla Firefox & 130 \\
    \hline
    Apple Safari & 17 \\
    \hline
    \end{tabularx}
    \caption{Browser supportati}
\end{table}
