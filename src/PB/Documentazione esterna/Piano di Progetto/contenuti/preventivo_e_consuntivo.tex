% Intestazione
\fancyhead[L]{5 \hspace{0.2cm} Preventivo e Consuntivo} % Testo a sinistra

% Sezione 
\section{Preventivo e Consuntivo}
\label{sec:preventivo e consuntivo}

\subsection{RTB}
\subsubsection{Periodo zero: 04/11/2024 - 20/11/2024}
Gli obiettivi fissati per il periodo zero sono:
\begin{itemize}
    \item Effettuare le correzioni consigliate dopo l'aggiudicazione del capitolato;
    \item Inizio stesura documentazione RTB: \textit{Glossario, Analisi dei requisiti, Norme di Progetto, Piano di Progetto, Piano di Qualifica};
    \item Comunicare all'azienda \emph{AzzurroDigitale} dell'effettiva aggiudicazione del \textit{Capitolato} ed organizzare un primo incontro conoscitivo;
    \item Implementare i giusti strumenti di lavoro come {\emph{Jira}}\textsubscript{\textit{\textbf{G}}}, {\emph{Fogli google}}\textsubscript{\textit{\textbf{G}}} e {\emph{Draw.io}}\textsubscript{\textit{\textbf{G}}}.
\end{itemize}

%Primo Periodo - CON FINE PRIMO SPRINT
\subsubsection{Primo periodo: 21/11/2024 - 05/12/2024}
Gli obiettivi fissati per il primo periodo sono:
\begin{itemize}
    \item Identificazione e studio delle tecnologie necessarie per il progetto;
    \item Effettuare l'analisi dei {\emph{Casi d'uso}}\textsubscript{\textit{\textbf{G}}};
    \item Continuazione della documentazione per il documento \textit{Piano di Qualifica};
    \item Conclusione della prima {\emph{sprint}}\textsubscript{\textit{\textbf{G}}} e successiva scrittura del preventivo, consuntivo, prospetto orario, prospetto economico e retrospettiva nel \textit{Piano di Progetto};
\end{itemize}

\subsubsubsection{Preventivo primo periodo: 21/11/2024 - 05/12/2024}
\begin{table}[h!]
    \centering
    \renewcommand{\arraystretch}{1.5}
    \begin{tabularx}{\textwidth}{|c|X|X|X|X|X|X|c|}\hline
    \rowcolor[HTML]{FFD700} 
    \textbf{Nominativi dei membri} & \textbf{Re} & \textbf{Am} & \textbf{An} & \textbf{Pg} & \textbf{Pr} & \textbf{Ve} & \textbf{Ore per membro} \\ \hline
    Federica Bolognini  & 1 & 0 & 2 & 0 & 2 & 2 & 7 \\ \hline
    Michael Fantinato   & 0 & 0 & 3 & 0 & 2 & 2 & 7 \\ \hline
    Giacomo Loat        & 0 & 7 & 0 & 0 & 0 & 0 & 7 \\ \hline
    Filippo Righetto    & 0 & 2 & 4 & 0 & 0 & 2 & 8 \\ \hline
    Riccardo Stefani    & 7 & 0 & 0 & 0 & 0 & 0 & 7 \\ \hline
    Davide Verzotto     & 0 & 1 & 4 & 0 & 0 & 2 & 7 \\ \hline
    \rowcolor[HTML]{FFD700} 
    \textbf{Ore totali per ruolo} & 8 & 10 & 13 & 0 & 4 & 8 & \textbf{Ore totali del gruppo: 43} \\ \hline
    \end{tabularx}
    \caption{Preventivo della suddivisione oraria per ruolo nel primo periodo}
\end{table}

\subsubsubsection{Consuntivo primo periodo: 21/11/2024 - 05/12/2024}
Tutti gli obiettivi predisposti all'inizio del periodo sono stati soddisfatti con successo.\\
In particolare, l'identificazione e lo studio delle tecnologie necessarie per il progetto sono stati completati positivamente sebbene abbiano richiesto più tempo del previsto. L'analisi dei \textit{Casi d'uso} è stata portata a termine. La continuazione della documentazione per il documento \textit{Piano di Qualifica} è stata svolta regolarmente. Infine, la prima {\emph{sprint}} è stata conclusa con successo, con la successiva redazione del preventivo, consuntivo, prospetto orario, prospetto economico e rischi occorsi nel \textit{Piano di Progetto}.

\newpage
\paragraph{Prospetto orario: }
Nel primo periodo si è registrato un consumo orario superiore a quello preventivato, con un eccesso di 11 ore di lavoro complessive per il gruppo. Questo incremento è stato determinato principalmente dalle ore assegnate ad Amministratore, Analista e Verificatore, mentre si è osservata una riduzione delle ore destinate al Programmatore.
\begin{table}[h!]
    \centering
    \renewcommand{\arraystretch}{1.5}
    \begin{tabularx}{\textwidth}{|c|X|X|X|X|X|X|c|}\hline
    \rowcolor[HTML]{FFD700} 
    \textbf{Nominativi dei membri} & \textbf{Re} & \textbf{Am} & \textbf{An} & \textbf{Pg} & \textbf{Pr} & \textbf{Ve} & \textbf{Ore per membro} \\ \hline
    Federica Bolognini & 1 & 1 & 2 & 0 & 0 & 2 & 6  \\ \hline
    Michael Fantinato  & 0 & 3 & 3 & 0 & 0 & 2 & 8  \\ \hline
    Giacomo Loat       & 0 & 7 & 2 & 0 & 0 & 2 & 11 \\ \hline
    Filippo Righetto   & 0 & 2 & 4 & 0 & 0 & 2 & 8  \\ \hline
    Riccardo Stefani   & 7 & 3 & 2 & 0 & 0 & 2 & 14 \\ \hline
    Davide Verzotto    & 0 & 1 & 4 & 0 & 0 & 2 & 7  \\ \hline
    \rowcolor[HTML]{FFD700} 
    \textbf{Ore totali per ruolo} & 8 & 17 & 17 & 0 & 0 & 12 & \textbf{Ore totali del gruppo: 54} \\ \hline
    \end{tabularx}
    \caption{Suddivisione oraria per ruolo nel primo periodo}
\end{table}

\paragraph{Prospetto economico primo periodo: 21/11/2024 - 05/12/2024: }
Il prospetto economico relativo al primo periodo evidenzia i costi sostenuti per ciascun membro del team, suddivisi per ruolo, e il saldo complessivo a fine periodo.\\
L'analisi dei costi si è concentrata sulle ore di lavoro effettivamente registrate, che hanno comportato un lieve incremento rispetto al preventivo iniziale, dovuto principalmente a un maggiore impegno nelle attività di analisi e amministrazione.\\
In dettaglio, il costo orario per ogni membro del team è stato calcolato in base al ruolo interpretato, con i costi totali che sono stati ottenuti considerando le ore di lavoro svolte.\\
Il totale delle spese sostenute per il primo periodo ammonta a 1185 \euro, con un saldo finale che riflette l'andamento positivo del progetto, nonostante gli aumenti orari in alcune aree.\\
Questo prospetto offre una visione chiara dell'impatto economico del periodo e consente di monitorare il progresso rispetto al budget complessivo del progetto.
\begin{table}[!h]
    \centering
    \renewcommand{\arraystretch}{1.5}
    \begin{tabularx}{\textwidth}{|c|X|X|X|X|X|X|c|}\hline
    \rowcolor[HTML]{FFD700} 
    \textbf{Costo} & \textbf{Re} & \textbf{Am} & \textbf{An} & \textbf{Pg} & \textbf{Pr} & \textbf{Ve} & \textbf{Totale} \\ \hline
    Costo orario & 30 \euro & 20 \euro & 25 \euro & 25 \euro & 15 \euro & 15 \euro & /  \\ \hline
    Costo totale & 240 \euro & 340 \euro & 425 \euro & 0 \euro & 0 \euro & 180 \euro & 1185 \euro \\ \hline
    \rowcolor[HTML]{FFD700} 
    \textbf{Saldo a fine periodo}  & 1740 \euro & 740 \euro & 1675 \euro & 3300 \euro & 1620 \euro & 1260 \euro & 10335 \euro \\ \hline
    \end{tabularx}
    \caption{Costi sostenuti durante il primo periodo e saldo rimanente}
\end{table}


\paragraph{Rischi occorsi primo periodo: 21/11/2024 - 05/12/2024: }
I rischi occorsi durante il primo periodo sono stati:
\begin{itemize}
    \item \S\bulref{sec:Complessità delle nuove tecnologie}{: Complessità delle nuove tecnologie};
    \item \S\bulref{sec:Rischi di comunicazione interna}{: Rischi di comunicazione interna};
    \item \S\bulref{sec:Rischi legati alla continuità del progetto}{: Rischi legati alla continuità del progetto}.
\end{itemize}Durante il progetto sono emerse criticità significative legate a diversi rischi. I rischi di comunicazione interna si sono manifestati principalmente nella mancata condivisione delle informazioni relative ai problemi riscontrati nello studio delle tecnologie: alcuni membri del team non hanno comunicato tempestivamente le difficoltà o le soluzioni adottate, causando ritardi e inefficienze.
I rischi legati alla continuità del progetto, invece, si sono concretizzati in assenze improvvise e nella discontinuità di alcune risorse, creando ulteriori ostacoli alla gestione operativa del lavoro di gruppo.
Infine, la complessità delle nuove tecnologie ha rappresentato una sfida particolarmente impegnativa, richiedendo uno sforzo significativo per adattare e integrare strumenti e risorse nel processo di sviluppo. La combinazione di questi fattori ha sottolineato l'importanza di una gestione più strutturata e preventiva per garantire il successo del progetto.

%Secondo Periodo - CON FINE SECONDO SPRINT
\subsubsection{Secondo periodo: 06/12/2024 - 19/12/2024}
\label{sec:prev_cons_secondo_periodo}
Gli obiettivi fissati per il secondo periodo sono:
    \begin{itemize}
        \item Proseguire con l’\emph{Analisi dei casi d’uso}, identificando e documentando i punti chiave attraverso il confronto con il \emph{proponente} e il dialogo con il professore Cardin;
        \item Trascrivere i \emph{requisiti}, trasformando i \emph{casi d’uso} in specifiche tecniche e funzionali;
        \item Per quanto riguarda le tecnologie:
        \begin{itemize}
            \item Verificare le tecnologie scelte e integrarle progressivamente nel \emph{PoC};
            \item Realizzare un \emph{PoC} che risponda a domande riguardanti dati presenti su \emph{GitHub}, \emph{Confluence} e \emph{Jira};
            \item Organizzare il \emph{PoC} in una struttura modulare e orientata alle classi;
            \item Implementare header che forniscano istruzioni al browser, incrementandoli in base alle esigenze;
            \item Affrontare le problematiche legate alla ricerca di similarità attraverso una migliore strutturazione dei dati.
        \end{itemize}
        \item Compilare una lista di documenti di riferimento chiave e organizzarli in \emph{Confluence}, \emph{Jira} e \emph{GitHub}, per garantire un contesto chiaro e accessibile al team.
    \end{itemize}
    

\subsubsubsection{Preventivo secondo periodo: 06/12/2024 - 19/12/2024}
\begin{table}[h!]
    \centering
    \renewcommand{\arraystretch}{1.5}
    \begin{tabularx}{\textwidth}{|c|X|X|X|X|X|X|c|}\hline
    \rowcolor[HTML]{FFD700} 
    \textbf{Nominativi dei membri} & \textbf{Re} & \textbf{Am} & \textbf{An} & \textbf{Pg} & \textbf{Pr} & \textbf{Ve} & \textbf{Ore per membro} \\ \hline
    Federica Bolognini  & 2 & 0 & 2 & 0 & 0 & 2 & 6 \\ \hline
    Michael Fantinato   & 0 & 4 & 0 & 0 & 0 & 2 & 6 \\ \hline
    Giacomo Loat        & 0 & 0 & 2 & 0 & 3 & 1 & 6 \\ \hline
    Filippo Righetto    & 0 & 4 & 1 & 0 & 1 & 1 & 7 \\ \hline
    Riccardo Stefani    & 3 & 0 & 4 & 0 & 0 & 1 & 8 \\ \hline
    Davide Verzotto     & 3 & 3 & 0 & 0 & 0 & 1 & 7 \\ \hline
    \rowcolor[HTML]{FFD700} 
    \textbf{Ore totali per ruolo} & 8 & 11 & 9 & 0 & 4 & 8 & \textbf{Ore totali del gruppo: 40} \\ \hline
    \end{tabularx}
    \caption{Preventivo della suddivisione oraria per ruolo nel secondo periodo}
\end{table}

\subsubsubsection{Consuntivo secondo periodo: 06/12/2024 - 19/12/2024}
Tutti gli obiettivi predisposti per il secondo periodo, che coincide con la seconda \emph{sprint}, sono stati soddisfatti con successo. Con un po' di fatica, è stata terminata la modellazione dei \emph{casi d'uso}, mentre la trascrizione dei \emph{requisiti}, trasformando i \emph{casi d’uso} in specifiche tecniche e funzionali, è stata completata positivamente.\\
Lo studio e la validazione delle tecnologie necessarie per il \emph{PoC} sono stati conclusi, e la loro integrazione iniziale è stata avviata con risultati soddisfacenti. Infine, è stato sviluppato il \emph{PoC} con una struttura modulare e organizzata in classi, rispondendo a domande riguardanti dati presenti su \emph{GitHub}, \emph{Confluence} e \emph{Jira}.\\
La documentazione di riferimento chiave è stata completata e organizzata in \emph{Confluence}, \emph{Jira} e \emph{GitHub}, garantendo maggiore chiarezza per il team.\\
La continuazione del \emph{Piano di Qualifica} è stata portata avanti regolarmente, integrando i progressi fatti durante il periodo. Anche per questa seconda fase, il preventivo, il consuntivo, il prospetto orario, il prospetto economico e i rischi occorsi sono stati redatti nel \textit{Piano di Progetto}, chiudendo con successo il periodo di lavoro.


\paragraph{Prospetto orario: }
Nel secondo periodo si è registrato un consumo orario superiore a quello preventivato, con un eccesso di 10 ore di lavoro complessive per il gruppo. Questo incremento è stato determinato principalmente dalle ore assegnate ad Analista e Progettista, mentre si è osservata una riduzione delle ore destinate ad Amministratore.
L'oneroso incremento del numero di ore impiegate dall'analista è dovuto all'elevato impegno speso dal gruppo durante la sprint nel documento di \emph{Analisi dei Requisiti}, che ha richiesto un maggiore sforzo rispetto a quanto preventivato, soprattutto a causa di difficoltà nella modellazione sintattica e semantica dei \emph{casi d'uso}.\\
\begin{table}[h!]
    \centering
    \renewcommand{\arraystretch}{1.5}
    \begin{tabularx}{\textwidth}{|c|X|X|X|X|X|X|c|}\hline
    \rowcolor[HTML]{FFD700} 
    \textbf{Nominativi dei membri} & \textbf{Re} & \textbf{Am} & \textbf{An} & \textbf{Pg} & \textbf{Pr} & \textbf{Ve} & \textbf{Ore per membro} \\ \hline
    Federica Bolognini & 0 & 0 & 6 & 0 & 0 & 1 & 7  \\ \hline
    Michael Fantinato  & 0 & 0 & 2 & 5 & 0 & 0 & 7  \\ \hline
    Giacomo Loat       & 1 & 3 & 3 & 0 & 2 & 1 & 10 \\ \hline
    Filippo Righetto   & 0 & 0 & 6 & 0 & 0 & 1 & 7  \\ \hline
    Riccardo Stefani   & 2 & 1 & 3 & 0 & 5 & 1 & 12 \\ \hline
    Davide Verzotto    & 0 & 0 & 5 & 0 & 0 & 2 & 7  \\ \hline
    \rowcolor[HTML]{FFD700} 
    \textbf{Ore totali per ruolo} & 3 & 4 & 25 & 5 & 7 & 6 & \textbf{Ore totali del gruppo: 50} \\ \hline
    \end{tabularx}
    \caption{Suddivisione oraria per ruolo nel secondo periodo}
\end{table}

\paragraph{Prospetto economico secondo periodo: 06/12/2024 - 19/12/2024: }
Il prospetto economico relativo al secondo periodo evidenzia i costi sostenuti per ciascun membro del team, suddivisi per ruolo, e il saldo complessivo a fine periodo.\\
L'analisi dei costi si è concentrata sulle ore di lavoro effettivamente registrate, che hanno comportato un incremento rispetto al preventivo iniziale, dovuto principalmente ad un maggiore 
impegno nelle attività di analisi e progettazione, in particolare nell'attività di analisi, poichè abbiamo incontrato difficoltà nella modellazione dei \emph{casi d'uso}. \\
In dettaglio, il costo orario per ogni membro del team è stato calcolato in base al ruolo interpretato, con i costi totali che sono stati ottenuti considerando le ore di lavoro svolte. \\
Il totale delle spese sostenute per il secondo periodo ammonta a 1115 \euro, con un saldo finale che riflette l'andamento positivo del progetto, nonostante gli aumenti orari in alcune aree.\\
Questo prospetto offre una visione chiara dell'impatto economico del periodo e consente di monitorare il progresso rispetto al budget complessivo del progetto.
\begin{table}[!h]
    \centering
    \renewcommand{\arraystretch}{1.5}
    \begin{tabularx}{\textwidth}{|c|X|X|X|X|X|X|c|}\hline
    \rowcolor[HTML]{FFD700} 
    \textbf{Costo} & \textbf{Re} & \textbf{Am} & \textbf{An} & \textbf{Pg} & \textbf{Pr} & \textbf{Ve} & \textbf{Totale} \\ \hline
    Costo orario & 30 \euro & 20 \euro & 25 \euro & 25 \euro & 15 \euro & 15 \euro & /  \\ \hline
    Costo totale & 90 \euro & 80 \euro & 625 \euro & 125 \euro & 105 \euro & 90 \euro & 1115 \euro \\ \hline
    \rowcolor[HTML]{FFD700} 
    \textbf{Saldo a fine periodo}  & 1650 \euro & 660 \euro  & 1050 \euro & 3175 \euro & 1515 \euro & 1170 \euro & 9220 \euro \\ \hline
    \end{tabularx}
    \caption{Costi sostenuti durante il secondo periodo e saldo rimanente}
\end{table}


\paragraph{Rischi occorsi secondo periodo: 06/12/2024 - 19/12/2024: }
Il rischio occorso durante il secondo periodo è stato:
\begin{itemize}
    \item \S\bulref{sec:Mancanza di risorse e documentazione}{: Mancanza di risorse e documentazione}.
\end{itemize}
Per mancanza di risorse e documentazioneo si intende principalmente una carenza di conoscenze specifiche necessarie per condurre l’analisi dei \emph{Casi d’Uso}.\\
Questo ha richiesto uno studio approfondito e un maggiore impegno per colmare i dubbi e garantire un’analisi corretta e completa. 


\newpage
%Terzo Periodo - CON FINE TERZO SPRINT
\subsubsection{Terzo periodo: 20/12/2024 - 03/01/2024}  
\label{sec:prev_cons_terzo_periodo}  

Gli obiettivi fissati per il terzo periodo sono:  
\begin{itemize}  
    \item Revisione dei \emph{casi d’uso} e dei \emph{requisiti}: 
    \begin{itemize}  
        \item Classificare i \emph{requisiti} in obbligatori, desiderabili e opzionali concentrandosi su quelli prioritari per il \emph{PoC};  
        \item Coinvolgere il \emph{proponente} per allineare le aspettative sui \emph{requisiti} chiave e sugli obiettivi di breve termine.  
    \end{itemize}  
    \item Continuazione dello sviluppo del \emph{PoC}:
    \begin{itemize}  
        \item Focalizzarsi sull’implementazione dei \emph{requisiti} obbligatori, lasciando in sospeso quelli opzionali e desiderabili.    
    \end{itemize}   
\end{itemize} 

\subsubsubsection{Preventivo terzo periodo: 20/12/2024 - 03/01/2025}
\begin{table}[h!]
    \centering
    \renewcommand{\arraystretch}{1.5}
    \begin{tabularx}{\textwidth}{|c|X|X|X|X|X|X|c|}\hline
    \rowcolor[HTML]{FFD700} 
    \textbf{Nominativi dei membri} & \textbf{Re} & \textbf{Am} & \textbf{An} & \textbf{Pg} & \textbf{Pr} & \textbf{Ve} & \textbf{Ore per membro} \\ \hline
    Federica Bolognini  & 1 & 2 & 4 & 0 & 0 & 1 & 8 \\ \hline
    Michael Fantinato   & 2 & 0 & 2 & 0 & 3 & 1 & 8 \\ \hline
    Giacomo Loat        & 0 & 0 & 1 & 0 & 4 & 1 & 6 \\ \hline
    Filippo Righetto    & 2 & 2 & 3 & 0 & 0 & 1 & 8 \\ \hline
    Riccardo Stefani    & 0 & 0 & 1 & 0 & 3 & 1 & 5 \\ \hline
    Davide Verzotto     & 3 & 0 & 4 & 0 & 0 & 2 & 9 \\ \hline
    \rowcolor[HTML]{FFD700} 
    \textbf{Ore totali per ruolo} & 8 & 4 & 15 & 0 & 10 & 7 & \textbf{Ore totali del gruppo: 44} \\ \hline
    \end{tabularx}
    \caption{Preventivo della suddivisione oraria per ruolo nel terzo periodo}
\end{table}

\subsubsubsection{Consuntivo terzo periodo: 20/12/2024 - 03/01/2025}
Tutti gli obiettivi predisposti per il terzo periodo, che coincide con la terza \emph{sprint}, sono stati soddisfatti con successo.
È stata effettuata la revisione dei \emph{casi d’uso} e dei \emph{requisiti}. È stato inoltre coinvolto il \emph{proponente} per allineare le aspettative sui \emph{requisiti} chiave e sugli obiettivi di breve termine.
Infine, la continuazione dello sviluppo del \emph{PoC} ha visto l'implementazione di alcuni \emph{requisiti} obbligatori, lasciando in sospeso quelli opzionali e desiderabili.

\newpage
\paragraph{Prospetto orario: }
Nel terzo periodo si è registrato un leggero superamento delle ore di lavoro rispetto a quanto preventivato, con un totale di 3 ore in eccesso per il gruppo.\\
Questo incremento è stato dovuto principalmente all’aumento delle ore assegnate al Verificatore e al Programmatore, mentre per gli altri ruoli il consumo orario è rimasto invariato rispetto a quanto pianificato.
\begin{table}[h!]
    \centering
    \renewcommand{\arraystretch}{1.5}
    \begin{tabularx}{\textwidth}{|c|X|X|X|X|X|X|c|}\hline
    \rowcolor[HTML]{FFD700} 
    \textbf{Nominativi dei membri} & \textbf{Re} & \textbf{Am} & \textbf{An} & \textbf{Pg} & \textbf{Pr} & \textbf{Ve} & \textbf{Ore per membro} \\ \hline
    Federica Bolognini & 5 & 0 & 5 & 0 & 0 & 0 & 10  \\ \hline
    Michael Fantinato  & 1 & 0 & 3 & 0 & 3 & 2 & 9  \\ \hline
    Giacomo Loat       & 0 & 3 & 0 & 0 & 4 & 0 & 7 \\ \hline
    Filippo Righetto   & 1 & 0 & 5 & 0 & 0 & 2 & 8  \\ \hline
    Riccardo Stefani   & 1 & 0 & 1 & 0 & 4 & 1 & 7 \\ \hline
    Davide Verzotto    & 0 & 1 & 1 & 0 & 0 & 4 & 6  \\ \hline
    \rowcolor[HTML]{FFD700} 
    \textbf{Ore totali per ruolo} & 8 & 4 & 15 & 0 & 11 & 9 & \textbf{Ore totali del gruppo: 47} \\ \hline
    \end{tabularx}
    \caption{Suddivisione oraria per ruolo nel terzo periodo}
\end{table}

\paragraph{Prospetto economico terzo periodo: 20/12/2024 - 03/01/2025: }
Il prospetto economico relativo al terzo periodo evidenzia i costi sostenuti per ciascun membro del team, suddivisi per ruolo, e il saldo complessivo a fine periodo.\\
L'analisi dei costi si è concentrata sulle ore di lavoro effettivamente registrate, che hanno comportato un lieve incremento rispetto al preventivo iniziale, dovuto principalmente a un maggiore impegno nelle attività di programmazione e verifica. \\
In dettaglio, il costo orario per ogni membro del team è stato calcolato in base al ruolo interpretato, con i costi totali che sono stati ottenuti considerando le ore di lavoro svolte. \\
Il totale delle spese sostenute per il terzo periodo ammonta a 995 \euro, con un saldo finale che riflette l'andamento positivo del progetto, nonostante gli aumenti orari in alcune aree.\\
Questo prospetto offre una visione chiara dell'impatto economico del periodo e consente di monitorare il progresso rispetto al budget complessivo del progetto.

\begin{table}[!h]
    \centering
    \renewcommand{\arraystretch}{1.5}
    \begin{tabularx}{\textwidth}{|c|X|X|X|X|X|X|c|}\hline
    \rowcolor[HTML]{FFD700} 
    \textbf{Costo} & \textbf{Re} & \textbf{Am} & \textbf{An} & \textbf{Pg} & \textbf{Pr} & \textbf{Ve} & \textbf{Totale} \\ \hline
    Costo orario & 30 \euro & 20 \euro & 25 \euro & 25 \euro & 15 \euro & 15 \euro & /  \\ \hline
    Costo totale & 240 \euro & 80 \euro & 375 \euro & 0 \euro & 165 \euro & 135 \euro & 995 \euro \\ \hline
    \rowcolor[HTML]{FFD700} 
    \textbf{Saldo a fine periodo}  & 1410 \euro & 740 \euro  & 675 \euro & 3175 \euro & 1350 \euro & 1035 \euro & 8385 \euro \\ \hline
    \end{tabularx}
    \caption{Costi sostenuti durante il terzo periodo e saldo rimanente}
\end{table}

\paragraph{Rischi occorsi terzo periodo: 20/12/2024 - 03/01/2025: }
I rischi occorsi durante il terzo periodo sono stati:
\begin{itemize}
    \item \S\bulref{sec:Complessità delle nuove tecnologie}{: Complessità delle nuove tecnologie};
    \item \S\bulref{sec:Rischi legati alla gestione del tempo e delle scadenze}{: Rischi legati alla gestione del tempo e delle scadenze}.
\end{itemize}
Un rischio che abbiamo identificato riguarda l'approfondimento di tecnologie non direttamente correlate al nostro progetto, in particolare {\emph{Docker}}\textsubscript{\textit{\textbf{G}}}.\\
Per risolverlo ci siamo confrontati con l'azienda per capire se Docker fosse un tema centrale del progetto e ci è stato confermato che, pur non essendo uno strumento necessario, non esula dagli obiettivi del progetto. Anzi, potrebbe favorire lo sviluppo.\\
Un altro rischio emerso è stato legato alla gestione del tempo e delle scadenze, a causa delle vacanze natalizie, che hanno influito sulla pianificazione e sull’avanzamento delle attività.

\newpage
%Quarto Periodo - CON FINE QUARTO SPRINT
\subsubsection{Quarto periodo: 04/01/2025 - 16/01/2025}  
\label{sec:prev_cons_quarto_periodo}  

Gli obiettivi fissati per il quarto periodo sono:  
\begin{itemize} 
    \item Portare a termine la redazione di alcuni documenti fondamentali per il progetto, tra cui: \emph{Analisi dei Requisiti}, \emph{Norme di Progetto} e \emph{Glossario};
    \item Ci siamo promessi di completare i documenti del \emph{Piano di Progetto} e \emph{Piano di Qualifica} dopo la fine della sprint, in quanto avere disponibili anche i dati di quest'ultima risulta essenziale per ottenere una visione completa e dettagliata dell'avanzamento del lavoro e delle strategie di gestione della qualità;  
    \item Finalizzare il \emph{PoC}, dimostrando al proponente il lavoro svolto fino a questo momento. La presentazione ha permesso di verificare le scelte progettuali effettuate e di raccogliere eventuali feedback utili per migliorare ulteriormente il prodotto;
    \item Valutare e programmare la consegna della \emph{RTB}.
\end{itemize} 

\subsubsubsection{Preventivo quarto periodo: 04/01/2025 - 16/01/2025}
\begin{table}[h!]
    \centering
    \renewcommand{\arraystretch}{1.5}
    \begin{tabularx}{\textwidth}{|c|X|X|X|X|X|X|c|}\hline
    \rowcolor[HTML]{FFD700} 
    \textbf{Nominativi dei membri} & \textbf{Re} & \textbf{Am} & \textbf{An} & \textbf{Pg} & \textbf{Pr} & \textbf{Ve} & \textbf{Ore per membro} \\ \hline
    Federica Bolognini  & 0 & 0 & 4 & 0 & 0 & 5 & 9 \\ \hline
    Michael Fantinato   & 4 & 0 & 2 & 0 & 4 & 0 & 10 \\ \hline
    Giacomo Loat        & 0 & 2 & 0 & 0 & 4 & 3 & 9 \\ \hline
    Filippo Righetto    & 3 & 4 & 0 & 0 & 0 & 2 & 9 \\ \hline
    Riccardo Stefani    & 3 & 0 & 2 & 0 & 4 & 0 & 9 \\ \hline
    Davide Verzotto     & 0 & 4 & 0 & 0 & 0 & 4 & 8 \\ \hline
    \rowcolor[HTML]{FFD700} 
    \textbf{Ore totali per ruolo} & 10 & 10 & 8 & 0 & 12 & 14 & \textbf{Ore totali del gruppo: 54} \\ \hline
    \end{tabularx}
    \caption{Preventivo della suddivisione oraria per ruolo nel quarto periodo}
\end{table}

\subsubsubsection{Consuntivo quarto periodo: 04/01/2025 - 16/01/2025}
Tutti gli obiettivi predisposti per il quarto periodo, che coincide con il quarto \emph{sprint}, sono stati soddisfatti con successo.

\paragraph{Prospetto orario: }
Nel quarto periodo si è registrata una lieve diminuzione delle ore di lavoro rispetto a quanto preventivato, con un totale di un'ora in meno per l'intero gruppo.
Ciò dimostra il miglioramento conseguito del team nella gestione del tempo con il passare degli sprint.\\
\begin{table}[h!]
    \centering
    \renewcommand{\arraystretch}{1.5}
    \begin{tabularx}{\textwidth}{|c|X|X|X|X|X|X|c|}\hline
    \rowcolor[HTML]{FFD700} 
    \textbf{Nominativi dei membri} & \textbf{Re} & \textbf{Am} & \textbf{An} & \textbf{Pg} & \textbf{Pr} & \textbf{Ve} & \textbf{Ore per membro} \\ \hline
    Federica Bolognini & 3 & 0 & 0 & 0 & 0 & 6 & 9  \\ \hline
    Michael Fantinato  & 0 & 4 & 0 & 0 & 4 & 0 & 8  \\ \hline
    Giacomo Loat       & 2 & 3 & 0 & 0 & 4 & 0 & 9 \\ \hline
    Filippo Righetto   & 2 & 0 & 2 & 0 & 0 & 6 & 10 \\ \hline
    Riccardo Stefani   & 2 & 1 & 2 & 0 & 4 & 0 & 9 \\ \hline
    Davide Verzotto    & 1 & 2 & 2 & 0 & 0 & 3 & 8  \\ \hline
    \rowcolor[HTML]{FFD700} 
    \textbf{Ore totali per ruolo} & 10 & 10 & 6 & 0 & 12 & 15 & \textbf{Ore totali del gruppo: 53} \\ \hline
    \end{tabularx}
    \caption{Suddivisione oraria per ruolo nel quarto periodo}
\end{table}

\paragraph{Prospetto economico quarto periodo: 04/01/2025 - 16/01/2025}  
Il prospetto economico relativo al quarto periodo evidenzia i costi sostenuti per ciascun membro del team, suddivisi per ruolo, e il saldo complessivo a fine periodo.\\  
L'analisi dei costi si è basata sulle ore di lavoro effettivamente registrate, le quali risultano leggermente inferiori rispetto al preventivo iniziale.\\  
In dettaglio, il costo orario per ogni membro del team è stato calcolato in base al ruolo interpretato, con i costi totali che sono stati ottenuti considerando le ore di lavoro svolte. \\
Il totale delle spese sostenute per il terzo periodo ammonta a 1055 \euro, con un saldo finale che riflette l'andamento positivo del progetto. \\
Questo prospetto offre una visione chiara dell'impatto economico del periodo e consente di monitorare il progresso rispetto al budget complessivo del progetto.

\begin{table}[!h]
    \centering
    \renewcommand{\arraystretch}{1.5}
    \begin{tabularx}{\textwidth}{|c|X|X|X|X|X|X|c|}\hline
    \rowcolor[HTML]{FFD700} 
    \textbf{Costo} & \textbf{Re} & \textbf{Am} & \textbf{An} & \textbf{Pg} & \textbf{Pr} & \textbf{Ve} & \textbf{Totale} \\ \hline
    Costo orario & 30 \euro & 20 \euro & 25 \euro & 25 \euro & 15 \euro & 15 \euro & /  \\ \hline
    Costo totale & 300 \euro & 200 \euro & 150 \euro & 0 \euro & 180 \euro & 225 \euro & 1055 \euro \\ \hline
    \rowcolor[HTML]{FFD700} 
    \textbf{Saldo a fine periodo}  & 1110 \euro & 540 \euro  & 525 \euro & 3175 \euro & 1170 \euro & 810 \euro & 7330 \euro \\ \hline
    \end{tabularx}
    \caption{Costi sostenuti durante il quarto periodo e saldo rimanente}
\end{table}

\paragraph{Rischi occorsi quarto periodo: 04/01/2025 - 16/01/2025: }
I rischi occorsi durante il quarto periodo sono stati:
\begin{itemize}
    \item \S\bulref{sec:Complessità delle nuove tecnologie}{: Complessità delle nuove tecnologie}.
\end{itemize}
In particolare, abbiamo riscontrato delle difficoltà nella suddivisione dei compiti di programmazione, perchè un programmatore che è stato incaricato di approfondire una determinata tematica non ha potuto chiedere aiuto riguardo ad essa perché i compagni di ruolo non avevano seguito il suo percorso di apprendimento ma bensì si avevano approfondito altri argomenti.
Per risolvere questo problema, abbiamo deciso di dedicare più tempo alla formazione e alla condivisione delle conoscenze, in modo da garantire che tutti i membri del team siano in grado di affrontare le sfide e le difficoltà che si presentano durante lo sviluppo del progetto.

\newpage
%Sommario finale RTB
\subsubsection{Sommario finale RTB}   
\label{sec:prev_cons_rtb} 
\subsubsubsection{Prospetto sulle tempistiche di lavoro} 

\begin{table}[h!]
    \centering
    \renewcommand{\arraystretch}{1.5}
    \begin{tabularx}{\textwidth}{|c|X|X|X|X|X|X|c|}\hline
    \rowcolor[HTML]{FFD700} 
    \textbf{Nominativi dei membri} & \textbf{Re} & \textbf{Am} & \textbf{An} & \textbf{Pg} & \textbf{Pr} & \textbf{Ve} & \textbf{Ore per membro} \\ \hline
    Federica Bolognini & 9 & 1 & 13 & 0 & 0 & 9 & 32  \\ \hline
    Michael Fantinato  & 1 & 7 & 8 & 5 & 7 & 4 & 32  \\ \hline
    Giacomo Loat       & 3 & 11 & 5 & 0 & 10 & 3 & 32 \\ \hline
    Filippo Righetto   & 3 & 2 & 17 & 0 & 0 & 11 & 33  \\ \hline
    Riccardo Stefani   & 12 & 2 & 8 & 0 & 13 & 4 & 39 \\ \hline
    Davide Verzotto    & 1 & 4 & 12 & 0 & 0 & 11 & 28  \\ \hline
    \rowcolor[HTML]{FFD700} 
    \textbf{Ore totali per ruolo} & 29 & 27 & 63 & 5 & 30 & 42 & \textbf{Ore totali del gruppo: 196} \\ \hline
    \end{tabularx}
    \caption{Prospetto orario dei membri durante l’RTB}
\end{table}

\begin{figure}[htbp]
    \centering
    \begin{tikzpicture}
        \pie[radius=3,
            text=pin,
            pin distance=0.5cm,
            sum=auto,
            before number=\footnotesize,
            after number=\%,
            color={
                yellow!90!orange,
                orange!70,
                yellow!85!red,
                orange!80!yellow,
                yellow!75!orange,
                orange!100!red
            }
        ]{
            21.43/Verificatore,
            15.31/Programmatore,
            2.55/Progettista,
            14.80/Responsabile,
            13.78/Amministratore,
            32.14/Analista
        }
    \end{tikzpicture}
    \caption{Grafico della distribuzione oraria per ruolo durante l’RTB}
\end{figure}

Rispetto all’orario specificato nel preventivo di costi e impegni orari presentato durante la fase di candidatura, la disponibilità oraria attuale è di 344 ore, a fronte delle 540 ore inizialmente previste.

\newpage
\subsubsubsection{Distribuzione delle ore nell’RTB} 
\begin{table}[h!]
    \centering
    \renewcommand{\arraystretch}{1.5}
    \begin{tabularx}{\textwidth}{|c|X|X|X|}
        \hline
        \rowcolor[HTML]{FFD700} 
        \textbf{Ruolo} & \textbf{Costo Orario} & \textbf{Costo RTB} & \textbf{Saldo per ruolo} \\
        \hline
        Responsabile & 30 \euro & 870 \euro & 1110 \euro \\
        \hline
        Amministratore & 20 \euro & 540 \euro & 540 \euro \\
        \hline
        Analista & 25 \euro & 1575 \euro & 525 \euro \\
        \hline
        Progettista & 25 \euro & 125 \euro & 3175 \euro \\
        \hline
        Programmatore & 15 \euro & 450 \euro & 1170 \euro \\
        \hline
        Verificatore & 15 \euro & 530 \euro & 810 \euro \\
        \hline
        \rowcolor[HTML]{FFD700} 
        \textbf{Totale} & / & 4190 \euro & 7330 \euro \\
        \hline
    \end{tabularx}
    \caption{Costo sostenuto durante l’RTB rispetto al preventivo consegnato}
\end{table}

\begin{figure}[htbp]
    \centering
    \begin{tikzpicture}
        \pie[radius=3,
            text=pin,
            pin distance=0.5cm,
            sum=auto,
            before number=\footnotesize,
            after number=\%,
            color={
                yellow!90!orange,
                orange!70,
                yellow!85!red,
                orange!80!yellow,
                yellow!75!orange,
                orange!100!red
            }
        ]{
            15.04/Verificatore,
            10.74/Programmatore,
            2.98/Progettista,
            20.76/Responsabile,
            12.89/Amministratore,
            37.59/Analista
        }
    \end{tikzpicture}
    \caption{Grafico della distribuzione dei costi durante l’RTB}
\end{figure}

Rispetto al costo totale indicato nel preventivo di costi e impegni orari consegnato, risulta un saldo in positivo di 7.330 \euro, netto del lavoro sostenuto durante il periodo di preparazione all’\emph{RTB}.

\newpage
\subsection{PB}
\subsubsubsection{Preventivo quinto periodo: 06/02/2025 - 20/02/2025}
\begin{table}[h!]
    \centering
    \renewcommand{\arraystretch}{1.5}
    \begin{tabularx}{\textwidth}{|c|X|X|X|X|X|X|c|}\hline
    \rowcolor[HTML]{FFD700} 
    \textbf{Nominativi dei membri} & \textbf{Re} & \textbf{Am} & \textbf{An} & \textbf{Pg} & \textbf{Pr} & \textbf{Ve} & \textbf{Ore per membro} \\ \hline
    Federica Bolognini  & 2 & 7 & 0 & 0 & 0 & 7 & 16 \\ \hline
    Michael Fantinato   & 0 & 0 & 0 & 10 & 5 & 0 & 15  \\ \hline
    Giacomo Loat        & 2 & 0 & 0 & 10 & 5 & 0 & 17  \\ \hline
    Filippo Righetto    & 3 & 0 & 7 & 0 & 0 & 7 & 17 \\ \hline
    Riccardo Stefani    & 0 & 0 & 5 & 10 & 0 & 0 & 15 \\ \hline
    Davide Verzotto     & 3 & 0 & 7 & 0 & 5 & 0 & 15  \\ \hline
    \rowcolor[HTML]{FFD700} 
    \textbf{Ore totali per ruolo} & 10 & 7 & 19 & 30 & 15 & 14 & \textbf{Ore totali del gruppo: 95} \\ \hline
    \end{tabularx}
    \caption{Preventivo della suddivisione oraria per ruolo nel quinto periodo}
\end{table}

\subsubsection{Quinto periodo: 06/02/2025 - 20/02/2025}
\label{sec:prev_cons_quinto_periodo}  
Gli obiettivi fissati per il quinto periodo sono:
\begin{itemize}
    \item Effettuare le correzioni consigliate dopo l' \emph{RTB};
    \item Proseguire la stesura della documentazione, integrando la \emph{Specifica Tecnica} e il \emph{Manuale Utente};
    \item Comunicare all'azienda \emph{AzzurroDigitale} le prossime sprint e i relativi incontri;
    \item Iniziare a modellare i vari scenari legati alla gestione delle sessioni, valutando l’uso del Singleton per garantire la persistenza della sessione, la creazione dinamica nel database e la generazione contestuale delle domande, anche in caso di refresh di \emph{Angular};
    \item Migliorare il salvataggio e recupero dello storico di sessione, ottimizzare la proposta di domande per rendere la conversazione più naturale ed efficace, potenziare l’aggiornamento automatico, introdurre la visualizzazione dell’esito e una gestione più efficiente delle modifiche ai file, in particolare con i file di \emph{GitHub}.
\end{itemize}

\subsubsubsection{Consuntivo quinto periodo: 06/02/2025 - 20/02/2025}
Tutti gli obiettivi predisposti all'inizio del periodo sono stati soddisfatti con successo.\\

\newpage
\paragraph{Prospetto orario: }
Nel quinto periodo si è registrato un consumo orario superiore a quello preventivato, con un eccesso di 4 ore di lavoro complessive per il gruppo. Questo incremento è stato determinato principalmente dalle ore assegnate al Progettista.
\begin{table}[h!]
    \centering
    \renewcommand{\arraystretch}{1.5}
    \begin{tabularx}{\textwidth}{|c|X|X|X|X|X|X|c|}\hline
    \rowcolor[HTML]{FFD700} 
    \textbf{Nominativi dei membri} & \textbf{Re} & \textbf{Am} & \textbf{An} & \textbf{Pg} & \textbf{Pr} & \textbf{Ve} & \textbf{Ore per membro} \\ \hline
    Federica Bolognini & 1 & 0 & 5 & 10 & 0 & 1 & 17  \\ \hline
    Michael Fantinato  & 2 & 2 & 1 & 10 & 0 & 0 & 15  \\ \hline
    Giacomo Loat       & 1 & 0 & 2 & 10 & 3 & 0 & 16 \\ \hline
    Filippo Righetto   & 1 & 2 & 5 & 7 & 0 & 1 & 16  \\ \hline
    Riccardo Stefani   & 0 & 0 & 5 & 5 & 2 & 2 & 12 \\ \hline
    Davide Verzotto    & 2 & 0 & 1 & 16 & 0 & 4 & 23  \\ \hline
    \rowcolor[HTML]{FFD700} 
    \textbf{Ore totali per ruolo} & 7 & 4 & 19 & 58 & 5 & 6 & \textbf{Ore totali del gruppo: 99} \\ \hline
    \end{tabularx}
    \caption{Suddivisione oraria per ruolo nel quinto periodo}
\end{table}

\paragraph{Prospetto economico quinto periodo: 06/02/2025 - 20/02/2025: }
Il prospetto economico relativo al quinto periodo evidenzia i costi sostenuti per ciascun membro del team, suddivisi per ruolo, e il saldo complessivo a fine periodo.\\
L'analisi dei costi si è concentrata sulle ore di lavoro effettivamente registrate, che hanno comportato un lieve incremento rispetto al preventivo iniziale, dovuto principalmente a un maggiore impegno nell'attività di progettazione.\\
Si è registrato inoltre un consumo orario inferiore a quello preventivato per tutti i ruoli tranne analista e progettista, dovuto principalmente agli impegni di studio della maggior parte dei membri del gruppo.\\

In dettaglio, il costo orario per ogni membro del team è stato calcolato in base al ruolo interpretato, con i costi totali che sono stati ottenuti considerando le ore di lavoro svolte.\\
Il totale delle spese sostenute per il quinto periodo ammonta a 2380 \euro, con un saldo finale che riflette l'andamento positivo del progetto, nonostante gli aumenti orari in alcune aree.\\
Questo prospetto offre una visione chiara dell'impatto economico del periodo e consente di monitorare il progresso rispetto al budget complessivo del progetto.
\begin{table}[!h]
    \centering
    \renewcommand{\arraystretch}{1.5}
    \begin{tabularx}{\textwidth}{|c|X|X|X|X|X|X|c|}\hline
    \rowcolor[HTML]{FFD700} 
    \textbf{Costo} & \textbf{Re} & \textbf{Am} & \textbf{An} & \textbf{Pg} & \textbf{Pr} & \textbf{Ve} & \textbf{Totale} \\ \hline
    Costo orario & 30 \euro & 20 \euro & 25 \euro & 25 \euro & 15 \euro & 15 \euro & /  \\ \hline
    Costo totale & 210 \euro & 80 \euro & 475 \euro & 1450 \euro & 75 \euro & 90 \euro & 2380 \euro \\ \hline
    \rowcolor[HTML]{FFD700} 
    \textbf{Saldo a fine periodo}  & 900 \euro & 460 \euro & 50 \euro & 1725 \euro & 1095 \euro & 720 \euro & 4950 \euro \\ \hline
    \end{tabularx}
    \caption{Costi sostenuti durante il quinto periodo e saldo rimanente}
\end{table}


\paragraph{Rischi occorsi quinto periodo: 06/02/2025 - 20/02/2025: }
I rischi occorsi durante il quinto periodo sono stati:
\begin{itemize}
    \item \S\bulref{sec:Rischi legati alla continuità del progetto}{: Rischi legati alla continuità del progetto}.
\end{itemize}
I rischi legati alla continuità del progetto si sono concretizzati a causa del poco tempo a disposizione, poiché alcuni membri del gruppo si sono dedicati al secondo appello di Ingegneria del Software. Di conseguenza, il tempo disponibile è stato concentrato principalmente sull'analisi dei requisiti e sulla progettazione.
\newpage
\subsubsubsection{Preventivo sesto periodo: 21/02/2025 - 06/03/2025}
\begin{table}[h!]
    \centering
    \renewcommand{\arraystretch}{1.5}
    \begin{tabularx}{\textwidth}{|c|X|X|X|X|X|X|c|}\hline
    \rowcolor[HTML]{FFD700} 
    \textbf{Nominativi dei membri} & \textbf{Re} & \textbf{Am} & \textbf{An} & \textbf{Pg} & \textbf{Pr} & \textbf{Ve} & \textbf{Ore per membro} \\ \hline
    Federica Bolognini  & 3 & 1 & 0 & 10 & 0 & 6 & 20 \\ \hline
    Michael Fantinato   & 0 & 0 & 0 & 9 & 10 & 0 & 19 \\ \hline
    Giacomo Loat        & 1 & 0 & 0 & 8 & 10 & 0 & 19 \\ \hline
    Filippo Righetto    & 1 & 2 & 0 & 10 & 0 & 4 & 17 \\ \hline
    Riccardo Stefani    & 0 & 0 & 0 & 5 & 10 & 0 & 15 \\ \hline
    Davide Verzotto     & 5 & 5 & 2 & 8 & 0 & 2 & 22 \\ \hline
    \rowcolor[HTML]{FFD700} 
    \textbf{Ore totali per ruolo} & 10 & 8 & 2 & 50 & 30 & 12 & \textbf{Ore totali del gruppo: 112} \\ \hline
    \end{tabularx}
    \caption{Preventivo della suddivisione oraria per ruolo nel sesto periodo}
\end{table}


\newpage
\subsubsection{Sesto periodo: 21/02/2025 - 06/03/2025}
\label{sec:prev_cons_sesto_periodo}  
Gli obiettivi fissati per il sesto periodo sono:
\begin{itemize}
    \item Correggere i diagrammi di progettazione affinché rispettino con precisione le esigenze del \emph{proponente}.
    \item Convertire i diagrammi del \emph{backend} dell'applicazione in codice \emph{Python}.
    \item Sviluppare i \emph{test di integrazione}.
    \item Sperimentare soluzioni grafiche per il frontend in \emph{Angular}.
    \item Definire una strategia per ottimizzare la gestione delle modifiche e delle eliminazioni dei documenti nel \emph{database vettoriale}.
\end{itemize}

\subsubsubsection{Consuntivo sesto periodo: 21/02/2025 - 06/03/2025}
Tutti gli obiettivi predisposti all'inizio del periodo sono stati soddisfatti con successo.\\

\paragraph{Prospetto orario: }
Nel sesto periodo si è registrato un consumo orario in linea con quanto preventivato, evidenziando un miglioramento nella capacità di collaborazione del team nel tempo. Il ruolo del progettista è quello che ha registrato il maggior consumo di ore, poiché l'attività principale è stata la progettazione. Un numero elevato di ore è stato dedicato anche al programmatore, poiché in questa sprint è stata avviata la fase di programmazione. Di conseguenza, gli altri ruoli hanno registrato un consumo di ore inferiore rispetto alle previsioni.
Inoltre, l'analisi dei requisiti è stata definitivamente completata, esattamente in concomitanza con l'esaurimento delle ore previste per il ruolo di analista.
\begin{table}[h!]
    \centering
    \renewcommand{\arraystretch}{1.5}
    \begin{tabularx}{\textwidth}{|c|X|X|X|X|X|X|c|}\hline
    \rowcolor[HTML]{FFD700} 
    \textbf{Nominativi dei membri} & \textbf{Re} & \textbf{Am} & \textbf{An} & \textbf{Pg} & \textbf{Pr} & \textbf{Ve} & \textbf{Ore per membro} \\ \hline
    Federica Bolognini & 1 & 1 & 0 & 9 & 0 & 6 & 17  \\ \hline
    Michael Fantinato  & 0 & 0 & 0 & 10 & 10 & 0 & 20  \\ \hline
    Giacomo Loat       & 1 & 0 & 0 & 9 & 10 & 0 & 20 \\ \hline
    Filippo Righetto   & 1 & 2 & 0 & 11 & 0 & 4 & 18  \\ \hline
    Riccardo Stefani   & 0 & 0 & 0 & 5 & 10 & 0 & 15 \\ \hline
    Davide Verzotto    & 1 & 2 & 2 & 13 & 4 & 0 & 22  \\ \hline
    \rowcolor[HTML]{FFD700} 
    \textbf{Ore totali per ruolo} & 4 & 5 & 2 & 57 & 34 & 10 & \textbf{Ore totali del gruppo: 112} \\ \hline
    \end{tabularx}
    \caption{Suddivisione oraria per ruolo nel sesto periodo}
\end{table}

\paragraph{Prospetto economico quinto periodo: 21/02/2025 - 06/03/2025: }
Il prospetto economico relativo al quinto periodo evidenzia i costi sostenuti per ciascun membro del team, suddivisi per ruolo, e il saldo complessivo a fine periodo.\\
L'analisi dei costi si è concentrata sulle ore di lavoro effettivamente registrate, che hanno comportato un lieve incremento rispetto al preventivo iniziale, dovuto principalmente a un maggiore impegno nell'attività di progettazione.\\
In dettaglio, il costo orario per ogni membro del team è stato calcolato in base al ruolo interpretato, con i costi totali che sono stati ottenuti considerando le ore di lavoro svolte.\\
Il totale delle spese sostenute per il quinto periodo ammonta a 2355 \euro.\\
Questo prospetto offre una visione chiara dell'impatto economico del periodo e consente di monitorare il progresso rispetto al budget complessivo del progetto.
\begin{table}[!h]
    \centering
    \renewcommand{\arraystretch}{1.5}
    \begin{tabularx}{\textwidth}{|c|X|X|X|X|X|X|c|}\hline
    \rowcolor[HTML]{FFD700} 
    \textbf{Costo} & \textbf{Re} & \textbf{Am} & \textbf{An} & \textbf{Pg} & \textbf{Pr} & \textbf{Ve} & \textbf{Totale} \\ \hline
    Costo orario & 30 \euro & 20 \euro & 25 \euro & 25 \euro & 15 \euro & 15 \euro & /  \\ \hline
    Costo totale & 120 \euro & 100 \euro & 50 \euro & 1425 \euro & 510 \euro & 150 \euro & 2355 \euro \\ \hline
    \rowcolor[HTML]{FFD700} 
    \textbf{Saldo a fine periodo}  & 780 \euro & 360 \euro & 0 \euro & 300 \euro & 585 \euro & 570 \euro & 2595 \euro \\ \hline
    \end{tabularx}
    \caption{Costi sostenuti durante il sesto periodo e saldo rimanente}
\end{table}

\newpage
\paragraph{Rischi occorsi sesto periodo: 21/02/2025 - 06/03/2025: }
I rischi occorsi durante il sesto periodo sono stati:
\begin{itemize}
    \item \S\bulref{sec:Complessità delle nuove tecnologie}{: Complessità delle nuove tecnologie}.
\end{itemize}
I rischi legati alla complessità delle nuove tecnologie derivano dal fatto che l’avvento della programmazione ci ha posto di fronte a sfide tecnologiche che abbiamo dovuto affrontare e risolvere.

\subsubsubsection{Preventivo settimo periodo: 07/02/2025 - 19/03/2025}
\begin{table}[h!]
    \centering
    \renewcommand{\arraystretch}{1.5}
    \begin{tabularx}{\textwidth}{|c|X|X|X|X|X|X|c|}\hline
    \rowcolor[HTML]{FFD700} 
    \textbf{Nominativi dei membri} & \textbf{Re} & \textbf{Am} & \textbf{An} & \textbf{Pg} & \textbf{Pr} & \textbf{Ve} & \textbf{Ore per membro} \\ \hline
    Federica Bolognini  & 5 & 4 & 0 & 0 & 0 & 3 & 12 \\ \hline
    Michael Fantinato   & 0 & 0 & 0 & 5 & 10 & 0 & 15 \\ \hline
    Giacomo Loat        & 0 & 0 & 0 & 5 & 10 & 0 & 15 \\ \hline
    Filippo Righetto    & 5 & 0 & 0 & 2 & 0 & 5 & 12 \\ \hline
    Riccardo Stefani    & 0 & 5 & 0 & 0 & 10 & 0 & 15 \\ \hline
    Davide Verzotto     & 3 & 0 & 0 & 0 & 0 & 5 & 8 \\ \hline
    \rowcolor[HTML]{FFD700} 
    \textbf{Ore totali per ruolo} & 13 & 9 & 0 & 12 & 30 & 13 & \textbf{Ore totali del gruppo: 77} \\ \hline
    \end{tabularx}
    \caption{Preventivo della suddivisione oraria per ruolo nel settimo periodo}
\end{table}

