% Intestazione
\fancyhead[L]{1 \hspace{0.2cm} Tecnologie coinvolte} % Testo a sinistra

\pagenumbering{arabic} % Numerazione araba per il contenuto 

\section{Tecnologie coinvolte}
\label{sec:tecnologie_coinvolte}

Questa sezione fornisce un’analisi esaustiva delle tecnologie impiegate nel progetto in questione, comprendendo le procedure, gli
strumenti e le librerie necessarie per lo sviluppo, il testing e la distribuzione del prodotto. Saranno discusse le tecnologie
utilizzate per implementare sia il backend che il frontend, la gestione dei dati e l’integrazione con i servizi previsti.


\subsection{Tecnologie utilizzate per la codifica}

\subsubsection{Strumenti per il backend}
\begin{table}[h!]
    \centering
    \renewcommand{\arraystretch}{1.6} % Per aumentare l'altezza delle righe
    \begin{tabularx}{\textwidth}{|p{2cm}|p{2cm}|X|} \hline
    \rowcolor[HTML]{FFD700} 
    \textbf{Nome} & \textbf{Versione} & \textbf{Descrizione} \\ \hline
    ... & ... & ... \\ \hline
    \end{tabularx}
\end{table}

\subsubsection{Strumenti per il frontend}
\begin{table}[h!]
    \centering
    \renewcommand{\arraystretch}{1.6} % Per aumentare l'altezza delle righe
    \begin{tabularx}{\textwidth}{|p{2cm}|p{2cm}|X|} \hline
    \rowcolor[HTML]{FFD700} 
    \textbf{Nome} & \textbf{Versione} & \textbf{Descrizione} \\ \hline
    ... & ... & ... \\ \hline
    \end{tabularx}
\end{table}

\subsubsection{Strumenti di gestione dei dati}
\label{subsec:strumenti_gestione_dati}
\begin{table}[h!]
    \centering
    \renewcommand{\arraystretch}{1.6} % Per aumentare l'altezza delle righe
    \begin{tabularx}{\textwidth}{|p{2cm}|p{2cm}|X|} \hline
    \rowcolor[HTML]{FFD700} 
    \textbf{Nome} & \textbf{Versione} & \textbf{Descrizione} \\ \hline
    Chroma & 0.6.4 & Chroma è un \emph{database vettoriale}\textsubscript{\textbf{\textit{G}}} progettato per memorizzare e
                    gestire vettori ad alta dimensionalità. 
                    È ottimizzato per operazioni di ricerca e recupero efficienti, dunque aiuta a ridurre il numero di documenti
                    di contesto da inviare all'\emph{LLM}\textsubscript{\textbf{\textit{G}}} selezionando solo quelli più rilevanti
                    per la query corrente grazie alla ricerca di \emph{similarità}\textsubscript{\textbf{\textit{G}}} tra 
                    quest'ultima e i documenti salvati come vettori. \\ \hline
    \end{tabularx}
\end{table}

\subsubsection{Strumenti di integrazione e di supporto}
\begin{table}[h!]
    \centering
    \renewcommand{\arraystretch}{1.6} % Per aumentare l'altezza delle righe
    \begin{tabularx}{\textwidth}{|p{2cm}|p{2cm}|X|} \hline
    \rowcolor[HTML]{FFD700} 
    \textbf{Nome} & \textbf{Versione} & \textbf{Descrizione} \\ \hline
    ... & ... & ... \\ \hline
    \end{tabularx}
\end{table}

\newpage

\subsection{Strumenti per l’analisi del codice}

\subsubsection{Strumenti per l’analisi statica}
\label{subsec:strumenti_analisi_statica}
\begin{table}[h!]
    \centering
    \renewcommand{\arraystretch}{1.6} % Per aumentare l'altezza delle righe
    \begin{tabularx}{\textwidth}{|p{2cm}|p{2cm}|X|} \hline
    \rowcolor[HTML]{FFD700} 
    \textbf{Nome} & \textbf{Versione} & \textbf{Descrizione} \\ \hline
    Pylint & 3.3.4 & Pylint è un analizzatore di codice statico e permette di
                    analizzare codice \emph{Python}\textsubscript{\textbf{\textit{G}}} senza eseguirlo. Controlla la presenza
                    di errori, applica uno standard di codifica e cerca di dare
                    suggerimenti su come il codice potrebbe essere modificato. \\ \hline
    SonarQube for IDE & 4.15.1 & Si tratta di un’estensione per \emph{IDE}\textsubscript{\textbf{\textit{G}}} 
                                che aiuta a rilevare
                                e correggere i problemi di qualità durante la scrittura
                                del codice, individuando i difetti in modo perchè possano
                                essere corretti prima del commit del codice. Non è specifico
                                per linguaggio, bensì supporta tanti linguaggi differenti. \\ \hline
    ESLint & 3.0.10 & ESLint è uno strumento di analisi del codice statico per identificare
                    e segnalare pattern trovati nel codice \emph{JavaScript}\textsubscript{\textbf{\textit{G}}}
                    e \emph{TypeScript}\textsubscript{\textbf{\textit{G}}}, cioè uno 
                    dei linguaggi utilizzati da \emph{Angular}\textsubscript{\textbf{\textit{G}}}. Aiuta a mantenere
                    uno stile di codifica
                    coerente e a prevenire errori comuni, fornendo suggerimenti per migliorare
                    la qualità del codice. \\ \hline
    \end{tabularx}
\end{table}

\subsubsection{Strumenti per l’analisi dinamica}
\begin{table}[h!]
    \centering
    \renewcommand{\arraystretch}{1.6} % Per aumentare l'altezza delle righe
    \begin{tabularx}{\textwidth}{|p{2cm}|p{2cm}|X|} \hline
    \rowcolor[HTML]{FFD700} 
    \textbf{Nome} & \textbf{Versione} & \textbf{Descrizione} \\ \hline
    ... & ... & ... \\ \hline
    \end{tabularx}
\end{table}
