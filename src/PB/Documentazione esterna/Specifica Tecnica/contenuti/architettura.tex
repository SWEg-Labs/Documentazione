% Intestazione
\fancyhead[L]{1 \hspace{0.2cm} Architettura} % Testo a sinistra

\pagenumbering{arabic} % Numerazione araba per il contenuto 

\section{Architettura}
\label{sec:architettura}

Questa sezione fornisce una descrizione dettagliata dell'architettura del prodotto software, illustrando le scelte progettuali adottate per garantire la corretta realizzazione del sistema. Saranno presentati le principali scelte e \emph{pattern architetturali}\textsubscript{\textbf{\textit{G}}}, e le \emph{componenti software}\textsubscript{\textbf{\textit{G}}} che compongono il sistema.
\subsection{Architettura logica}
L’architettura adottata nella realizzazione dell’applicativo si basa sul modello di architettura esagonale. Il livello di \emph{business}\textsubscript{\textbf{\textit{G}}} del software è quindi indipendente dagli altri componenti, ovvero nulla presente all’esterno della logica di business può conoscere la sua \emph{implementazione}\textsubscript{\textbf{\textit{G}}}. Questo principio alla base dell’architettura esagonale ci permette di ottenere un prodotto software facilmente testabile e \emph{manutenibile}\textsubscript{\textbf{\textit{G}}}. Le componenti che devono restare indipendenti sono rappresentate dal \emph{Domain business model}\textsubscript{\textbf{\textit{G}}}, che non comunica mai direttamente con l’esterno.
Gli elementi che permettono il funzionamento dell’architettura esagonale sono i seguenti:
\begin{itemize}
    \item \textbf{Port}: Possono essere presenti in numero variabile. Una port definisce un set di operazioni e permette alla business logic di interagire con l’esterno. Si tratta degli unici componenti comunicanti con il Domain Business Model. Ci sono due tipi di port:
    \begin{itemize}
        \item \textbf{Inbound Port}: si tratta di API esposte dalla business logic, implementatate proprio da quest’ultima in modo da poter essere invocate da un’applicazione esterna;
        \item \textbf{Outbound Port}: definiscono i metodi tramite i quali la business logic può invocare i sistemi esterni. 
    \end{itemize}
    \item \textbf{\emph{Adapter}\textsubscript{\textbf{\textit{G}}}}: si tratta di componenti che permettono di adattare i dati provenienti dall’esterno in modo che possano essere utilizzati dalla business logic. Sono presenti tra le port e la business logic. Un elemento fondamentale dell'architettura esagonale è che la logica di business non dipende dagli adapter; al contrario, sono questi ultimi a dipendere da essa. Analogamente alle port, gli adapter si suddividono in due categorie:
    \begin{itemize}
        \item \textbf{Inbound Adapter}: gestiscono la conversione dei dati provenienti dall'esterno invocando una inbound port. È possibile che più inbound adapter utilizzino la stessa inbound port;
        \item \textbf{Outbound Adapter}: implementano i metodi definiti da una outbound port, permettendo alla business logic di interagire con i sistemi esterni.
    \end{itemize}
\end{itemize}

È stato scelto di utilizzare un'architettura esagonale per i seguenti motivi:
\begin{itemize}
    \item Facilità di test: l'architettura esagonale permette di testare facilmente la business logic, in quanto è possibile sostituire gli adapter con degli \emph{stub}\textsubscript{\textbf{\textit{G}}} o dei \emph{mock}\textsubscript{\textbf{\textit{G}}};
    \item \emph{Manutenibilità}\textsubscript{\textbf{\textit{G}}}: l'architettura esagonale permette di mantenere la business logic indipendente dagli altri componenti, facilitando la manutenzione del codice;
    \item \emph{Scalabilità}\textsubscript{\textbf{\textit{G}}}: l'architettura esagonale permette di aggiungere nuovi adapter senza dover modificare la business logic.
\end{itemize}

\subsection{Architettura di Deployment}


