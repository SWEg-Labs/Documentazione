% Intestazione
\fancyhead[L]{1 \hspace{0.2cm} Introduzione} % Testo a sinistra

\pagenumbering{arabic} % Numerazione araba per il contenuto 

\section{Introduzione}
\label{sec:introduzione}

\subsection{Scopo del documento}
Questo documento fornisce linee guida per gli sviluppatori incaricati dell'estensione o della manutenzione del prodotto.\\
Al suo interno sono raccolte tutte le informazioni sui linguaggi e le tecnologie adottate, sull’architettura del sistema e sulle
scelte progettuali effettuate.

\subsection{Scopo del prodotto}
Nell’ultimo anno vi è stato un cambiamento repentino nello sviluppo e nell’applicazione
dell’\emph{Intelligenza Artificiale}\textsubscript{\textbf{\textit{G}}} all’elaborazione e raccomandazione dei contenuti alla
generazione
di essi, come immagini, testi e tracce audio.
Il \emph{capitolato}\textsubscript{\textbf{\textit{G}}} C9, "BuddyBot", pone come obiettivo la realizzazione di un applicativo che
permetta di porre interrogazioni in
linguaggio naturale sullo stato attuale dei progetti software in lavorazione, ricevendo una risposta il quanto più precisa. Tale
risposta dovrà essere generata tramite un \emph{LLM}\textsubscript{\textbf{\textit{G}}} collegato. Tale software sarà fruibile
attraverso un'\emph{applicazione web}\textsubscript{\textbf{\textit{G}}}, dove l'utente potrà interrogare il chatbot sullo stato
attuale del codice e della documentazione dei progetti software nelle piattaforme utilizzate per il loro sviluppo.

\subsection{Glossario}
Al fine di prevenire ed evitare possibili ambiguità nei termini e acronimi presenti all’interno della documentazione, è stato
realizzato un glossario nel file \emph{glossario\_v2.0.0.pdf} in grado di dare una definizione precisa per ogni vocabolo potenzialmente
ambiguo. All’interno di ogni documento i termini specifici, che quindi hanno una definizione all’interno del 
\emph{Glossario}\textsubscript{\textbf{\textit{G}}}, saranno
contrassegnati con una \textsubscript{\textbf{\textit{G}}} aggiunta a pedice e trascritti in corsivo. Tale prassi sarà rispettata
solamente per la prima occorrenza del termine in una determinata sezione del documento.

\subsection{Miglioramenti al documento}
La revisione e l’evoluzione del documento sono aspetti fondamentali per garantirne la \emph{qualità}\textsubscript{\textbf{\textit{G}}}
e l’adeguatezza nel tempo.
Questo consente di apportare modifiche in base alle esigenze concordate tra i membri del gruppo e il
\emph{proponente}\textsubscript{\textbf{\textit{G}}}.
Di conseguenza, questa versione del documento non può essere considerata definitiva o completa, in quanto soggetta a possibili
aggiornamenti futuri.

\subsection{Riferimenti}
\label{sec:riferimenti}

\subsubsection{Riferimenti normativi}
\begin{itemize}
    \item \bulhref{https://sweg-labs.github.io/Documentazione/output/PB/Documentazione\%20interna/norme_progetto_v2.0.0.pdf}{Norme di Progetto v.2.0.0}; \\
    \item \textbf{Capitolato d'appalto C9 - BuddyBot (slide 3-18)}: \\
    \bulhref{https://www.math.unipd.it/~tullio/IS-1/2024/Progetto/C9.pdf}{https://www.math.unipd.it/~tullio/IS-1/2024/Progetto/C9.pdf}
    \emph{(Ultimo accesso: 03/04/2025)};\\
    \item \textbf{Regolamento progetto didattico (slide 2-25)}:\\
    \bulhref{https://www.math.unipd.it/~tullio/IS-1/2024/Dispense/PD1.pdf}{https://www.math.unipd.it/~tullio/IS-1/2024/Dispense/PD1.pdf}
    \emph{(Ultimo accesso: 03/04/2025)};\\
\end{itemize}

\subsubsection{Riferimenti informativi}
\begin{itemize}
    \item \bulhref{https://sweg-labs.github.io/Documentazione/output/PB/Documentazione\%20esterna/glossario_v2.0.0.pdf}{Glossario v.2.0.0}; \\
\end{itemize}
