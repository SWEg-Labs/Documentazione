% Intestazione
\fancyhead[L]{J} % Testo a sinistra

\section{}

\hypertarget{sec:jasmine}{}
\subsection*{Jasmine}
Jasmine è un framework di testing per JavaScript, progettato per essere semplice e intuitivo. È spesso utilizzato per lo sviluppo di
applicazioni web, in particolare per il testing di codice front-end. Jasmine offre una sintassi chiara e leggibile, che facilita
la scrittura e l'esecuzione di test automatizzati, e fornisce una serie di funzionalità per la gestione delle asserzioni, dei mock
e dei test asincroni. Jasmine è compatibile con diversi runner di test e framework di sviluppo, come Karma e AngularJS.

\hypertarget{sec:javascript}{}
\subsection*{JavaScript}
Linguaggio di programmazione ad alto livello utilizzato per aggiungere interattività alle pagine web, grazie alla capacità di essere
orientato agli eventi esterni ed interagire con essi in modo asincrono. Esso viene utilizzato sia nella programmazione lato client
(front-end), che si occupa della parte grafica e dell’interfaccia utente, sia nella programmazione lato server (back-end), che
gestisce la logica e la comunicazione del sistema. JavaScript è uno dei linguaggi fondamentali del web, insieme a HTML e CSS, ed
è supportato da tutti i principali browser.

\subsection*{Jira}
Jira è uno strumento di gestione dei progetti e di issue tracking sviluppato da Atlassian, ampiamente utilizzato per la pianificazione, il monitoraggio e 
il controllo di progetti, in particolare nell'ambito dello sviluppo software. Nato come strumento di gestione dei bug e dei problemi, Jira è diventato uno 
dei principali strumenti per il project management, soprattutto per le organizzazioni che adottano metodologie agili come Scrum e Kanban.

\newpage


% Intestazione
\fancyhead[L]{K} % Testo a sinistra

\section{}

\hypertarget{sec:Kebab Case}{}
\subsection*{Kebab Case}
È una convenzione di scrittura in cui le parole sono separate da un trattino (-) e sono tutte in minuscolo.

\newpage


% Intestazione
\fancyhead[L]{L} % Testo a sinistra

\section{}

\hypertarget{sec:langchain}{}
\subsection*{LangChain}
LangChain è un framework Python e JavaScript open-source progettato per sviluppare applicazioni che sfruttano i Large Language Models 
(LLM). Offre una struttura modulare e flessibile per la creazione di catene di elaborazione (chains) che combinano modelli di 
linguaggio con altre fonti di dati e strumenti.

\subsection*{\LaTeX}
Linguaggio di marcatura per la preparazione di testi, basato sul programma di composizione tipografica TeX. LaTeX è ampiamente utilizzato per la creazione 
di documenti scientifici e tecnici grazie alla sua capacità di gestire formule matematiche complesse e alla sua alta qualità tipografica.

\hypertarget{sec:LLM}{}
\subsection*{Large Language Model (LLM)}
Il termine Large Language Model si riferisce a modelli di linguaggio avanzati e complessi che sono stati addestrati su enormi quantità di dati testuali. 
Questi modelli, basati su tecniche di intelligenza artificiale come il deep learning, sono in grado di comprendere e generare testo in linguaggio naturale 
in modo più sofisticato rispetto a modelli più piccoli (Small Language Model, SML).

\hypertarget{sec:logging}{}
\subsection*{Logging}
Il processo di registrazione di eventi, errori e altre informazioni rilevanti all'interno di un'applicazione. Questi dati vengono 
solitamente salvati in un file di log, un database o inviati a un servizio esterno per analisi successive. Il logging è fondamentale 
per la risoluzione dei problemi, il monitoraggio delle prestazioni e la sicurezza delle applicazioni.

\newpage