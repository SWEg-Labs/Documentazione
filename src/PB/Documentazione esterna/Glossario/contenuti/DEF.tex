% Intestazione
\fancyhead[L]{D} % Testo a sinistra

\section{}

\hypertarget{sec:database_relazionale}{}
\subsection*{Database relazionale}
Un database relazionale è un tipo di database che organizza i dati in tabelle strutturate, con righe e colonne. Ogni tabella rappresenta un'entità e le 
relazioni tra le tabelle sono definite tramite chiavi primarie e chiavi esterne. Questo modello facilita la gestione e l'interrogazione dei dati utilizzando 
il linguaggio SQL.

\hypertarget{sec:database_vettoriale}{}
\subsection*{Database vettoriale}
Un database vettoriale è un tipo di sistema di archiviazione progettato per gestire e interrogare dati rappresentati come vettori, spesso in spazi ad 
alta dimensionalità. Questi vettori sono comunemente utilizzati per rappresentare oggetti, concetti o entità in un formato numerico, solitamente prodotto 
da algoritmi di machine learning, reti neurali o tecniche di embedding.

\hypertarget{sec:demo}{}
\subsection*{Demo}
Una demo (abbreviazione di "dimostrazione") è una presentazione interattiva o registrata di un software o di una funzionalità specifica all'interno di un 
progetto. Viene usata per mostrare lo stato attuale dello sviluppo, illustrando come funzionano le caratteristiche implementate e come si comporta il 
sistema rispetto ai requisiti stabiliti. Le demo sono solitamente presentate a stakeholder, clienti, o team interni per ottenere feedback, verificare 
funzionalità e orientare lo sviluppo futuro. Le demo sono particolarmente utili nei processi iterativi e agili, dove è importante monitorare e comunicare 
costantemente i progressi del progetto.

\hypertarget{sec:diagramma_Gantt}{}
\subsection*{Diagramma di Gantt}
Strumento di visualizzazione temporale utilizzato nella gestione dei progetti per rappresentare le attività pianificate nel tempo. È composto da una barra 
orizzontale che rappresenta l’arco temporale totale del progetto e da barre orizzontali più piccole che rappresentano le singole attività del progetto. 
Ogni barra è posizionata lungo l’asse temporale in base alle date di inizio e fine previste per l’attività.

\hypertarget{sec:diagramma_UML}{}
\subsection*{Diagramma UML}
Acronimo di Unified Modeling Language, un diagramma UML diagramma utilizzato per modellare, descrivere e visualizzare sistemi software e processi di sviluppo 
software. È uno standard industriale nel campo dell’ingegneria del software e fornisce una serie di diagrammi, ognuno dei quali si concentra su un aspetto 
specifico del sistema o del processo. Sono diagrammi UML ad esempio i diagrammi dei casi d’uso, i diagrammi delle classi e i diagrammi delle funzionalità.

\subsection*{Diagramma UML dei casi d'uso}
Diagramma UML che rappresenta le interazioni tra utenti (attori) e il sistema, descrivendo come gli utenti utilizzano il sistema per raggiungere obiettivi 
specifici. Ogni caso d'uso rappresenta una funzione o un'attività significativa, utile per descrivere i requisiti funzionali. È spesso il primo passo nella 
progettazione di un sistema software e aiuta a identificare le funzioni principali e il modo in cui il sistema interagisce con gli utenti.

\subsection*{Diagramma UML delle classi}
Diagramma UML che descrive la struttura statica di un sistema, mostrando le classi, i loro attributi, i metodi e le relazioni tra di esse (come ereditarietà, 
associazioni e aggregazioni). Questo diagramma è fondamentale per la programmazione orientata agli oggetti poiché fornisce una rappresentazione visiva della 
struttura del codice, aiutando a comprendere le interconnessioni tra le varie entità e a definire i componenti principali.

\hypertarget{sec:diario_di_bordo}{}
\subsection*{Diario di Bordo}
Nel contesto del progetto didattico, il “diario di bordo” è un’attività settimanale in cui ogni gruppo di progetto documenta e presenta pubblicamente il progresso del proprio lavoro. 
Questa attività include la descrizione delle attività svolte, delle difficoltà incontrate, dei dubbi e delle incertezze, e dei piani per il periodo successivo. 
L’obiettivo è promuovere una discussione aperta e riflessiva per migliorare la consapevolezza e la gestione del progetto collaborativo.

\subsection*{Discord}
Piattaforma VoIP (Voice over IP: tecnologia che rende possibile effettuare una conversazione sfruttando una connessione internet), messaggistica istantanea 
e distribuzione digitale progettata per la comunicazione.

\hypertarget{sec:docker}{}
\subsection*{Docker}
Docker è un software progettato per eseguire processi informatici in ambienti isolabili, minimali e facilmente distribuibili chiamati 
container, con l’obiettivo di semplificare i processi di deployment di applicazioni software.

\hypertarget{sec:docstring}{}
\subsection*{Docstring}
Una docstring è una stringa speciale inserita all'interno del codice sorgente di un programma per documentare una funzione, un modulo 
o una classe. In pratica, è una descrizione dettagliata e concisa dello scopo, dei parametri, del valore di ritorno e di eventuali 
eccezioni sollevate da un elemento di codice.

\subsection*{Draw.io}
Draw.io (ora chiamato diagrams.net) è uno strumento gratuito per la creazione di diagrammi, disponibile sia come applicazione web che come app desktop per 
vari sistemi operativi. Viene utilizzato ampiamente per progettare e documentare diagrammi di flusso, architetture software, diagrammi UML, organigrammi, 
mappe mentali, wireframe, e altri tipi di rappresentazioni visive.

\newpage


% Intestazione
\fancyhead[L]{E} % Testo a sinistra

\section{}

\hypertarget{sec:efficacia}{}
\subsection*{Efficacia}
Capacità di raggiungere gli obiettivi desiderati o di produrre gli effetti previsti.

\hypertarget{sec:efficienza}{}
\subsection*{Efficienza}
Capacità di svolgere un compito, un'attività o un processo nel modo più economico e con il minimo spreco di risorse.

\hypertarget{sec:embedding}{}
\subsection*{Embedding (modello di)}
Un Modello di embedding è un modello di intelligenza artificiale progettato per trasformare dati, come parole, frasi o immagini, in rappresentazioni 
vettoriali numeriche in uno spazio multidimensionale. Questi vettori catturano caratteristiche semantiche o strutturali dei dati, rendendoli utili per 
confronti, analisi o classificazioni basate su similarità. I modelli di embedding sono comunemente utilizzati in applicazioni come la ricerca semantica, 
il clustering e il machine learning.

\hypertarget{sec:eslint}{}
\subsection*{ESLint}
ESLint è uno strumento ampiamente utilizzato per l'analisi statica del codice JavaScript (e relativi framework come React, Vue, e Node.js), 
con l'obiettivo di individuare errori, applicare standard di codifica e migliorare la qualità complessiva del codice.
\newpage


% Intestazione
\fancyhead[L]{F} % Testo a sinistra

\section{}

\hypertarget{sec:fastapi}{}
\subsection*{FastAPI}
FastAPI è un framework web Python moderno e performante, progettato per creare API RESTful in modo rapido e intuitivo. Offre una sintassi 
concisa e tipo-hintata, che riduce significativamente il numero di bug e aumenta la velocità di sviluppo. FastAPI è ideale per 
costruire microservizi e backend per applicazioni web, grazie alla sua eccellente scalabilità e integrazione con strumenti di 
test e documentazione.

\hypertarget{sec:feature}{}
\subsection*{Feature}
Una feature è una specifica funzionalità o capacità di un sistema software che ne arricchisce il comportamento e le possibilità d'uso. Le feature sono 
aggiunte o miglioramenti che rispondono a esigenze degli utenti, migliorano l’esperienza d’uso, o risolvono specifici problemi. L'introduzione o il 
miglioramento di una feature segue un ciclo di sviluppo completo, dalla raccolta dei requisiti fino alla verifica e al testing.

\hypertarget{sec:feedback}{}
\subsection*{Feedback}
Il feedback è un'informazione, un'opinione o una valutazione che viene fornita in risposta a un'azione, a una prestazione o a un 
comportamento. Ha lo scopo di evidenziare punti di forza e aree di miglioramento, permettendo a chi lo riceve di comprendere meglio come il 
proprio operato viene percepito e come può migliorare o adattarsi. Il processo di feedback è un processo per cui il risultato dell’azione 
di un sistema si riflette sul sistema stesso per correggerne o modificarne il comportamento.

\hypertarget{sec:file-.env}{}
\subsection*{File .env}
Un file di testo che contiene variabili di ambiente utilizzate per configurare un'applicazione. Consente di separare le impostazioni 
specifiche di un ambiente (es. sviluppo, produzione) dal codice sorgente, facilitando la gestione e la sicurezza delle informazioni 
sensibili come chiavi API, password e URL.

\hypertarget{sec:flask}{}
\subsection*{Flask}
Flask è un leggero framework web Python per lo sviluppo di applicazioni web. Conosciuto per la sua flessibilità e semplicità d'uso, 
Flask offre una base solida su cui costruire applicazioni personalizzate. Permette agli sviluppatori di scegliere le librerie e gli 
strumenti che meglio si adattano al loro progetto, senza imporre una struttura rigida.

\subsection*{Fogli Google}
Fogli Google è un'applicazione web di Google, parte della suite di produttività Google Workspace, che consente di creare, modificare e condividere fogli 
di calcolo online. È uno strumento particolarmente apprezzato per il suo accesso immediato da browser, le funzionalità collaborative in tempo reale, e 
l'integrazione con altri servizi Google. Così come ogni foglio di calcolo, include la possibilità di creare grafici e diagrammi basati su dati tabellari.

\hypertarget{sec:framework}{}
\subsection*{Framework}
Infrastruttura software che fornisce un’architettura generale per lo sviluppo di applicazioni. Un framework è progettato per facilitare il processo di 
sviluppo fornendo strutture, librerie, linee guida e pattern comuni che possono essere seguiti dagli sviluppatori. L’obiettivo è semplificare il lavoro 
degli sviluppatori fornendo un ambiente predefinito in cui possono costruire le loro applicazioni.

\hypertarget{sec:front-end}{}
\subsection*{Front-end}
Termine riferito alla parte grafica e funzionale dell’interfaccia utente con cui l’utente interagisce direttamente. 
Esso comprende elementi grafici, un’interfaccia utente, la disposizione dei pulsanti, gli elementi di input e la 
validazione dei dati inseriti dall’utente. Gli strumenti più comuni per lo sviluppo del front-end includono HTML, 
CSS e JavaScript.

\hypertarget{sec:funzionalità}{}
\subsection*{Funzionalità}
Nella programmazione, indica una capacità specifica offerta da un sistema o un'applicazione di soddisfare un determinato requisito o necessità.

\newpage