% Intestazione
\fancyhead[L]{P} % Testo a sinistra

\section{}

\hypertarget{sec:Pascal Case}{}
\subsection*{Pascal Case}
È una convenzione di scrittura utilizzata nella programmazione per nominare variabili, funzioni, classi e altri identificatori. 
In Pascal Case, ogni parola della stringa inizia con una lettera maiuscola e non ci sono spazi o separatori tra le parole.

\hypertarget{sec:pattern_architetturale}{}
\subsection*{Pattern architetturale}
Soluzione generale e riutilizzabile a un problema che si verifica comunemente nell’ambito dello sviluppo del software. I pattern 
architetturali forniscono un approccio testato e comprovato per la progettazione e l’implementazione del software.


\subsection*{Piano di Progetto}
Documento formale che delinea in dettaglio la pianificazione, l’esecuzione, il monitoraggio e il controllo di tutte le attività coinvolte nella 
realizzazione di un progetto. Questo documento fornisce una roadmap chiara e organizzata, comprensiva di obiettivi, risorse, scadenze e strategie di 
gestione dei rischi. Essenziale per la gestione efficace di un progetto, il piano di progetto serve come guida per il team di lavoro e gli stakeholder, 
fornendo una struttura che facilita il coordinamento delle attività e l’assegnazione delle risorse.

\subsection*{Piano di Qualifica}
Documento che stabilisce gli standard di qualità, i processi e le attività di testing che saranno implementati durante lo sviluppo di un progetto. 
Contiene una descrizione dettagliata delle strategie di testing, delle metriche di valutazione e dei criteri di accettazione del prodotto finale. 
L’obiettivo principale del Piano di Qualifica è garantire che il prodotto soddisfi gli standard di qualità prefissati e che il processo di sviluppo 
segua procedure coerenti ed efficaci.

\hypertarget{sec:piattaforma}{}
\subsection*{Piattaforma}
In ambito informatico, un ambiente software che supporta l’esecuzione di applicazioni o servizi.

\hypertarget{sec:pinecone}{}
\subsection*{Pinecone}
Pinecone è una piattaforma cloud-native progettata per l'indicizzazione e il retrieval di dati vettoriali. Consente di costruire e gestire motori di 
ricerca vettoriale altamente scalabili, spesso utilizzati in applicazioni di intelligenza artificiale come motori di raccomandazione, ricerca semantica 
e assistenti virtuali. Pinecone offre un'infrastruttura ottimizzata per il retrieval basato su similarità, supportando scenari di alto volume e 
garantendo basse latenze, con un'interfaccia API intuitiva per integrarsi facilmente con applicazioni esistenti.

\hypertarget{sec:postgres}{}
\subsection*{Postgres (PostgreSQL)}
PostgreSQL (o Postgres) è un sistema di gestione di database relazionali open-source, noto per la sua stabilità, flessibilità e conformità agli standard SQL. 
Supporta funzionalità avanzate come transazioni ACID, indicizzazione full-text, query JSONB per dati semi-strutturati e estensibilità tramite funzioni 
definite dall'utente. Postgres è ampiamente utilizzato in applicazioni che richiedono scalabilità, sicurezza e gestione efficiente di dati complessi, ed è 
compatibile con molte estensioni per database vettoriali e applicazioni di machine learning.

\hypertarget{sec:preventivo}{Preventivo}
\subsection*{Preventivo}
Stima dei costi e delle risorse necessarie per completare un determinato lavoro o progetto.

\hypertarget{sec:processo}{Processo}
\subsection*{Processo}
Insieme di attività correlate e coese che trasformano ingressi (bisogni) in uscite (prodotti) secondo regole date,
consumando risorse nel farlo.

\hypertarget{sec:prodotto_software}{}
\subsection*{Prodotto software}
Un prodotto software è un’applicazione, un sistema o un programma
informatico risultante dal processo di sviluppo del software. In altre parole, è il risultato
tangibile e funzionante di attività di progettazione, sviluppo, test e manutenzione svolte dal
team di sviluppo.

\hypertarget{sec:PB}{}
\subsection*{Product Baseline (PB)}
E la seconda revisione di avanzamento del progetto didattico. Comprende un prodotto software con design definitivo, 
chiamato \emph{Minimum Viable Product (MVP)}.

\hypertarget{sec:progettazione}{progettazione}
\subsection*{Progettazione}
Attività nel processo di sviluppo che mira a definire l'architettura di un prodotto in grado di soddisfare le esigenze di tutti gli stakeholder. 
Questa fase fornisce documentazione dettagliata sulla struttura del prodotto, specifiche tecniche e scelte tecnologiche adottate per la sua realizzazione.

\hypertarget{sec:PoC}{}
\subsection*{Proof of Concept (PoC)}
Versione preliminare di un’applicazione o di una soluzione software che viene sviluppata per dimostrare la fattibilità tecnica di un’idea o di un concetto. 
Viene utilizzata per testare rapidamente l’efficacia di un approccio, identificare eventuali limitazioni delle tecnologie scelte e valutare se l’idea può 
essere realizzata in modo pratico.

\hypertarget{sec:Prompt}{}
\subsection*{Prompt}
Si riferisce al testo o alla richiesta che appare su un'interfaccia, chiedendo all'utente di fornire un input. 

\hypertarget{sec:proponente}{}
\subsection*{Proponente}
Nel contesto dell’ingegneria del software, colui che presenta un’idea, un progetto o una proposta e ne sostiene la realizzazione. Il gruppo \emph{SWEg Labs} 
ha come proponente l’azienda \emph{AzzurroDigitale}.

\hypertarget{sec:pull_request}{Pull request}
\subsection*{Pull request}
Nel sistema di versionamento integrato da GitHub, operazione che permette di proporre modifiche al codice sorgente di un progetto. 
Una pull request consente agli sviluppatori di richiedere l’integrazione delle loro modifiche nel branch principale del repository, 
facilitando la revisione e la collaborazione.

\hypertarget{sec:pylint}{}
\subsection*{PyLinit}
Pylint è uno dei più diffusi linters per il linguaggio Python, utilizzato per rilevare errori nel codice, garantire aderenza 
alle convenzioni di stile (come PEP 8) e migliorare la qualità e la manutenibilità del codice.

\hypertarget{sec:pytest}{}
\subsection*{Pytest}
Pytest è un framework di testing per il linguaggio di programmazione Python. È progettato per essere semplice da usare e
permette di scrivere test in modo chiaro e conciso. Pytest supporta test di unità, test funzionali e test di integrazione,
offrendo funzionalità avanzate come l'assertion introspection, la parametrizzazione dei test e l'integrazione con altri
strumenti di testing e di sviluppo.

\hypertarget{sec:python}{}
\subsection*{Python}
Un linguaggio di programmazione open-source, interpretato, dinamico e multi-paradigma, ampiamente utilizzato per lo sviluppo di software, applicazioni web, 
analisi dati, intelligenza artificiale e automazione. Creato da Guido van Rossum nel 1991, si distingue per la sua sintassi semplice e leggibile, che 
favorisce la produttività e la manutenzione del codice. Supporta programmazione procedurale, orientata agli oggetti e funzionale. Dotato di una vasta 
libreria standard e di un ecosistema ricco di framework e pacchetti, è adatto sia a principianti sia a sviluppatori esperti per applicazioni front-end e 
back-end.

\hypertarget{sec:python_crontab}{}
\subsection*{Python Crontab}
Python Crontab è un modulo Python che semplifica la gestione dei crontab. Un crontab è un file di configurazione utilizzato nei 
sistemi operativi Unix-like (come Linux e macOS) per pianificare l'esecuzione di comandi a intervalli di tempo regolari. In pratica, 
un crontab è una tabella (tabella dei comandi) che indica al sistema operativo quali comandi eseguire e a quale ora.

\newpage


% Intestazione
\fancyhead[L]{Q} % Testo a sinistra

\section{}

\hypertarget{sec:qdrant}{}
\subsection*{Qdrant}
Qdrant è un motore open-source progettato per la gestione e la ricerca di dati vettoriali. Ottimizzato per il nearest neighbor search, Qdrant è utilizzato 
per applicazioni basate su similarità, come motori di ricerca semantica, raccomandazioni e clustering. Fornisce un'API RESTful, supporta query scalabili in 
tempo reale e integra funzionalità per filtrare e classificare dati multidimensionali. La sua architettura è progettata per essere facilmente scalabile e 
per garantire alte prestazioni con dataset di grandi dimensioni.

\hypertarget{sec:Qualità}{}
\subsection*{Qualità}
Insieme delle caratteristiche di un’entità che ne determinano la capacità di soddisfare esigenze sia espresse che implicite. Si parla di qualità del prodotto software in termini di:
\begin{itemize}
    \item \textbf{Qualità Intrinseca}: conformità ai requisiti, idoneità all’uso;
    \item \textbf{Qualità Relativa}: soddisfazione del cliente;
    \item \textbf{Qualità Quantitativa}: misurazione oggettiva, per confronto.
\end{itemize}

\hypertarget{sec:question_answering}{}
\subsection*{Question answering (Modello di)}
Un sistema di intelligenza artificiale progettato per rispondere a domande poste in linguaggio naturale. Nell'ambito dell'ingegneria 
del software e dell'elaborazione del linguaggio naturale (NLP), un modello di question answering utilizza tecniche avanzate di 
machine learning e deep learning per comprendere il contesto della domanda e fornire risposte accurate e pertinenti. Questi modelli 
possono essere addestrati su grandi quantità di dati testuali e sono in grado di estrarre informazioni da documenti, articoli, 
database e altre fonti di conoscenza per rispondere alle domande degli utenti.


\newpage


% Intestazione
\fancyhead[L]{R} % Testo a sinistra

\section{}

\hypertarget{sec:refactoring}{}
\subsection*{Refactoring}
Il processo di riorganizzazione del codice sorgente di un programma senza modificarne il comportamento esterno. L'obiettivo è 
migliorare la leggibilità, la manutenibilità e la struttura del codice, senza alterarne la funzionalità.

\subsection*{Repository}
In termini informatici, un luogo o un archivio dove vengono conservati e gestiti dati, documenti o, nel contesto del software, il codice sorgente di un 
progetto. Nell'ambito dei sistemi di controllo delle versioni come Git, un repository è una struttura dati che archivia anche la cronologia completa delle 
modifiche apportate al codice sorgente di un progetto.

\subsection*{Requirement and Technology Retrospective (RTB)}
La prima revisione di avanzamento del progetto didattico. Fissa i requisiti da soddisfare in accordo con il proponente, motiva le tecnologie, i framework 
e le librerie adottate dimostrandone adeguatezza e compatibilità tramite il Proof of Concept (PoC).

\subsection*{Retrospettiva}
Vedi \bulhyperlink{sec:sprint_reptrospective}{Sprint Retrospective}.

\newpage