% Intestazione
\fancyhead[L]{2 \hspace{0.2cm} Descrizione generale} % Testo a sinistra

\section{Descrizione generale}
\label{sec:descrizione_generale}

\subsection{Obiettivi del prodotto}
L'obiettivo del prodotto è sviluppare "BuddyBot", un assistente virtuale che utilizza l'\emph{intelligenza artificiale}\textsubscript{\textit{\textbf{G}}} 
per rispondere in modo efficiente e accurato a domande poste in linguaggio naturale. BuddyBot dovrà essere in grado 
di reperire informazioni specifiche da fonti designate, rendendole disponibili agli utenti tramite una chat intuitiva. 
La piattaforma web associata consentirà agli utenti di interagire con il sistema in modo semplice e immediato, migliorando
l'accesso alle informazioni e l'efficacia del supporto fornito.

\subsection{Funzioni del prodotto}
L'applicazione sviluppata permetterà agli utenti di accedere rapidamente alle informazioni aziendali raccolte da varie 
fonti e di ottenere risposte in linguaggio naturale attraverso una piattaforma web con interfaccia chat. Questo sistema 
si integrerà con diverse \emph{API}\textsubscript{\textit{\textbf{G}}}  di terze parti per la raccolta e l'elaborazione delle informazioni, e successivamente queste ultime verranno elaborate
da altre API di terze parti dedicate all'\emph{intelligenza artificiale}\textsubscript{\textit{\textbf{G}}}  (come \emph{ChatGPT}\textsubscript{\textit{\textbf{G}}} o altri \emph{LLM}\textsubscript{\textit{\textbf{G}}}), centralizzandole e facilitandone la consultazione. 
BuddyBot supporterà l'interazione tramite linguaggio naturale per migliorare la produttività del team, l'\emph{onboarding}\textsubscript{\textit{\textbf{G}}} di nuovi 
membri e l'accesso alle risorse aziendali.\\
Le funzionalità implementate nell'applicazione includono:
\begin{itemize}
    \item \textbf{Accesso rapido alle informazioni}: l'applicazione consente agli utenti di accedere velocemente alle informazioni desiderate tramite una chat in linguaggio naturale, riducendo i tempi di ricerca e migliorando la reattività del team;
    \item \textbf{Centralizzazione dei dati}: raccolta e aggregazione di dati provenienti da \emph{GitHub}\textsubscript{\textit{\textbf{G}}}, \emph{Jira}\textsubscript{\textit{\textbf{G}}}, \emph{Confluence}\textsubscript{\textit{\textbf{G}}}, \emph{Telegram}\textsubscript{\textit{\textbf{G}}} (facoltativo) e \emph{Slack}\textsubscript{\textit{\textbf{G}}} (facoltativo), offrendo una vista unificata delle risorse aziendali;
    \item \textbf{Integrazione con API esterne}: utilizzo di API di terze parti per ottenere informazioni dalle fonti indicate e successiva elaborazione delle informazioni tramite API di intelligenza artificiale come ChatGPT o altri LLM;
    \item \textbf{Supporto per il linguaggio naturale}: interpretazione delle domande in linguaggio naturale poste dagli utenti e fornitura di risposte pertinenti grazie alla tecnologia di intelligenza artificiale integrata;
    \item \textbf{Facilitazione dell'onboarding}: supporto ai nuovi membri del team tramite risposte mirate che permettono loro di conoscere le risorse aziendali e di integrarsi più rapidamente;
    \item \textbf{Interfaccia web con chat integrata}: accesso a una piattaforma web user-friendly con interfaccia chat per l'interazione diretta con BuddyBot.
\end{itemize}

\subsection{Caratteristiche degli utenti}
I membri del team di sviluppo utilizzeranno BuddyBot per accedere rapidamente a documentazione e aggiornamenti su piattaforme come 
\emph{GitHub}, \emph{Jira} e \emph{Confluence}, ottimizzando il flusso di lavoro senza dover navigare tra diversi strumenti.\\
I nuovi membri del team troveranno in BuddyBot un supporto per l'\emph{onboarding}, utilizzandolo per 
orientarsi nelle risorse aziendali e ottenere risposte a domande comuni, facilitando così la loro integrazione.

\subsection{Piattaforma di esecuzione}
Il prodotto si presenterà sotto forma di \emph{applicazione web}\textsubscript{\textit{\textbf{G}}} e sarà consultabile dalla maggior
parte dei \emph{browser}\textsubscript{\textit{\textbf{G}}}: in particolare la sua esecuzione sarà garantita nella versione 131.0.6778.204/.205 di \emph{Google Chrome}\textsubscript{\textit{\textbf{G}}}, ovvero l'ultima disponibile al 19 dicembre 2024.
