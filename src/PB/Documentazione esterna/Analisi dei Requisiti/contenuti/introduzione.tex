% Intestazione
\fancyhead[L]{1 \hspace{0.2cm} Introduzione} % Testo a sinistra

\pagenumbering{arabic} % Numerazione araba per il contenuto 

\section{Introduzione}
\label{sec:introduzione}

Questo documento è stato redatto con l'intento di offrire una trattazione esaustiva e dettagliata 
dei requisiti e dei casi d'uso individuati dal gruppo \emph{SWEg Labs} nel corso dello sviluppo
del progetto "BuddyBot". La raccolta di questi dati è il frutto di un'analisi approfondita
del documento di presentazione del \emph{capitolato}\textsubscript{\textit{\textbf{G}}}, di intense discussioni interne al gruppo di lavoro, 
nonchè di colloqui attivi con il \emph{proponente}\textsubscript{\textit{\textbf{G}}}, \emph{AzzurroDigitale}\textsubscript{\textit{\textbf{G}}}.

L'obiettivo è garantire una comprensione completa ed accurata dei requisiti di progetto,
fornendo una base solida per la pianificazione e l'implementazione delle successive fasi di lavoro.

Nel documento adottiamo la sintassi \emph{UML}\textsubscript{\textit{\textbf{G}}} al fine di formalizzare la rappresentazione e
renderla comprensibile a tutti i potenziali utenti. In particolare, i casi d'uso seguono una
struttura logica e vengono descritti in dettaglio attraverso i seguenti punti:
\begin{itemize}
    \item \textbf{Nominativo}: includiamo il titolo del \emph{caso d'uso}\textsubscript{\textit{\textbf{G}}};
    \item \textbf{Attori Principali}: identifichiamo chi sono gli \emph{attori}\textsubscript{\textit{\textbf{G}}} che eseguono le azioni all'interno 
                del caso d'uso;
    \item \textbf{Precondizioni}: specifichiamo lo stato del programma prima dell'esecuzione del caso d'uso;
    \item \textbf{Trigger}: indichiamo il fattore scatenante che dà inizio al caso d'uso;
    \item \textbf{Postcondizioni}: definiamo lo stato del programma dopo il completamento dello scenario del caso d'uso;
    \item \textbf{\emph{Scenario Principale}\textsubscript{\textit{\textbf{G}}}}: descriviamo in modo dettagliato le azioni svolte durante
                l'esecuzione del caso d'uso, delineando il percorso seguito tra le condizioni iniziali e i risultati ottenuti;
    \item \textbf{Scenari alternativi:} descriviamo gli scenari che diramano dallo scenario principale o le situazioni nelle quali lo svolgimento delle 
                azioni dello scenario principale sia impossibilitato dalla comparsa di condizioni di errore;
    \item \textbf{\emph{Sottocasi d'uso}\textsubscript{\textit{\textbf{G}}}}: in alcune circostanze può essere necessaria la definizione di uno
                o più sottocasi d'uso, che andranno ad utilizzare la stessa struttura dei casi d'uso, e potranno essere 
                identificati mediante un numero progressivo nella forma:
                \begin{center}
                    X.Y
                \end{center}
                dove X è il caso d'uso da cui derivano e Y un numero progressivo ad identificare il sottocaso;
    \item \textbf{Casi che ereditano}: in alcuni casi, un caso d'uso può ereditare funzionalità da un altro caso d'uso, cioè
                i figli condividono tutte le funzionalità del padre e in più ne possiedono di proprie. Qui 
                indichiamo i casi d'uso che ereditano dal caso d'uso corrente;
    \item \textbf{Eredita da}: qui indichiamo il caso d'uso da cui il caso d'uso corrente eredita funzionalità;
\end{itemize}

\subsection{Scopo del prodotto}
Nel corso dell'ultimo anno si è verificato un repentino e significativo mutamento nel panorama
dello sviluppo e nell'implementazione dell'\emph{Intelligenza Artificiale}\textsubscript{\textit{\textbf{G}}}.
Questa trasformazione ha attraversato varie sfaccettature della tecnologia, segnando una transizione dall'uso
dell'Intelligenza Artificiale principalmente per l'elaborazione e la raccomandazione di contenuti, a un'era in
cui tali sistemi sono capaci di generare contenuti originali. \\
Il \emph{capitolato}\textsubscript{\textit{\textbf{G}}} C9, "BuddyBot", ha come obiettivo la realizzazione di un assistente virtuale (chatbot) 
capace di raccogliere rapidamente informazioni dalle fonti indicate e di fornirle in risposta a domande poste in 
linguaggio naturale tramite chat.\\
Tale assistente virtuale sarà fruibile attraverso una piccola piattaforma web, dove l'utente potrà interagire con l'\emph{IA}\textsubscript{\textit{\textbf{G}}} 
per ottenere le risposte desiderate.

\subsection{Glossario}
Al fine di evitare possibili ambiguità relative al linguaggio utilizzato nei documenti, viene fornito un \emph{Glossario}\textsubscript{\textit{\textbf{G}}}
(attualmente alla sua versione \emph{1.0.0}), nel quale sono contenute le definizioni di termini complessi o aventi uno 
specifico significato. Tali termini, ove necessario, sono segnati in corsivo e marcati con il simbolo G a pedice
(esempio: \emph{Way of Working}\textsubscript{\textit{\textbf{G}}}).

\subsection{Miglioramenti al documento}
La maturità e i miglioramenti sono aspetti fondamentali nella stesura di un documento.
Questo permette di apportare agevolmente modifiche in base alle esigenze concordate tra i
membri del gruppo e il \emph{proponente}\textsubscript{\textit{\textbf{G}}} nel corso del tempo. Di conseguenza, questa versione del
documento non può essere considerata definitiva o completa, poichè è soggetta a evoluzioni future.

\subsection{Riferimenti}
\subsubsection{Riferimenti normativi}
\begin{itemize}
    \item \bulhref{https://sweg-labs.github.io/Documentazione/output/PB/Documentazione\%20interna/norme_progetto_v2.0.0.pdf}{Norme di Progetto v.2.0.0};
    \item \bulhref{https://sweg-labs.github.io/Documentazione/output/PB/Documentazione\%20esterna/piano_qualifica_v2.0.0.pdf}{Piano di qualifica v.2.0.0};
    \item \bulhref{https://www.math.unipd.it/~tullio/IS-1/2024/Progetto/C9.pdf}{Capitolato d'appalto C9 - BuddyBot};
    \item \bulhref{https://www.math.unipd.it/~tullio/IS-1/2024/Dispense/PD1.pdf}{Slide PD1 del corso di Ingegneria del Software - Regolamento del Progetto Didattico};
\end{itemize}

\subsubsection{Riferimenti informativi}
\begin{itemize}
    \item \bulhref{https://sweg-labs.github.io/Documentazione/output/PB/Documentazione\%20esterna/glossario_v2.0.0.pdf}{Glossario v.2.0.0}; 
    \item \bulhref{https://sweg-labs.github.io/Documentazione/\#rtb_verbali_interni}{Verbali interni};
    \item \bulhref{https://sweg-labs.github.io/Documentazione/\#rtb_verbali_esterni}{Verbali esterni};
    \item \bulhref{https://www.math.unipd.it/~tullio/IS-1/2024/Dispense/T05.pdf}{Slide T05 del corso di Ingegneria del Software - Analisi dei Requisiti};
    \item \bulhref{https://www.math.unipd.it/~rcardin/swea/2022/Diagrammi Use Case.pdf}{Diagrammi dei casi d'uso};
\end{itemize}
