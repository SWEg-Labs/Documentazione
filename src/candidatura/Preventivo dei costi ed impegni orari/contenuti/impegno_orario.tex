\section{Impegno orario}

Dopo aver calcolato la media dell’orario di impegno tra i componenti del gruppo, abbiamo definito le ore effettive che possiamo dedicare al progetto. Successivamente, abbiamo elaborato il preventivo, stabilendo una suddivisione equa delle ore di lavoro tra i diversi ruoli necessari per l’esecuzione del progetto. La tabella seguente riporta le ore di lavoro assegnate a ciascun ruolo, le ore totali, i relativi costi e il costo totale per ogni mansione.

\vspace{0.5cm}

\begin{table}[h]
    \centering
    \resizebox{1.2\textwidth}{!}{ 
    \begin{tabular}{|c|c|c|c|c|}
        \hline
        \rowcolor[gray]{0.9}
        \textbf{Ruolo} & \textbf{Ore Individuali} & \textbf{Ore Totali} & \textbf{Costo (\euro/h)} & \textbf{Costo Totale (\euro)} \\
        \hline
        Responsabile & 11 & 66 & 30 & 1980 \\
        \hline
        Amministratore & 9 & 54 & 20 & 1.080 \\
        \hline
        Analista & 11 & 66 & 25 & 1.650 \\
        \hline
        Progettista & 22 & 132 & 25 & 3.300 \\
        \hline
        Programmatore & 25 & 150 & 15 & 2.250 \\
        \hline
        Verificatore & 14 & 84 & 15 & 1.260 \\
        \hline
        \rowcolor[gray]{0.9}
        & \textbf{Ore tot. per membro: 92} & \textbf{Ore tot.: 552} &  & \textbf{Costo tot.: 11.520} \\
        \hline
    \end{tabular}
    }
    \caption{Ripartizione oraria e dettaglio dei costi}
\end{table}

\vspace{0.5cm}

\subsection{Partizione oraria dei ruoli}
A seguito di un consulto via email con l'azienda, è emerso che lo sviluppo e la progettazione rappresentano le aree su cui focalizzarsi per il buon svolgimento del progetto. Per questo motivo, si è deciso di dedicare a queste attività la maggior parte delle risorse orarie disponibili, poiché saranno anche le più dispendiose. È dunque fondamentale, in una prima fase, riservare la giusta attenzione a queste aree, mentre le altre attività, pur mantenendo un ruolo importante, avranno una distribuzione di ore ottimizzata per supportare sviluppo e progettazione.
