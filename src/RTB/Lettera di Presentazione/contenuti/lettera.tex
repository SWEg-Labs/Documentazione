% Intestazione
\fancyhead[R]{\includegraphics[width=0.16\textwidth]{sweg_logo_sito_inverted.png}} % Immagine a destra

% Piè di pagina
\fancyfoot[C]{\thepage} % Numero di pagina al centro

\pagenumbering{arabic} % Numerazione araba per il contenuto


Egregio prof. Vardanega, \\
Egregio prof. Cardin, \\
\vspace{0.5cm}

con la presente, il gruppo \emph{SWEg Labs} è lieto di comunicarVi la propria intenzione di candidarsi alla revisione denominata \emph{Requirements and Technology Baseline}\textsubscript{\textit{\textbf{G}}} per il capitolato:\\
\begin{center}
    \bulhref{https://www.math.unipd.it/~tullio/IS-1/2024/Progetto/C9.pdf}{C9: BuddyBot}
\end{center}
proposto dall’azienda \emph{AzzurroDigitale}\textsubscript{\textit{\textbf{G}}}.\\
La documentazione prodotta durante questa fase del progetto è consultabile all’interno della cartella "src"
la cui cartella è denominata "RTB”, raggiungibile al seguente link: 
\begin{center}
    \bulhref{https://github.com/SWEg-Labs/Documentazione}{https://github.com/SWEg-Labs/Documentazione}
\end{center}
All’interno di questa sezione, oltre alla presente lettera di presentazione è possibile trovare:
\begin{itemize}
\item La cartella “Documentazione Interna” che contiene il documento \emph{Norme di Progetto}\textsubscript{\textit{\textbf{G}}} nella versione 1.0.0, insieme alla sottocartella “Verbali Interni” che raccoglie i verbali redatti durante le riunioni interne del gruppo.
\item La cartella “Documentazione Esterna” che comprende il \emph{Piano di Qualifica}\textsubscript{\textit{\textbf{G}}} nella versione 1.0.0, il \emph{Piano di Progetto}\textsubscript{\textit{\textbf{G}}} nella versione 1.0.0, l' \emph{Analisi dei Requisiti}\textsubscript{\textit{\textbf{G}}} nella versione 1.0.0, e il \emph{Glossario}\textsubscript{\textit{\textbf{G}}} nella versione 1.0.0. In aggiunta, vi è la sottocartella “Verbali Esterni” che contiene i verbali degli incontri con il proponente, da lui stesso approvati e firmati.
\end{itemize}
Per agevolare la consultazione dei documenti, il gruppo si è munito di un sito web:
\begin{center}
    \bulhref{https://sweg-labs.github.io/Documentazione}{https://sweg-labs.github.io/Documentazione}
\end{center}
Viene inoltre reso disponibile il codice sorgente del \emph{Proof of Concept}\textsubscript{\textit{\textbf{G}}}: 
\begin{center}
    \bulhref{https://github.com/SWEg-Labs/BuddyBot}{https://github.com/SWEg-Labs/BuddyBot}
\end{center}
Con la presente lettera desideriamo inoltre aggiornarVi sugli impegni presi, confermando il costo finale per la realizzazione del progetto di 11.520 \euro, e allo stesso tempo ribadendo l’impegno preso in fase di candidatura di completare il progetto entro il 22/04/2025.\\
\vspace{1.5cm}

Cordiali saluti,\\
Gruppo \emph{SWEg Labs}

