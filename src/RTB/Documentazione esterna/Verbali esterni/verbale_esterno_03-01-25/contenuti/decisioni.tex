% Intestazione
\fancyhead[L]{4 \hspace{0.2cm} Decisioni} % Testo a sinistra


\section{Decisioni}

Durante la riunione sono state prese le seguenti decisioni:

\vspace{0.5cm}

\begin{table}[htbp]
    \centering
    \rowcolors{2}{lightgray}{white}
    \begin{tabular}{|c|p{0.8\textwidth}|}
        \hline
        \rowcolor[gray]{0.75}
        \multicolumn{1}{|c|}{\textbf{Codice}} & \multicolumn{1}{|c|}{\textbf{Descrizione}}\\
        \hline
        VE 7.1 & È stato deciso di ricercare una soluzione riguardo il problema della similarità, basandosi sulla possibilità di dare maggiore priorità ai documenti di lunghezza maggiore, oppure di attribuire tag di provenienza ai documenti e chiedere all'utente dove effettuare la ricerca. \\
        VE 7.2 & È stato deciso che non è necessario permettere al chatbot di leggere i file pdf. \\
        VE 7.3 & Sono stati stabiliti gli elementi da ottenere tramite il getter per Github. \\
        VE 7.4 & Sono state decise le funzionalità e la struttura dello storico della chat. \\
        VE 7.5 & Sono stati stabiliti quali sarebbero i requisiti di sicurezza, i quali sono però opzionali se l'app dovesse rimanere locale. \\
        VE 7.6 & È stato deciso che il gruppo utilizzerà Docker per eseguire il software in container. \\
        VE 7.7 & È stato stabilito che il gruppo ricevera la API Key di OpenAI il giorno martedì 7 Gennaio. \\
        VE 7.8 & È stato deciso che Davide si occuperà della scrittura del verbale esterno. \\
        VE 7.9 & Avendo l'autorizzazione del professore, è stato deciso di mantenere gli incontri tra il gruppo e l'azienda \emph{proponente}\textsubscript{\textbf{\textit{G}}} con la frequenza di uno ogni 2 settimane, quantomeno fino al termine della sessione di esami. \\
        \hline
    \end{tabular}
\end{table}