% Intestazione
\fancyhead[L]{3 \hspace{0.2cm} Diario della riunione} % Testo a sinistra

\section{Diario della riunione}

\begin{itemize}
    \item È stato presentato il Proof of Concept, vedendo in dettaglio le funzionalità implementate dallo scorso incontro;
    \item È stata discussa, come possibile miglioria per le future \emph{demo}\textsubscript{\textbf{\textit{G}}}, la possibilità di porre
    al chatbot una domanda che necessiti di coinvolgere più piattaforme per ottenere la risposta, tra 
    \emph{GitHub}\textsubscript{\textbf{\textit{G}}}, \emph{Jira}\textsubscript{\textbf{\textit{G}}} e 
    \emph{Confluence}\textsubscript{\textbf{\textit{G}}};
    \item È stato aggiornato il proponente sullo stato della documentazione del progetto, cioè è stato spiegato che
    ci mancano da completare i documenti di \emph{Piano di Progetto}\textsubscript{\textbf{\textit{G}}} e
    \emph{Piano di Qualifica}\textsubscript{\textbf{\textit{G}}};
    \item Per quanto riguarda la documentazione da consegnare al proponente per la RTB, è stato chiarito che ciò che verrà consegnato
    sarà il repository del PoC ed il sito contenente la documentazione del progetto;
    \item Per quanto riguarda la documentazione da consegnare al proponente per la \emph{PB}\textsubscript{\textbf{\textit{G}}}, è stato chiarito che ciò che verrà consegnato
    sarà:
    \begin{itemize}
        \item Un documento di analisi funzionale e tecnica, coincidente con la \emph{Specifica Tecnica}\textsubscript{\textbf{\textit{G}}}
        richiesta dal corso di Ingegneria del Software;
        \item I \emph{Test}\textsubscript{\textbf{\textit{G}}} implementati;
        \item Il verbale di collaudo, cioè il verbale di fine progetto;
        \item Il documento \emph{Manuale Utente}\textsubscript{\textbf{\textit{G}}};
        \item Il repository del codice sorgente dell'\emph{MVP}\textsubscript{\textbf{\textit{G}}};
        \item Il link al sito web della documentazione.
    \end{itemize}
    \item È stata comunicata al proponente la nostra aspettativa temporale per la revisione RTB, cioè ci aspettiamo di svolgere
    il primo incontro con il professor Cardin entro la fine di gennaio ed il secondo incontro con il professor Vardanega entro
    la metà di febbraio;
    \item E' stato proposto al proponente di interrompere per qualche settimana le riunioni esterne, in quanto il gruppo
    dovrà intraprendere la revisione RTB e sostenere gli esami della sessione invernale. Il proponente ha accettato la proposta,
    ed è stata fissata provvisoriamente la prossima riunione esterna per giovedì 13 febbraio 2025. Il gruppo si impegnerà
    a comunicare al proponente eventuali cambiamenti di data.
\end{itemize}