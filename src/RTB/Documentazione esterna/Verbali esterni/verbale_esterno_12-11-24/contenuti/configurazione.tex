\usepackage{titling} % Required for inserting the subtitle
\usepackage{graphicx} % Required for inserting images
\usepackage{tabularx} % Per l'ambiente tabularx (tabelle)
\usepackage{calc} % Sempre per le tabelle
\usepackage[hidelinks]{hyperref} % Per i collegamenti ipertestuali, ad esempio sulla table of contents
\usepackage[italian]{babel} % Per la lingua italiana nelle scritte automatiche
\usepackage[table]{xcolor} % Per colorare il testo e le celle delle tabelle
\usepackage{colortbl} % Per colorare le celle delle tabelle
\usepackage{lipsum} % Per generare lorem ipsum
\usepackage[normalem]{ulem} % Per sottolineare il testo
\usepackage{array} % Per la visualizzazione fluttuante di array di domande e risposte
\usepackage{ragged2e} % Pacchetto necessario per \justifying che giustifica il testo di tabelle
\usepackage{tikz} % Per spostare elementi nel documento in modo facile e veloce
\usepackage[a4paper, top=2.5cm, bottom=2.5cm, left=2.5cm, right=2.5cm]{geometry} % Per i margini della pagina
\usepackage{fancyhdr} % Per l'intestazione e il piè di pagina
\usepackage{amsmath} % Per scrivere formule matematiche, in particolare per il pedice G
\usepackage{comment} % Per commentare blocchi di testo
\usepackage{titlesec} % Per la definizione di nuove suddivisioni e sottosezioni, come subsubsubsection
\usepackage{eurosym} % Per il simbolo dell'euro
\usepackage{pgf-pie} % Per creare grafici a torta

\newcommand{\bulhref}[2]{\href{#1}{\textbf{\uline{#2}}}} % Nuovo comando per sottolineare e ingrassettare i collegamenti ipertestuali a siti e risorse esterne
\newcommand{\bulref}[1]{\textbf{\uline{\ref{#1}}}} % Nuovo comando per sottolineare i riferimenti a labels (come immagini, tabelle e sezioni) ma senza ipertesto
\newcommand{\bulhyperlink}[2]{\hyperlink{#1}{\textbf{\uline{#2}}}} % Nuovo comando per sottolineare e ingrassettare i collegamenti ipertestuali a targets interni al documento
\setlength{\parindent}{0pt} % Rimuove il rientro automatico dei paragrafi
\usetikzlibrary{calc} % Libreria per il calcolo delle coordinate di TikZ
\pagestyle{fancy} % Stile della pagina, per l'intestazione e il piè di pagina
\renewcommand{\footrulewidth}{0.4pt} % Inserimento della linea orizzontale sopra il piè di pagina
\definecolor{lightgray}{gray}{0.95} % Definizione del colore grigio chiaro

% Per l'intestazione
\setlength{\headheight}{23.62976pt} % Altezza dell'intestazione
\addtolength{\topmargin}{-11.62976pt} % Spazio tra il margine superiore e l'intestazione




% Creazione del nuovo comando subsubsubsection


\titleclass{\subsubsubsection}{straight}[\subsection]

\newcounter{subsubsubsection}[subsubsection]
\renewcommand\thesubsubsubsection{\thesubsubsection.\arabic{subsubsubsection}}
\renewcommand\theparagraph{\thesubsubsubsection.\arabic{paragraph}} % optional; useful if paragraphs are to be numbered

\titleformat{\subsubsubsection}
  {\normalfont\normalsize\bfseries}{\thesubsubsubsection}{1em}{}
\titlespacing*{\subsubsubsection}
{0pt}{3.25ex plus 1ex minus .2ex}{1.5ex plus .2ex}

\makeatletter
\renewcommand\paragraph{\@startsection{paragraph}{5}{\z@}%
  {3.25ex \@plus1ex \@minus.2ex}%
  {-1em}%
  {\normalfont\normalsize\bfseries}}
\renewcommand\subparagraph{\@startsection{subparagraph}{6}{\parindent}%
  {3.25ex \@plus1ex \@minus .2ex}%
  {-1em}%
  {\normalfont\normalsize\bfseries}}
\def\toclevel@subsubsubsection{4}
\def\toclevel@paragraph{5}
%\def\toclevel@paragraph{6}
\def\toclevel@subparagraph{6}
\def\l@subsubsubsection{\@dottedtocline{4}{7em}{4em}}
\def\l@paragraph{\@dottedtocline{5}{10em}{5em}}
\def\l@subparagraph{\@dottedtocline{6}{14em}{6em}}
\makeatother

\setcounter{secnumdepth}{4}
\setcounter{tocdepth}{4}
