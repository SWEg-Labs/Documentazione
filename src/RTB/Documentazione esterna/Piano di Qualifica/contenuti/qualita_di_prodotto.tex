% Intestazione
\fancyhead[L]{1 \hspace{0.2cm} qualità del prodotto} % Testo a sinistra

\section{Qualità di prodotto}
\label{sec:Qualità di prodotto}

\subsection{Qualità dell’architettura}


\subsection{Qualità della documentazione}
\begin{table}[h!]
    \centering
    \renewcommand{\arraystretch}{1.5} % Per aumentare l'altezza delle righe
    \begin{tabularx}{\textwidth}{|X|X|X|}\hline
    \rowcolor[HTML]{FFD700}
    \textbf{Obiettivo} & \textbf{Descrizione} & \textbf{Metriche} \\ \hline
    Correttezza linguistica & I documenti non devono avere errori grammaticali. & MPD1 \\ \hline
    Leggibilità & Il contenuto dei documenti deve essere comprensibileall’utente. & MPD2 \\ \hline
    \end{tabularx}
    \caption{\textbf{RT1}: qualità della documentazione}
\end{table}

\subsubsection{Metriche utilizzate}
\begin{table}[h!]
    \centering
    \renewcommand{\arraystretch}{1.5} % Per aumentare l'altezza delle righe
    \begin{tabularx}{\textwidth}{|X|X|X|X|}\hline
    \rowcolor[HTML]{FFD700}
    \textbf{Codice} & \textbf{Nome metrica} & \textbf{Valore accettabile} & \textbf{Valore ottimale} \\ \hline
    MPD1 & Errori ortografici & 5\% & 0\% \\ \hline
    MPD2 & Indice di Gulpease & \( \geq 40 \) & 60 \\ \hline
    \end{tabularx}
    \caption{\textbf{RT1}: metriche}
\end{table}

\subsection{Qualità del software}
