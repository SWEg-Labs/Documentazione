% Intestazione
\fancyhead[L]{3 \hspace{0.2cm} Piano di qualità} % Testo a sinistra

\section{Piano di qualità}
\label{sec:Piano di qualità}

\subsection{Qualità di processo}
\label{sec:Qualità di processo}

\subsubsection{Varianza di Budget}
\textbf{Codice:} MPC-1 \\
\textbf{Processo:} Fornitura. \\
\textbf{Formula:}
\[
100 \cdot \frac{\text{Budget Consuntivato} - \text{Budget Preventivato}}{\text{Budget Preventivato}}
\]
\textbf{Descrizione:} Questa metrica valuta la percentuale di variazione del budget tra preventivo e consuntivo in uno sprint. Il valore è positivo quando viene preventivato un budget inferiore a quello effettivamente utilizzato, mentre è negativo quando viene preventivato un budget maggiore a quello effettivamente utilizzato.

\subsubsection{Varianza dell’impegno orario}
\textbf{Codice:} MPC-2 \\
\textbf{Processo:} Fornitura. \\
\textbf{Formula:}
\[
100 \cdot \frac{\text{Ore Consuntivate} - \text{Ore Preventivate}}{\text{Ore Preventivate}}
\]
\textbf{Descrizione:} Questa metrica valuta la percentuale di variazione dell’impegno orario complessivo tra preventivo e consuntivo in uno sprint. Il valore è positivo quando viene preventivato un impegno orario inferiore a quello effettivamente svolto, mentre è negativo quando viene preventivato un impegno orario maggiore a quello effettivamente svolto.

\subsubsection{Earned Value}
\textbf{Codice:} MPC-3 \\
\textbf{Processo:} Fornitura. \\
\textbf{Formula:}
\[
\text{Budget Preventivato} \cdot \% \text{Completamento di Attività Sprint}
\]
\textbf{Descrizione:} Questa metrica rappresenta il valore effettivo del lavoro realizzato alla fine di uno sprint. Se l’Earned Value è maggiore dell’Actual Cost, significa che è stato speso meno del previsto. Al contrario, se è minore, significa che è stato speso di più del previsto.

\subsubsection{Actual Cost}
\textbf{Codice:} MPC-4 \\
\textbf{Processo:} Fornitura. \\
\textbf{Formula:}
\[
\sum_{i=1}^{n\_Sprint} \text{Budget Consuntivato}_i
\]
\textbf{Descrizione:} Questa metrica rappresenta il costo totale effettivamente sostenuto in base al lavoro eseguito nello sprint.

\subsubsection{Planned Value}
\textbf{Codice:} MPC-5 \\
\textbf{Processo:} Fornitura. \\
\textbf{Formula:}
\[
\text{Actual Cost}_{\text{sprint-1}} + \text{Budget Preventivato}_{\text{sprint}}
\]
\textbf{Descrizione:} Rappresenta il totale dei costi pianificati allo sprint e viene calcolata prima che esso inizi.

\subsubsection{Cost Variance}
\textbf{Codice:} MPC-6 \\
\textbf{Processo:} Fornitura. \\
\textbf{Formula:}
\[
\text{Earned Value} - \text{Actual Cost}
\]
\textbf{Descrizione:} Rappresenta lo scostamento dai costi pianificati. Un valore positivo indica che il lavoro effettivamente prodotto è costato meno di quanto preventivato, mentre un valore negativo indica il contrario.

\subsubsection{Schedule Variance}
\textbf{Codice:} MPC-7 \\
\textbf{Processo:} Fornitura. \\
\textbf{Formula:}
\[
\text{Earned Value} - \text{Planned Value}
\]
\textbf{Descrizione:} Indica lo scostamento dai tempi pianificati.

\subsubsection{Cost Performance Index}
\textbf{Codice:} MPC-8 \\
\textbf{Processo:} Fornitura. \\
\textbf{Formula:}
\[
\frac{\text{Earned Value}}{\text{Actual Cost}}
\]
\textbf{Descrizione:} Rappresenta l’efficienza economica del progetto.

\subsubsection{Schedule Performance Index}
\textbf{Codice:} MPC-9 \\
\textbf{Processo:} Fornitura. \\
\textbf{Formula:}
\[
\frac{\text{Earned Value}}{\text{Planned Value}}
\]
\textbf{Descrizione:} Rappresenta l’efficienza temporale del progetto.

\subsubsection{Estimate to Complete}
\textbf{Codice:} MPC-10 \\
\textbf{Processo:} Fornitura. \\
\textbf{Formula:}
\[
\frac{\text{Budget at Completion} - \text{Earned Value}}{\text{Cost Performance Index}}
\]
\textbf{Descrizione:} Indica il costo totale ancora da sostenere per il completamento del progetto.

\subsubsection{Estimate at Completion}
\textbf{Codice:} MPC-11 \\
\textbf{Processo:} Fornitura. \\
\textbf{Formula:}
\[
\text{Actual Cost} + \text{Estimate to Complete}
\]
\textbf{Descrizione:} Indica il costo totale alla fine del progetto in base all’andamento attuale.

\subsubsection{Budget at Completion}
\textbf{Codice:} MPC-12 \\
\textbf{Processo:} Fornitura. \\
\textbf{Descrizione:} Indica il budget totale del progetto.

\subsubsection{Code Coverage}
\textbf{Codice:} MPC-13 \\
\textbf{Processo:} Sviluppo. \\
\textbf{Descrizione:} Percentuale di codice attraversato dai test rispetto al totale della codebase.

\subsubsection{Misure di mitigazione insufficienti}
\textbf{Codice:} MPC-14 \\
\textbf{Processo:} Risoluzione dei problemi. \\
\textbf{Descrizione:} Indica il numero totale di misure di mitigazione previste che si sono rivelate insufficienti.

\subsubsection{Rischi inattesi}
\textbf{Codice:} MPC-15 \\
\textbf{Processo:} Risoluzione dei problemi. \\
\textbf{Descrizione:} Indica il numero totale di rischi inattesi (non analizzati) che si sono verificati.

\begin{table}[h!]
    \centering
    \renewcommand{\arraystretch}{1.5}
    \setlength{\tabcolsep}{5pt}
    \begin{tabularx}{\textwidth}{|l|X|c|c|}
    \hline
    \rowcolor[HTML]{FFD700}
    \textbf{Codice} & \textbf{Nome} & \textbf{Accettabile} & \textbf{Preferibile} \\ \hline
    MPC-1 & Varianza di Budget & ±10\% & ±0\% \\ \hline
    MPC-2 & Varianza dell’impegno orario & ±5\% & ±0\% \\ \hline
    MPC-3 & Earned Value & \(\geq\) MPC-4 & - \\ \hline
    MPC-4 & Actual Cost & - & - \\ \hline
    MPC-5 & Planned Value & - & - \\ \hline
    MPC-6 & Cost Variance & ±150 & 0 \\ \hline
    MPC-7 & Schedule Variance & ±150 & 0 \\ \hline
    MPC-8 & Cost Performance Index & 1 ±0.1 & 1 \\ \hline
    MPC-9 & Schedule Performance Index & 1 ±0.1 & 1 \\ \hline
    MPC-10 & Estimate to Complete & - & - \\ \hline
    MPC-11 & Estimate at Completion & ±5\% di MPC-12 & MPC-12 \\ \hline
    MPC-12 & Budget at Completion & - & - \\ \hline
    MPC-13 & Code Coverage & 75\% & 100\% \\ \hline
    MPC-14 & Misure di mitigazione insufficienti & 3 & 0 \\ \hline
    MPC-15 & Rischi inattesi & 3 & 0 \\ \hline
    \end{tabularx}
    \caption{Metriche di qualità di processo}
    \label{tab:metriche-qualita-processo}
\end{table}



\subsection{Qualità di prodotto}
\label{sec:Qualità di prodotto}

\subsubsection{Indice di Gulpease}
\textbf{Codice:} MPD-1 \\
\textbf{Processo:} Documentazione. \\
\textbf{Formula:}
\[
89 + \frac{300 \cdot (\text{Numero Frasi}) - 10 \cdot (\text{Numero Lettere})}{\text{Numero Parole}}
\]
\textbf{Descrizione:} Misura il grado di leggibilità di un testo su una scala da 1 a 100. Un valore minimo accettabile è 40, mentre un valore preferibile è 60 o più.

\subsubsection{Requisiti obbligatori soddisfatti}
\textbf{Codice:} MPD-2 \\
\textbf{Processo:} Sviluppo. \\
\textbf{Formula:}
\[
100 \cdot \frac{\text{Numero Requisiti Obbligatori Soddisfatti}}{\text{Numero Requisiti Obbligatori}}
\]
\textbf{Descrizione:} Indica la percentuale di requisiti obbligatori soddisfatti. Deve raggiungere il 100\%.

\subsubsection{Requisiti desiderabili soddisfatti}
\textbf{Codice:} MPD-3 \\
\textbf{Processo:} Sviluppo. \\
\textbf{Formula:}
\[
100 \cdot \frac{\text{Numero Requisiti Desiderabili Soddisfatti}}{\text{Numero Requisiti Desiderabili}}
\]
\textbf{Descrizione:} Indica la percentuale di requisiti desiderabili soddisfatti. Il valore accettabile è 0\%, mentre il valore preferibile è 100\%.

\subsubsection{Requisiti opzionali soddisfatti}
\textbf{Codice:} MPD-4 \\
\textbf{Processo:} Sviluppo. \\
\textbf{Formula:}
\[
100 \cdot \frac{\text{Numero Requisiti Opzionali Soddisfatti}}{\text{Numero Requisiti Opzionali}}
\]
\textbf{Descrizione:} Indica la percentuale di requisiti opzionali soddisfatti. Il valore accettabile è 0\%, mentre il valore preferibile è 100\%.


\subsection{Qualità della documentazione}
\begin{table}[h!]
    \centering
    \renewcommand{\arraystretch}{1.5} % Per aumentare l'altezza delle righe
    \begin{tabularx}{\textwidth}{|X|X|X|}\hline
    \rowcolor[HTML]{FFD700}
    \textbf{Obiettivo} & \textbf{Descrizione} & \textbf{Metriche} \\ \hline
    Correttezza linguistica & I documenti non devono avere errori grammaticali. & MPD1 \\ \hline
    Leggibilità & Il contenuto dei documenti deve essere comprensibileall’utente. & MPD2 \\ \hline
    \end{tabularx}
    \caption{\textbf{RT1}: qualità della documentazione}
\end{table}

\subsubsection{Metriche utilizzate}
\begin{table}[h!]
    \centering
    \renewcommand{\arraystretch}{1.5} % Per aumentare l'altezza delle righe
    \begin{tabularx}{\textwidth}{|X|X|X|X|}\hline
    \rowcolor[HTML]{FFD700}
    \textbf{Codice} & \textbf{Nome metrica} & \textbf{Valore accettabile} & \textbf{Valore ottimale} \\ \hline
    MPD1 & Errori ortografici & 5\% & 0\% \\ \hline
    MPD2 & Indice di Gulpease & \( \geq 40 \) & 60 \\ \hline
    \end{tabularx}
    \caption{\textbf{RT1}: metriche}
\end{table}

\subsection{Qualità del software}
