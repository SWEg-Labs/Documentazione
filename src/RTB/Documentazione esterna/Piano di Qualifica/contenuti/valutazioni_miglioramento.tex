% Intestazione
\fancyhead[L]{6 \hspace{0.2cm} Valutazioni per il miglioramento} % Testo a sinistra


\section{Valutazioni per il miglioramento}
\label{sec:Valutazioni per il miglioramento}


\subsection{Scopo}
Valutazioni periodiche sono svolte da parte del gruppo per identificare problemi e possibili
risoluzioni. L’obiettivo è facilitare un sistema di miglioramento continuo nel corso del progetto, 
prevenendo la ripetizione degli stessi errori in futuro. 
Queste valutazioni riflettono tre categorie, ovvero:
\begin{itemize}
    \item Valutazioni sugli strumenti utilizzati;
    \item Valutazione sull'organizzazione;
    \item Valutazione sui ruoli.
\end{itemize}

\subsection{Valutazioni sugli strumenti utilizzati}
\begin{table}[h!]
    \centering
    \begin{tabularx}{\textwidth}{|X|p{0.1\textwidth}|X|}\hline
    \rowcolor[HTML]{FFD700}
    \textbf{Problema} & \textbf{Rischio associato} & \textbf{Soluzione} \\ 
    \hline
    Ricerca tecnologia utilizzata è superficiale e richiede maggior approfondimento 
    & Alto & Prima di procedere con l'applicazione della tecnologia è essenziale condurre
    uno studio completo e dettagliato su di essa. \\ 
    \hline
    Milestone definite in modo troppo ampio & Medio 
    & Dividere le milestone in frammenti più gestibili e dimensioni ridotte. \\ 
    \hline
    Bassa compatibilità tra i diversi strumenti utilizzati. & Alto &
    Scegliere strumenti che offrono integrazioni native o sviluppare script per automatizzare l'integrazione. \\ 
    \hline
    Gli strumenti di testing automatico non coprono tutte le funzionalità del software & Alto &
    Integrare strumenti di testing aggiuntivi e sviluppare test personalizzati per coprire le funzionalità mancanti. \\
    \hline
    \end{tabularx}
    \caption{Valutazioni sugli strumenti utilizzati}
    \label{tab:valutazioni_strumenti}
\end{table}

\subsection{Valutazione sull'organizzazione}
\begin{table}[h!]
    \centering
    \begin{tabularx}{\textwidth}{|X|p{0.1\textwidth}|X|}\hline
    \rowcolor[HTML]{FFD700}
    \textbf{Problema} & \textbf{Rischio associato} & \textbf{Soluzione} \\ 
    \hline
    Il gruppo ha affrontato diverse sfide nell’esecuzione dei compiti in modo asincrono & Medio
    & Il gruppo, in collaborazione con il Responsabile, ha deciso di implementare un maggior monitoraggio delle
    attività nel breve termine. \\
    \hline
    Mancanza di comunicazione tra i membri del gruppo & Alto & 
    Organizzare riunioni periodiche e utilizzare strumenti di comunicazione come Telegram e Discord. \\ 
    \hline
    Mancanza di coordinamento tra i membri del gruppo & Alto & 
    Assegnare un responsabile per ogni attività e definire chiaramente i compiti e le scadenze. \\ 
    \hline
    Mancanza di trasparenza nelle decisioni prese & Medio & 
    Documentare le decisioni prese e condividerle con il gruppo. \\ 
    \hline
    Riunioni troppo frequenti e poco produttive & Basso & 
    Ridurre il numero di riunioni e migliorare l'efficienza con un ordine del giorno chiaro e obiettivi specifici. \\
    \hline
    \end{tabularx}
    \caption{Valutazioni sull'organizzazione}
    \label{tab:valutazioni_organizzazione}
\end{table}
\newpage
\subsection{Valutazione sui ruoli}
\begin{table}[h!]
    \centering
    \begin{tabularx}{\textwidth}{|X|p{0.1\textwidth}|X|}\hline
    \rowcolor[HTML]{FFD700}
    \textbf{Problema} & \textbf{Rischio associato} & \textbf{Soluzione} \\ 
    \hline
    L’analisi dei requisiti è stata superficiale e non ha approfondito adeguatamente le
    esigenze e le specifiche del progetto & Alto & 
    Abbiamo intensificato il confronto con i professori e i proponenti per migliorare la
    ricerca e ottenere una comprensione più approfondita dei requisiti del progetto. \\ 
    \hline
    Il responsabile del progetto è sovraccarico di lavoro e non riesce a seguire tutte le attività. & Medio & 
    Delegare alcune responsabilità agli altri membri del team e utilizzare strumenti di gestione del tempo per migliorare l'efficienza. \\ 
    \hline
    Sovraccarico di lavoro per alcuni membri del gruppo & Alto & 
    Ridistribuire i compiti in modo equo e bilanciare il carico di lavoro tra i membri del gruppo. \\ 
    \hline
    Mancanza di feedback costruttivo tra i membri del team. & Basso & 
    Implementare un sistema di feedback regolare e costruttivo, come le revisioni del codice e le retrospettive di progetto. \\
    \hline
    \end{tabularx}
    \caption{Valutazioni sui ruoli}
    \label{tab:valutazioni_ruoli}
\end{table}