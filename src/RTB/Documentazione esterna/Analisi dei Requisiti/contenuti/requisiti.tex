
\fancyhead[L]{4 \hspace{0.2cm} Requisiti} % Testo a sinistra

\section{Requisiti}
\label{sec:Requisiti}   

\subsection{Requisiti funzionali}
\label{sec:requisiti_funzionali}
Questa sezione delinea i requisiti funzionali del sistema. Gli obiettivi e le azioni chiave che
l'utente deve essere in grado di compiere sono presentati in modo chiaro, fornendo una base
solida per la \emph{progettazione}\textsubscript{\textit{\textbf{G}}} del sistema.
\begin{table}[h!]
    \centering
    \renewcommand{\arraystretch}{1.6} % Per aumentare l'altezza delle righe
    \begin{tabularx}{\textwidth}{|p{2cm}|p{3cm}|X|p{4cm}|} \hline
    \rowcolor[HTML]{FFD700} 
    \textbf{Codice} & \textbf{Rilevanza} & \textbf{Descrizione} & \textbf{Fonti} \\ \hline
    ROF1 & Obbligatorio & L'utente deve poter inserire un'interrogazione in linguaggio naturale nel sistema. & \bulhyperlink{UC1}{UC1} \\ \hline
    ROF2 & Obbligatorio & Deve essere presente una barra di input per poter inserire l'interrogazione dell'utente. & \bulhyperlink{UC1.1}{UC1.1} \\ \hline
    ROF3 & Obbligatorio & Deve essere presente un apposito pulsante per inviare al sistema le interrogazioni inserite. & \bulhyperlink{UC1.2}{UC1.2} \\ \hline
    ROF4 & Obbligatorio & Quando l'interrogazione viene invata al sistema, deve essere generata una risposta. & \bulhyperlink{UC2}{UC2} \\ \hline
    ROF5 & Obbligatorio & Nel caso il sistema fallisca nel generare una risposta per via di un problema intenrno, deve far visualizzare all'utente un messaggio di errore, chiedendo di riprovare più tardi. & \bulhyperlink{UC3}{UC3}, \bulhyperlink{UC4}{UC4} \\ \hline
    ROF6 & Obbligatorio & Nel caso in cui l'utente inserisca un'interrogazione che non riguarda i contenuti del database associato, il sistema non deve generare risposta, bensì deve avvisare l'utente che la domanda inserita è fuori contesto. & \bulhyperlink{UC3}{UC3}, \bulhyperlink{UC5}{UC5} \\ \hline
    ROF7 & Obbligatorio & Nel caso il sistema non riesca a trovare le informazioni richieste dall'utente, deve far visualizzare all'utente un messaggio di errore in cui viene segnalata la mancanza dell'informazione richiesta. & \bulhyperlink{UC3}{UC3}, \bulhyperlink{UC6}{UC6} \\ \hline
    ROF8 & Desiderabile & L'utente deve poter visualizzare i file da cui il sistema ha preso i dati per la risposta &\bulhyperlink{UC2.1}{UC2.1} \\ \hline
    RDF9 & Desiderabile & Nel caso l'utente tenti di accedere ai file utilizzati per genereare la risposta ma questo non sia possibile, deve essere visualizzato un messaggio di errore. &\bulhyperlink{UC15}{UC15} \\ \hline
    RDF10 & Desiderabile & Deve essere presente un pulsante al cui click la risposta del chatbot viene copiata nel dispositivo dell'utente. & \bulhyperlink{UC7}{UC7} \\ \hline
    \end{tabularx}
    \end{table}

    \vspace{0.5cm}
    \newpage
% Seconda parte della tabella
    \begin{table}[h!]
    \renewcommand{\arraystretch}{1.6} % Per aumentare l'altezza delle righe
    \begin{tabularx}{\textwidth}{|p{2cm}|p{3cm}|X|p{4cm}|} \hline
    \rowcolor[HTML]{FFD700} 
    \textbf{Codice} & \textbf{Rilevanza} & \textbf{Descrizione} & \textbf{Fonti} \\ \hline
    RDF11 & Desiderabile & Nel caso la risposta contenga uno snippet di codice, deve essere presente un pulsante che copi il singolo snippet nel dispositivo dell'utente. & \bulhyperlink{UC8}{UC8} \\ \hline
    RDF12 & Desiderabile & Deve essere presente un sistema di archiviazione delle domande e delle risposte in un database relazionale. & \bulhyperlink{UC9}{UC9} \\ \hline
    RDF13 & Desiderabile & L'utente deve poter visualizzare lo storico della chat, recuperato dal database relazionale. & \bulhyperlink{UC9}{UC9} \\ \hline
    RDF14 & Desiderabile & Nel caso il sistema fallisca nel recuperare lo storico della chat, deve essere fatto visualizzare un messaggio di errore all'utente spiegando che non è stato possibile recuperare lo storico. & \bulhyperlink{UC10}{UC10} \\ \hline
    ROF15 & Obbligatorio & Uno scheduler deve collegarsi al sistema e periodicamente aggiornare il database vettoriale con i dati più recenti. & \bulhyperlink{UC11}{UC11} \\ \hline
    ROF16 & Obbligatorio & Il sistema deve potersi collegare a GitHub tramite API per richiedere i dati aggiornati. & \bulhyperlink{UC11.1}{UC11.1} \\ \hline
    ROF17 & Obbligatorio & Il sistema deve potersi collegare a Jira tramite API per richiedere i dati aggiornati. & \bulhyperlink{UC11.2}{UC11.2} \\ \hline
    ROF18 & Obbligatorio & Il sistema deve potersi collegare a Confluence tramite API per richiedere i dati aggiornati. & \bulhyperlink{UC11.3}{UC11.3} \\ \hline
    ROF19 & Obbligatorio & Il sistema deve poter convertire i dati ottenuti in formato vettoriale. & \bulhyperlink{UC11.4.1}{UC11.4.1} \\ \hline
    ROF20 & Obbligatorio & Il sistema deve poter aggiornare il database vettoriale con i nuovi dati ottenuti. & \bulhyperlink{UC11.4}{UC11.4} \\ \hline
    RZF21 & Opzionale & Se la conversazione non è ancora avviata l'utente deve poter visualizzare e selezionare alcune domande di partenza proposte. & \bulhyperlink{UC12}{UC12},\bulhyperlink{UC12.1}{UC12.1} \\ \hline
    RZF22 & Opzionale & Durante la conversazione, all'utente devono venire suggerite alcune interrogazioni da porre al sistema. & \bulhyperlink{UC13}{UC13},\bulhyperlink{UC13.1}{UC13.1} \\ \hline
    RZF23 & Opzionale & Nel caso il sistema vada in errore nel tentativo di proporre altre domande per proseguire la conversazione, deve venire mostrato un messaggio che comunica l'errore all'utente e invita a fare altre domande. & \bulhyperlink{UC14}{UC14} \\ \hline
\end{tabularx}

    \caption{Requisiti funzionali}
    \label{tab:Requisiti_funzionali}
\end{table}

\vspace{0.5cm}
\newpage
% Seconda parte della tabella
\begin{table}[h!]
\renewcommand{\arraystretch}{1.6} % Per aumentare l'altezza delle righe
\begin{tabularx}{\textwidth}{|p{2cm}|p{3cm}|X|p{4cm}|} \hline
    \rowcolor[HTML]{FFD700} 
    \textbf{Codice} & \textbf{Rilevanza} & \textbf{Descrizione} & \textbf{Fonti} \\ \hline
    RZF24 & Opzionale & Il sistema deve registrare data e ora degli aggiornamenti del database vettoriale, in modo da poter scrivere un log di aggiornamento. & \bulhyperlink{UC16}{UC16} \\ \hline
    RZF25 & Opzionale & Deve essere possibile per l'utente visualizzare il log di aggiornamento del database vettoriale. & \bulhyperlink{UC16}{UC16} \\ \hline
    RZF26 & Opzionale & Il sistema deve mostrare un messaggio di errore nel caso non riesca a recuperare i log di aggiornamento vettoriale. & \bulhyperlink{UC17}{UC17} \\ \hline
    RZF27 & Opzionale & Il sistema deve poter mostrare all'utente se il database vettoriale è aggiornato o meno rispetto ai dati più recenti di Github, Jira e Confluence. & \bulhyperlink{UC18}{UC18},\bulhyperlink{UC19}{UC19}, \bulhyperlink{UC20}{UC20} \\ \hline
\end{tabularx}

\caption{Requisiti funzionali}
\label{tab:Requisiti_funzionali}
\end{table}

\subsection{Requisiti qualitativi}
\label{sec:Requisiti_qualitativi}
I requisiti qualitativi del sistema sono trattati in questo sotto-capitolo. Questa sezione
delinea le specifiche qualitative che devono essere rispettate per garantire la qualità del
sistema.
\begin{table}[h!]
    \centering
    \renewcommand{\arraystretch}{1.6} % Per aumentare l'altezza delle righe
    \begin{tabularx}{\textwidth}{|>{\centering\arraybackslash}c|>{\centering\arraybackslash}c|>{\centering\arraybackslash}X|>{\centering\arraybackslash}p{3cm}|} \hline
    \rowcolor[HTML]{FFD700} 
    \textbf{Codice} & \textbf{Rilevanza} & \textbf{Descrizione} & \textbf{Fonti} \\ \hline
    ROQ1 & Obbligatorio & Devono essere rispettate tutte le norme definite in \emph{Norme di Progetto}\textsubscript{\textbf{G}}. & Verbale Interno \\ \hline
    ROQ2 & Obbligatorio & Devono essere rispettate le \emph{metriche}\textsubscript{\textbf{G}} e i vincoli definiti in \emph{Piano di Qualifica}\textsubscript{\textbf{G}}. & Verbale Interno \\ \hline
    ROQ3 & Obbligatorio & Deve essere fornito un documento che riporti le attività di \emph{bug}\textsubscript{\textbf{G}} reporting svolte. & Capitolato \\ \hline
    ROQ4 & Obbligatorio & Deve essere fornito al \emph{proponente} il codice sorgente in un \emph{repository}\textsubscript{\textbf{G}} \emph{GitHub}\textsubscript{\textbf{G}}. & Capitolato \\ \hline
    ROQ5 & Obbligatorio & Deve essere fornito il \emph{Manuale Utente}\textsubscript{\textbf{G}}. & Capitolato \\ \hline
    \end{tabularx}
    \caption{Requisiti qualitativi}
    \label{tab:Requisiti_qualitativi}
\end{table}


\newpage
\subsection{Requisiti di vincolo}
\label{sec:req_vincolo}
Qui sono presentati i requisiti di vincolo, che rappresentano le restrizioni e le condizioni
che devono essere soddisfatte durante lo sviluppo e l'implementazione del sistema. Questa
sezione fornisce le linee guida fondamentali che devono essere rispettate per garantire la
coerenza e l'\emph{efficienza}\textsubscript{\textit{\textbf{G}}} del prodotto.
\begin{table}[h!]
    \centering
    \renewcommand{\arraystretch}{1.6} % Per aumentare l'altezza delle righe
    \begin{tabularx}{\textwidth}{|>{\centering\arraybackslash}c|>{\centering\arraybackslash}c|>{\centering\arraybackslash}X|>{\centering\arraybackslash}p{3cm}|} \hline
    \rowcolor[HTML]{FFD700} 
    \textbf{Codice} & \textbf{Rilevanza} & \textbf{Descrizione} & \textbf{Fonti} \\ \hline
    ROV 1 & Obbligatorio & L'applicazione deve garantire la compatibilità con l'ultima versione di \emph{Google Chrome}\textsubscript{\textit{\textbf{G}}} & Verbale esterno \\ \hline
    ROV 2 & Obbligatorio & Il sistema deve garantire piena integrazione con \emph{API}\textsubscript{\textit{\textbf{G}}} di \emph{Confluence}\textsubscript{\textit{\textbf{G}}} & Verbale esterno \\ \hline
    ROV 3 & Obbligatorio & Il sistema deve garantire piena integrazione con \emph{API}\textsubscript{\textit{\textbf{G}}} di \emph{Jira}\textsubscript{\textit{\textbf{G}}} & Verbale esterno\\ \hline
    ROV 4 & Obbligatorio & Il sistema deve garantire piena integrazione con \emph{API}\textsubscript{\textit{\textbf{G}}} di \emph{GitHub}\textsubscript{\textit{\textbf{G}}} & Verbale esterno \\ \hline
    \end{tabularx}
    \caption{Requisiti di vincolo}
    \label{tab:Requisiti_di_vincolo}
\end{table}

\newpage
\subsection{Requisiti implementativi}
\label{sec:Requisiti_implementativi}
\begin{table}[h!]
    \centering
    \renewcommand{\arraystretch}{1.6} % Per aumentare l'altezza delle righe
    \begin{tabularx}{\textwidth}{|>{\centering\arraybackslash}c|>{\centering\arraybackslash}c|>{\centering\arraybackslash}X|>{\centering\arraybackslash}p{3cm}|} \hline
    \rowcolor[HTML]{FFD700} 
    \textbf{Codice} & \textbf{Rilevanza} & \textbf{Descrizione} & \textbf{Fonti} \\ \hline
    prova & prova & prova & prova \\ \hline
    prova & prova & prova & prova \\ \hline
    \end{tabularx}
    \caption{Requisiti implementativi}
    \label{tab:Requisiti_implementativi}
\end{table}

\subsection{Requisiti sistema operativo}

\subsection{Requisiti prestazionali}
\label{sec:req_prestazionali}
Trattandosi di una \emph{applicazione web}\textsubscript{\textit{\textbf{G}}}, i requisiti prestazionali saranno influenzati principalmente dalla connessione Internet, 
le prestazioni del dispositivo e dall'ottimizzazione specifica per il browser \emph{Google Chrome}, l'unico supportato ufficialmente.

\subsection{Requisiti di sicurezza}

\subsection{Tracciamento}

\subsubsection{Fonte - Requisiti}
\label{sec:fonte_requisito}
\begin{table}[h!]
    \centering
    \renewcommand{\arraystretch}{1.6} % Per aumentare l'altezza delle righe
    \begin{tabularx}{0.8\textwidth}{|>{\centering\arraybackslash}p{2.8cm}|>{\centering\arraybackslash}X|} \hline
    \rowcolor[HTML]{FFD700} 
    \textbf{Fonte} & \textbf{Requisiti} \\ \hline
    Capitolato & ROQ3, ROQ4, ROQ5 \\ \hline
    Verbali interni & ROQ1, ROQ2 \\ \hline
    Verbali esterni & ROV1, ROV2, ROV3, ROV4 \\ \hline
    \bulhyperlink{UC1}{UC1} & ROF1 \\ \hline
    \bulhyperlink{UC2}{UC2} & ROF5 \\ \hline
    \bulhyperlink{UC3}{UC3} & ROF6 \\ \hline
    \bulhyperlink{UC3.1}{UC3.1} & ROF7 \\ \hline
    \bulhyperlink{UC3.2}{UC3.2} & ROF8 \\ \hline
    \bulhyperlink{UC5}{UC5} & ROF10 \\ \hline
    \bulhyperlink{UC5.2}{UC5.2} & ROF11 \\ \hline
    \bulhyperlink{UC8}{UC8}, \bulhyperlink{UC8.1}{UC8.1}, \bulhyperlink{UC8.2}{UC8.2}, \bulhyperlink{UC8.3}{UC8.3}& ROF14 \\ \hline
    \bulhyperlink{UC9}{UC9} & ROF15 \\ \hline
    \bulhyperlink{UC17}{UC17} & ROF16 \\ \hline
    \bulhyperlink{UC11}{UC11}, \bulhyperlink{UC11.1}{UC11.1}, \bulhyperlink{UC11.2}{UC11.2}, \bulhyperlink{UC11.3}{UC11.3}& ROF17 \\ \hline
    \bulhyperlink{UC16}{UC16} & ROF18 \\ \hline
    \end{tabularx}
    \caption{Tracciamento Fonte - Requisiti}
    \label{tab:Tracciamento_fonte_requisiti}
\end{table}

\label{sec:fonte_requisito}
\begin{table}[h!]
    \centering
    \renewcommand{\arraystretch}{1.6} % Per aumentare l'altezza delle righe
    \begin{tabularx}{0.8\textwidth}{|>{\centering\arraybackslash}p{2.8cm}|>{\centering\arraybackslash}X|} \hline
    \rowcolor[HTML]{FFD700} 
    \textbf{Fonte} & \textbf{Requisiti} \\ \hline
    \bulhyperlink{UC12}{UC12}, \bulhyperlink{UC12.1}{UC12.1} & ROF19 \\ \hline
    \bulhyperlink{UC13}{UC13}, \bulhyperlink{UC13.1}{UC13.1}, \bulhyperlink{UC13.2}{UC13.2} & ROF20 \\ \hline
    \bulhyperlink{UC18}{UC18} & ROF21 \\ \hline
    \bulhyperlink{UC14}{UC14}, \bulhyperlink{UC14.1}{UC14.1} & ROF22 \\ \hline
    \bulhyperlink{UC15}{UC15} & ROF23 \\ \hline
    \end{tabularx}
    \caption{Tracciamento Fonte - Requisiti}
    \label{tab:Tracciamento_fonte_requisiti}
\end{table}


\subsubsection{Requisito - Fonti}
\begin{table}[h!]
    \centering
    \renewcommand{\arraystretch}{1.6} % Per aumentare l'altezza delle righe
    \begin{tabularx}{0.8\textwidth}{|>{\centering\arraybackslash}p{2.8cm}|>{\centering\arraybackslash}X|} \hline
    \rowcolor[HTML]{FFD700} 
    \textbf{Requisito} & \textbf{Fonti} \\ \hline
    ROF1 & \bulhyperlink{UC1}{UC1}\\ \hline
    ROF2 & \bulhyperlink{UC1.1}{UC1.1}\\ \hline
    ROF3 & \bulhyperlink{UC1.2}{UC1.2}\\ \hline
    ROF4 & \bulhyperlink{UC1.3}{UC1.3}\\ \hline
    ROF5 & \bulhyperlink{UC2}{UC2}\\ \hline
    ROF6 & \bulhyperlink{UC3}{UC3}, \bulhyperlink{UC4}{UC4}\\ \hline
    ROF7 & \bulhyperlink{UC3}{UC3}, \bulhyperlink{UC5}{UC5}\\ \hline
    ROF8 & \bulhyperlink{UC3}{UC3}, \bulhyperlink{UC6}{UC6}\\ \hline
    ROF9 & \bulhyperlink{UC2.1}{UC2.1}\\ \hline
    RDF10 & \bulhyperlink{UC15}{UC15} \\ \hline
    RDF11 & \bulhyperlink{UC7}{UC7}\\ \hline
    RDF12 & \bulhyperlink{UC8}{UC8}\\ \hline
    RDF13 & \bulhyperlink{UC9}{UC9}\\ \hline
    RDF14 & \bulhyperlink{UC9}{UC9}\\ \hline
    RDF15 & \bulhyperlink{UC10}{UC10}\\ \hline
    ROF16 & \bulhyperlink{UC11}{UC11}\\ \hline
    ROF17 & \bulhyperlink{UC11.1}{UC11.1}\\ \hline
    ROF18 & \bulhyperlink{UC11.2}{UC11.2}\\ \hline
    ROF19 & \bulhyperlink{UC11.3}{UC11.3}\\ \hline
    ROF20 & \bulhyperlink{UC11.4.1}{UC11.4.1}\\ \hline
    ROF21 & \bulhyperlink{UC11.4}{UC11.4}\\ \hline
    RZF22 & \bulhyperlink{UC12}{UC12},\bulhyperlink{UC12.1}{UC12.1}\\ \hline
    \end{tabularx}
    \caption{Tracciamento Requisito - Fonti}
    \label{tab:Tracciamento_requisiti_fonti}
\end{table}


\begin{table}[h!]
    \centering
    \renewcommand{\arraystretch}{1.6} % Per aumentare l'altezza delle righe
    \begin{tabularx}{0.8\textwidth}{|>{\centering\arraybackslash}p{2.8cm}|>{\centering\arraybackslash}X|} \hline
    \rowcolor[HTML]{FFD700} 
    \textbf{Requisito} & \textbf{Fonti} \\ \hline
    RZF23 & \bulhyperlink{UC13}{UC13},\bulhyperlink{UC13.1}{UC13.1}\\ \hline
    RZF24 & \bulhyperlink{UC14}{UC14}\\ \hline
    RZF25 & \bulhyperlink{UC16}{UC16}\\ \hline
    RZF26 & \bulhyperlink{UC16}{UC16}\\ \hline
    RZF27 & \bulhyperlink{UC17}{UC17}\\ \hline
    RZF28 & \bulhyperlink{UC18}{UC18},\bulhyperlink{UC19}{UC19}, \bulhyperlink{UC20}{UC20}\\ \hline
    ROQ1 & Verbale Interno\\ \hline
    ROQ2 & Verbale Interno\\ \hline
    ROQ3 & Capitolato \\ \hline
    ROQ4 & Capitolato \\ \hline
    ROQ5 & Capitolato \\ \hline
    ROV1 & Verbale esterno\\ \hline
    ROV2 & Verbale esterno\\ \hline
    ROV3 & Verbale esterno\\ \hline
    ROV4 & Verbale esterno\\ \hline
    \end{tabularx}
    \caption{Tracciamento Requisito - Fonti}
    \label{tab:Tracciamento_requisiti_fonti}
\end{table}