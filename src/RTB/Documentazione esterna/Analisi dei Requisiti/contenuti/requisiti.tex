
\fancyhead[L]{4 \hspace{0.2cm} Requisiti} % Testo a sinistra

\section{Requisiti}
\label{sec:Requisiti}   

\subsection{Requisiti funzionali}
Questa sezione delinea i requisiti funzionali del sistema. Gli obiettivi e le azioni chiave che
l'utente deve essere in grado di compiere sono presentati in modo chiaro, fornendo una base
solida per la \emph{progettazione}\textsubscript{\textit{\textbf{G}}} del sistema.
\begin{table}[h!]
    \centering
    \renewcommand{\arraystretch}{1.6} % Per aumentare l'altezza delle righe
    \begin{tabularx}{\textwidth}{|p{2cm}|p{3cm}|X|p{4cm}|} \hline
    \rowcolor[HTML]{FFD700} 
    \textbf{Codice} & \textbf{Rilevanza} & \textbf{Descrizione} & \textbf{Fonti} \\ \hline
    ROF1 & Obbligatorio & L'utente deve poter inserire un'interrogazione in linguaggio naturale nel sistema. & UC1 \\ \hline
    %ROF2 & Obbligatorio & Deve essere presente una barra di input per poter inserire l'interrogazione dell'utente. & UC1.1 \\ \hline
    %ROF3 & Obbligatorio & Deve essere presente un apposito tasto per inviare al sistema le interrogazioni inserite. & UC1.2 \\ \hline
    % UC1.1 e UC 1.2 sono use case? o rispondono alla domanda "come"? Dal momento che riguarda il come l'utente inserisce i dati?
    %ROF4 & Obbligatorio & ???????????Non è chiaro: cosa si intende con "validato"?. & UC1.3 \\ \hline
    ROF5 & Obbligatorio & Nel caso in cui l'utente inserisca un'interrogazione che non riguarda i contenuti del database associato, il sistema non deve generare risposta, e avvisare l'utente di conseguenza. & UC2 \\ \hline
    ROF6 & Obbligatorio & Quando l'interrogazione viene invata al sistema, deve essere generata una risposta. & UC3 \\ \hline
    ROF7 & Obbligatorio & La risposta deve essere generata tramite ricerca all'interno del database vettoriale. & UC3.1 \\ \hline
    ROF8 & Obbligatorio & La risposta deve essere in linguaggio naturale. & UC3.2 \\ \hline
    %ROF9 & Obbligatorio & ???????????Sentire Michael. & UC4 \\ \hline
    ROF10 & Obbligatorio & La risposta generata deve essere visualizzata all'utente, ed essere identificabile come tale. & UC5 \\ \hline
    ROF11 & Obbligatorio & La risposta deve poter contenere del codice, e nel caso renderlo identificabile come tale. & UC5.1, UC5.2, UC5.3 \\ \hline
    %ROF12 & Obbligatorio & ????? Verificare cosa risponde il gruppo, non sono sicuro esista. & UC6 \\ \hline
    %ROF13 & Obbligatorio & ???????   Nel caso la risposta contenga del codice, per ogni snippet deve essere presente un tasto apposito che se selezionato copia in automatico il codice contenuto nello snippet. & UC7 \\ \hline
    ROF14 & Obbligatorio & La risposta deve essere generata prendendo in considerazione i dati di contesto provenienti da GitHub,Jira e Confluence. & UC8, UC8.1, UC8.2, UC8.3 \\ \hline
    ROF15 & Obbligatorio & L'utente deve poter visualizzare uno storico di sessione contenente le domande fatte precedentemente e le risposte. & UC9 \\ \hline



    \end{tabularx}
    \end{table}

    \vspace{0.5cm}
    \newpage
% Seconda parte della tabella
    \begin{table}[h!]
    \renewcommand{\arraystretch}{1.6} % Per aumentare l'altezza delle righe
    \begin{tabularx}{\textwidth}{|p{2cm}|p{3cm}|X|p{4cm}|} \hline
    \rowcolor[HTML]{FFD700} 
    \textbf{Codice} & \textbf{Rilevanza} & \textbf{Descrizione} & \textbf{Fonti} \\ \hline
    ROF16 & Obbligatorio & In caso di errore nel recuperare lo storico di sessione, all'utente deve venir notificato che c'è stato un errore e non è possibile recuperarlo. & UC17 \\ \hline
    ROF17 & Obbligatorio & Le risposte generate devono riguardare (se possibile) i dati aggiornati. & UC11, UC11.1, UC11.2, UC11.3 \\ \hline
    ROF18 & Obbligatorio & Nel caso in cui vengono rilevati nuovi dati più aggiornati, ma non è possibile usarli, il sistema deve avvisare l'utente che le informazioni restituite potrebbero essere basate su dati obsoleti. & UC16 \\ \hline
    ROF19 & Obbligatorio & Se la conversazione non è ancora avviata l'utente deve poter visualizzare e selezionare alcune domande di partenza proposte. & UC12, UC12.1 \\ \hline
    ROF20 & Obbligatorio & Il sistema deve poter ricondurre l'utente ai file utilizzati per rispondere alla sua interrogazione. & UC13, UC13.1, UC13.2 \\ \hline
    ROF21 & Obbligatorio & Nel caso il sistema non riesca a individuare i file utilizzati per generare la risposta, verrà avvisato l'utente con un apposito messaggio. & UC18 \\ \hline
    ROF22 & Obbligatorio & Durante la conversazione, all'utente devono venire suggerite alcune interrogazioni da porre al sistema. & UC14, UC14.1 \\ \hline
    ROF23 & Obbligatorio & Nel caso il sistema vada in errore nel tentativo di proporre altre domande, deve venire mostrato un messaggio che comunica l'errore all'utente e invita a fare altre domande. & UC15 \\ \hline
    \end{tabularx}

    \caption{Requisiti funzionali}
    \label{tab:Requisiti_funzionali}
\end{table}


\newpage
\subsection{Requisiti qualitativi}
\label{sec:Requisiti_qualitativi}
I requisiti qualitativi del sistema sono trattati in questo sotto-capitolo. Questa sezione
delinea le specifiche qualitative che devono essere rispettate per garantire la qualità del
sistema.
\begin{table}[h!]
    \centering
    \renewcommand{\arraystretch}{1.6} % Per aumentare l'altezza delle righe
    \begin{tabularx}{\textwidth}{|>{\centering\arraybackslash}c|>{\centering\arraybackslash}c|>{\centering\arraybackslash}X|>{\centering\arraybackslash}p{3cm}|} \hline
    \rowcolor[HTML]{FFD700} 
    \textbf{Codice} & \textbf{Rilevanza} & \textbf{Descrizione} & \textbf{Fonti} \\ \hline
    ROQ1 & Obbligatorio & Devono essere rispettate tutte le norme definite in \emph{Norme di Progetto}\textsubscript{\textbf{G}}. & Verbale Interno \\ \hline
    ROQ2 & Obbligatorio & Devono essere rispettate le \emph{metriche}\textsubscript{\textbf{G}} e i vincoli definiti in \emph{Piano di Qualifica}\textsubscript{\textbf{G}}. & Verbale Interno \\ \hline
    ROQ3 & Obbligatorio & Deve essere fornito un documento che riporti le attività di \emph{bug}\textsubscript{\textbf{G}} reporting svolte. & Capitolato \\ \hline
    ROQ4 & Obbligatorio & Deve essere fornito al \emph{proponente} il codice sorgente in un \emph{repository}\textsubscript{\textbf{G}} \emph{GitHub}\textsubscript{\textbf{G}}. & Capitolato \\ \hline
    ROQ5 & Obbligatorio & Deve essere fornito il \emph{Manuale Utente}\textsubscript{\textbf{G}}. & Capitolato \\ \hline
    \end{tabularx}
    \caption{Requisiti qualitativi}
    \label{tab:Requisiti_qualitativi}
\end{table}



\subsection{Requisiti di vincolo}
Qui sono presentati i requisiti di vincolo, che rappresentano le restrizioni e le condizioni
che devono essere soddisfatte durante lo sviluppo e l'implementazione del sistema. Questa
sezione fornisce le linee guida fondamentali che devono essere rispettate per garantire la
coerenza e l'\emph{efficienza}\textsubscript{\textit{\textbf{G}}} del prodotto.
\begin{table}[h!]
    \centering
    \renewcommand{\arraystretch}{1.6} % Per aumentare l'altezza delle righe
    \begin{tabularx}{\textwidth}{|>{\centering\arraybackslash}c|>{\centering\arraybackslash}c|>{\centering\arraybackslash}X|>{\centering\arraybackslash}p{3cm}|} \hline
    \rowcolor[HTML]{FFD700} 
    \textbf{Codice} & \textbf{Rilevanza} & \textbf{Descrizione} & \textbf{Fonti} \\ \hline
    prova & prova & prova & prova \\ \hline
    prova & prova & prova & prova \\ \hline
    \end{tabularx}
    \caption{Requisiti di vincolo}
    \label{tab:Requisiti_di_vincolo}
\end{table}

\newpage
\subsection{Requisiti implementativi}
\label{sec:Requisiti_implementativi}
\begin{table}[h!]
    \centering
    \renewcommand{\arraystretch}{1.6} % Per aumentare l'altezza delle righe
    \begin{tabularx}{\textwidth}{|>{\centering\arraybackslash}c|>{\centering\arraybackslash}c|>{\centering\arraybackslash}X|>{\centering\arraybackslash}p{3cm}|} \hline
    \rowcolor[HTML]{FFD700} 
    \textbf{Codice} & \textbf{Rilevanza} & \textbf{Descrizione} & \textbf{Fonti} \\ \hline
    prova & prova & prova & prova \\ \hline
    prova & prova & prova & prova \\ \hline
    \end{tabularx}
    \caption{Requisiti implementativi}
    \label{tab:Requisiti_implementativi}
\end{table}

\subsection{Requisiti sistema operativo}

\subsection{Requisiti prestazionali}
Trattandosi di una \emph{applicazione web}\textsubscript{\textit{\textbf{G}}}, i requisiti prestazionali saranno influenzati principalmente dalla connessione Internet, 
le prestazioni del dispositivo e lo specifico \emph{browser}\textsubscript{\textit{\textbf{G}}} utilizzato dell'utente.

\subsection{Requisiti di sicurezza}

\subsection{Tracciamento}

\subsubsection{Fonte - Requisiti}
\begin{table}[h!]
    \centering
    \renewcommand{\arraystretch}{1.6} % Per aumentare l'altezza delle righe
    \begin{tabularx}{0.8\textwidth}{|>{\centering\arraybackslash}p{2.8cm}|>{\centering\arraybackslash}X|} \hline
    \rowcolor[HTML]{FFD700} 
    \textbf{Fonte} & \textbf{Requisiti} \\ \hline
    prova & prova \\ \hline
    prova & prova \\ \hline
    \end{tabularx}
    \caption{Tracciamento Fonte - Requisiti}
    \label{tab:Tracciamento_fonte_requisiti}
\end{table}


\subsubsection{Requisito - Fonti}
\begin{table}[h!]
    \centering
    \renewcommand{\arraystretch}{1.6} % Per aumentare l'altezza delle righe
    \begin{tabularx}{0.8\textwidth}{|>{\centering\arraybackslash}p{2.8cm}|>{\centering\arraybackslash}X|} \hline
    \rowcolor[HTML]{FFD700} 
    \textbf{Requisito} & \textbf{Fonti} \\ \hline
    prova & prova \\ \hline
    prova & prova \\ \hline
    \end{tabularx}
    \caption{Tracciamento Requisito - Fonti}
    \label{tab:Tracciamento_requisiti_fonti}
\end{table}