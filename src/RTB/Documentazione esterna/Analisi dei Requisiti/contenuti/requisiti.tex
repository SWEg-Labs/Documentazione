\fancyhead[L]{4 \hspace{0.2cm} Requisiti} % Testo a sinistra

\section{Requisiti}
\label{sec:Requisiti}

\subsection{Requisiti funzionali}
Questa sezione delinea i requisiti funzionali del sistema. Gli obiettivi e le azioni chiave che
l'utente deve essere in grado di compiere sono presentati in modo chiaro, fornendo una base
solida per la \emph{progettazione}\textsubscript{\textit{\textbf{G}}} del sistema.
    \begin{table}[h!]
        \centering
        \renewcommand{\arraystretch}{1.6} % Per aumentare l'altezza delle righe
        \begin{tabularx}{\textwidth}{|>{\centering\arraybackslash}c|>{\centering\arraybackslash}c|>{\centering\arraybackslash}X|>{\centering\arraybackslash}p{3cm}|} \hline
        \rowcolor[HTML]{FFD700} 
        \textbf{Codice} & \textbf{Rilevanza} & \textbf{Descrizione} & \textbf{Fonti} \\ \hline
        prova & prova & prova & prova \\ \hline
        prova & prova & prova & prova \\ \hline
        \end{tabularx}
        \caption{Requisiti funzionali}
        \label{tab:Requisiti_funzionali}
    \end{table}
    

\subsection{Requisiti qualitativi}
\label{sec:Requisiti_qualitativi}
I requisiti qualitativi del sistema sono trattati in questo sotto-capitolo. Questa sezione
delinea le specifiche qualitative che devono essere rispettate per garantire la qualità del
sistema.
\begin{table}[h!]
    \centering
    \renewcommand{\arraystretch}{1.6} % Per aumentare l'altezza delle righe
    \begin{tabularx}{\textwidth}{|>{\centering\arraybackslash}c|>{\centering\arraybackslash}c|>{\centering\arraybackslash}X|>{\centering\arraybackslash}p{3cm}|} \hline
    \rowcolor[HTML]{FFD700} 
    \textbf{Codice} & \textbf{Rilevanza} & \textbf{Descrizione} & \textbf{Fonti} \\ \hline
    ROQ1 & Obbligatorio & Devono essere rispettate tutte le norme definite in \emph{Norme di Progetto}. & Verbale Interno \\ \hline
    ROQ2 & Obbligatorio & Devono essere rispettate le \emph{metriche}\textsubscript{\textbf{G}} e i vincoli definiti in \emph{Piano di Qualifica}. & Verbale Interno \\ \hline
    ROQ3 & Obbligatorio & Deve essere fornito un documento che riporti le attività di \emph{bug}\textsubscript{\textbf{G}} reporting svolte. & Capitolato \\ \hline
    ROQ4 & Obbligatorio & Deve essere fornito al \emph{proponente} il codice sorgente in un \emph{repository}\textsubscript{\textbf{G}} GitHub. & Capitolato \\ \hline
    ROQ5 & Obbligatorio & Deve essere fornito il Manuale Utente. & Capitolato \\ \hline
    \end{tabularx}
    \caption{Requisiti qualitativi}
    \label{tab:Requisiti_qualitativi}
\end{table}

\subsection{Requisiti di vincolo}
Qui sono presentati i requisiti di vincolo, che rappresentano le restrizioni e le condizioni
che devono essere soddisfatte durante lo sviluppo e l'implementazione del sistema. Questa
sezione fornisce le linee guida fondamentali che devono essere rispettate per garantire la
coerenza e l'\emph{efficienza}\textsubscript{\textit{\textbf{G}}} del prodotto.
\begin{table}[h!]
    \centering
    \renewcommand{\arraystretch}{1.6} % Per aumentare l'altezza delle righe
    \begin{tabularx}{\textwidth}{|>{\centering\arraybackslash}c|>{\centering\arraybackslash}c|>{\centering\arraybackslash}X|>{\centering\arraybackslash}p{3cm}|} \hline
    \rowcolor[HTML]{FFD700} 
    \textbf{Codice} & \textbf{Rilevanza} & \textbf{Descrizione} & \textbf{Fonti} \\ \hline
    prova & prova & prova & prova \\ \hline
    prova & prova & prova & prova \\ \hline
    \end{tabularx}
    \caption{Requisiti di vincolo}
    \label{tab:Requisiti_di_vincolo}
\end{table}

\subsubsection{Requisiti sistema operativo}

\subsubsection{Requisiti prestazionali}
Trattandosi di una \emph{applicazione web}\textsubscript{\textit{\textbf{G}}}, i requisiti prestazionali saranno influenzati principalmente dalla connessione Internet, 
le prestazioni del dispositivo e lo specifico \emph{browser}\textsubscript{\textit{\textbf{G}}} utilizzato dell'utente.

\subsubsection{Requisiti di sicurezza}

\subsection{Tracciamento}

\subsubsection{Fonte - Requisiti}
\begin{table}[h!]
    \centering
    \renewcommand{\arraystretch}{1.6} % Per aumentare l'altezza delle righe
    \begin{tabularx}{0.8\textwidth}{|>{\centering\arraybackslash}p{2.8cm}|>{\centering\arraybackslash}X|} \hline
    \rowcolor[HTML]{FFD700} 
    \textbf{Fonte} & \textbf{Requisiti} \\ \hline
    prova & prova \\ \hline
    prova & prova \\ \hline
    \end{tabularx}
    \caption{Tracciamento Fonte - Requisiti}
    \label{tab:Tracciamento_fonte_requisiti}
\end{table}


\subsubsection{Requisito - Fonti}
\begin{table}[h!]
    \centering
    \renewcommand{\arraystretch}{1.6} % Per aumentare l'altezza delle righe
    \begin{tabularx}{0.8\textwidth}{|>{\centering\arraybackslash}p{2.8cm}|>{\centering\arraybackslash}X|} \hline
    \rowcolor[HTML]{FFD700} 
    \textbf{Requisito} & \textbf{Fonti} \\ \hline
    prova & prova \\ \hline
    prova & prova \\ \hline
    \end{tabularx}
    \caption{Tracciamento Requisito - Fonti}
    \label{tab:Tracciamento_requisiti_fonti}
\end{table}