% Intestazione
\fancyhead[L]{3 \hspace{0.2cm} Casi d'uso} % Testo a sinistra

\setcounter{secnumdepth}{5} % Per il 5° livello di sezione


\section{Casi d'uso}
\label{sec:casi_uso}

\subsection{Scopo}

Lo scopo di questa sezione è descrivere in maniera dettagliata i casi d’uso individuati dal
gruppo, in riferimento alle funzionalità dell’applicazione.


\subsection{Attori}

L’applicazione prevede la presenza di \dots


\subsection{Lista casi d'uso}




% TEMPLATE

\begin{comment}
\hypertarget{UC1}{}
\subsubsection{UC1: Inserimento di interrogazione in linguaggio naturale}

\begin{figure}[h]
    \centering
    \includegraphics[width=\textwidth]{placeholder.png}
    \caption{Visualizzazione una risposta di BuddyBot generata utilizzando dati di contesto aggiornati}
\end{figure}

\begin{itemize}
    \item \textbf{Attori} principali}: \dots;
    \item \textbf{Precondizioni}: \dots;
    % Oppure, se ci sono più precondizioni:
    \item \textbf{Precondizioni}: 
    \begin{itemize}
        \item \dots;
        \item \dots.
    \end{itemize}
    \item \textbf{Trigger}: \dots;
    \item \textbf{Postcondizioni}: \dots;
    \item \textbf{Scenario principale}:
    \begin{enumerate}
        \item \dots;
        \item \dots.
    \end{enumerate}
    \item \textbf{Sottocasi d'uso}:
    \begin{itemize}
        \item \dots;
        \item \dots.
    \end{itemize}
    \item \textbf{Scenario alternativo}:
    \begin{enumerate}
        \item \dots;
        \item \dots.
    \end{enumerate}
\end{itemize}
\end{comment}




\hypertarget{UC1}{}
\subsubsection{UC1: Inserimento di interrogazione in linguaggio naturale}

\begin{figure}[h]
    \centering
    \includegraphics[width=\textwidth]{placeholder.png}
    \caption{Visualizzazione una risposta di BuddyBot generata utilizzando dati di contesto aggiornati}
\end{figure}

\begin{itemize}
    \item \textbf{\emph{Attori}\textsubscript{\textbf{\textit{G}}} principali}: \dots;
    \item \textbf{Precondizioni}: \dots;
    \item \textbf{\emph{Trigger}\textsubscript{\textbf{\textit{G}}}}: \dots;
    \item \textbf{Postcondizioni}: \dots;
    \item \textbf{\emph{Scenario principale}\textsubscript{\textbf{\textit{G}}}}:
    \begin{enumerate}
        \item \dots.
    \end{enumerate}
    \item \textbf{\emph{Sottocasi d'uso}\textsubscript{\textbf{\textit{G}}}}:
    \begin{itemize}
        \item \dots.
    \end{itemize}
    \item \textbf{\emph{Scenario alternativo}\textsubscript{\textbf{\textit{G}}}}:
    \begin{enumerate}
        \item \dots.
    \end{enumerate}
\end{itemize}



\hypertarget{UC3}{}
\subsubsection{UC3: Generazione della risposta}



\hypertarget{UC5}{}
\subsubsection{UC5: Visualizzazione della risposta generata}

\begin{figure}[h]
    \centering
    \includegraphics[width=\textwidth]{Diagramma UC5 - Visualizzazione della risposta generata.png}
    \caption{Visualizzazione della risposta generata}
\end{figure}

\begin{itemize}
    \item \textbf{Attori principali}: Utente;
    \item \textbf{Precondizioni}: BuddyBot ha generato correttamente la risposta alla domanda dell'utente;
    \item \textbf{Trigger}: L'utente desidera visualizzare la risposta alla domanda che ha posto;
    \item \textbf{Postcondizioni}: L'utente visualizza la risposta generata da BuddyBot;
    \item \textbf{Scenario principale}:
    \begin{enumerate}
        \item \bulhyperlink{UC3}{UC3}: BuddyBot ha generato correttamente la risposta alla domanda dell'utente;
        \item L'utente visualizza la risposta generata da BuddyBot.
    \end{enumerate}
    \item \textbf{Sottocasi d'uso}:
    \begin{itemize}
        \item \bulhyperlink{UC5.1}{UC5.1}: Visualizzazione dell'introduzione della risposta;
        \item \bulhyperlink{UC5.2}{UC5.2}: Visualizzazione di uno snippet del contenuto che è stato trovato (es.: codice);
        \item \bulhyperlink{UC5.3}{UC5.3}: Visualizzazione di un messaggio di resoconto e conclusione.
    \end{itemize}
    \item \textbf{Scenario alternativo}:
    \begin{enumerate}
        \item \bulhyperlink{UC6}{UC6}: Visualizzazione di una risposta negativa (informazione non trovata).
    \end{enumerate}
\end{itemize}



\hypertarget{UC5.1}{}
\subsubsubsection{UC5.1: Visualizzazione dell'introduzione della risposta}

\begin{itemize}
    \item \textbf{Attori principali}: Utente;
    \item \textbf{Precondizioni}: BuddyBot ha generato correttamente la risposta alla domanda dell'utente;
    \item \textbf{Trigger}: L'utente desidera visualizzare l'introduzione della risposta alla domanda che ha posto;
    \item \textbf{Postcondizioni}: L'utente visualizza l'introduzione della risposta generata da BuddyBot;
    \item \textbf{Scenario principale}:
    \begin{enumerate}
        \item \bulhyperlink{UC3}{UC3}: BuddyBot ha generato correttamente la risposta alla domanda dell'utente;
        \item L'utente visualizza l'introduzione della risposta generata da BuddyBot.
    \end{enumerate}
\end{itemize}



\hypertarget{UC5.2}{}
\subsubsubsection{UC5.2: Visualizzazione di uno snippet del contenuto che è stato trovato (es.: codice)}

\begin{itemize}
    \item \textbf{Attori coinvolti}: Utente;
    \item \textbf{Precondizioni}: 
    \begin{itemize}
        \item L'utente ha visualizzato l'introduzione della risposta generata da BuddyBot;
        \item La risposta generata da BuddyBot contiene uno snippet di codice.
    \end{itemize}
    \item \textbf{Trigger}: L'utente desidera visualizzare lo snippet di codice contenuto nella risposta alla domanda che ha posto;
    \item \textbf{Postcondizioni}: L'utente visualizza uno snippet di codice contenuto della risposta generata da BuddyBot;
    \item \textbf{Scenario principale}:
    \begin{enumerate}
        \item \bulhyperlink{UC5.1}{UC5.1}: L'utente ha visualizzato l'introduzione della risposta generata da BuddyBot;
        \item L'utente visualizza uno snippet del contenuto che è stato trovato (es.: codice).
    \end{enumerate}
\end{itemize}



\hypertarget{UC5.3}{}
\subsubsubsection{UC5.3: Visualizzazione di un messaggio di resoconto e conclusione}

\begin{itemize}
    \item \textbf{Attori coinvolti}: Utente;
    \item \textbf{Precondizioni}: L'utente ha visualizzato almeno l'introduzione della risposta generata da BuddyBot;
    \item \textbf{Trigger}: L'utente desidera visualizzare la conclusione della risposta alla domanda che ha posto;
    \item \textbf{Postcondizioni}: L'utente visualizza un messaggio di resoconto e conclusione contenuto nella risposta generata da BuddyBot;
    \item \textbf{Scenario principale}:
    \begin{enumerate}
        \item \bulhyperlink{UC5.1}{UC5.1}: L'utente ha visualizzato l'introduzione della risposta generata da BuddyBot;
        \item \bulhyperlink{UC5.2}{UC5.2}: Se presente, l'utente ha visualizzato lo snippet di codice contenuto nella risposta;
        \item L'utente visualizza un messaggio di resoconto e conclusione che termina la risposta.
    \end{enumerate}
\end{itemize}



\hypertarget{UC6}{}
\subsubsection{UC6: Visualizzazione di una risposta negativa (informazione non trovata)}

\begin{itemize}
    \item \textbf{Attori coinvolti}: Utente;
    \item \textbf{Precondizioni}: BuddyBot non ha trovato nei documenti di contesto l'informazione che l'utente ha domandato;
    \item \textbf{Trigger}: L'utente desidera visualizzare la risposta alla domanda che ha posto;
    \item \textbf{Postcondizioni}: L'utente visualizza come risposta un messaggio di BuddyBot in cui gli viene segnalata la mancanza dell'informazione 
    richiesta nei dati forniti come contesto;
    \item \textbf{Scenario principale}:
    \begin{enumerate}
        \item BuddyBot non ha trovato nei documenti di contesto l'informazione che l'utente ha domandato;
        \item L'utente visualizza come risposta un messaggio di BuddyBot in cui gli viene segnalata la mancanza dell'informazione richiesta 
        nei dati forniti come contesto.
    \end{enumerate}
\end{itemize}






\hypertarget{UC8}{}
\subsubsection{UC8: Visualizzare una risposta generata con dati di contesto provenienti da 
\emph{GitHub}\textsubscript{\textbf{\textit{G}}}, \emph{Jira}\textsubscript{\textbf{\textit{G}}} e 
\emph{Confluence}\textsubscript{\textbf{\textit{G}}}}






\hypertarget{UC11}{}
\subsubsection{UC11: Visualizzazione una risposta di BuddyBot generata utilizzando dati di contesto aggiornati}

\begin{figure}[h]
    \centering
    \includegraphics[width=\textwidth]{placeholder.png}
    \caption{Visualizzazione una risposta di BuddyBot generata utilizzando dati di contesto aggiornati}
\end{figure}

\begin{itemize}
    \item \textbf{Attori principali}: Utente;
    \item \textbf{Precondizioni}: 
    \begin{itemize}
        \item L'utente ha inserito una interrogazione in linguaggio naturale;
        \item BuddyBot ha generato correttamente la risposta alla domanda dell'utente.
    \end{itemize}
    \item \textbf{Trigger}: L'utente desidera visualizzare una risposta basata su dati aggiornati alla domanda che ha posto;
    \item \textbf{Postcondizioni}: L'utente visualizza una risposta generata su dati di contesto aggiornati;
    \item \textbf{Scenario principale}:
    \begin{enumerate}
        \item \bulhyperlink{UC1}{UC1}: L'utente ha inserito una interrogazione in linguaggio naturale;
        \item \bulhyperlink{UC3}{UC3}: BuddyBot ha generato correttamente la risposta alla domanda dell'utente;
        \item La risposta è stata generata basandosi su dati aggiornati;
        \item \bulhyperlink{UC5}{UC5}: L'utente visualizza la risposta.
    \end{enumerate}
    \item \textbf{Sottocasi d'uso}:
    \begin{itemize}
        \item \bulhyperlink{UC11.1}{UC11.1}: Visualizzazione una risposta di BuddyBot generata utilizzando dati di contesto di 
        GitHub aggiornati;
        \item \bulhyperlink{UC11.2}{UC11.2}: Visualizzazione di una risposta di BuddyBot generata utilizzando dati di contesto di 
        Jira aggiornati;
        \item \bulhyperlink{UC11.3}{UC11.3}: Visualizzazione di una risposta di BuddyBot generata utilizzando dati di contesto di 
        Confluence aggiornati;
    \end{itemize}
    \item \textbf{Scenario alternativo}:
    \begin{enumerate}
        \item \bulhyperlink{UC17}{UC17}: Visualizzazione dell’avviso che i dati potrebbero essere obsoleti.
    \end{enumerate}
\end{itemize}



\hypertarget{UC11.1}{}
\subsubsubsection{UC11.1: Visualizzazione una risposta di BuddyBot generata utilizzando dati di contesto di GitHub aggiornati}

\begin{itemize}
    \item \textbf{Attori principali}: Utente, GitHub;
    \item \textbf{Precondizioni}: 
    \begin{itemize}
        \item L'utente ha inserito una interrogazione in linguaggio naturale che riguarda GitHub;
        \item BuddyBot ha generato correttamente la risposta alla domanda dell'utente.
    \end{itemize}
    \item \textbf{Trigger}: L'utente desidera visualizzare una risposta basata su dati di GitHub aggiornati alla domanda che ha posto;
    \item \textbf{Postcondizioni}: L'utente visualizza una risposta basata sui dati di contesto di GitHub aggiornati;
    \item \textbf{Scenario principale}: 
    \begin{enumerate}
        \item \bulhyperlink{UC1}{UC1}: L'utente ha inserito una interrogazione in linguaggio naturale;
        \item L'interrogazione in linguaggio naturale riguarda GitHub;
        \item \bulhyperlink{UC3}{UC3}: BuddyBot ha generato correttamente la risposta alla domanda dell'utente;
        \item La risposta è stata generata basandosi su dati aggiornati di GitHub;
        \item \bulhyperlink{UC5}{UC5}: L'utente visualizza la risposta.
    \end{enumerate}
\end{itemize}



\hypertarget{UC11.2}{}
\subsubsubsection{UC11.2: Visualizzazione di una risposta di BuddyBot generata utilizzando dati di contesto di Jira aggiornati}

\begin{itemize}
    \item \textbf{Attori principali}: Utente, Jira;
    \item \textbf{Precondizioni}: 
    \begin{itemize}
        \item L'utente ha inserito una interrogazione in linguaggio naturale che riguarda Jira;
        \item BuddyBot ha generato correttamente la risposta alla domanda dell'utente.
    \end{itemize}
    \item \textbf{Trigger}: L'utente desidera visualizzare una risposta basata su dati di Jira aggiornati alla domanda che ha posto;
    \item \textbf{Postcondizioni}: L'utente visualizza una risposta basata sui dati di contesto di Jira aggiornati;
    \item \textbf{Scenario principale}: 
    \begin{enumerate}
        \item \bulhyperlink{UC1}{UC1}: L'utente ha inserito una interrogazione in linguaggio naturale;
        \item L'interrogazione in linguaggio naturale riguarda Jira;
        \item \bulhyperlink{UC3}{UC3}: BuddyBot ha generato correttamente la risposta alla domanda dell'utente;
        \item La risposta è stata generata basandosi su dati aggiornati di Jira;
        \item \bulhyperlink{UC5}{UC5}: L'utente visualizza la risposta.
    \end{enumerate}
\end{itemize}



\hypertarget{UC11.3}{}
\subsubsubsection{UC11.3: Visualizzazione di una risposta di BuddyBot generata utilizzando dati di contesto di Confluence aggiornati}

\begin{itemize}
    \item \textbf{Attori principali}: Utente, Confluence;
    \item \textbf{Precondizioni}: 
    \begin{itemize}
        \item L'utente ha inserito una interrogazione in linguaggio naturale che riguarda Confluence;
        \item BuddyBot ha generato correttamente la risposta alla domanda dell'utente.
    \end{itemize}
    \item \textbf{Trigger}: L'utente desidera visualizzare una risposta basata su dati di Confluence aggiornati alla domanda che ha posto;
    \item \textbf{Postcondizioni}: L'utente visualizza una risposta basata sui dati di contesto di Confluence aggiornati;
    \item \textbf{Scenario principale}: 
    \begin{enumerate}
        \item \bulhyperlink{UC1}{UC1}: L'utente ha inserito una interrogazione in linguaggio naturale;
        \item L'interrogazione in linguaggio naturale riguarda Confluence;
        \item \bulhyperlink{UC3}{UC3}: BuddyBot ha generato correttamente la risposta alla domanda dell'utente;
        \item La risposta è stata generata basandosi su dati aggiornati di Confluence;
        \item \bulhyperlink{UC5}{UC5}: L'utente visualizza la risposta.
    \end{enumerate}
\end{itemize}



\hypertarget{UC17}{}
\subsubsection{UC17: Visualizzazione dell’avviso che i dati potrebbero essere obsoleti}

\begin{itemize}
    \item \textbf{Attori principali}: Utente;
    \item \textbf{Precondizioni}: 
    \begin{itemize}
        \item L'utente ha inserito una interrogazione in linguaggio naturale;
        \item BuddyBot ha generato correttamente la risposta alla domanda dell'utente.
    \end{itemize}
    \item \textbf{Trigger}: L'utente desidera visualizzare una risposta basata su dati aggiornati alla domanda che ha posto;
    \item \textbf{Postcondizioni}: L'utente visualizza la risposta del bot anticipata da una frase che segnala che la
    risposta potrebbe essere basata su dati obsoleti;
    \item \textbf{Scenario principale}: 
    \begin{enumerate}
        \item \bulhyperlink{UC1}{UC1}: L'utente ha inserito una interrogazione in linguaggio naturale;
        \item \bulhyperlink{UC3}{UC3}: BuddyBot ha generato correttamente la risposta alla domanda dell'utente;
        \item L'utente, prima della risposta, visualizza una frase che segnala che la risposta potrebbe essere basata su dati obsoleti;
        \item \bulhyperlink{UC5}{UC5}: L'utente visualizza la risposta.
    \end{enumerate}
\end{itemize}