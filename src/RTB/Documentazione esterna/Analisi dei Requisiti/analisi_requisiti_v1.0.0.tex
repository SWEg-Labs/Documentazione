% Configurazione
\documentclass{article}

\usepackage{titling} % Required for inserting the subtitle
\usepackage{graphicx} % Required for inserting images
\usepackage{tabularx} % Per l'ambiente tabularx (tabelle)
\usepackage{calc} % Sempre per le tabelle
\usepackage[hidelinks]{hyperref} % Per i collegamenti ipertestuali, ad esempio sulla table of contents
\usepackage[italian]{babel} % Per la lingua italiana nelle scritte automatiche
\usepackage[table]{xcolor} % Per colorare il testo e le celle delle tabelle
\usepackage{colortbl} % Per colorare le celle delle tabelle
\usepackage{lipsum} % Per generare lorem ipsum
\usepackage[normalem]{ulem} % Per sottolineare il testo
\usepackage{array} % Per la visualizzazione fluttuante di array di domande e risposte
\usepackage{ragged2e} % Pacchetto necessario per \justifying che giustifica il testo di tabelle
\usepackage{tikz} % Per spostare elementi nel documento in modo facile e veloce
\usepackage[a4paper, top=2.5cm, bottom=2.5cm, left=2.5cm, right=2.5cm]{geometry} % Per i margini della pagina
\usepackage{fancyhdr} % Per l'intestazione e il piè di pagina
\usepackage{amsmath} % Per scrivere formule matematiche, in particolare per il pedice G

\newcommand{\ulhref}[2]{\href{#1}{\uline{#2}}} % Nuovo comando per sottolineare i link
\newcommand{\ulref}[1]{\uline{\ref{#1}}} % Nuovo comando per sottolineare i collegamenti a immagini e tabelle
\setlength{\parindent}{0pt} % Rimuove il rientro automatico dei paragrafi
\usetikzlibrary{calc} % Libreria per il calcolo delle coordinate di TikZ
\pagestyle{fancy} % Stile della pagina, per l'intestazione e il piè di pagina
\renewcommand{\footrulewidth}{0.4pt} % Inserimento della linea orizzontale in basso
\setlength{\headsep}{1.4cm} % Spazio tra l'intestazione e il testo
\definecolor{lightgray}{gray}{0.95} % Definizione del colore grigio chiaro

\graphicspath{ {immagini/} {../../../shared/immagini/} }



% Struttura
\begin{document}

\thispagestyle{plain} % Niente intestazione e piè di pagina


\begin{tikzpicture}[remember picture, overlay]
    % Punto di partenza al centro orizzontale nella metà superiore
    \coordinate (top_center) at ($(current page.north)!0.3!(current page.south)$);

    % UniPD: Logo e descrizione
    \node at (top_center) [anchor=north, xshift=-3cm, yshift=4.85cm] 
        {\includegraphics[width=0.15\textwidth]{Logo Universita di Padova.png}};
    \node at (top_center) [anchor=north, xshift=1.7cm, yshift=4.5cm]
        {\textcolor{red}{\textbf{Università degli Studi di Padova}}};
    \node at (top_center) [anchor=north, xshift=1.7cm, yshift=4cm]
        {\textcolor{red}{Laurea: Informatica}};
    \node at (top_center) [anchor=north, xshift=1.7cm, yshift=3.5cm]
        {\textcolor{red}{Corso: Ingegneria del Software}};
    \node at (top_center) [anchor=north, xshift=1.7cm, yshift=3cm]
        {\textcolor{red}{Anno Accademico: 2024/2025}};

    % SWEg Labs: Logo e descrizione
    \node at (top_center) [anchor=north, xshift=-2.85cm, yshift=1.5cm] 
        {\includegraphics[width=0.16\textwidth]{Logo SWEg.png}};
    \node at (top_center) [anchor=north, xshift=1.7cm, yshift=0.5cm]
        {\textbf{Gruppo: SWEg Labs}};
    \node at (top_center) [anchor=north, xshift=1.7cm, yshift=0cm]
        {Email: \textsf{gruppo.sweg@gmail.com}};
\end{tikzpicture}


\vspace{10cm}

{
\centering
\Huge\bfseries Manuale Utente\par
\vspace{0.5cm}
\Large Versione 1.0.0\par
}

\vspace{2cm}
% Intestazione
\fancyhead[L]{1 \hspace{0.2cm} Informazioni generali} % Testo a sinistra

\pagenumbering{arabic} % Numerazione araba per il contenuto


\section{Informazioni generali}

\begin{itemize}
    \item \textbf{Tipo di riunione}: interna
    \item \textbf{Luogo}: meeting \emph{Discord}\textsubscript{\textit{\textbf{G}}}
    \item \textbf{Data}: 04/11/2024
    \item \textbf{Ora inizio}: 17:30
    \item \textbf{Ora fine}: 18:45
    \item \textbf{Responsabile}: Riccardo Stefani
    \item \textbf{Scriba}: Davide Verzotto
    \item \textbf{Partecipanti}:
    \begin{itemize}
        \item Federica Bolognini
        \item Michael Fantinato
        \item Giacomo Loat
        \item Filippo Righetto
        \item Riccardo Stefani
        \item Davide Verzotto
    \end{itemize}
\end{itemize}


\newpage
% Intestazione
\fancyhead[L]{Registro delle modifiche} % Testo a sinistra
\fancyhead[R]{\includegraphics[width=0.16\textwidth]{sweg_logo_sito_inverted.png}} % Immagine a destra

% Piè di pagina
\fancyfoot[L]{Analisi dei Requisiti}       % Testo a sinistra
\fancyfoot[C]{\thepage}                % Numero di pagina al centro
\fancyfoot[R]{Versione 1.0.0}          % Testo a destra

\pagenumbering{roman} % Numerazione romana per l'indice


\section*{Registro delle modifiche}

\begin{table}[h]
    \centering
    \rowcolors{2}{lightgray}{white}
    \begin{tabular}{|c|c|p{5cm}|p{3cm}|p{3cm}|}
        \hline
        \rowcolor[gray]{0.75}
        \textbf{Versione} & \textbf{Data} & \multicolumn{1}{|c|}{\textbf{Descrizione}} & 
        \multicolumn{1}{|c|}{\textbf{Autore}} & \multicolumn{1}{|c|}{\textbf{Verifica}}\\
        \hline
        1.0.0 & ... & Approvazione del documento & Filippo Righetto & Filippo Righetto\\
        \hline
        ... & ... & Verifica del documento & ... & ...\\
        \hline
        0.2.4 & 10-12-24 & Scrittura del caso d'uso \bulhyperlink{UC2}{UC2} & Giacomo Loat & Michael Fantinato \\
        \hline
        0.2.3 & 09-12-24 & Scrittura del caso d'uso \bulhyperlink{UC1}{UC1} & Federica Bolognini & ... \\
        \hline
        0.2.2 & 09-12-24 & Creazione del template per la trascrizione dei casi d'uso in \S\bulref{sec:casi_uso} & Riccardo Stefani & Federica Bolognini\\
        \hline
        0.2.1 & 08-12-24 & Scrittura dei casi d'uso \bulhyperlink{UC5}{UC5}, \bulhyperlink{UC6}{UC6}, \bulhyperlink{UC11}{UC11} e 
        \bulhyperlink{UC16}{UC16} & Riccardo Stefani & Giacomo Loat\\
        \hline
        0.2.0 & 06-12-24 & Verifica del documento allo stato attuale & Riccardo Stefani & Riccardo Stefani\\
        \hline
        0.1.2 & 18-11-24 & Inizio scrittura sezione \S\bulref{sec:Requisiti} & Filippo Righetto & Davide Verzotto\\
        \hline
        0.1.1 & 10-11-24 & Scrittura della sezione \S\bulref{sec:introduzione} di introduzione e della sezione \S\bulref{sec:descrizione_generale} 
        riguardante la descrizione generale & Filippo Righetto & Riccardo Stefani\\
        \hline
        0.1.0 & 05-11-24 & Creazione del documento & Riccardo Stefani & Giacomo Loat\\
        \hline
    \end{tabular}
    \caption{Registro delle modifiche}
\end{table}

\newpage
% Intestazione
\fancyhead[L]{Indice} % Testo a sinistra
\fancyhead[R]{\includegraphics[width=0.16\textwidth]{sweg_logo_sito_inverted.png}} % Immagine a destra

% Piè di pagina
\fancyfoot[L]{Verbale interno}       % Testo a sinistra
\fancyfoot[C]{\thepage}                % Numero di pagina al centro
\fancyfoot[R]{16/11/24}          % Testo a destra

\pagenumbering{roman} % Numerazione romana per l'indice


\tableofcontents
\newpage
% Intestazione
\fancyhead[L]{Elenco delle figure} % Testo a sinistra

\listoffigures


\newpage
% Intestazione
\fancyhead[L]{Elenco delle tabelle} % Testo a sinistra

\listoftables
\newpage

% Intestazione
\fancyhead[L]{1 \hspace{0.2cm} Introduzione} % Testo a sinistra

\pagenumbering{arabic} % Numerazione araba per il contenuto 


\section{Introduzione}
Questo documento è stato redatto con l'intento di offrire una trattazione esaustiva e dettagliata 
dei requisiti e dei casi d'uso individuati dal gruppo \textit{sweg labs} nel corso dello sviluppo
del progetto “BuddyBot”. La raccolta di questi dati è il frutto di un'analisi approfondita
del documento di presentazione del \textit{capitolato\textsubscript{G}}, di intense discussioni interne al gruppo di lavoro, 
nonchè di colloqui attivi con il \textit{proponente\textsubscript{G}}, \textit{Azzurrodigitale}.

L'obiettivo è garantire una comprensione completa ed accurata dei requisiti di progetto,
fornendo una base solida per la pianificazione e l'implementazione delle successive fasi di lavoro.

Nel documento adottiamo la sintassi \textit{UML\textsubscript{G}} al fine di formalizzare la rappresentazione e
renderla comprensibile a tutti i potenziali utenti. In particolare, i casi d'uso seguono una
struttura logica e vengono descritti in dettaglio attraverso i seguenti punti:
\begin{itemize}
    \item \textbf{Nominativo:} includiamo il titolo del \textit{caso d'uso\textsubscript{G}} e un breve commento esplicativo;
    \item \textbf{Attori Principali:} identifichiamo chi sono gli \textit{attori\textsubscript{G}} che eseguono le azioni all’interno 
                del caso d'uso;
    \item \textbf{Precondizioni:} specifichiamo lo stato del programma prima dell'esecuzione del caso d'uso;
    \item \textbf{Postcondizioni:} definiamo lo stato del programma dopo il completamento dello scenario del caso d'uso;
    \item \textbf{\textit{Scenario Principale\textsubscript{G}}:} descriviamo in modo dettagliato le azioni svolte durante
                l'esecuzione del caso d'uso, delineando il percorso seguito tra le condizioni iniziali e irisultati ottenuti;
    \item \textbf{Scenari alternativi:} descriviamo gli scenari che diramano dallo scenario principale o le situazioni nelle quali lo svolgimento delle 
                azioni dello scenario principale siaimpossibilitato dalla comparsa di condizioni di errore;
    \item \textbf{\textit{Sottocasi d'uso\textsubscript{G}}:} in alcune circostanze può essere necessaria la definizione di uno
                o più sottocasi d'uso, che andranno ad utilizzare la stessa struttura dei casi d'uso, e potranno essere 
                identificati mediante un numero progressivo nella forma:
                \begin{center}
                    X.Y
                \end{center}
    dove X `e il caso d'uso da cui derivano e Y un numero progressivo ad identificare il sottocaso.
    \item \textbf{Inclusioni:} descrivono funzionalità in comune fra più casi d'uso;
    \item \textbf{Specializzazioni:} possono essere di due tipologie:
    \begin{enumerate}
        \item di attori, dove i figli condividono tutte le funzionalit`a del padre e in pi`u ne
            possiedono di proprie;
        \item di casi d'uso, dove i figli possono aggiungere funzionalit`a rispetto ai padri o
            modificarne il comportamento.
    \end{enumerate}
\end{itemize}

\subsection{Scopo del prodotto}
Nel corso dell'ultimo anno si è verificato un repentino e significativo mutamento nel panorama
dello sviluppo e nell'implementazione dell'\textit{Intelligenza Artificiale\textsubscript{G}}.
Questa trasformazione ha interessato diverse sfaccettature della tecnologia, e si è verificata con il passaggio da un
ruolo prevalentemente incentrato sull'elaborazione e sulla raccomandazione dei contenuti ad
una fase in cui l'Intelligenza Artificiale assume attivamente la responsabilità di generare
contenuti originali. Questa nuova fase ha visto l'emergere di sistemi in grado di creare non
solo testi, ma anche immagini e tracce audio con un livello di sofisticazione che sfida le
precedenti aspettative. \\
Il capitolato\textsubscript{G} C9, 'BuddyBot,' ha come obiettivo la realizzazione di un assistente virtuale (chatbot) 
capace di raccogliere rapidamente informazioni dalle fonti indicate e di fornirle in risposta a domande poste in 
linguaggio naturale tramite chat.\\
Tale assistente virtuale sarà fruibile attraverso una piccola piattaforma web, dove l'utente potrà interagire con l'IA 
per ottenere le risposte desiderate.

\subsection{Glossario}
Al fine di evitare possibili ambiguità relative al linguaggio utilizzato nei documenti, viene fornito un \textit{Glossario}
(attualmente alla sua versione \textit{1.0.0}), nel quale sono contenute le definizioni di termini complessi o aventi uno 
specifico significato. Tali termini, ove necessario, sono segnati in corsivo e marcati con il simbolo G a pedice
(\textit{esempio Way of Working\textsubscript{G}}).

\subsection{Miglioramenti al documento}
La maturità e i miglioramenti sono aspetti fondamentali nella stesura di un documento.
Questo permette di apportare agevolmente modifiche in base alle esigenze concordate tra i
membri del gruppo e il \textit{proponente\textsubscript{G}} nel corso del tempo. Di conseguenza, questa versione del
documento non pu`o essere considerata definitiva o completa, poichè è soggetta a evoluzioni future.

\subsection{Riferimenti}
\subsubsection{Riferimenti normativi}
\begin{itemize}
    \item \href{https://www.sito2.com}{Norme di Progetto (manca link)}
    \item \href{https://github.com/SWEg-Labs/Documentazione/blob/adea4950d9135916b22ef3af717e955f2c11f975/output/RTB/Documentazione%20esterna/piano_qualifica_v1.0.0.pdf}{Piano di qualifica (v 1.0.0)}
    \item \href{https://www.math.unipd.it/~tullio/IS-1/2024/Progetto/C9.pdf}{Capitolato d'appalto C9 - BuddyBot}
    \item \href{https://www.math.unipd.it/~tullio/IS-1/2024/Dispense/PD1.pdf}{Slide PD1 del corso di Ingegneria del Software - Regolamento del Progetto Didattico}
  \end{itemize}

\subsubsection{Riferimenti informativi}
\begin{itemize}
    \item \href{https://github.com/SWEg-Labs/Documentazione/blob/adea4950d9135916b22ef3af717e955f2c11f975/output/RTB/Documentazione%20esterna/glossario_v1.0.0.pdf}{Glossario (v 1.0.0)}
    \item \href{https://github.com/SWEg-Labs/Documentazione/tree/adea4950d9135916b22ef3af717e955f2c11f975/output/RTB/Documentazione%20interna/Verbali%20interni}{Verbali interni}
    \item \href{https://github.com/SWEg-Labs/Documentazione/tree/adea4950d9135916b22ef3af717e955f2c11f975/output/RTB/Documentazione%20esterna/Verbali%20esterni}{Verbali esterni}
    \item \href{https://www.math.unipd.it/~tullio/IS-1/2024/Dispense/T05.pdf}{Slide T05 del corso di Ingegneria del Software - Analisi dei Requisiti}
    \item \href{https://www.sito3.com}{Diagrammi dei casi d'uso (non ho trovato il documento)}
  \end{itemize}

\newpage
% Intestazione
\fancyhead[L]{2 \hspace{0.2cm} Descrizione generale} % Testo a sinistra
 
\pagenumbering{arabic} % Numerazione araba per il contenuto


\section{Descrizione generale}

\subsection{Obiettivi del prodotto}
L'obiettivo del prodotto è sviluppare "BuddyBot," un assistente virtuale che utilizza l'intelligenza artificiale 
per rispondere in modo efficiente e accurato a domande poste in linguaggio naturale. BuddyBot dovrà essere in grado 
di reperire informazioni specifiche da fonti designate, rendendole disponibili agli utenti tramite una chat intuitiva. 
La piattaforma web associata consentirà agli utenti di interagire con il sistema in modo semplice e immediato, migliorando
l'accesso alle informazioni e l'efficacia del supporto fornito.

\subsection{Funzioni del prodotto}


\subsection{Caratteristiche degli utenti}
\subsection{Piattaforma di esecuzione}
Il prodotto si presenter`a sotto forma di applicazione webG e sar`a consultabile dalla maggior
parte dei \textit{browser\textsubscript{G}}.

\end{document}