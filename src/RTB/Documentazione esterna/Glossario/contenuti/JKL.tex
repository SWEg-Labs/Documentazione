% Intestazione
\fancyhead[L]{J} % Testo a sinistra

\section*{J}
\addcontentsline{toc}{section}{J}

\subsection*{Jira}
Jira è uno strumento di gestione dei progetti e di issue tracking sviluppato da Atlassian, ampiamente utilizzato per la pianificazione, il monitoraggio e 
il controllo di progetti, in particolare nell'ambito dello sviluppo software. Nato come strumento di gestione dei bug e dei problemi, Jira è diventato uno 
dei principali strumenti per il project management, soprattutto per le organizzazioni che adottano metodologie agili come Scrum e Kanban.

\newpage


% Intestazione
\fancyhead[L]{K} % Testo a sinistra

\section*{K}
\addcontentsline{toc}{section}{K}

\dots

\newpage


% Intestazione
\fancyhead[L]{L} % Testo a sinistra

\section*{L}
\addcontentsline{toc}{section}{L}

\subsection*{\LaTeX}
Linguaggio di marcatura per la preparazione di testi, basato sul programma di composizione tipografica TeX. LaTeX è ampiamente utilizzato per la creazione 
di documenti scientifici e tecnici grazie alla sua capacità di gestire formule matematiche complesse e alla sua alta qualità tipografica.

\newpage