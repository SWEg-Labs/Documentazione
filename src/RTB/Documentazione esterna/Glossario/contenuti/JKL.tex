% Intestazione
\fancyhead[L]{J} % Testo a sinistra

\section{}

\subsection*{Jira}
Jira è uno strumento di gestione dei progetti e di issue tracking sviluppato da Atlassian, ampiamente utilizzato per la pianificazione, il monitoraggio e 
il controllo di progetti, in particolare nell'ambito dello sviluppo software. Nato come strumento di gestione dei bug e dei problemi, Jira è diventato uno 
dei principali strumenti per il project management, soprattutto per le organizzazioni che adottano metodologie agili come Scrum e Kanban.

\newpage


% Intestazione
\fancyhead[L]{K} % Testo a sinistra

\section{}

\dots

\newpage


% Intestazione
\fancyhead[L]{L} % Testo a sinistra

\section{}

\hypertarget{sec:langchain}{}
\subsection*{LangChain}
LangChain è un framework Python e JavaScript open-source progettato per sviluppare applicazioni che sfruttano i Large Language Models 
(LLM). Offre una struttura modulare e flessibile per la creazione di catene di elaborazione (chains) che combinano modelli di 
linguaggio con altre fonti di dati e strumenti.

\subsection*{\LaTeX}
Linguaggio di marcatura per la preparazione di testi, basato sul programma di composizione tipografica TeX. LaTeX è ampiamente utilizzato per la creazione 
di documenti scientifici e tecnici grazie alla sua capacità di gestire formule matematiche complesse e alla sua alta qualità tipografica.

\hypertarget{sec:LLM}{}
\subsection*{Large Language Model (LLM)}
Il termine Large Language Model si riferisce a modelli di linguaggio avanzati e complessi che sono stati addestrati su enormi quantità di dati testuali. 
Questi modelli, basati su tecniche di intelligenza artificiale come il deep learning, sono in grado di comprendere e generare testo in linguaggio naturale 
in modo più sofisticato rispetto a modelli più piccoli (Small Language Model, SML).

\newpage