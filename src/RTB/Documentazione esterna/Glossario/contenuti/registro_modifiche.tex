% Intestazione
\fancyhead[L]{Registro delle modifiche} % Testo a sinistra
\fancyhead[R]{\includegraphics[width=0.16\textwidth]{sweg_logo_sito_inverted.png}} % Immagine a destra

% Piè di pagina
\fancyfoot[L]{Glossario}       % Testo a sinistra
\fancyfoot[C]{\thepage}                % Numero di pagina al centro
\fancyfoot[R]{Versione 1.0.0}          % Testo a destra

\pagenumbering{roman} % Numerazione romana per l'indice


\section*{Registro delle modifiche}

\begin{table}[h]
    \centering
    \rowcolors{2}{lightgray}{white}
    \begin{tabular}{|c|c|p{5cm}|p{3cm}|p{3cm}|}
        \hline
        \rowcolor[gray]{0.75}
        \textbf{Versione} & \textbf{Data} & \multicolumn{1}{|c|}{\textbf{Descrizione}} & 
        \multicolumn{1}{|c|}{\textbf{Autore}} & \multicolumn{1}{|c|}{\textbf{Verifica}}\\
        \hline
        1.0.0 & ... & Approvazione del documento & ... & ...\\
        \hline
        ... & ... & Verifica del documento & ... & ...\\
        \hline
        ... & ... & ... & ... & ...\\
        \hline
        0.2.0 & ... & Verifica del documento allo stato attuale & ... & ...\\
        \hline
        0.1.12 & 18-11-24 & Aggiunto il termine \bulhyperlink{sec:progettazione}{Progettazione} & Filippo Righetto & Federica Bolognini\\
        \hline
        0.1.11 & 17-11-24 & Aggiunti i termini \bulhyperlink{sec:specifiche_funzionali}{Specifiche funzionali}, \bulhyperlink{sec:specifiche_tecniche}{Specifiche tecniche},
        \bulhyperlink{sec:contesto_applicativo}{Contesto applicativo}, \bulhyperlink{sec:architettura}{Architettura}, \bulhyperlink{sec:accoppiamento}{Accoppiamento},
        \bulhyperlink{sec:pattern_architetturale}{Pattern architetturale}, \bulhyperlink{sec:pull_request}{Pull request}, \bulhyperlink{sec:build}{Build}, 
        \bulhyperlink{sec:feedbacl}{Feedback}, \bulhyperlink{sec:google_meet}{Google Meet} e \bulhyperlink{sec:issue_tracking_system}{Issue Tracking System (ITS)}
        & Giacomo Loat & Federica Bolognini\\
        \hline
    \end{tabular}
\end{table}

\newpage

\begin{table}[h]
    \centering
    \rowcolors{2}{lightgray}{white}
    \begin{tabular}{|c|c|p{5cm}|p{3cm}|p{3cm}|}
        \hline
        \rowcolor[gray]{0.75}
        \textbf{Versione} & \textbf{Data} & \multicolumn{1}{|c|}{\textbf{Descrizione}} & 
        \multicolumn{1}{|c|}{\textbf{Autore}} & \multicolumn{1}{|c|}{\textbf{Verifica}}\\
        \hline
        0.1.10 & 16-11-24 & Aggiunti i termini \bulhyperlink{sec:codifica}{Codifica}, \bulhyperlink{ISO/IEC 9126}{ISO/IEC 9126}, 
        \bulhyperlink{sec:efficienza}{Efficienza}, \bulhyperlink{sec:efficacia}{Efficacia}, \bulhyperlink{sec:prodotto_software}{Prodotto software} 
        e \bulhyperlink{sec:test}{Test} & Riccardo Stefani & Federica Bolognini\\
        \hline
        0.1.9 & 16-11-24 & Aggiunti i termini \bulhyperlink{sec:angular}{Angular}, \bulhyperlink{sec:nestjs}{NestJS}, \bulhyperlink{sec:nodejs}{Node.js}, 
        \bulhyperlink{sec:spring}{Spring}, \bulhyperlink{sec:spring_boot}{Spring Boot} e \bulhyperlink{sec:telegram}{Telegram} 
        & Riccardo Stefani & Michael Fantinato\\
        \hline
        0.1.8 & 13-11-24 & Aggiunti i termini \bulhyperlink{sec:metrica}{Metrica} & Michael Fantinato & Filippo Righetto\\
        \hline
        0.1.9 & 13-11-24 & Aggiunto il termine \bulhyperlink{ISO/IEC/IEEE 12207}{ISO/IEC/IEEE 12207} & Federica Bolognini & Filippo Righetto\\
        \hline
        0.1.8 & 13-11-24 & Aggiunti i termini \bulhyperlink{sec:demo}{Demo} e \bulhyperlink{sec:snippet}{Snippet} & Riccardo Stefani & Filippo Righetto\\
        \hline
        0.1.7 & 12-11-24 & Modificato il termine \bulhyperlink{sec:processo}{Processo} & Giacomo Loat & Riccardo Stefani\\
        \hline
        0.1.6 & 12-11-24 & Aggiunti i termini \bulhyperlink{sec:diario_di_bordo}{Diario di Bordo}, \bulhyperlink{sec:modello_agile}{Agile (modello di sviluppo)},
        \bulhyperlink{sec:modello_incrementale}{Incrementale (modello di sviluppo)}, \bulhyperlink{sec:diagramma_di_burndown}{Burndown (diagramma di)}, \bulhyperlink{sec:preventivo}{Preventivo}
        & Giacomo Loat & Riccardo Stefani\\
        \hline
        0.1.5 & 11-11-24 & Aggiunti i termini \bulhyperlink{sec:capitolato}{Capitolato}, \bulhyperlink{sec:caso_uso}{Caso d'uso}, \bulhyperlink{sec:attore}{Attore},
        \bulhyperlink{sec:scenario_uso}{Scenario d'uso}, \bulhyperlink{sec:intelligenza_artificiale}{Intelligenza artificiale}, \bulhyperlink{sec:ia}{IA}, 
        \bulhyperlink{sec:api}{API}, \bulhyperlink{sec:applicazione_web}{Applicazione web}, \bulhyperlink{sec:browser}{Browser}, \bulhyperlink{sec:confluence}{Confluence},
        \bulhyperlink{sec:slack}{Slack} e \bulhyperlink{sec:onboarding}{Onboarding} & Filippo Righetto & Riccardo Stefani\\
        \hline
        0.1.4 & 11-11-24 & Aggiunto il termine \bulhyperlink{ISO/IEC 31000:2018}{ISO/IEC 31000:2018} & Federica Bolognini & Michael Fantinato\\
        \hline
        0.1.3 & 10-11-24 & Aggiunti i termini \bulhyperlink{sec:baseline}{Baseline}, \bulhyperlink{sec:branch}{Branch}, \bulhyperlink{sec:feature}{Feature}, 
        \bulhyperlink{sec:git}{Git}, \bulhyperlink{sec:ide}{IDE}, \bulhyperlink{sec:PB}{Product Baseline (PB)}, \bulhyperlink{sec:proponente}{Proponente}, 
        \bulhyperlink{sec:PoC}{Proof of Concept (PoC)}, \bulhyperlink{sec:merge}{Merge}, \bulhyperlink{sec:MVP}{Minimum Viable Product (MVP)}, 
        \bulhyperlink{sec:snake_case}{Snake Case} e \bulhyperlink{sec:VSC}{Visual Studio Code (VSC)} & Riccardo Stefani & Michael Fantinato\\
        \hline
        0.1.2 & 09-11-24 & Aggiunti i termini \bulhyperlink{sec:ciclo_di_vita}{Ciclo di vita del software},
        \bulhyperlink{sec:processo}{Processo}, \bulhyperlink{sec:verifica}{Verifica} e \bulhyperlink{sec:way_of_working}{Way of Working} 
        & Riccardo Stefani & Giacomo Loat\\
        \hline
        0.1.1 & 08-11-24 & Inserimento dei primi termini, tratti dai primi due verbali interni della RTB & Riccardo Stefani & Filippo Righetto\\
        \hline
        0.1.0 & 05-11-24 & Creazione del documento & Riccardo Stefani & Filippo Righetto\\
        \hline
    \end{tabular}
    \caption{Registro delle modifiche}
\end{table}
