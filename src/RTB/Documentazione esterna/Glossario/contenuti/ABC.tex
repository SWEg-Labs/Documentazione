\appendix % Cambia la numerazione delle sezioni da numeri a lettere


% Intestazione
\fancyhead[L]{A} % Testo a sinistra

\section{}
%\addcontentsline{toc}{section}{A}

\subsection*{Analisi dei Requisiti}
Processo fondamentale dello sviluppo di un prodotto software che si concentra sulla raccolta, analisi e definizione delle necessità e delle aspettative 
degli utenti finali, degli stakeholder e del sistema nel suo complesso. Questo processo mira a comprendere e documentare in modo chiaro e completo le 
esigenze, le funzionalità, le prestazioni e i vincoli che il sistema deve soddisfare. L’obiettivo principale dell’analisi dei requisiti è fornire una 
base solida per tutte le fasi successive dello sviluppo del software, assicurando che il prodotto finale soddisfi le esigenze degli utenti e raggiunga 
gli obiettivi del progetto.

\subsection*{Analisi dei Rischi}
L'analisi dei rischi è il processo di identificazione, valutazione e priorizzazione dei rischi in un progetto, sistema o attività, al fine di ridurre 
o gestire il loro impatto potenziale. Viene utilizzata per prevedere gli eventi negativi che potrebbero influenzare il successo di un progetto e per 
determinare le azioni preventive o correttive da intraprendere.

\subsection*{AzzurroDigitale}
AzzurroDigitale è una società italiana con sede a Padova. Si occupa di digitalizzazione di processi sia con prodotti proprietari che di terze parti, e 
ha l’obiettivo di accompagnare le aziende manifatturiere nella transizione 5.0.

\newpage


% Intestazione
\fancyhead[L]{B} % Testo a sinistra

\section{}

\subsection*{Backlog}
Insieme di compiti/attività da completare per un certo obiettivo. All’interno del framework Scrum, ne esistono due tipi principali: il product backlog, 
che è la lista delle funzionalità da implementare, e lo sprint backlog, che contiene le attività da svolgere durante un particolare sprint.
Un’attività interna al backlog porta valore ad un progetto perchè possiede:
\begin{itemize}
    \item Stato, che segnala se l’attività è stata completata, in corso o non ancora iniziata.
    \item Priorità, che indica l’importanza dell’attività rispetto alle altre.
    \item Assegnatario, cioè una persona incaricata a svolgere l’attività. Questa assegnazione non è vincolante, infatti se un membro del team ha terminato 
    la sua attività può prendersi a carico un’altra attività presente nel backlog anche se non era stata inizialmente assegnata a lui.
    \item Scadenza, cioè un termine temporale entro il quale l’attività deve essere svolta.
\end{itemize}

\subsection*{Best Practices}
Nello sviluppo software, metodologie che attraverso l’esperienza e la sperimentazione sono state identificate come modi efficaci e raccomandati di 
affrontare determinati problemi o compiti nel processo di sviluppo del software. Queste pratiche sono considerate migliori (best) perché hanno dimostrato 
di portare a risultati di alta qualità, facilitando la manutenzione del codice e promuovendo una migliore collaborazione nel team di sviluppo.

\newpage


% Intestazione

\fancyhead[L]{C} % Testo a sinistra

\section{}

\subsection*{Checklist}
Lista dettagliata di elementi, attività o criteri specifici che devono essere controllati, esaminati o completati durante le diverse fasi del ciclo di vita 
del software. E' utilizzata come strumento di controllo e verifica.

\hypertarget{sec:ciclo_di_vita}{}
\subsection*{Ciclo di vita del software}
Serie di fasi attraverso le quali un software passa dal suo concepimento iniziale fino al suo ritiro o dismissione. È un concetto chiave nell’ingegneria 
del software e fornisce una struttura organizzativa per il processo di sviluppo del software.

\newpage