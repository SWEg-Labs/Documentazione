% Intestazione
\fancyhead[L]{P} % Testo a sinistra

\section*{P}
\addcontentsline{toc}{section}{P}

\subsection*{Piano di Progetto}
Documento formale che delinea in dettaglio la pianificazione, l’esecuzione, il monitoraggio e il controllo di tutte le attività coinvolte nella 
realizzazione di un progetto. Questo documento fornisce una roadmap chiara e organizzata, comprensiva di obiettivi, risorse, scadenze e strategie di 
gestione dei rischi. Essenziale per la gestione efficace di un progetto, il piano di progetto serve come guida per il team di lavoro e gli stakeholder, 
fornendo una struttura che facilita il coordinamento delle attività e l’assegnazione delle risorse.

\subsection*{Piano di Qualifica}
Documento che stabilisce gli standard di qualità, i processi e le attività di testing che saranno implementati durante lo sviluppo di un progetto. 
Contiene una descrizione dettagliata delle strategie di testing, delle metriche di valutazione e dei criteri di accettazione del prodotto finale. 
L’obiettivo principale del Piano di Qualifica è garantire che il prodotto soddisfi gli standard di qualità prefissati e che il processo di sviluppo 
segua procedure coerenti ed efficaci.

\newpage


% Intestazione
\fancyhead[L]{Q} % Testo a sinistra

\section*{Q}
\addcontentsline{toc}{section}{Q}

\dots

\newpage


% Intestazione
\fancyhead[L]{R} % Testo a sinistra

\section*{R}
\addcontentsline{toc}{section}{R}

\subsection*{Repository}
In termini informatici, un luogo o un archivio dove vengono conservati e gestiti dati, documenti o, nel contesto del software, il codice sorgente di un 
progetto. Nell'ambito dei sistemi di controllo delle versioni come Git, un repository è una struttura dati che archivia anche la cronologia completa delle 
modifiche apportate al codice sorgente di un progetto.

\subsection*{Requirement and Technology Retrospective (RTB)}
La prima revisione di avanzamento del progetto didattico. Fissa i requisiti da soddisfare in accordo con il proponente, motiva le tecnologie, i framework 
e le librerie adottate dimostrandone adeguatezza e compatibilità tramite il Proof of Concept (PoC).

\subsection*{Retrospettiva}
Vedi \emph{Sprint Retrospective}\textsubscript{\textit{\textbf{G}}}.

\newpage