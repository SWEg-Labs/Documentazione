% Intestazione
\fancyhead[L]{P} % Testo a sinistra

\section{}

\hypertarget{sec:pattern_architetturale}{Pattern architetturale}
\subsection*{Pattern architetturale}
Soluzione generale e riutilizzabile a un problema che si verifica comunemente nell’ambito dello sviluppo del software. I pattern 
architetturali forniscono un approccio testato e comprovato per la progettazione e l’implementazione del software.


\subsection*{Piano di Progetto}
Documento formale che delinea in dettaglio la pianificazione, l’esecuzione, il monitoraggio e il controllo di tutte le attività coinvolte nella 
realizzazione di un progetto. Questo documento fornisce una roadmap chiara e organizzata, comprensiva di obiettivi, risorse, scadenze e strategie di 
gestione dei rischi. Essenziale per la gestione efficace di un progetto, il piano di progetto serve come guida per il team di lavoro e gli stakeholder, 
fornendo una struttura che facilita il coordinamento delle attività e l’assegnazione delle risorse.

\subsection*{Piano di Qualifica}
Documento che stabilisce gli standard di qualità, i processi e le attività di testing che saranno implementati durante lo sviluppo di un progetto. 
Contiene una descrizione dettagliata delle strategie di testing, delle metriche di valutazione e dei criteri di accettazione del prodotto finale. 
L’obiettivo principale del Piano di Qualifica è garantire che il prodotto soddisfi gli standard di qualità prefissati e che il processo di sviluppo 
segua procedure coerenti ed efficaci.

\hypertarget{sec:preventivo}{Preventivo}
\subsection*{Preventivo}
Stima dei costi e delle risorse necessarie per completare un determinato lavoro o progetto.

\hypertarget{sec:processo}{Processo}
\subsection*{Processo}
Insieme di attività correlate e coese che trasformano ingressi (bisogni) in uscite (prodotti) secondo regole date,
consumando risorse nel farlo.

\hypertarget{sec:prodotto_software}{}
\subsection*{Prodotto software}
Un prodotto software è un’applicazione, un sistema o un programma
informatico risultante dal processo di sviluppo del software. In altre parole, è il risultato
tangibile e funzionante di attività di progettazione, sviluppo, test e manutenzione svolte dal
team di sviluppo.

\hypertarget{sec:PB}{}
\subsection*{Product Baseline (PB)}
E la seconda revisione di avanzamento del progetto didattico. Comprende un prodotto software con design definitivo, 
chiamato \emph{Minimum Viable Product (MVP)}\textsubscript{\textit{\textbf{G}}}.

\hypertarget{sec:PoC}{}
\subsection*{Proof of Concept (PoC)}
Versione preliminare di un’applicazione o di una soluzione software che viene sviluppata per dimostrare la fattibilità tecnica di un’idea o di un concetto. 
Viene utilizzata per testare rapidamente l’efficacia di un approccio, identificare eventuali limitazioni delle tecnologie scelte e valutare se l’idea può 
essere realizzata in modo pratico.

\hypertarget{sec:proponente}{}
\subsection*{Proponente}
Nel contesto dell’ingegneria del software, colui che presenta un’idea, un progetto o una proposta e ne sostiene la realizzazione. Il gruppo \emph{SWEg Labs} 
ha come proponente l’azienda \emph{AzzurroDigitale}\textsubscript{\textit{\textbf{G}}}.

\hypertarget{sec:pull_request}{Pull request}
\subsection*{Pull request}
Nel sistema di versionamento integrato da GitHub, operazione che permette di proporre modifiche al codice sorgente di un progetto. 
Una pull request consente agli sviluppatori di richiedere l’integrazione delle loro modifiche nel branch principale del repository, 
facilitando la revisione e la collaborazione.

\newpage


% Intestazione
\fancyhead[L]{Q} % Testo a sinistra

\section{}

\hypertarget{sec:Qualità}{}
\subsection*{Qualità}
Insieme delle caratteristiche di un’entità che ne determinano la capacità di soddisfare esigenze sia espresse che implicite. Si parla di qualità del prodotto software in termini di:
\begin{itemize}
    \item \textbf{Qualità Intrinseca}: conformità ai requisiti, idoneità all’uso;
    \item \textbf{Qualità Relativa}: soddisfazione del cliente;
    \item \textbf{Qualità Quantitativa}: misurazione oggettiva, per confronto.
\end{itemize}


\newpage


% Intestazione
\fancyhead[L]{R} % Testo a sinistra

\section{}

\subsection*{Repository}
In termini informatici, un luogo o un archivio dove vengono conservati e gestiti dati, documenti o, nel contesto del software, il codice sorgente di un 
progetto. Nell'ambito dei sistemi di controllo delle versioni come Git, un repository è una struttura dati che archivia anche la cronologia completa delle 
modifiche apportate al codice sorgente di un progetto.

\subsection*{Requirement and Technology Retrospective (RTB)}
La prima revisione di avanzamento del progetto didattico. Fissa i requisiti da soddisfare in accordo con il proponente, motiva le tecnologie, i framework 
e le librerie adottate dimostrandone adeguatezza e compatibilità tramite il Proof of Concept (PoC).

\subsection*{Retrospettiva}
Vedi \emph{Sprint Retrospective}\textsubscript{\textit{\textbf{G}}}.

\newpage