% Intestazione
\fancyhead[L]{S} % Testo a sinistra

\section{}

\hypertarget{sec:scenario_uso}{}
\subsection*{Scenario d'uso}
Uno scenario d'uso è una descrizione narrativa di come un utente interagisce con un sistema software per raggiungere un determinato obiettivo. Rappresenta 
una sequenza di azioni e reazioni tra l'utente e il sistema in un contesto specifico e fornisce una visione dettagliata del comportamento atteso del software 
in una situazione reale.

\subsection*{Script}
Uno script è un insieme di istruzioni scritte in un linguaggio di scripting (come Python, JavaScript, Bash, ecc.) progettato per automatizzare compiti 
specifici o per eseguire operazioni in modo diretto su un sistema, applicazione o ambiente web. A differenza di un programma complesso, uno script tende 
a essere più leggero e adatto per operazioni mirate e ripetibili, senza richiedere una fase di compilazione lunga o complessa.

\hypertarget{sec:slack}{}
\subsection*{Slack}
Piattaforma di messaggistica istantanea e collaborazione progettata per team e organizzazioni. Permette di comunicare tramite canali tematici, inviare 
messaggi diretti, condividere file, e integrare altre applicazioni e strumenti di lavoro. Slack favorisce la comunicazione in tempo reale, migliorando 
la produttività e la gestione dei progetti, e viene utilizzato soprattutto in ambienti di lavoro agili e collaborativi.

\hypertarget{sec:snake_case}{}
\subsection*{Snake Case}
Lo Snake Case è uno stile di scrittura di nomi in cui le parole sono separate da un trattino basso o underscore (\_) e tutte le lettere sono generalmente 
minuscole (es.: nome\_variabile\_uno). Viene spesso contrapposto allo stile Camel Case, in cui invece le parole successive alla prima vengono concatenate 
con l’iniziale maiuscola (es.: nomeVariabileUno).

\hypertarget{sec:snippet}{}
\subsection*{Snippet}
Uno snippet è un piccolo blocco di codice riutilizzabile che svolge una funzione specifica o rappresenta un esempio di utilizzo di una determinata 
funzionalità. Gli snippet sono utilizzati per risparmiare tempo durante la scrittura del codice, poiché forniscono pezzi di codice già pronti che possono 
essere facilmente inseriti e adattati nel contesto del progetto.

\subsection*{Sprint}
Periodo di tempo prefissato entro il quale lavorare producendo dei risultati documentati. Gli sprint sono al centro delle metodologie Agile, atte a produrre 
risultati piccoli e in maniera costante.

\subsection*{Sprint Retrospective}
Incontro finalizzato ad analizzare l’andamento dello sprint quando questo è terminato, per migliorare la performance futura del team di sviluppo. La 
riunione retrospettiva dello sprint è quindi propedeutica allo sprint successivo.

\newpage


% Intestazione
\fancyhead[L]{T} % Testo a sinistra

\section{}

\subsection*{Tracciamento}
Il tracciamento è il processo di monitoraggio e documentazione dell'evoluzione e delle modifiche di un sistema, dei suoi requisiti e dei suoi componenti, 
all’interno di un progetto. Questo processo è essenziale per mantenere una visione chiara dello stato attuale del sistema, delle sue modifiche, e delle 
relazioni tra diversi elementi (come requisiti, attività di design, e componenti di codice).

\newpage


% Intestazione
\fancyhead[L]{U} % Testo a sinistra

\section{}

\dots

\newpage


% Intestazione
\fancyhead[L]{V} % Testo a sinistra

\section{}

\hypertarget{sec:verifica}{}
\subsection*{Verifica}
Processo che include un insieme di attività volte a garantire che il lavoro svolto durante lo sviluppo del software rispetti gli standard, i requisiti e 
le aspettative stabilite. La verifica è essenziale per garantire che il software sia di alta qualità, risponda alle esigenze degli utenti e riduca il 
rischio di errori e difetti. Si svolge a periodi regolari durante il corso del progetto.

\subsection*{Versionamento}
Processo che realizza il cosiddetto “controllo di versione”, stabilendo la storia cronologica delle azioni fatte per una certa attività, tracciando i 
cambiamenti occorsi e permettendo di tornare a uno stadio precedente qualora necessario.

\hypertarget{sec:VSC}{}
\subsection*{Visual Studio Code (VSC)}
\emph{IDE}\textsubscript{\textit{\textbf{G}}} libero e gratuito sviluppato da Microsoft per Windows, Linux e macOS. Permette sia di scrivere codice 
sorgente per un prodotto software sia di scrivere testo di documentazione.

\newpage