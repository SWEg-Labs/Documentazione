% Intestazione
\fancyhead[L]{S} % Testo a sinistra

\section{}

\subsection*{Script}
Uno script è un insieme di istruzioni scritte in un linguaggio di scripting (come Python, JavaScript, Bash, ecc.) progettato per automatizzare compiti 
specifici o per eseguire operazioni in modo diretto su un sistema, applicazione o ambiente web. A differenza di un programma complesso, uno script tende 
a essere più leggero e adatto per operazioni mirate e ripetibili, senza richiedere una fase di compilazione lunga o complessa.

\subsection*{Sprint}
Periodo di tempo prefissato entro il quale lavorare producendo dei risultati documentati. Gli sprint sono al centro delle metodologie Agile, atte a produrre 
risultati piccoli e in maniera costante.

\subsection*{Sprint Retrospective}
Incontro finalizzato ad analizzare l’andamento dello sprint quando questo è terminato, per migliorare la performance futura del team di sviluppo. La 
riunione retrospettiva dello sprint è quindi propedeutica allo sprint successivo.

\newpage


% Intestazione
\fancyhead[L]{T} % Testo a sinistra

\section{}

\subsection*{Tracciamento}
Il tracciamento è il processo di monitoraggio e documentazione dell'evoluzione e delle modifiche di un sistema, dei suoi requisiti e dei suoi componenti, 
all’interno di un progetto. Questo processo è essenziale per mantenere una visione chiara dello stato attuale del sistema, delle sue modifiche, e delle 
relazioni tra diversi elementi (come requisiti, attività di design, e componenti di codice).

\newpage


% Intestazione
\fancyhead[L]{U} % Testo a sinistra

\section{}

\dots

\newpage


% Intestazione
\fancyhead[L]{V} % Testo a sinistra

\section{}

\hypertarget{sec:verifica}{}
\subsection*{Verifica}
Processo che include un insieme di attività volte a garantire che il lavoro svolto durante lo sviluppo del software rispetti gli standard, i requisiti e 
le aspettative stabilite. La verifica è essenziale per garantire che il software sia di alta qualità, risponda alle esigenze degli utenti e riduca il 
rischio di errori e difetti. Si svolge a periodi regolari durante il corso del progetto.

\subsection*{Versionamento}
Processo che realizza il cosiddetto “controllo di versione”, stabilendo la storia cronologica delle azioni fatte per una certa attività, tracciando i 
cambiamenti occorsi e permettendo di tornare a uno stadio precedente qualora necessario.

\newpage