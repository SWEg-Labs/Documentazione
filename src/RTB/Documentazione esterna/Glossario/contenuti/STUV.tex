% Intestazione
\fancyhead[L]{S} % Testo a sinistra

\hypertarget{sec:scenario_alternativo}{}
\subsection*{Scenario alternativo}
Uno scenario alternativo nei casi d'uso è una sequenza di azioni o eventi che deviano dal flusso principale a causa di condizioni o circostanze specifiche. 
Questi scenari descrivono come il sistema dovrebbe comportarsi quando si verificano situazioni diverse rispetto al percorso standard, garantendo che tutte 
le possibilità siano considerate nel processo di progettazione.

\hypertarget{sec:scenario_uso}{}
\subsection*{Scenario d'uso (scenario principale)}
Uno scenario d'uso è una descrizione narrativa di come un utente interagisce con un sistema software per raggiungere un determinato obiettivo. Rappresenta 
una sequenza di azioni e reazioni tra l'utente e il sistema in un contesto specifico e fornisce una visione dettagliata del comportamento atteso del software 
in una situazione reale.

\subsection*{Script}
Uno script è un insieme di istruzioni scritte in un linguaggio di scripting (come Python, JavaScript, Bash, ecc.) progettato per automatizzare compiti 
specifici o per eseguire operazioni in modo diretto su un sistema, applicazione o ambiente web. A differenza di un programma complesso, uno script tende 
a essere più leggero e adatto per operazioni mirate e ripetibili, senza richiedere una fase di compilazione lunga o complessa.

\hypertarget{sec:similarità}{}
\subsection*{Similarità}
La similarità nell'apprendimento automatico indica quanto due o più elementi (immagini, testi, numeri) sono simili tra loro. 
Questa misura, fondamentale per raggruppare dati simili, classificare nuovi dati e fare raccomandazioni, si basa spesso sul calcolo 
della similarità tra vettori che rappresentano numericamente gli elementi. Esistono diverse misure di similarità, ciascuna adatta a 
specifici tipi di dati e problemi.

\hypertarget{sec:slack}{}
\subsection*{Slack}
Piattaforma di messaggistica istantanea e collaborazione progettata per team e organizzazioni. Permette di comunicare tramite canali tematici, inviare 
messaggi diretti, condividere file, e integrare altre applicazioni e strumenti di lavoro. Slack favorisce la comunicazione in tempo reale, migliorando 
la produttività e la gestione dei progetti, e viene utilizzato soprattutto in ambienti di lavoro agili e collaborativi.

\hypertarget{sec:snake_case}{}
\subsection*{Snake Case}
Lo Snake Case è uno stile di scrittura di nomi in cui le parole sono separate da un trattino basso o underscore (\_) e tutte le lettere sono generalmente 
minuscole (es.: nome\_variabile\_uno). Viene spesso contrapposto allo stile Camel Case, in cui invece le parole successive alla prima vengono concatenate 
con l’iniziale maiuscola (es.: nomeVariabileUno).

\hypertarget{sec:snippet}{}
\subsection*{Snippet}
Uno snippet è un piccolo blocco di codice riutilizzabile che svolge una funzione specifica o rappresenta un esempio di utilizzo di una determinata 
funzionalità. Gli snippet sono utilizzati per risparmiare tempo durante la scrittura del codice, poiché forniscono pezzi di codice già pronti che possono 
essere facilmente inseriti e adattati nel contesto del progetto.

\hypertarget{sec:sottocaso_d'uso}{}
\subsection*{Sottocaso d'uso}
Un Sottocaso d'uso è una specificazione di un caso d'uso che descrive un flusso alternativo o dettagliato di azioni, mantenendo una relazione diretta con 
il caso d'uso principale. Viene utilizzato per suddividere comportamenti complessi in scenari più gestibili, evidenziando variazioni o estensioni del 
flusso principale, pur rimanendo parte integrante del caso d'uso generale.

\hypertarget{sec:specifiche_funzionali}{Specifiche funzionali}
\subsection*{Specifiche funzionali}
Le specifiche funzionali sono una descrizione dettagliata delle funzionalità e dei comportamenti che un sistema software deve avere per soddisfare i 
requisiti degli utenti. Le specifiche funzionali definiscono cosa il software deve fare, come deve rispondere a determinate azioni e quali risultati
deve produrre in risposta a determinati input.

\hypertarget{sec:specifiche_tecniche}{Specifiche tecniche}
\subsection*{Specifiche tecniche}
Le specifiche tecniche sono una descrizione dettagliata delle caratteristiche tecniche e delle prestazioni che un sistema software deve avere per
soddisfare i requisiti tecnici e funzionali. Le specifiche tecniche definiscono come il software deve essere progettato, implementato e testato per
garantire che funzioni correttamente e risponda alle esigenze degli utenti.

\hypertarget{sec:spring}{}
\subsection*{Spring}
Framework Java che fornisce un ecosistema modulare per sviluppare applicazioni lato server. Facilita la gestione delle dipendenze, configurazioni, 
sicurezza, accesso ai dati e integrazioni, promuovendo un'architettura scalabile e flessibile basata su principi come l'inversione del controllo (IoC) e 
la programmazione orientata agli aspetti (AOP).

\hypertarget{sec:spring_boot}{}
\subsection*{Spring Boot}
Framework Java che semplifica lo sviluppo di applicazioni Spring, ovvero applicazioni basate sul framework Spring per la gestione di componenti, dipendenze 
e funzionalità lato server come sicurezza, accesso ai dati e integrazione. Spring Boot offre configurazioni predefinite, un server embedded e una struttura 
modulare, permettendo di creare applicazioni pronte all'uso con meno configurazioni manuali.

\subsection*{Sprint}
Periodo di tempo prefissato entro il quale lavorare producendo dei risultati documentati. Gli sprint sono al centro delle metodologie Agile, atte a produrre 
risultati piccoli e in maniera costante.

\hypertarget{sec:sprint_reptrospective}{}
\subsection*{Sprint Retrospective}
Incontro finalizzato ad analizzare l’andamento dello sprint quando questo è terminato, per migliorare la performance futura del team di sviluppo. La 
riunione retrospettiva dello sprint è quindi propedeutica allo sprint successivo.

\hypertarget{sec:stakeholder}{}
\subsection*{Stakeholder}
Individui, gruppi o entità che hanno un interesse o un coinvolgimento in un progetto, un’organizzazione o un’iniziativa specifica. Essi possono influenzare 
o essere influenzati dalle attività e dalle decisioni associate al progetto o all’organizzazione. La gestione degli stakeholder è una parte fondamentale del 
processo di pianificazione ed esecuzione di progetti ed è essenziale per il successo complessivo di un’iniziativa.

\hypertarget{sec:streamlit}{}
\subsection*{Streamlit}
Streamlit è un framework open-source in Python progettato per creare applicazioni web interattive e data-driven in modo rapido e intuitivo. È utilizzato 
principalmente per sviluppare dashboard e strumenti di visualizzazione dati, grazie a una sintassi semplice che permette di trasformare script Python in 
applicazioni web complete con poche righe di codice. Streamlit è particolarmente apprezzato nella comunità scientifica e tra i data scientist per la sua 
capacità di integrare facilmente grafici, widget interattivi e modelli di machine learning in un'interfaccia user-friendly.

\hypertarget{sec:supabase}{}
\subsection*{Supabase}
Supabase è una piattaforma open-source che fornisce un backend completo per applicazioni web e mobili, simile a Firebase. Basato su PostgreSQL, offre un 
database relazionale, un sistema di autenticazione, un'API RESTful generata automaticamente, storage per file e supporto per database vettoriali. Questa 
funzionalità consente di gestire embedding e lavorare con dati multidimensionali, rendendolo adatto ad applicazioni di machine learning e intelligenza 
artificiale. Supabase è pensato per semplificare lo sviluppo di applicazioni, fornendo strumenti pronti all'uso per gestire dati, utenti e file, con la 
possibilità di eseguire query in tempo reale. È molto apprezzato per la sua flessibilità, l'integrazione con standard open-source e la facilità d'uso.

\newpage


% Intestazione
\fancyhead[L]{T} % Testo a sinistra

\section{}

\hypertarget{sec:telegram}{}
\subsection*{Telegram}
Applicazione multipiattaforma che permette una facile comunicazione con gruppi/canali, organizzando la comunicazione sulla base di strumenti e 
funzionalità automatiche offerte (bot), condividendo facilmente file e messaggi in un canale unico di comunicazione.

\hypertarget{sec:test}{}
\subsection*{Test}
In ingegneria del software, processo sistematico e controllato per valutare un sistema software o una sua componente allo scopo di individuare eventuali 
difetti, errori o comportamenti indesiderati.

\subsection*{Tracciamento}
Il tracciamento è il processo di monitoraggio e documentazione dell'evoluzione e delle modifiche di un sistema, dei suoi requisiti e dei suoi componenti, 
all’interno di un progetto. Questo processo è essenziale per mantenere una visione chiara dello stato attuale del sistema, delle sue modifiche, e delle 
relazioni tra diversi elementi (come requisiti, attività di design, e componenti di codice).

\hypertarget{sec:trigger}{}
\subsection*{Trigger}
Un trigger (dall'inglese "grilletto") è un meccanismo che innesca automaticamente un'azione o un evento in risposta a un particolare 
stimolo o condizione. Nell'ambito informatico, un trigger è una procedura predefinita che viene eseguita in modo automatico quando 
si verifica un evento specifico.

\hypertarget{sec:txtai}{}
\subsection*{Txtai}
Txtai è una piattaforma open-source per la ricerca semantica e l'indicizzazione di dati basati su embedding. Fornisce strumenti per creare motori di 
ricerca, sistemi di domande e risposte, classificazione di documenti e clustering, utilizzando modelli di machine learning. Txtai supporta vari formati 
di dati, integra API intuitive e può essere facilmente utilizzato per costruire applicazioni di AI scalabili.

\hypertarget{sec:typescript}{}
\subsection*{TypeScript}
TypeScript è un linguaggio di programmazione open-source sviluppato da Microsoft, basato su JavaScript ma con l'aggiunta di un sistema di tipi statici. 
Progettato per migliorare la scalabilità e la manutenibilità del codice, TypeScript permette di rilevare errori durante la scrittura del codice grazie 
alla tipizzazione, mantenendo la compatibilità con il runtime di JavaScript. È particolarmente utilizzato in applicazioni web e progetti di grandi 
dimensioni, offrendo una migliore esperienza di sviluppo e strumenti avanzati come il completamento automatico e il refactoring.

\newpage


% Intestazione
\fancyhead[L]{U} % Testo a sinistra

\section{}

\hypertarget{sec:uml}{}
\subsection*{UML, diagramma}
Vedi \bulhyperlink{sec:diagramma_UML}{Diagramma UML}.

\newpage


% Intestazione
\fancyhead[L]{V} % Testo a sinistra

\section{}

\hypertarget{sec:verifica}{}
\subsection*{Verifica}
Processo che include un insieme di attività volte a garantire che il lavoro svolto durante lo sviluppo del software rispetti gli standard, i requisiti e 
le aspettative stabilite. La verifica è essenziale per garantire che il software sia di alta qualità, risponda alle esigenze degli utenti e riduca il 
rischio di errori e difetti. Si svolge a periodi regolari durante il corso del progetto.

\subsection*{Versionamento}
Processo che realizza il cosiddetto “controllo di versione”, stabilendo la storia cronologica delle azioni fatte per una certa attività, tracciando i 
cambiamenti occorsi e permettendo di tornare a uno stadio precedente qualora necessario.

\hypertarget{sec:VSC}{}
\subsection*{Visual Studio Code (VSC)}
IDE libero e gratuito sviluppato da Microsoft per Windows, Linux e macOS. Permette sia di scrivere codice 
sorgente per un prodotto software sia di scrivere testo di documentazione.

\newpage