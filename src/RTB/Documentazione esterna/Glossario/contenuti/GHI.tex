% Intestazione
\fancyhead[L]{G} % Testo a sinistra

\section*{G}
\addcontentsline{toc}{section}{G}

\subsection*{GitHub}
Servizio di hosting per progetti software. Il sito è principalmente utilizzato da sviluppatori che caricano il codice sorgente di programmi in dei 
repository e lo rendono scaricabile e migliorabile da altri sviluppatori. Questi ultimi possono interagire con i proprietari dei repository tramite un 
sistema per inviare segnalazioni di bug o richieste di funzionalità (issue tracker), un sistema per copiare il software in una versione modificabile 
(fork), un sistema per proporre modifiche agli sviluppatori originali (pull request) e un sistema di discussione legato al codice del repository (commenti).

\subsection*{GitHub Pages}
GitHub Pages è un servizio di hosting gratuito offerto da GitHub che permette agli utenti di creare e pubblicare facilmente siti web statici direttamente 
dai loro repository GitHub. È utilizzato comunemente per creare siti di documentazione, pagine personali o di progetto, blog e portali di portfolio. 
GitHub Pages è particolarmente apprezzato perché permette di ospitare un sito senza costi e con aggiornamenti automatici ogni volta che il repository 
viene modificato.

\subsection*{GitHub Projects}
GitHub Projects è uno strumento di gestione dei progetti integrato in GitHub, ideato per aiutare sviluppatori e team a organizzare, pianificare e tracciare 
il lavoro sui progetti direttamente all'interno dell'ambiente GitHub. Si basa su un sistema flessibile di "project board" simile a Kanban, che offre un modo 
visuale per coordinare i task e monitorare l’avanzamento del lavoro. GitHub Projects è uno strumento ideale per team che lavorano su progetti di sviluppo 
software, in quanto permette di gestire l’intero processo di sviluppo all'interno di GitHub stesso. Grazie alla sua integrazione nativa con il codice, le 
issues e le pull requests, aiuta a mantenere sincronizzati i task e a ridurre il contesto di cambiamento per gli sviluppatori, migliorando il flusso di 
lavoro e facilitando la collaborazione su GitHub.

\subsection*{Glossario}
Elenco organizzato di termini tecnici, acronimi e definizioni utilizzati nel contesto del progetto. Questo documento fornisce una chiara comprensione dei 
concetti e dei linguaggi specifici impiegati nel progetto, aiutando a ridurre ambiguità e fraintendimenti tra i membri del team e gli stakeholder.

\subsection*{Gulpease, indice di}
Indice di leggibilità di un testo specificamente in lingua italiana, che utilizza il numero delle parole, delle frasi e delle lettere per facilitare il 
calcolo automatico della leggibilità.

\newpage


% Intestazione
\fancyhead[L]{H} % Testo a sinistra

\section*{H}
\addcontentsline{toc}{section}{H}

\dots

\newpage


% Intestazione
\fancyhead[L]{I} % Testo a sinistra

\section*{I}
\addcontentsline{toc}{section}{I}

\subsection*{Issue}
Una issue su GitHub (e altre piattaforme di gestione del codice e dei progetti) è un elemento utilizzato per tracciare problemi, richieste di funzionalità, 
idee o miglioramenti relativi a un progetto. È uno strumento fondamentale per organizzare il lavoro collaborativo e garantire che tutti i membri del team 
siano aggiornati sui task e le priorità. Ogni issue rappresenta un singolo elemento che richiede attenzione o azione, e fornisce un luogo centralizzato per 
discuterlo, seguirlo e risolverlo.

\newpage