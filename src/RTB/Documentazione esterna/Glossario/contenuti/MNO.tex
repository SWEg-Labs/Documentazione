% Intestazione
\fancyhead[L]{M} % Testo a sinistra

\section{}

\hypertarget{sec:markup}{}
\subsection*{Markup}
Un sistema di codifica che utilizza tag o annotazioni per strutturare e formattare documenti digitali (es. HTML o XML).

\hypertarget{sec:merge}{}
\subsection*{Merge}
Operazione fondamentale nei sistemi di controllo delle versioni. Essa è utilizzata per combinare le modifiche apportate in due branch separati in un 
singolo branch.

\hypertarget{sec:metrica}{}
\subsection*{Metrica}
Misura quantitativa utilizzata per valutare, quantificare e analizzare diversi aspetti del processo di sviluppo del software, del prodotto software stesso o della gestione del
progetto. Le metriche forniscono dati numerici che consentono di valutare l’andamento del
progetto, la qualit`a del software, l’efficacia dei processi e altri aspetti rilevanti.

\subsection*{Milestone}
In ingegneria del software e nella gestione dei progetti, punto di riferimento o traguardo significativo che sancisce il termine di un periodo nel ciclo 
di vita di un progetto. Le milestone rappresentano generalmente eventi chiave, compimenti o obiettivi importanti che indicano il progresso del progetto. 
L’obiettivo che ci si pone durante una milestone è realizzare una baseline.

\hypertarget{sec:MVP}{}
\subsection*{Minimum Viable Product (MVP)}
Versione ridotta di un prodotto, che incorpora solo le funzioni essenziali per soddisfare le esigenze di base. Viene utilizzato per rilasciare un prodotto 
come test e ricevere feedback dall’utenza per migliorare poi il prodotto finito con tutte le funzionalità.

\newpage


% Intestazione
\fancyhead[L]{N} % Testo a sinistra

\section{}

\hypertarget{sec:nestjs}{}
\subsection*{NestJS}
Framework Node.js per lo sviluppo di applicazioni lato server, basato su TypeScript. Utilizza un'architettura modulare e concetti come iniezione delle 
dipendenze e decoratori, ispirandosi a framework come Angular, per creare applicazioni scalabili e facilmente mantenibili.

\hypertarget{sec:nextjs}{}
\subsection*{Next.js}
Un framework open-source full stack basato su React, progettato per lo sviluppo di applicazioni web moderne. Consente la gestione sia del frontend sia 
del backend, grazie al supporto integrato per API e funzioni serverless.

\hypertarget{sec:nodejs}{}
\subsection*{Node.js}
Runtime JavaScript open-source e multipiattaforma, progettato per l'esecuzione di codice JavaScript lato server. Basato sul motore V8 di Google Chrome, 
permette di creare applicazioni veloci, scalabili e basate su eventi, sfruttando un modello non bloccante per la gestione delle operazioni I/O.

\subsection*{Norme di Progetto}
Insieme di linee guida, procedure e regole stabilite per regolare e standardizzare l’approccio, il processo e l’output del lavoro all’interno del progetto. 
Queste norme possono riguardare diversi aspetti del progetto, come la gestione del codice, la documentazione, la comunicazione e la gestione dei rischi. 
L’obiettivo delle norme di progetto è promuovere la coerenza, la qualità e l’efficienza nel processo di sviluppo del software, consentendo al team di 
lavorare in modo più efficace e collaborativo.

\newpage


% Intestazione
\fancyhead[L]{O} % Testo a sinistra

\section{}

\hypertarget{sec:onboarding}{}
\subsection*{Onboarding}
Il processo di integrazione e formazione di un nuovo dipendente, collaboratore o utente in un'organizzazione o piattaforma. L'onboarding ha l'obiettivo di 
fornire tutte le informazioni e le risorse necessarie per adattarsi al nuovo ambiente di lavoro, comprendere la cultura aziendale, acquisire competenze 
specifiche e diventare operativi nel più breve tempo possibile. 


\newpage