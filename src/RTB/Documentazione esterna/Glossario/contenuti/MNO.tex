% Intestazione
\fancyhead[L]{M} % Testo a sinistra

\section{}

\hypertarget{sec:merge}{}
\subsection*{Merge}
Operazione fondamentale nei sistemi di controllo delle versioni. Essa è utilizzata per combinare le modifiche apportate in due branch separati in un 
singolo branch.

\subsection*{Milestone}
In ingegneria del software e nella gestione dei progetti, punto di riferimento o traguardo significativo che sancisce il termine di un periodo nel ciclo 
di vita di un progetto. Le milestone rappresentano generalmente eventi chiave, compimenti o obiettivi importanti che indicano il progresso del progetto. 
L’obiettivo che ci si pone durante una milestone è realizzare una baseline.

\hypertarget{sec:MVP}{}
\subsection*{Minimum Viable Product (MVP)}
Versione ridotta di un prodotto, che incorpora solo le funzioni essenziali per soddisfare le esigenze di base. Viene utilizzato per rilasciare un prodotto 
come test e ricevere feedback dall’utenza per migliorare poi il prodotto finito con tutte le funzionalità.

\newpage


% Intestazione
\fancyhead[L]{N} % Testo a sinistra

\section{}

\subsection*{Norme di Progetto}
Insieme di linee guida, procedure e regole stabilite per regolare e standardizzare l’approccio, il processo e l’output del lavoro all’interno del progetto. 
Queste norme possono riguardare diversi aspetti del progetto, come la gestione del codice, la documentazione, la comunicazione e la gestione dei rischi. 
L’obiettivo delle norme di progetto è promuovere la coerenza, la qualità e l’efficienza nel processo di sviluppo del software, consentendo al team di 
lavorare in modo più efficace e collaborativo.

\newpage


% Intestazione
\fancyhead[L]{O} % Testo a sinistra

\section{}

\dots

\newpage