% Intestazione
\fancyhead[L]{D} % Testo a sinistra

\section{}

\hypertarget{sec:demo}{}
\subsection*{Demo}
Una demo (abbreviazione di "dimostrazione") è una presentazione interattiva o registrata di un software o di una funzionalità specifica all'interno di un 
progetto. Viene usata per mostrare lo stato attuale dello sviluppo, illustrando come funzionano le caratteristiche implementate e come si comporta il 
sistema rispetto ai requisiti stabiliti. Le demo sono solitamente presentate a stakeholder, clienti, o team interni per ottenere feedback, verificare 
funzionalità e orientare lo sviluppo futuro. Le demo sono particolarmente utili nei processi iterativi e agili, dove è importante monitorare e comunicare 
costantemente i progressi del progetto.

\subsection*{Diagramma di Gantt}
Strumento di visualizzazione temporale utilizzato nella gestione dei progetti per rappresentare le attività pianificate nel tempo. È composto da una barra 
orizzontale che rappresenta l’arco temporale totale del progetto e da barre orizzontali più piccole che rappresentano le singole attività del progetto. 
Ogni barra è posizionata lungo l’asse temporale in base alle date di inizio e fine previste per l’attività.

\subsection*{Diagramma UML}
Acronimo di Unified Modeling Language, un diagramma UML diagramma utilizzato per modellare, descrivere e visualizzare sistemi software e processi di sviluppo 
software. È uno standard industriale nel campo dell’ingegneria del software e fornisce una serie di diagrammi, ognuno dei quali si concentra su un aspetto 
specifico del sistema o del processo. Sono diagrammi UML ad esempio i diagrammi dei casi d’uso, i diagrammi delle classi e i diagrammi delle funzionalità.

\subsection*{Diagramma UML dei casi d'uso}
Diagramma UML che rappresenta le interazioni tra utenti (attori) e il sistema, descrivendo come gli utenti utilizzano il sistema per raggiungere obiettivi 
specifici. Ogni caso d'uso rappresenta una funzione o un'attività significativa, utile per descrivere i requisiti funzionali. È spesso il primo passo nella 
progettazione di un sistema software e aiuta a identificare le funzioni principali e il modo in cui il sistema interagisce con gli utenti.

\subsection*{Diagramma UML delle classi}
Diagramma UML che descrive la struttura statica di un sistema, mostrando le classi, i loro attributi, i metodi e le relazioni tra di esse (come ereditarietà, 
associazioni e aggregazioni). Questo diagramma è fondamentale per la programmazione orientata agli oggetti poiché fornisce una rappresentazione visiva della 
struttura del codice, aiutando a comprendere le interconnessioni tra le varie entità e a definire i componenti principali.

\hypertarget{sec:diario_di_bordo}{}
\subsection*{Diario di Bordo}
Nel contesto del progetto didattico, il “diario di bordo” è un’attività settimanale in cui ogni gruppo di progetto documenta e presenta pubblicamente il progresso del proprio lavoro. 
Questa attività include la descrizione delle attività svolte, delle difficoltà incontrate, dei dubbi e delle incertezze, e dei piani per il periodo successivo. 
L’obiettivo è promuovere una discussione aperta e riflessiva per migliorare la consapevolezza e la gestione del progetto collaborativo.

\subsection*{Discord}
Piattaforma VoIP (Voice over IP: tecnologia che rende possibile effettuare una conversazione sfruttando una connessione internet), messaggistica istantanea 
e distribuzione digitale progettata per la comunicazione.

\subsection*{Draw.io}
Draw.io (ora chiamato diagrams.net) è uno strumento gratuito per la creazione di diagrammi, disponibile sia come applicazione web che come app desktop per 
vari sistemi operativi. Viene utilizzato ampiamente per progettare e documentare diagrammi di flusso, architetture software, diagrammi UML, organigrammi, 
mappe mentali, wireframe, e altri tipi di rappresentazioni visive.

\newpage


% Intestazione
\fancyhead[L]{E} % Testo a sinistra

\section{}

\hypertarget{sec:efficacia}{}
\subsection*{Efficacia}
Capacità di raggiungere gli obiettivi desiderati o di produrre gli effetti previsti.

\hypertarget{sec:efficienza}{}
\subsection*{Efficienza}
Capacità di svolgere un compito, un'attività o un processo nel modo più economico e con il minimo spreco di risorse.

\newpage


% Intestazione
\fancyhead[L]{F} % Testo a sinistra

\section{}

\hypertarget{sec:feature}{}
\subsection*{Feature}
Una feature è una specifica funzionalità o capacità di un sistema software che ne arricchisce il comportamento e le possibilità d'uso. Le feature sono 
aggiunte o miglioramenti che rispondono a esigenze degli utenti, migliorano l’esperienza d’uso, o risolvono specifici problemi. L'introduzione o il 
miglioramento di una feature segue un ciclo di sviluppo completo, dalla raccolta dei requisiti fino alla verifica e al testing.

\subsection*{Fogli Google}
Fogli Google è un'applicazione web di Google, parte della suite di produttività Google Workspace, che consente di creare, modificare e condividere fogli 
di calcolo online. È uno strumento particolarmente apprezzato per il suo accesso immediato da browser, le funzionalità collaborative in tempo reale, e 
l'integrazione con altri servizi Google. Così come ogni foglio di calcolo, include la possibilità di creare grafici e diagrammi basati su dati tabellari.

\newpage