% Configurazione
\documentclass{article}

\usepackage{titling} % Required for inserting the subtitle
\usepackage{graphicx} % Required for inserting images
\usepackage{tabularx} % Per l'ambiente tabularx (tabelle)
\usepackage{calc} % Sempre per le tabelle
\usepackage[hidelinks]{hyperref} % Per i collegamenti ipertestuali, ad esempio sulla table of contents
\usepackage[italian]{babel} % Per la lingua italiana nelle scritte automatiche
\usepackage[table]{xcolor} % Per colorare il testo e le celle delle tabelle
\usepackage{colortbl} % Per colorare le celle delle tabelle
\usepackage{lipsum} % Per generare lorem ipsum
\usepackage[normalem]{ulem} % Per sottolineare il testo
\usepackage{array} % Per la visualizzazione fluttuante di array di domande e risposte
\usepackage{ragged2e} % Pacchetto necessario per \justifying che giustifica il testo di tabelle
\usepackage{tikz} % Per spostare elementi nel documento in modo facile e veloce
\usepackage[a4paper, top=2.5cm, bottom=2.5cm, left=2.5cm, right=2.5cm]{geometry} % Per i margini della pagina
\usepackage{fancyhdr} % Per l'intestazione e il piè di pagina
\usepackage{amsmath} % Per scrivere formule matematiche, in particolare per il pedice G

\newcommand{\ulhref}[2]{\href{#1}{\uline{#2}}} % Nuovo comando per sottolineare i link
\newcommand{\ulref}[1]{\uline{\ref{#1}}} % Nuovo comando per sottolineare i collegamenti a immagini e tabelle
\setlength{\parindent}{0pt} % Rimuove il rientro automatico dei paragrafi
\usetikzlibrary{calc} % Libreria per il calcolo delle coordinate di TikZ
\pagestyle{fancy} % Stile della pagina, per l'intestazione e il piè di pagina
\renewcommand{\footrulewidth}{0.4pt} % Inserimento della linea orizzontale in basso
\setlength{\headsep}{1.4cm} % Spazio tra l'intestazione e il testo
\definecolor{lightgray}{gray}{0.95} % Definizione del colore grigio chiaro

\graphicspath{ {immagini/} {../../../shared/immagini/} }



% Struttura
\begin{document}

\thispagestyle{plain} % Niente intestazione e piè di pagina


\begin{tikzpicture}[remember picture, overlay]
    % Punto di partenza al centro orizzontale nella metà superiore
    \coordinate (top_center) at ($(current page.north)!0.3!(current page.south)$);

    % UniPD: Logo e descrizione
    \node at (top_center) [anchor=north, xshift=-3cm, yshift=4.85cm] 
        {\includegraphics[width=0.15\textwidth]{Logo Universita di Padova.png}};
    \node at (top_center) [anchor=north, xshift=1.7cm, yshift=4.5cm]
        {\textcolor{red}{\textbf{Università degli Studi di Padova}}};
    \node at (top_center) [anchor=north, xshift=1.7cm, yshift=4cm]
        {\textcolor{red}{Laurea: Informatica}};
    \node at (top_center) [anchor=north, xshift=1.7cm, yshift=3.5cm]
        {\textcolor{red}{Corso: Ingegneria del Software}};
    \node at (top_center) [anchor=north, xshift=1.7cm, yshift=3cm]
        {\textcolor{red}{Anno Accademico: 2024/2025}};

    % SWEg Labs: Logo e descrizione
    \node at (top_center) [anchor=north, xshift=-2.85cm, yshift=1.5cm] 
        {\includegraphics[width=0.16\textwidth]{Logo SWEg.png}};
    \node at (top_center) [anchor=north, xshift=1.7cm, yshift=0.5cm]
        {\textbf{Gruppo: SWEg Labs}};
    \node at (top_center) [anchor=north, xshift=1.7cm, yshift=0cm]
        {Email: \textsf{gruppo.sweg@gmail.com}};
\end{tikzpicture}


\vspace{10cm}

{
\centering
\Huge\bfseries Manuale Utente\par
\vspace{0.5cm}
\Large Versione 1.0.0\par
}

\vspace{2cm}
% Intestazione
\fancyhead[L]{1 \hspace{0.2cm} Informazioni generali} % Testo a sinistra

\pagenumbering{arabic} % Numerazione araba per il contenuto


\section{Informazioni generali}

\begin{itemize}
    \item \textbf{Tipo di riunione}: interna
    \item \textbf{Luogo}: meeting \emph{Discord}\textsubscript{\textit{\textbf{G}}}
    \item \textbf{Data}: 04/11/2024
    \item \textbf{Ora inizio}: 17:30
    \item \textbf{Ora fine}: 18:45
    \item \textbf{Responsabile}: Riccardo Stefani
    \item \textbf{Scriba}: Davide Verzotto
    \item \textbf{Partecipanti}:
    \begin{itemize}
        \item Federica Bolognini
        \item Michael Fantinato
        \item Giacomo Loat
        \item Filippo Righetto
        \item Riccardo Stefani
        \item Davide Verzotto
    \end{itemize}
\end{itemize}


\newpage
% Intestazione
\fancyhead[L]{Registro delle modifiche} % Testo a sinistra
\fancyhead[R]{\includegraphics[width=0.16\textwidth]{sweg_logo_sito_inverted.png}} % Immagine a destra

% Piè di pagina
\fancyfoot[L]{Analisi dei Requisiti}       % Testo a sinistra
\fancyfoot[C]{\thepage}                % Numero di pagina al centro
\fancyfoot[R]{Versione 1.0.0}          % Testo a destra

\pagenumbering{roman} % Numerazione romana per l'indice


\section*{Registro delle modifiche}

\begin{table}[h]
    \centering
    \rowcolors{2}{lightgray}{white}
    \begin{tabular}{|c|c|p{5cm}|p{3cm}|p{3cm}|}
        \hline
        \rowcolor[gray]{0.75}
        \textbf{Versione} & \textbf{Data} & \multicolumn{1}{|c|}{\textbf{Descrizione}} & 
        \multicolumn{1}{|c|}{\textbf{Autore}} & \multicolumn{1}{|c|}{\textbf{Verifica}}\\
        \hline
        1.0.0 & ... & Approvazione del documento & Filippo Righetto & Filippo Righetto\\
        \hline
        ... & ... & Verifica del documento & ... & ...\\
        \hline
        0.2.4 & 10-12-24 & Scrittura del caso d'uso \bulhyperlink{UC2}{UC2} & Giacomo Loat & Michael Fantinato \\
        \hline
        0.2.3 & 09-12-24 & Scrittura del caso d'uso \bulhyperlink{UC1}{UC1} & Federica Bolognini & ... \\
        \hline
        0.2.2 & 09-12-24 & Creazione del template per la trascrizione dei casi d'uso in \S\bulref{sec:casi_uso} & Riccardo Stefani & Federica Bolognini\\
        \hline
        0.2.1 & 08-12-24 & Scrittura dei casi d'uso \bulhyperlink{UC5}{UC5}, \bulhyperlink{UC6}{UC6}, \bulhyperlink{UC11}{UC11} e 
        \bulhyperlink{UC16}{UC16} & Riccardo Stefani & Giacomo Loat\\
        \hline
        0.2.0 & 06-12-24 & Verifica del documento allo stato attuale & Riccardo Stefani & Riccardo Stefani\\
        \hline
        0.1.2 & 18-11-24 & Inizio scrittura sezione \S\bulref{sec:Requisiti} & Filippo Righetto & Davide Verzotto\\
        \hline
        0.1.1 & 10-11-24 & Scrittura della sezione \S\bulref{sec:introduzione} di introduzione e della sezione \S\bulref{sec:descrizione_generale} 
        riguardante la descrizione generale & Filippo Righetto & Riccardo Stefani\\
        \hline
        0.1.0 & 05-11-24 & Creazione del documento & Riccardo Stefani & Giacomo Loat\\
        \hline
    \end{tabular}
    \caption{Registro delle modifiche}
\end{table}

\newpage
% Intestazione
\fancyhead[L]{Indice} % Testo a sinistra
\fancyhead[R]{\includegraphics[width=0.16\textwidth]{sweg_logo_sito_inverted.png}} % Immagine a destra

% Piè di pagina
\fancyfoot[L]{Verbale interno}       % Testo a sinistra
\fancyfoot[C]{\thepage}                % Numero di pagina al centro
\fancyfoot[R]{16/11/24}          % Testo a destra

\pagenumbering{roman} % Numerazione romana per l'indice


\tableofcontents
\newpage

% Intestazione
\fancyhead[L]{1 \hspace{0.2cm} Introduzione} % Testo a sinistra

\pagenumbering{arabic} % Numerazione araba per il contenuto 


\section{Introduzione}
Questo documento è stato redatto con l'intento di offrire una trattazione esaustiva e dettagliata 
dei requisiti e dei casi d'uso individuati dal gruppo \textit{sweg labs} nel corso dello sviluppo
del progetto “BuddyBot”. La raccolta di questi dati è il frutto di un'analisi approfondita
del documento di presentazione del \textit{capitolato\textsubscript{G}}, di intense discussioni interne al gruppo di lavoro, 
nonchè di colloqui attivi con il \textit{proponente\textsubscript{G}}, \textit{Azzurrodigitale}.

L'obiettivo è garantire una comprensione completa ed accurata dei requisiti di progetto,
fornendo una base solida per la pianificazione e l'implementazione delle successive fasi di lavoro.

Nel documento adottiamo la sintassi \textit{UML\textsubscript{G}} al fine di formalizzare la rappresentazione e
renderla comprensibile a tutti i potenziali utenti. In particolare, i casi d'uso seguono una
struttura logica e vengono descritti in dettaglio attraverso i seguenti punti:
\begin{itemize}
    \item \textbf{Nominativo:} includiamo il titolo del \textit{caso d'uso\textsubscript{G}} e un breve commento esplicativo;
    \item \textbf{Attori Principali:} identifichiamo chi sono gli \textit{attori\textsubscript{G}} che eseguono le azioni all’interno 
                del caso d'uso;
    \item \textbf{Precondizioni:} specifichiamo lo stato del programma prima dell'esecuzione del caso d'uso;
    \item \textbf{Postcondizioni:} definiamo lo stato del programma dopo il completamento dello scenario del caso d'uso;
    \item \textbf{\textit{Scenario Principale\textsubscript{G}}:} descriviamo in modo dettagliato le azioni svolte durante
                l'esecuzione del caso d'uso, delineando il percorso seguito tra le condizioni iniziali e irisultati ottenuti;
    \item \textbf{Scenari alternativi:} descriviamo gli scenari che diramano dallo scenario principale o le situazioni nelle quali lo svolgimento delle 
                azioni dello scenario principale siaimpossibilitato dalla comparsa di condizioni di errore;
    \item \textbf{\textit{Sottocasi d'uso\textsubscript{G}}:} in alcune circostanze può essere necessaria la definizione di uno
                o più sottocasi d'uso, che andranno ad utilizzare la stessa struttura dei casi d'uso, e potranno essere 
                identificati mediante un numero progressivo nella forma:
                \begin{center}
                    X.Y
                \end{center}
    dove X `e il caso d'uso da cui derivano e Y un numero progressivo ad identificare il sottocaso.
    \item \textbf{Inclusioni:} descrivono funzionalità in comune fra più casi d'uso;
    \item \textbf{Specializzazioni:} possono essere di due tipologie:
    \begin{enumerate}
        \item di attori, dove i figli condividono tutte le funzionalit`a del padre e in pi`u ne
            possiedono di proprie;
        \item di casi d'uso, dove i figli possono aggiungere funzionalit`a rispetto ai padri o
            modificarne il comportamento.
    \end{enumerate}
\end{itemize}

\subsection{Scopo del prodotto}
Nel corso dell'ultimo anno si è verificato un repentino e significativo mutamento nel panorama
dello sviluppo e nell'implementazione dell'\textit{Intelligenza Artificiale\textsubscript{G}}.
Questa trasformazione ha interessato diverse sfaccettature della tecnologia, e si è verificata con il passaggio da un
ruolo prevalentemente incentrato sull'elaborazione e sulla raccomandazione dei contenuti ad
una fase in cui l'Intelligenza Artificiale assume attivamente la responsabilità di generare
contenuti originali. Questa nuova fase ha visto l'emergere di sistemi in grado di creare non
solo testi, ma anche immagini e tracce audio con un livello di sofisticazione che sfida le
precedenti aspettative. \\
Il capitolato\textsubscript{G} C9, 'BuddyBot,' ha come obiettivo la realizzazione di un assistente virtuale (chatbot) 
capace di raccogliere rapidamente informazioni dalle fonti indicate e di fornirle in risposta a domande poste in 
linguaggio naturale tramite chat.\\
Tale assistente virtuale sarà fruibile attraverso una piccola piattaforma web, dove l'utente potrà interagire con l'IA 
per ottenere le risposte desiderate.

\subsection{Glossario}
Al fine di evitare possibili ambiguità relative al linguaggio utilizzato nei documenti, viene fornito un \textit{Glossario}
(attualmente alla sua versione \textit{1.0.0}), nel quale sono contenute le definizioni di termini complessi o aventi uno 
specifico significato. Tali termini, ove necessario, sono segnati in corsivo e marcati con il simbolo G a pedice
(\textit{esempio Way of Working\textsubscript{G}}).

\subsection{Miglioramenti al documento}
La maturità e i miglioramenti sono aspetti fondamentali nella stesura di un documento.
Questo permette di apportare agevolmente modifiche in base alle esigenze concordate tra i
membri del gruppo e il \textit{proponente\textsubscript{G}} nel corso del tempo. Di conseguenza, questa versione del
documento non pu`o essere considerata definitiva o completa, poichè è soggetta a evoluzioni future.

\subsection{Riferimenti}
\subsubsection{Riferimenti normativi}
\begin{itemize}
    \item \href{https://www.sito2.com}{Norme di Progetto (manca link)}
    \item \href{https://github.com/SWEg-Labs/Documentazione/blob/adea4950d9135916b22ef3af717e955f2c11f975/output/RTB/Documentazione%20esterna/piano_qualifica_v1.0.0.pdf}{Piano di qualifica (v 1.0.0)}
    \item \href{https://www.math.unipd.it/~tullio/IS-1/2024/Progetto/C9.pdf}{Capitolato d'appalto C9 - BuddyBot}
    \item \href{https://www.math.unipd.it/~tullio/IS-1/2024/Dispense/PD1.pdf}{Slide PD1 del corso di Ingegneria del Software - Regolamento del Progetto Didattico}
  \end{itemize}

\subsubsection{Riferimenti informativi}
\begin{itemize}
    \item \href{https://github.com/SWEg-Labs/Documentazione/blob/adea4950d9135916b22ef3af717e955f2c11f975/output/RTB/Documentazione%20esterna/glossario_v1.0.0.pdf}{Glossario (v 1.0.0)}
    \item \href{https://github.com/SWEg-Labs/Documentazione/tree/adea4950d9135916b22ef3af717e955f2c11f975/output/RTB/Documentazione%20interna/Verbali%20interni}{Verbali interni}
    \item \href{https://github.com/SWEg-Labs/Documentazione/tree/adea4950d9135916b22ef3af717e955f2c11f975/output/RTB/Documentazione%20esterna/Verbali%20esterni}{Verbali esterni}
    \item \href{https://www.math.unipd.it/~tullio/IS-1/2024/Dispense/T05.pdf}{Slide T05 del corso di Ingegneria del Software - Analisi dei Requisiti}
    \item \href{https://www.sito3.com}{Diagrammi dei casi d'uso (non ho trovato il documento)}
  \end{itemize}

\newpage

\appendix % Cambia la numerazione delle sezioni da numeri a lettere


% Intestazione
\fancyhead[L]{A} % Testo a sinistra

\section{}
%\addcontentsline{toc}{section}{A}

\hypertarget{sec:accoppiamento}{}
\subsection*{Accoppiamento}
In informatica, l'accoppiamento è il grado di dipendenza tra due o più componenti di un sistema software. Un accoppiamento elevato indica una forte
interdipendenza tra le componenti, mentre un accoppiamento basso indica una minore dipendenza. Un basso accoppiamento è generalmente preferibile, poiché
rende il sistema più flessibile, modulare e facile da mantenere.

\hypertarget{sec:accessibilità}{}
\subsection*{Accessibilità}
Capacità di rendere le informazioni sono più facilmente fruibili, condivisibili e adattabili alle esigenze di ciascun utente.

\hypertarget{sec:adapter}{}
\subsection*{Adapter}
Design pattern strutturale che consente di convertire l'interfaccia di una classe in un'altra interfaccia che il client si aspetta.
L'adapter permette a classi con interfacce incompatibili di lavorare insieme, facilitando l'integrazione di componenti software eterogenee.

\hypertarget{sec_adapter_arch_esagonale}{}
\subsection*{Adapter (Architettura esagonale)}
Nell'architettura esagonale, l'adapter è un componente che si occupa di adattare le richieste provenienti dall'esterno del sistema alle interfacce interne,
e viceversa. L'adapter permette di isolare il core del sistema dalle dipendenze esterne, garantendo una maggiore flessibilità e facilità di manutenzione.
In particolare, esso converte i tipi di dato di business in oggetti Entity, e li fornisce alla classe Repository per l'interazione con il database o con
lo strumento di persistenza dei dati utilizzato. In caso di recupero di dati dal database, l'adapter converte gli oggetti Entity che
arrivano dal repository in tipi di dato di business, e li fornisce alla business logic per l'elaborazione.

\hypertarget{sec:modello_agile}{}
\subsection*{Agile, modello di sviluppo}
Approccio alla gestione dei progetti e allo sviluppo del software che enfatizza la flessibilità, la collaborazione e il miglioramento continuo. 
Ha come principi fondamentali consegna continua, collaborazione stretta tra sviluppatori e clienti, adattabilità ai cambiamenti, 
comunicazione diretta tra le parti e valutazione e miglioramento continuo del processo e del prodotto.

\hypertarget{sec:alpine_linux}{}
\subsection*{Alpine Linux}
Alpine Linux è una distribuzione Linux minimalista, progettata per garantire un'elevata efficienza delle risorse e una robusta sicurezza. 
Offre un ambiente operativo leggero e performante, ideale per l'utilizzo in container e microservizi. La sua 
impronta ridotta consente tempi di avvio rapidi e una minore superficie di attacco, mentre il design minimalista favorisce una gestione semplificata 
e una maggiore affidabilità, rendendola una scelta privilegiata in ambienti di produzione dove le prestazioni e la sicurezza sono fondamentali.


\subsection*{Analisi dei Requisiti}
Processo fondamentale dello sviluppo di un prodotto software che si concentra sulla raccolta, analisi e definizione delle necessità e delle aspettative 
degli utenti finali, degli stakeholder e del sistema nel suo complesso. Questo processo mira a comprendere e documentare in modo chiaro e completo le 
esigenze, le funzionalità, le prestazioni e i vincoli che il sistema deve soddisfare. L’obiettivo principale dell’analisi dei requisiti è fornire una 
base solida per tutte le fasi successive dello sviluppo del software, assicurando che il prodotto finale soddisfi le esigenze degli utenti e raggiunga 
gli obiettivi del progetto.

\subsection*{Analisi dei Rischi}
L'analisi dei rischi è il processo di identificazione, valutazione e priorizzazione dei rischi in un progetto, sistema o attività, al fine di ridurre 
o gestire il loro impatto potenziale. Viene utilizzata per prevedere gli eventi negativi che potrebbero influenzare il successo di un progetto e per 
determinare le azioni preventive o correttive da intraprendere.

\hypertarget{sec:analisi_dinamica}{}
\subsection*{Analisi dinamica}
Esecuzione del software per verificarne il comportamento, identificare malfunzionamenti e valutare il rispetto dei requisiti in un ambiente reale o simulato.

\hypertarget{sec:analisi_statica}{}
\subsection*{Analisi statica}
Controlli eseguiti senza avviare il software, utilizzati per verificare la qualità del codice, individuare errori e garantire conformità agli standard.

\hypertarget{sec:analogico}{}
\subsection*{Analogico}


\hypertarget{sec:angular}{}
\subsection*{Angular}
Framework open-source sviluppato da Google per creare applicazioni web dinamiche e scalabili. Basato su TypeScript, utilizza un'architettura a componenti, 
il data binding bidirezionale e strumenti integrati per lo sviluppo di interfacce utente robuste e performanti.

\hypertarget{sec:api}{}
\subsection*{API}
Acronimo di Application Programming Interface, è un insieme di regole e protocolli che consente a diverse applicazioni software di comunicare tra loro 
per scambiare dati, caratteristiche e funzionalità. Le API fungono da intermediari, permettendo lo scambio di informazioni tra software diversi, semplificando 
e accelerando lo sviluppo di applicazioni.

\hypertarget{sec:application_logic}{}
\subsection*{Application logic}
Parte di un'applicazione software che implementa la logica di business e le regole di funzionamento del sistema in una applicazione specifica.
L'application logic è responsabile della validazione dei dati che arrivano in input dall'applicazione esterna, chiamati DTO, e li adatta
verso tipi di business per poterli trasmettere alla business logic che ci compie le operazioni di dominio. I dati ricevuti in output dalla business logic
vengono poi adattati verso tipi di DTO per poterli trasmettere all'applicazione esterna.

\hypertarget{sec:applicazione_web}{}
\subsection*{Applicazione web}
Applicazione accessibile via web tramite un browser (Safari, Chrome, Firefox, Edge etc.) e può funzionare sia da mobile che da desktop. Si tratta di un'architettura 
tipicamente di tipo client-server, che offre determinati servizi all'utente.

\hypertarget{sec:architettura}{}
\subsection*{Architettura}
In ambito informatico, l'organizzazione di base di un sistema, espressa dai suoi componenti, dalle relazioni tra di loro e con l'ambiente, 
e dai principi che ne guidano il progetto e l'evoluzione. Essa comprende le strutture del sistema, necessarie per ragionare su di esso, 
che includono elementi software, le relazioni tra di essi e le loro proprietà.

\hypertarget{sec:architettura_esagonale}{}
\subsection*{Architettura esagonale}
Pattern architetturale che promuove la separazione delle responsabilità all'interno di un'applicazione, organizzando il codice in base a un'architettura
a esagono. Questo approccio favorisce la modularità, la manutenibilità e la scalabilità del software, consentendo di isolare le componenti del sistema e
di ridurre le dipendenze tra di esse.

\hypertarget{sec:architettura_monolitica}{}
\subsection*{Architettura monolitica}
Modello architetturale in cui un'applicazione software è progettata come un'unica unità monolitica, in cui tutte le funzionalità e i componenti sono
raggruppati insieme e distribuiti come un'entità unica. Questo approccio è caratterizzato da una forte interdipendenza tra i moduli, rendendo il sistema
più difficile da scalare e mantenere. Tuttavia, l'architettura monolitica è spesso più semplice da sviluppare e testare rispetto ad altre architetture
più complesse.

\hypertarget{sec:attore}{}
\subsection*{Attore}
Nel contesto dell'analisi dei requisiti e del design del software, il termine attore rappresenta chiunque interagisca con il sistema. Gli attori possono 
essere utenti, altri sistemi o qualsiasi altra entità che abbia un ruolo nelle interazioni con il sistema modellato.

\subsection*{AzzurroDigitale}
AzzurroDigitale è una società italiana con sede a Padova. Si occupa di digitalizzazione di processi sia con prodotti proprietari che di terze parti, e 
ha l’obiettivo di accompagnare le aziende manifatturiere nella transizione 5.0.

\newpage


% Intestazione
\fancyhead[L]{B} % Testo a sinistra

\section{}

\hypertarget{sec:back-end}{}
\subsection*{Back-end}
Termine che si riferisce alla parte di un'applicazione web o di un sistema software che gestisce le funzioni e le 
elaborazioni necessarie per rendere possibile l'utilizzo dell'applicazione stessa. Esso comprende tutti i componenti, 
i server e i sistemi di archiviazione dati che non sono accessibili direttamente dall'utente finale, ma che lavorano 
dietro le quinte per fornire funzionalità e dati al frontend. Normali attività gestite dal back-end includono la 
gestione dei database, l'elaborazione delle richieste dei client, la logica di business e la sicurezza 
dell'applicazione.

\subsection*{Backlog}
Insieme di compiti/attività da completare per un certo obiettivo. All’interno del framework Scrum, ne esistono due tipi principali: il product backlog, 
che è la lista delle funzionalità da implementare, e lo sprint backlog, che contiene le attività da svolgere durante un particolare sprint.
Un’attività interna al backlog porta valore ad un progetto perchè possiede:
\begin{itemize}
    \item Stato, che segnala se l’attività è stata completata, in corso o non ancora iniziata.
    \item Priorità, che indica l’importanza dell’attività rispetto alle altre.
    \item Assegnatario, cioè una persona incaricata a svolgere l’attività. Questa assegnazione non è vincolante, infatti se un membro del team ha terminato 
    la sua attività può prendersi a carico un’altra attività presente nel backlog anche se non era stata inizialmente assegnata a lui.
    \item Scadenza, cioè un termine temporale entro il quale l’attività deve essere svolta.
\end{itemize}

\hypertarget{sec:baseline}{}
\subsection*{Baseline}
Nel contesto dell'ingegneria del software, stato di avanzamento che rappresenta un insieme di punti di arrivo che ci si pone come obiettivo di raggiungere 
in una milestone, dimostrando che l'incremento delle modifiche condotte ha portato a un risultato.

\subsection*{Best Practices}
Nello sviluppo software, metodologie che attraverso l’esperienza e la sperimentazione sono state identificate come modi efficaci e raccomandati di 
affrontare determinati problemi o compiti nel processo di sviluppo del software. Queste pratiche sono considerate migliori (best) perché hanno dimostrato 
di portare a risultati di alta qualità, facilitando la manutenzione del codice e promuovendo una migliore collaborazione nel team di sviluppo.

\hypertarget{sec:Bot}{}
\subsection*{Bot}
Un bot è un software progettato per automatizzare attività ripetitive o interagire con gli utenti. 
Funziona seguendo regole predefinite o usando algoritmi di intelligenza artificiale.

\hypertarget{sec:branch}{}
\subsection*{Branch}
Letteralmente "ramo", indica un’entità che si sviluppa o si dirama da un punto principale. Nel contesto di un sistema di controllo delle versioni, 
un branch rappresenta una linea di sviluppo separata. Può essere utilizzato per sviluppare nuove funzionalità, risolvere bug o implementare modifiche 
senza influenzare direttamente il ramo principale del codice, noto come master o main.

\hypertarget{sec:browser}{}
\subsection*{Browser}
Applicazione software progettata per consentire agli utenti di navigare in Internet, visualizzare pagine web e accedere a contenuti online.

\hypertarget{sec:build}{Build}
\subsection*{Build}
Definito anche come costruzione, è il processo di compilazione di un progetto software, in cui il codice sorgente viene trasformato in un 
formato eseguibile. La build può includere una serie di attività come la compilazione del codice sorgente, la creazione di file di 
configurazione e la generazione di file di installazione o di pacchetti per la distribuzione del software. Normalmente viene automatizzato 
tramite l’utilizzo di strumenti affidabili e riproducibili, gestendo le dipendenze del progetto e automatizzando il processo di rilascio e 
il controllo di versione.

\hypertarget{sec:diagramma_di_burndown}{Burndown (diagramma di)}
\subsection*{Burndown, diagramma di}
Strumento grafico utilizzato nella gestione agile dei progetti per tracciare la quantità di lavoro rimanente nel tempo. 
Esso mostra la diminuzione progressiva (“burn down”) delle attività o dei punti stima rimanenti nel corso del tempo, 
consentendo al team di progetto di valutare il proprio progresso e adattare la pianificazione in base alle esigenze. 
A differenza del diagramma di Gantt, il diagramma di Burndown si concentra sulla visualizzazione dell’avanzamento reale rispetto al piano temporale.

\hypertarget{sec:business_logic}{}
\subsection*{Business logic}
Parte di un'applicazione software che implementa le regole di business e la logica di funzionamento del sistema. La business logic è responsabile
dell'elaborazione dei dati, della gestione delle regole aziendali e delle operazioni di calcolo necessarie per il funzionamento dell'applicazione.


\newpage


% Intestazione

\fancyhead[L]{C} % Testo a sinistra

\section{}

\hypertarget{sec:Camel Case}{}
\subsection*{Camel Case}
È una convenzione di scrittura usata nella programmazione per nominare variabili, funzioni e altri identificatori. 
Si distingue perché la prima parola inizia con una lettera minuscola, mentre le parole successive iniziano con una lettera maiuscola, senza spazi o separatori.

\hypertarget{sec:capitolato}{}
\subsection*{Capitolato}
Documento privato tra chi commissiona il lavoro e il gruppo (ditta) che lo esegue, in cui viene esposto un problema che il proponente necessita di risolvere 
e specifica le norme e i vincoli da rispettare per lo sviluppo del prodotto software specifico.

\hypertarget{sec:caso_uso}{}
\subsection*{Caso d'uso}
Descrizione dettagliata di come un utente (attore) interagisce con l'applicazione per il compimento di un'attività specifica. È uno strumento utilizzato nel 
contesto dello sviluppo software per individuare i requisiti funzionali del prodotto e per fornire una visuale chiara delle interazioni che possono avvenire 
all'interno dell'applicazione

\hypertarget{sec:chatgpt}{}
\subsection*{ChatGPT}
ChatGPT è un modello di intelligenza artificiale sviluppato da OpenAI, basato sulla famiglia di modelli GPT (Generative Pre-trained Transformer). 
È progettato per comprendere e generare testo in linguaggio naturale, rendendolo utile in numerosi scenari, come chatbot, assistenti virtuali, generazione 
di contenuti e risposte automatizzate. ChatGPT utilizza una vasta base di conoscenza pre-addestrata e può essere ulteriormente personalizzato per 
applicazioni specifiche, offrendo interazioni conversazionali fluide e contestualmente rilevanti.

\subsection*{Checklist}
Lista dettagliata di elementi, attività o criteri specifici che devono essere controllati, esaminati o completati durante le diverse fasi del ciclo di vita 
del software. E' utilizzata come strumento di controllo e verifica.

\hypertarget{sec:chroma}{}
\subsection*{Chroma}
Chroma è un database open-source ottimizzato per la gestione di dati vettoriali, progettato per supportare applicazioni di intelligenza artificiale e 
machine learning. È utilizzato principalmente per il retrieval di informazioni basato su similarità, come la ricerca di embedding, e integra funzionalità 
avanzate per lavorare con modelli di linguaggio (LLM). Fornisce un'API semplice per memorizzare, indicizzare e interrogare dati multidimensionali, 
rendendolo adatto a scenari come motori di raccomandazione, sistemi di domande e risposte, o clustering.

\hypertarget{sec:ciclo_di_vita}{}
\subsection*{Ciclo di vita del software}
Serie di fasi attraverso le quali un software passa dal suo concepimento iniziale fino al suo ritiro o dismissione. È un concetto chiave nell'ingegneria 
del software e fornisce una struttura organizzativa per il processo di sviluppo del software.

\hypertarget{sec:codifica}{}
\subsection*{Codifica}
Fase del processo di sviluppo software in cui gli sviluppatori traducono i requisiti e il design del sistema in linguaggio di programmazione, creando il 
codice sorgente. Durante questa fase, gli sviluppatori seguono gli standard di codifica e le buone prassi per assicurare la leggibilità, l’efficienza e la 
manutenibilità del codice.

\hypertarget{sec:committente}{}
\subsection*{Committente}
Nell’ambito dell’ingegneria del software, il termine committente si riferisce alla persona, all’organizzazione o all’entità che richiede e 
finanzia lo sviluppo di un sistema software. Il committente è colui che ha un interesse diretto nel progetto e assume il ruolo
di chi determina le aspettative per il prodotto software finale. Il committente svolge un ruolo chiave nel processo di sviluppo del software, 
in quanto fornisce la visione iniziale del progetto, identifica le esigenze degli utenti finali e stabilisce i criteri di successo.

\hypertarget{sec:componente software}{}
\subsection*{Componente software}
Unità logica e funzionale di un sistema software, progettata per svolgere specifiche funzioni o compiti. Le componenti software sono i mattoni
fondamentali di un'applicazione, e possono essere riutilizzate, combinate e integrate per creare sistemi più complessi.

\hypertarget{sec:confluence}{}
\subsection*{Confluence}
Piattaforma di collaborazione e gestione della conoscenza, utilizzata per creare, condividere e collaborare su documenti, progetti e informazioni all'interno 
di un team o di un'organizzazione. Confluence offre funzionalità come la creazione di wiki aziendali, la gestione di progetti, la documentazione tecnica e la 
collaborazione in tempo reale.

\hypertarget{sec:consuntivo}{}
\subsection*{Consuntivo}
Bilancio dei risultati ottenuti a rendiconto di un certo periodo temporale di attività, in termini di tempo e risorse.

\hypertarget{sec:contesto_applicativo}{}
\subsection*{Contesto applicativo}
Ambito o scenario in cui un'applicazione software è progettata per essere utilizzata. Il contesto applicativo definisce le condizioni, le esigenze e le
caratteristiche specifiche dell'ambiente in cui l'applicazione deve operare, influenzando il design, le funzionalità e le prestazioni del software.

\hypertarget{sec:controller}{}
\subsection*{Controller}
Componente di un'applicazione web che gestisce le richieste degli utenti, coordina il flusso di dati e controlla il comportamento degli altri componenti.
Il controller è responsabile di interpretare le azioni dell'utente, di invocare i servizi necessari e di restituire una risposta appropriata all'interfaccia utente.

\hypertarget{sec:controllo_versione}{}
\subsection*{Controllo di versione}
Strumento che consente di gestire e tracciare le modifiche apportate al codice sorgente o ad altri file di progetto.

\hypertarget{sec:cron}{}
\subsection*{Cron}
Nel contesto informatico, cron è un demone (un processo che si esegue in background) presente in molti sistemi operativi Unix-like 
(come Linux e macOS) che permette di pianificare l'esecuzione di comandi a intervalli di tempo regolari o a date e orari specifici.

\hypertarget{sec:css}{}
\subsection*{CSS}
Acronimo di Cascading Style Sheets, linguaggio di stile utilizzato per definire l’aspetto e lo stile delle pagine web HTML. CSS 
permette di separare il contenuto della pagina dalla sua presentazione, consentendo un maggiore controllo sulla formattazione e lo 
stile degli elementi della pagina. Con questo possono essere definiti degli stili, i quali possono definire proprietà come il colore, 
il font, la dimensione, la posizione e l’animazione.

\newpage
% Intestazione
\fancyhead[L]{D} % Testo a sinistra

\section*{D}
\addcontentsline{toc}{section}{D}

\subsection*{Diagramma di Gantt}
Strumento di visualizzazione temporale utilizzato nella gestione dei progetti per rappresentare le attività pianificate nel tempo. È composto da una barra 
orizzontale che rappresenta l’arco temporale totale del progetto e da barre orizzontali più piccole che rappresentano le singole attività del progetto. 
Ogni barra è posizionata lungo l’asse temporale in base alle date di inizio e fine previste per l’attività.

\subsection*{Diagramma UML}
Acronimo di Unified Modeling Language, un diagramma UML diagramma utilizzato per modellare, descrivere e visualizzare sistemi software e processi di sviluppo 
software. È uno standard industriale nel campo dell’ingegneria del software e fornisce una serie di diagrammi, ognuno dei quali si concentra su un aspetto 
specifico del sistema o del processo. Sono diagrammi UML ad esempio i diagrammi dei casi d’uso, i diagrammi delle classi e i diagrammi delle funzionalità.

\subsection*{Diagramma UML dei casi d'uso}
Diagramma UML che rappresenta le interazioni tra utenti (attori) e il sistema, descrivendo come gli utenti utilizzano il sistema per raggiungere obiettivi 
specifici. Ogni caso d'uso rappresenta una funzione o un'attività significativa, utile per descrivere i requisiti funzionali. È spesso il primo passo nella 
progettazione di un sistema software e aiuta a identificare le funzioni principali e il modo in cui il sistema interagisce con gli utenti.

\subsection*{Diagramma UML delle classi}
Diagramma UML che descrive la struttura statica di un sistema, mostrando le classi, i loro attributi, i metodi e le relazioni tra di esse (come ereditarietà, 
associazioni e aggregazioni). Questo diagramma è fondamentale per la programmazione orientata agli oggetti poiché fornisce una rappresentazione visiva della 
struttura del codice, aiutando a comprendere le interconnessioni tra le varie entità e a definire i componenti principali.

\subsection*{Discord}
Piattaforma VoIP (Voice over IP: tecnologia che rende possibile effettuare una conversazione sfruttando una connessione internet), messaggistica istantanea 
e distribuzione digitale progettata per la comunicazione.

\subsection*{Draw.io}
Draw.io (ora chiamato diagrams.net) è uno strumento gratuito per la creazione di diagrammi, disponibile sia come applicazione web che come app desktop per 
vari sistemi operativi. Viene utilizzato ampiamente per progettare e documentare diagrammi di flusso, architetture software, diagrammi UML, organigrammi, 
mappe mentali, wireframe, e altri tipi di rappresentazioni visive.

\newpage


% Intestazione
\fancyhead[L]{E} % Testo a sinistra

\section*{E}
\addcontentsline{toc}{section}{E}

\dots

\newpage


% Intestazione
\fancyhead[L]{F} % Testo a sinistra

\section*{F}
\addcontentsline{toc}{section}{F}

\subsection*{Fogli Google}
Fogli Google è un'applicazione web di Google, parte della suite di produttività Google Workspace, che consente di creare, modificare e condividere fogli 
di calcolo online. È uno strumento particolarmente apprezzato per il suo accesso immediato da browser, le funzionalità collaborative in tempo reale, e 
l'integrazione con altri servizi Google. Così come ogni foglio di calcolo, include la possibilità di creare grafici e diagrammi basati su dati tabellari.

\newpage
% Intestazione
\fancyhead[L]{G} % Testo a sinistra

\section{}

\subsection*{Gantt, diagramma di}
Vedi \bulhyperlink{sec:diagramma_Gantt}{Diagramma di Gantt}.

\hypertarget{sec:git}{}
\subsection*{Git}
Software per il controllo di versione distribuito utilizzabile tramite interfaccia a riga di comando.

\hypertarget{sec:git_flow}{}
\subsection*{Git Flow}
Git Flow è un modello di branching di Git che definisce un insieme di regole e procedure per gestire il ciclo di vita di un progetto 
software. Offre una struttura ben definita per la collaborazione tra sviluppatori e la gestione dei rilasci, garantendo un flusso di 
lavoro efficiente e organizzato.

\subsection*{GitHub}
Servizio di hosting per progetti software. Il sito è principalmente utilizzato da sviluppatori che caricano il codice sorgente di programmi in dei 
repository e lo rendono scaricabile e migliorabile da altri sviluppatori. Questi ultimi possono interagire con i proprietari dei repository tramite un 
sistema per inviare segnalazioni di bug o richieste di funzionalità (issue tracker), un sistema per copiare il software in una versione modificabile 
(fork), un sistema per proporre modifiche agli sviluppatori originali (pull request) e un sistema di discussione legato al codice del repository (commenti).

\subsection*{GitHub Pages}
GitHub Pages è un servizio di hosting gratuito offerto da GitHub che permette agli utenti di creare e pubblicare facilmente siti web statici direttamente 
dai loro repository GitHub. È utilizzato comunemente per creare siti di documentazione, pagine personali o di progetto, blog e portali di portfolio. 
GitHub Pages è particolarmente apprezzato perché permette di ospitare un sito senza costi e con aggiornamenti automatici ogni volta che il repository 
viene modificato.

\subsection*{GitHub Projects}
GitHub Projects è uno strumento di gestione dei progetti integrato in GitHub, ideato per aiutare sviluppatori e team a organizzare, pianificare e tracciare 
il lavoro sui progetti direttamente all'interno dell'ambiente GitHub. Si basa su un sistema flessibile di "project board" simile a Kanban, che offre un modo 
visuale per coordinare i task e monitorare l’avanzamento del lavoro. GitHub Projects è uno strumento ideale per team che lavorano su progetti di sviluppo 
software, in quanto permette di gestire l’intero processo di sviluppo all'interno di GitHub stesso. Grazie alla sua integrazione nativa con il codice, le 
issues e le pull requests, aiuta a mantenere sincronizzati i task e a ridurre il contesto di cambiamento per gli sviluppatori, migliorando il flusso di 
lavoro e facilitando la collaborazione su GitHub.

\subsection*{Glossario}
Elenco organizzato di termini tecnici, acronimi e definizioni utilizzati nel contesto del progetto. Questo documento fornisce una chiara comprensione dei 
concetti e dei linguaggi specifici impiegati nel progetto, aiutando a ridurre ambiguità e fraintendimenti tra i membri del team e gli stakeholder.

\hypertarget{sec:google_chrome}{}
\subsection*{Google Chrome}
Browser web che consente di accedere a Internet e interagire con siti e applicazioni online.


\hypertarget{sec:google_meet}{}
\subsection*{Google Meet}
Servizio di videoconferenza sviluppato da Google. È parte di Google Workspace e offre funzionalità di videochiamata, chat, condivisione d
ello schermo e registrazione delle riunioni. Google Meet è ampiamente utilizzato per incontri di lavoro, lezioni online, webinar e conferenze, 
grazie alla sua facilità d’uso, alla stabilità della connessione e alla possibilità di partecipare senza dover scaricare alcun software.

\subsection*{Gulpease, indice di}
Indice di leggibilità di un testo specificamente in lingua italiana, che utilizza il numero delle parole, delle frasi e delle lettere per facilitare il 
calcolo automatico della leggibilità.

\newpage


% Intestazione
\fancyhead[L]{H} % Testo a sinistra

\section{}

\hypertarget{sec:hosting}{}
\subsection*{Hosting}
Servizio che fornisce l’infrastruttura necessaria per rendere accessibili siti web, applicazioni o dati attraverso Internet.

\hypertarget{sec:html}{}
\subsection*{HTML}
Acronimo di HyperText Markup Language, linguaggio utilizzato per la creazione e l’impaginazione di documenti ipertestuali disponibili 
sul web. Consente di strutturare il contenuto delle pagine web utilizzando tag e attributi, permettendo la visualizzazione di testo, 
immagini, link e altri elementi multimediali nei browser web.

\newpage


% Intestazione
\fancyhead[L]{I} % Testo a sinistra

\section{}

\hypertarget{sec:ia}{}
\subsection*{IA}
Acronimo di intelligenza artificiale

\hypertarget{sec:ide}{}
\subsection*{IDE}
Acronimo di Integrated Development Environment, è un ambiente di sviluppo, ovvero un software che, in fase di programmazione, supporta i programmatori 
nello sviluppo e debugging del codice sorgente di un programma, segnalando errori di sintassi del codice direttamente in fase di scrittura, oltre a fornire 
una serie di strumenti e funzionalità di supporto alla fase stessa di sviluppo e debugging.

\hypertarget{sec:modello_incrementale}{Incrementale (modello di sviluppo)}
\subsection*{Incrementale, modello di sviluppo}
Approccio alla creazione di software che suddivide il progetto in piccoli segmenti o incrementi. 
Ogni incremento rappresenta una versione funzionante del software che include nuove funzionalità o miglioramenti rispetto alla versione precedente.

\hypertarget{sec:implementazione}{}
\subsection*{Implementazione}
L'implementazione è il processo mediante il quale un progetto, un piano o un sistema viene realizzato e reso operativo, traducendo 
specifiche o requisiti in soluzioni concrete e funzionanti. In termini generali, implica la costruzione, l'integrazione e la 
configurazione degli elementi necessari per ottenere il risultato desiderato. Oltre a ciò, l'implementazione include la responsabilità 
di restituire un utensile usabile ai fini attesi, ossia fornire un prodotto che soddisfi le esigenze per cui è stato progettato, 
assicurandosi che sia funzionale, accessibile e adatto all'uso previsto dagli utenti finali.

\hypertarget{sec:intelligenza_artificiale}{}
\subsection*{Intelligenza artificiale}
Disciplina che studia come realizzare sistemi informatici in grado di simulare il pensiero umano. Sistemi basati su di essa hanno dunque l'abilità di mostrare 
capacità umane quali il ragionamento, l'apprendimento, la pianificazione e la creatività.

\hypertarget{ISO/IEC 31000:2018}{}
\subsection*{ISO/IEC 31000:2018}
La norma ISO 31000 "Risk management - Principles and guidelines", 
in italiano UNI ISO 31000 Gestione del rischio - Principi e linee guida. 
È una guida che fornisce principi e linee guida generali per la gestione del rischio. 
Può essere utilizzata da qualsiasi organizzazione pubblica, privata o sociale, associazione, gruppo o individuo, e non è specifica per nessuna industria o settore.
La ISO 31000 può essere applicata nel corso dell'intero ciclo di vita di un'organizzazione, ed essere adottata per molte attività come la definizione di strategie e decisioni, operazioni, processi, funzioni, progetti, prodotti, servizi e beni.
Può inoltre essere applicata a qualsiasi tipo di rischio, sia per conseguenze di tipo positivo che negativo. 

\hypertarget{ISO/IEC 9126}{}
\subsection*{ISO/IEC 9126}
La norma ISO/IEC 9126 è uno standard internazionale per la valutazione della qualità del software.
Definisce un modello di qualità software che si basa su sei caratteristiche generali di qualità (funzionalità, affidabilità, usabilità, efficienza,
manutenibilità, portabilità) e su 27 sotto-caratteristiche.
Lo standard è stato sostituito dalla norma ISO/IEC 25010:2011, che definisce un modello di qualità software più ampio e aggiornato.

\hypertarget{ISO/IEC/IEEE 12207}{}
\subsection*{ISO/IEC/IEEE 12207}
Lo standard ISO 12207 stabilisce un processo di ciclo di vita del software, compreso processi ed attività relative alle specifiche ed alla configurazione di un sistema.
Ad ogni processo corrisponde un insieme di risultati (outcome): in totale ci sono 43 processi, 133 attività, 325 sottoattività e 236 risultati (la nuova ISO/IEC 12207:2008 definisce 43 sistemi e processi software).
Lo standard ha come obiettivo principale quello di fornire una struttura comune che permetta a clienti, fornitori, sviluppatori, tecnici, manager di usare gli stessi termini e lo stesso linguaggio per definire gli stessi processi.
La struttura dello standard è stata concepita per essere flessibile e modulare in modo che sia adattabile alle necessità di chiunque lo utilizzi.

\subsection*{Issue}
Una issue su GitHub (e altre piattaforme di gestione del codice e dei progetti) è un elemento utilizzato per tracciare problemi, richieste di funzionalità, 
idee o miglioramenti relativi a un progetto. È uno strumento fondamentale per organizzare il lavoro collaborativo e garantire che tutti i membri del team 
siano aggiornati sui task e le priorità. Ogni issue rappresenta un singolo elemento che richiede attenzione o azione, e fornisce un luogo centralizzato per 
discuterlo, seguirlo e risolverlo.

\hypertarget{sec:issue_tracking_system}{Issue Tracking System (ITS)}
\subsection*{Issue Tracking System (ITS)}
Un Issue Tracking System (ITS) è un software o un sistema di gestione che consente di creare, monitorare e gestire issue 
(problemi, task, richieste o bug) all'interno di un progetto. Questo strumento è utilizzato principalmente per facilitare la collaborazione 
tra membri del team e per garantire che tutte le problematiche e le richieste siano gestite in modo organizzato, trasparente e tracciabile. 
Un ITS è spesso impiegato da team di sviluppo software, ma può essere utile anche in contesti di supporto clienti, gestione di progetto, e 
qualsiasi ambito dove occorra tracciare richieste di lavoro o problemi.

\newpage
% Intestazione
\fancyhead[L]{J} % Testo a sinistra

\section{}

\subsection*{Jira}
Jira è uno strumento di gestione dei progetti e di issue tracking sviluppato da Atlassian, ampiamente utilizzato per la pianificazione, il monitoraggio e 
il controllo di progetti, in particolare nell'ambito dello sviluppo software. Nato come strumento di gestione dei bug e dei problemi, Jira è diventato uno 
dei principali strumenti per il project management, soprattutto per le organizzazioni che adottano metodologie agili come Scrum e Kanban.

\newpage


% Intestazione
\fancyhead[L]{K} % Testo a sinistra

\section{}

\dots

\newpage


% Intestazione
\fancyhead[L]{L} % Testo a sinistra

\section{}

\hypertarget{sec:langchain}{}
\subsection*{LangChain}
LangChain è un framework Python e JavaScript open-source progettato per sviluppare applicazioni che sfruttano i Large Language Models 
(LLM). Offre una struttura modulare e flessibile per la creazione di catene di elaborazione (chains) che combinano modelli di 
linguaggio con altre fonti di dati e strumenti.

\subsection*{\LaTeX}
Linguaggio di marcatura per la preparazione di testi, basato sul programma di composizione tipografica TeX. LaTeX è ampiamente utilizzato per la creazione 
di documenti scientifici e tecnici grazie alla sua capacità di gestire formule matematiche complesse e alla sua alta qualità tipografica.

\hypertarget{sec:LLM}{}
\subsection*{Large Language Model (LLM)}
Il termine Large Language Model si riferisce a modelli di linguaggio avanzati e complessi che sono stati addestrati su enormi quantità di dati testuali. 
Questi modelli, basati su tecniche di intelligenza artificiale come il deep learning, sono in grado di comprendere e generare testo in linguaggio naturale 
in modo più sofisticato rispetto a modelli più piccoli (Small Language Model, SML).

\hypertarget{sec:logging}{}
\subsection*{Logging}
Il processo di registrazione di eventi, errori e altre informazioni rilevanti all'interno di un'applicazione. Questi dati vengono 
solitamente salvati in un file di log, un database o inviati a un servizio esterno per analisi successive. Il logging è fondamentale 
per la risoluzione dei problemi, il monitoraggio delle prestazioni e la sicurezza delle applicazioni.

\newpage
% Intestazione
\fancyhead[L]{M} % Testo a sinistra

\section{}

\hypertarget{sec:merge}{}
\subsection*{Merge}
Operazione fondamentale nei sistemi di controllo delle versioni. Essa è utilizzata per combinare le modifiche apportate in due branch separati in un 
singolo branch.

\hypertarget{sec:metrica}{}
\subsection*{Metrica}
Misura quantitativa utilizzata per valutare, quantificare e analizzare diversi aspetti del processo di sviluppo del software, del prodotto software stesso o della gestione del
progetto. Le metriche forniscono dati numerici che consentono di valutare l’andamento del
progetto, la qualit`a del software, l’efficacia dei processi e altri aspetti rilevanti.

\subsection*{Milestone}
In ingegneria del software e nella gestione dei progetti, punto di riferimento o traguardo significativo che sancisce il termine di un periodo nel ciclo 
di vita di un progetto. Le milestone rappresentano generalmente eventi chiave, compimenti o obiettivi importanti che indicano il progresso del progetto. 
L’obiettivo che ci si pone durante una milestone è realizzare una baseline.

\hypertarget{sec:MVP}{}
\subsection*{Minimum Viable Product (MVP)}
Versione ridotta di un prodotto, che incorpora solo le funzioni essenziali per soddisfare le esigenze di base. Viene utilizzato per rilasciare un prodotto 
come test e ricevere feedback dall’utenza per migliorare poi il prodotto finito con tutte le funzionalità.

\newpage


% Intestazione
\fancyhead[L]{N} % Testo a sinistra

\section{}

\hypertarget{sec:nestjs}{}
\subsection*{NestJS}
Framework Node.js per lo sviluppo di applicazioni lato server, basato su TypeScript. Utilizza un'architettura modulare e concetti come iniezione delle 
dipendenze e decoratori, ispirandosi a framework come Angular, per creare applicazioni scalabili e facilmente mantenibili.

\hypertarget{sec:nodejs}{}
\subsection*{Node.js}
Runtime JavaScript open-source e multipiattaforma, progettato per l'esecuzione di codice JavaScript lato server. Basato sul motore V8 di Google Chrome, 
permette di creare applicazioni veloci, scalabili e basate su eventi, sfruttando un modello non bloccante per la gestione delle operazioni I/O.

\subsection*{Norme di Progetto}
Insieme di linee guida, procedure e regole stabilite per regolare e standardizzare l’approccio, il processo e l’output del lavoro all’interno del progetto. 
Queste norme possono riguardare diversi aspetti del progetto, come la gestione del codice, la documentazione, la comunicazione e la gestione dei rischi. 
L’obiettivo delle norme di progetto è promuovere la coerenza, la qualità e l’efficienza nel processo di sviluppo del software, consentendo al team di 
lavorare in modo più efficace e collaborativo.

\newpage


% Intestazione
\fancyhead[L]{O} % Testo a sinistra

\section{}

\hypertarget{sec:onboarding}{}
\subsection*{Onboarding}
Il processo di integrazione e formazione di un nuovo dipendente, collaboratore o utente in un'organizzazione o piattaforma. L'onboarding ha l'obiettivo di 
fornire tutte le informazioni e le risorse necessarie per adattarsi al nuovo ambiente di lavoro, comprendere la cultura aziendale, acquisire competenze 
specifiche e diventare operativi nel più breve tempo possibile. 


\newpage
% Intestazione
\fancyhead[L]{P} % Testo a sinistra

\section{}

\subsection*{Piano di Progetto}
Documento formale che delinea in dettaglio la pianificazione, l’esecuzione, il monitoraggio e il controllo di tutte le attività coinvolte nella 
realizzazione di un progetto. Questo documento fornisce una roadmap chiara e organizzata, comprensiva di obiettivi, risorse, scadenze e strategie di 
gestione dei rischi. Essenziale per la gestione efficace di un progetto, il piano di progetto serve come guida per il team di lavoro e gli stakeholder, 
fornendo una struttura che facilita il coordinamento delle attività e l’assegnazione delle risorse.

\subsection*{Piano di Qualifica}
Documento che stabilisce gli standard di qualità, i processi e le attività di testing che saranno implementati durante lo sviluppo di un progetto. 
Contiene una descrizione dettagliata delle strategie di testing, delle metriche di valutazione e dei criteri di accettazione del prodotto finale. 
L’obiettivo principale del Piano di Qualifica è garantire che il prodotto soddisfi gli standard di qualità prefissati e che il processo di sviluppo 
segua procedure coerenti ed efficaci.

\hypertarget{sec:preventivo}{Preventivo}
\subsection*{Preventivo}
Stima dei costi e delle risorse necessarie per completare un determinato lavoro o progetto.

\hypertarget{sec:processo}{}
\subsection*{Processo}
Insieme strutturato di attività necessarie per lo sviluppo di un sistema software.

\hypertarget{sec:PB}{}
\subsection*{Product Baseline (PB)}
E la seconda revisione di avanzamento del progetto didattico. Comprende un prodotto software con design definitivo, 
chiamato \emph{Minimum Viable Product (MVP)}\textsubscript{\textit{\textbf{G}}}.

\hypertarget{sec:PoC}{}
\subsection*{Proof of Concept (PoC)}
Versione preliminare di un’applicazione o di una soluzione software che viene sviluppata per dimostrare la fattibilità tecnica di un’idea o di un concetto. 
Viene utilizzata per testare rapidamente l’efficacia di un approccio, identificare eventuali limitazioni delle tecnologie scelte e valutare se l’idea può 
essere realizzata in modo pratico.

\hypertarget{sec:proponente}{}
\subsection*{Proponente}
Nel contesto dell’ingegneria del software, colui che presenta un’idea, un progetto o una proposta e ne sostiene la realizzazione. Il gruppo \emph{SWEg Labs} 
ha come proponente l’azienda \emph{AzzurroDigitale}\textsubscript{\textit{\textbf{G}}}.

\newpage


% Intestazione
\fancyhead[L]{Q} % Testo a sinistra

\section{}

\hypertarget{sec:Qualità}{}
\subsection*{Qualità}
Insieme delle caratteristiche di un’entità che ne determinano la capacità di soddisfare esigenze sia espresse che implicite. Si parla di qualità del prodotto software in termini di:
\begin{itemize}
    \item \textbf{Qualità Intrinseca}: conformità ai requisiti, idoneità all’uso;
    \item \textbf{Qualità Relativa}: soddisfazione del cliente;
    \item \textbf{Qualità Quantitativa}: misurazione oggettiva, per confronto.
\end{itemize}


\newpage


% Intestazione
\fancyhead[L]{R} % Testo a sinistra

\section{}

\subsection*{Repository}
In termini informatici, un luogo o un archivio dove vengono conservati e gestiti dati, documenti o, nel contesto del software, il codice sorgente di un 
progetto. Nell'ambito dei sistemi di controllo delle versioni come Git, un repository è una struttura dati che archivia anche la cronologia completa delle 
modifiche apportate al codice sorgente di un progetto.

\subsection*{Requirement and Technology Retrospective (RTB)}
La prima revisione di avanzamento del progetto didattico. Fissa i requisiti da soddisfare in accordo con il proponente, motiva le tecnologie, i framework 
e le librerie adottate dimostrandone adeguatezza e compatibilità tramite il Proof of Concept (PoC).

\subsection*{Retrospettiva}
Vedi \emph{Sprint Retrospective}\textsubscript{\textit{\textbf{G}}}.

\newpage
% Intestazione
\fancyhead[L]{S} % Testo a sinistra

\section{}

\hypertarget{sec:scalabilità}{}
\subsection*{Scalabilità}
La scalabilità è la capacità di un sistema, di un'applicazione o di un'infrastruttura di crescere e adattarsi in modo flessibile e affidabile per gestire un
aumento della domanda o del carico di lavoro. La scalabilità può essere orizzontale (aggiungendo più risorse o nodi) o verticale (aggiungendo più potenza
di elaborazione o risorse a un singolo nodo) e può essere progettata per essere elastica, automatica o manuale in base alle esigenze del sistema.

\hypertarget{sec:scenario_alternativo}{}
\subsection*{Scenario alternativo}
Uno scenario alternativo nei casi d'uso è una sequenza di azioni o eventi che deviano dal flusso principale a causa di condizioni o circostanze specifiche. 
Questi scenari descrivono come il sistema dovrebbe comportarsi quando si verificano situazioni diverse rispetto al percorso standard, garantendo che tutte 
le possibilità siano considerate nel processo di progettazione.

\hypertarget{sec:scenario_uso}{}
\subsection*{Scenario d'uso (scenario principale)}
Uno scenario d'uso è una descrizione narrativa di come un utente interagisce con un sistema software per raggiungere un determinato obiettivo. Rappresenta 
una sequenza di azioni e reazioni tra l'utente e il sistema in un contesto specifico e fornisce una visione dettagliata del comportamento atteso del software 
in una situazione reale.

\hypertarget{sec:scheduler}{}
\subsection*{Scheduler}
Uno scheduler è un componente software che gestisce l'esecuzione di processi, attività o operazioni in un sistema informatico. Può essere utilizzato per
programmare l'avvio e la terminazione di processi, assegnare risorse, gestire la priorità delle attività e garantire che le operazioni vengano eseguite in
modo efficiente e sincronizzato. Gli scheduler sono ampiamente utilizzati nei sistemi operativi, nei database, nei server web e in altre applicazioni per
gestire il tempo e le risorse in modo ottimale.

\subsection*{Script}
Uno script è un insieme di istruzioni scritte in un linguaggio di scripting (come Python, JavaScript, Bash, ecc.) progettato per automatizzare compiti 
specifici o per eseguire operazioni in modo diretto su un sistema, applicazione o ambiente web. A differenza di un programma complesso, uno script tende 
a essere più leggero e adatto per operazioni mirate e ripetibili, senza richiedere una fase di compilazione lunga o complessa.

\hypertarget{sec:service}{}
\subsection*{Service}
Un service è un componente software che fornisce funzionalità specifiche o servizi a un'applicazione o a un sistema. I service sono progettati per
essere riutilizzabili, modulari e indipendenti dal contesto, consentendo di separare la logica di business, l'accesso ai dati e altre funzionalità
in unità distinte e ben definite. I service sono ampiamente utilizzati nell'architettura orientata ai servizi (SOA), nei microservizi e in altri
approcci di progettazione software per promuovere la modularità, la scalabilità e la manutenibilità del codice.

\hypertarget{sec:similarità}{}
\subsection*{Similarità}
La similarità nell'apprendimento automatico indica quanto due o più elementi (immagini, testi, numeri) sono simili tra loro. 
Questa misura, fondamentale per raggruppare dati simili, classificare nuovi dati e fare raccomandazioni, si basa spesso sul calcolo 
della similarità tra vettori che rappresentano numericamente gli elementi. Esistono diverse misure di similarità, ciascuna adatta a 
specifici tipi di dati e problemi.

\hypertarget{sec:singleton}{}
\subsection*{Singleton}
Il Singleton è un design pattern creazionale che garantisce che una classe abbia una sola istanza e fornisce un punto di accesso globale a tale istanza.
Questo pattern è utile quando si desidera limitare il numero di istanze di una classe e fornire un modo per accedere a quella singola istanza da qualsiasi
parte del programma. Il Singleton è ampiamente utilizzato per la gestione delle risorse condivise, la configurazione globale e la creazione di oggetti
unici in un'applicazione.

\hypertarget{sec:slack}{}
\subsection*{Slack}
Piattaforma di messaggistica istantanea e collaborazione progettata per team e organizzazioni. Permette di comunicare tramite canali tematici, inviare 
messaggi diretti, condividere file, e integrare altre applicazioni e strumenti di lavoro. Slack favorisce la comunicazione in tempo reale, migliorando 
la produttività e la gestione dei progetti, e viene utilizzato soprattutto in ambienti di lavoro agili e collaborativi.

\hypertarget{sec:snake_case}{}
\subsection*{Snake Case}
Lo Snake Case è uno stile di scrittura di nomi in cui le parole sono separate da un trattino basso o underscore (\_) e tutte le lettere sono generalmente 
minuscole (es.: nome\_variabile\_uno). Viene spesso contrapposto allo stile Camel Case, in cui invece le parole successive alla prima vengono concatenate 
con l’iniziale maiuscola (es.: nomeVariabileUno).

\hypertarget{sec:snippet}{}
\subsection*{Snippet}
Uno snippet è un piccolo blocco di codice riutilizzabile che svolge una funzione specifica o rappresenta un esempio di utilizzo di una determinata 
funzionalità. Gli snippet sono utilizzati per risparmiare tempo durante la scrittura del codice, poiché forniscono pezzi di codice già pronti che possono 
essere facilmente inseriti e adattati nel contesto del progetto.

\hypertarget{sec:sottocaso_d'uso}{}
\subsection*{Sottocaso d'uso}
Un Sottocaso d'uso è una specificazione di un caso d'uso che descrive un flusso alternativo o dettagliato di azioni, mantenendo una relazione diretta con 
il caso d'uso principale. Viene utilizzato per suddividere comportamenti complessi in scenari più gestibili, evidenziando variazioni o estensioni del 
flusso principale, pur rimanendo parte integrante del caso d'uso generale.

\hypertarget{sec:specifica_tecnica}{}
\subsection*{Specifica tecnica}
Si tratta di un documento formale con lo scopo di servire da linea guida per gli sviluppatori che andranno ad estendere o mantenere 
il prodotto: al suo interno si trovano tutte le informazioni riguardanti i linguaggi e le tecnologie utilizzate, l’architettura del 
sistema e le scelte progettuali effettuate per il prodotto.

\hypertarget{sec:specifiche_funzionali}{Specifiche funzionali}
\subsection*{Specifiche funzionali}
Le specifiche funzionali sono una descrizione dettagliata delle funzionalità e dei comportamenti che un sistema software deve avere per soddisfare i 
requisiti degli utenti. Le specifiche funzionali definiscono cosa il software deve fare, come deve rispondere a determinate azioni e quali risultati
deve produrre in risposta a determinati input.

\hypertarget{sec:specifiche_tecniche}{Specifiche tecniche}
\subsection*{Specifiche tecniche}
Le specifiche tecniche sono una descrizione dettagliata delle caratteristiche tecniche e delle prestazioni che un sistema software deve avere per
soddisfare i requisiti tecnici e funzionali. Le specifiche tecniche definiscono come il software deve essere progettato, implementato e testato per
garantire che funzioni correttamente e risponda alle esigenze degli utenti.

\hypertarget{sec:spring}{}
\subsection*{Spring}
Framework Java che fornisce un ecosistema modulare per sviluppare applicazioni lato server. Facilita la gestione delle dipendenze, configurazioni, 
sicurezza, accesso ai dati e integrazioni, promuovendo un'architettura scalabile e flessibile basata su principi come l'inversione del controllo (IoC) e 
la programmazione orientata agli aspetti (AOP).

\hypertarget{sec:spring_boot}{}
\subsection*{Spring Boot}
Framework Java che semplifica lo sviluppo di applicazioni Spring, ovvero applicazioni basate sul framework Spring per la gestione di componenti, dipendenze 
e funzionalità lato server come sicurezza, accesso ai dati e integrazione. Spring Boot offre configurazioni predefinite, un server embedded e una struttura 
modulare, permettendo di creare applicazioni pronte all'uso con meno configurazioni manuali.

\hypertarget{sec:SonarQube for IDE}{}
\subsection*{Sonarqube for IDE}
Estensione che integra le funzionalità di analisi della qualità del codice direttamente all'interno degli ambienti di sviluppo (IDE) più diffusi.
Questa integrazione permette agli sviluppatori di rilevare problemi nel codice (come bug, vulnerabilità, code smells e problemi di mantenibilità) in tempo reale.

\hypertarget{sec:Sprint}{}
\subsection*{Sprint}
Periodo di tempo prefissato entro il quale lavorare producendo dei risultati documentati. Gli sprint sono al centro delle metodologie Agile, atte a produrre 
risultati piccoli e in maniera costante.

\hypertarget{sec:sprint_reptrospective}{}
\subsection*{Sprint Retrospective}
Incontro finalizzato ad analizzare l’andamento dello sprint quando questo è terminato, per migliorare la performance futura del team di sviluppo. La 
riunione retrospettiva dello sprint è quindi propedeutica allo sprint successivo.

\hypertarget{sec:stakeholder}{}
\subsection*{Stakeholder}
Individui, gruppi o entità che hanno un interesse o un coinvolgimento in un progetto, un’organizzazione o un’iniziativa specifica. Essi possono influenzare 
o essere influenzati dalle attività e dalle decisioni associate al progetto o all’organizzazione. La gestione degli stakeholder è una parte fondamentale del 
processo di pianificazione ed esecuzione di progetti ed è essenziale per il successo complessivo di un’iniziativa.

\hypertarget{sec:streamlit}{}
\subsection*{Streamlit}
Streamlit è un framework open-source in Python progettato per creare applicazioni web interattive e data-driven in modo rapido e intuitivo. È utilizzato 
principalmente per sviluppare dashboard e strumenti di visualizzazione dati, grazie a una sintassi semplice che permette di trasformare script Python in 
applicazioni web complete con poche righe di codice. Streamlit è particolarmente apprezzato nella comunità scientifica e tra i data scientist per la sua 
capacità di integrare facilmente grafici, widget interattivi e modelli di machine learning in un'interfaccia user-friendly.

\hypertarget{sec:stub}{}
\subsection*{Stub}
Uno stub è un componente software che simula il comportamento di un modulo o di una funzionalità specifica in un sistema, al fine di testare altre parti
del software o di verificare l'interazione tra i componenti. Gli stub vengono utilizzati per sostituire temporaneamente i moduli non ancora implementati
o per simulare condizioni specifiche, consentendo di testare il software in modo isolato e controllato.

\hypertarget{sec:supabase}{}
\subsection*{Supabase}
Supabase è una piattaforma open-source che fornisce un backend completo per applicazioni web e mobili, simile a Firebase. Basato su PostgreSQL, offre un 
database relazionale, un sistema di autenticazione, un'API RESTful generata automaticamente, storage per file e supporto per database vettoriali. Questa 
funzionalità consente di gestire embedding e lavorare con dati multidimensionali, rendendolo adatto ad applicazioni di machine learning e intelligenza 
artificiale. Supabase è pensato per semplificare lo sviluppo di applicazioni, fornendo strumenti pronti all'uso per gestire dati, utenti e file, con la 
possibilità di eseguire query in tempo reale. È molto apprezzato per la sua flessibilità, l'integrazione con standard open-source e la facilità d'uso.

\newpage


% Intestazione
\fancyhead[L]{T} % Testo a sinistra

\section{}

\hypertarget{sec:telegram}{}
\subsection*{Telegram}
Applicazione multipiattaforma che permette una facile comunicazione con gruppi/canali, organizzando la comunicazione sulla base di strumenti e 
funzionalità automatiche offerte (bot), condividendo facilmente file e messaggi in un canale unico di comunicazione.

\hypertarget{sec:test}{}
\subsection*{Test}
In ingegneria del software, processo sistematico e controllato per valutare un sistema software o una sua componente allo scopo di individuare eventuali 
difetti, errori o comportamenti indesiderati.

\hypertarget{sec:test_integrazione}{}
\subsection*{Test di integrazione}
Fase del processo di testing software in cui i singoli moduli o componenti di un’applicazione vengono combinati e testati come un
gruppo. L’obiettivo è verificare che le diverse parti del sistema interagiscano correttamente e che l’intera applicazione funzioni
come previsto. Questo tipo di test è cruciale per identificare e risolvere problemi di interoperabilità tra le componenti e garantire
la coesione del sistema nel suo complesso.

\hypertarget{sec:test_unità}{}
\subsection*{Test di unità}
Pratica dell’ingegneria del software che aiuta a garantire la qualità e l’affidabilità del codice attraverso la verifica delle
singole unità. Tali unità si riferiscono alle parti più piccole e atomiche del software, come funzioni, metodi, classi o moduli.
L’obiettivo principale del test di unità è isolare e testare ogni singola unità di codice in modo indipendente dagli altri componenti
del sistema. Questo consente agli sviluppatori di individuare e risolvere eventuali difetti o bug in modo più efficiente e tempestivo.

\hypertarget{sec:tdd}{}
\subsection*{TDD}
Il Test-Driven Development (TDD) è una metodologia di sviluppo software in cui i test vengono scritti prima del codice di produzione. 
Il processo segue un ciclo iterativo di tre fasi: scrittura di un test fallimentare, sviluppo del codice minimo per superarlo e successiva refactorizzazione per migliorare la qualità del codice.

\subsection*{Tracciamento}
Il tracciamento è il processo di monitoraggio e documentazione dell'evoluzione e delle modifiche di un sistema, dei suoi requisiti e dei suoi componenti, 
all’interno di un progetto. Questo processo è essenziale per mantenere una visione chiara dello stato attuale del sistema, delle sue modifiche, e delle 
relazioni tra diversi elementi (come requisiti, attività di design, e componenti di codice).

\hypertarget{sec:trigger}{}
\subsection*{Trigger}
Un trigger (dall'inglese "grilletto") è un meccanismo che innesca automaticamente un'azione o un evento in risposta a un particolare 
stimolo o condizione. Nell'ambito informatico, un trigger è una procedura predefinita che viene eseguita in modo automatico quando 
si verifica un evento specifico.

\hypertarget{sec:txtai}{}
\subsection*{Txtai}
Txtai è una piattaforma open-source per la ricerca semantica e l'indicizzazione di dati basati su embedding. Fornisce strumenti per creare motori di 
ricerca, sistemi di domande e risposte, classificazione di documenti e clustering, utilizzando modelli di machine learning. Txtai supporta vari formati 
di dati, integra API intuitive e può essere facilmente utilizzato per costruire applicazioni di AI scalabili.

\hypertarget{sec:typescript}{}
\subsection*{TypeScript}
TypeScript è un linguaggio di programmazione open-source sviluppato da Microsoft, basato su JavaScript ma con l'aggiunta di un sistema di tipi statici. 
Progettato per migliorare la scalabilità e la manutenibilità del codice, TypeScript permette di rilevare errori durante la scrittura del codice grazie 
alla tipizzazione, mantenendo la compatibilità con il runtime di JavaScript. È particolarmente utilizzato in applicazioni web e progetti di grandi 
dimensioni, offrendo una migliore esperienza di sviluppo e strumenti avanzati come il completamento automatico e il refactoring.

\newpage


% Intestazione
\fancyhead[L]{U} % Testo a sinistra

\section{}

\hypertarget{sec:uml}{}
\subsection*{UML, diagramma}
Vedi \bulhyperlink{sec:diagramma_UML}{Diagramma UML}.

\hypertarget{sec:use case}{}
\subsection*{Use Case}
Unità logica che rappresenta un caso d’uso specifico del sistema. Definisce un’operazione di business e viene implementato da un Service per eseguire la logica di business associata.

\newpage


% Intestazione
\fancyhead[L]{V} % Testo a sinistra

\section{}

\hypertarget{sec:verifica}{}
\subsection*{Verifica}
Processo che include un insieme di attività volte a garantire che il lavoro svolto durante lo sviluppo del software rispetti gli standard, i requisiti e 
le aspettative stabilite. La verifica è essenziale per garantire che il software sia di alta qualità, risponda alle esigenze degli utenti e riduca il 
rischio di errori e difetti. Si svolge a periodi regolari durante il corso del progetto.

\subsection*{Versionamento}
Processo che realizza il cosiddetto “controllo di versione”, stabilendo la storia cronologica delle azioni fatte per una certa attività, tracciando i 
cambiamenti occorsi e permettendo di tornare a uno stadio precedente qualora necessario.

\hypertarget{sec:VSC}{}
\subsection*{Visual Studio Code (VSC)}
IDE libero e gratuito sviluppato da Microsoft per Windows, Linux e macOS. Permette sia di scrivere codice 
sorgente per un prodotto software sia di scrivere testo di documentazione.

\newpage
% Intestazione
\fancyhead[L]{W} % Testo a sinistra

\section{}

\hypertarget{sec:way_of_working}{}
\subsection*{Way of working}
Il termine “Way of working” (modo di lavorare) si riferisce al modo in cui un individuo, un team o un’organizzazione svolge le proprie attività lavorative. 
Questo concetto può includere processi, metodologie, abitudini, strumenti e culture aziendali che influenzano la gestione del lavoro e la collaborazione. 
Un modo di lavorare efficace ed efficiente può migliorare la produttività, la qualità del lavoro e la soddisfazione dei membri del team, promuovendo un 
ambiente di lavoro positivo e collaborativo. \\
Il way of working deve essere:
\begin{itemize}
    \item \emph{Sistematico}, cioè non deve dipendere dalle singole persone, bensì deve essere adattabile anche a futuri membri del team.
    \item \emph{Disciplinato}, cioè deve far seguire norme di comportamento che blocchino l'istinto e favoriscano il ragionamento.
    \item \emph{Quantificabile}, cioè si deve poter verificare concretamente la sua implementazione.
\end{itemize}

\hypertarget{sec:whisper.ai}{}
\subsection*{Whisper.ai}
Sistema di riconoscimento automatico del parlato (ASR) sviluppato da OpenAI.
È progettato per trascrivere il linguaggio parlato in testo con alta precisione.

\newpage


% Intestazione
\fancyhead[L]{X} % Testo a sinistra

\section{}

\dots

\newpage


% Intestazione
\fancyhead[L]{Y} % Testo a sinistra

\section{}

\dots

\newpage


% Intestazione
\fancyhead[L]{Z} % Testo a sinistra

\section{}

\dots


\end{document}
