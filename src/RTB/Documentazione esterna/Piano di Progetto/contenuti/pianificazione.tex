% Intestazione
\fancyhead[L]{4 \hspace{0.2cm} Pianificazione} % Testo a sinistra

% Sezione 
\section{Pianificazione}
\label{sec:pianificazione}
Il gruppo SWEg Labs ha deciso di strutturare il proprio progetto in modo sistematico, 
organizzando le attività secondo scadenze ben definite, indicate all'inizio di ogni sezione. Le attività saranno suddivise in base alla fase di revisione e per argomento, garantendo così un approccio strutturato e focalizzato sui diversi stadi del progetto.\\
Un \emph{Diagramma di Burndown}\textsubscript{\textit{\textbf{G}}} fornirà una rappresentazione grafica dell'avanzamento del lavoro nel tempo, facilitando il monitoraggio costante del progresso e la gestione delle risorse.\\
Questo strumento consentirà al team di mantenere una visione d'insieme sull’avanzamento del progetto e di apportare eventuali aggiustamenti per rispettare le tempistiche pianificate
Verrà redatto durante ogni sprint un consuntivo dettagliato, che includerà sia i costi che le ore impiegate da ciascun membro del team. Tale analisi offrirà una chiara panoramica delle risorse investite e dei risultati ottenuti, rendendo possibile una valutazione complessiva dell’efficienza e dell’efficacia del progetto.
Il progetto del gruppo SWEg Labs sarà suddiviso nelle seguenti fasi:
\begin{itemize}
    \item \textbf{Studio preliminare dei capitolati}: Analisi dei capitolati per valutare vantaggi e svantaggi di ciascuno, al fine di identificare quello che meglio si adatta alle competenze e agli interessi del gruppo, garantendo una scelta in linea con le esigenze progettuali.
    \item \textbf{\emph{Requirements and Technology Baseline}}\textsubscript{\textit{\textbf{G}}}: Definizione dei requisiti e delle tecnologie di base necessarie per il progetto.
    \item \textbf{\emph{Product Baseline}}\textsubscript{\textit{\textbf{G}}}: Sviluppo del prodotto secondo i requisiti definiti.
\end{itemize}

Questa suddivisione consentirà al gruppo di avanzare in modo ordinato, 
assicurando che ogni fase del progetto sia completata in maniera ottimale per garantire il massimo livello di qualità e soddisfazione del cliente.


