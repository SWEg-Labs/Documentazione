% Intestazione
\fancyhead[L]{5 \hspace{0.2cm} Preventivo e Consuntivo} % Testo a sinistra

% Sezione 
\section{Preventivo e Consuntivo}
\label{sec:preventivo e consuntivo}

\subsection{RTB}
\subsubsection{Periodo zero: 04/11/2024 - 20/11/2024}
Gli obiettivi fissati per il periodo zero sono:
\begin{itemize}
    \item Effettuare le correzioni consigliate dopo l'aggiudicazione del capitolato;
    \item Inizio stesura documentazione RTB: \textit{Glossario, Analisi dei requisiti, Norme di Progetto, Piano di Progetto, Piano di Qualifica};
    \item Comunicare all'azienda AzzurroDigitale dell'effettiva aggiudicazione del \textit{Capitolato} ed organizzare un primo incontro conoscitivo;
    \item Implementare i giusti strumenti di lavoro come {\emph{Jira}}\textsubscript{\textit{\textbf{G}}}, {\emph{Fogli google}}\textsubscript{\textit{\textbf{G}}}, {\emph{Draw.io}}\textsubscript{\textit{\textbf{G}}}.
\end{itemize}

%Primo Periodo - CON FINE PRIMO SPRINT
\subsubsection{Primo periodo: 21/11/2024 - 05/12/2024}
Gli obiettivi fissati per il primo periodo sono:
\begin{itemize}
    \item Identificazione e studio delle tecnologie necessarie per il progetto;
    \item Effettuare l'analisi dei {\emph{Casi d'uso}}\textsubscript{\textit{\textbf{G}}};
    \item Continuazione della documentazione per il documento \textit{Piano di Qualifica};
    \item Conclusione della prima {\emph{sprint}}\textsubscript{\textit{\textbf{G}}} e successiva scriturra del preventivo, consuntivo, prospetto orario, prospetto economico e retrospettiva nel \textit{Piano di Progetto};
\end{itemize}

\subsubsubsection{Preventivo primo periodo: 21/11/2024 - 05/12/2024}
\begin{table}[h!]
    \centering
    \renewcommand{\arraystretch}{1.5}
    \begin{tabularx}{\textwidth}{|c|X|X|X|X|X|X|c|}\hline
    \rowcolor[HTML]{FFD700} 
    \textbf{Nominativi dei membri} & \textbf{Re} & \textbf{Am} & \textbf{An} & \textbf{Pg} & \textbf{Pr} & \textbf{Ve} & \textbf{Ore per membro} \\ \hline
    Federica Bolognini  & 1 & 0 & 2 & 0 & 2 & 2 & 7 \\ \hline
    Michael Fantinato   & 0 & 0 & 3 & 0 & 2 & 2 & 7 \\ \hline
    Giacomo Loat        & 0 & 7 & 0 & 0 & 0 & 0 & 7 \\ \hline
    Filippo Righetto    & 0 & 2 & 4 & 0 & 0 & 2 & 8 \\ \hline
    Riccardo Stefani    & 7 & 0 & 0 & 0 & 0 & 0 & 7 \\ \hline
    Davide Verzotto     & 0 & 1 & 4 & 0 & 0 & 2 & 7 \\ \hline
    \rowcolor[HTML]{FFD700} 
    \textbf{Ore totali per ruolo} & 8 & 10 & 13 & 0 & 4 & 8 & \textbf{Ore totali del gruppo: 43} \\ \hline
    \end{tabularx}
    \caption{Preventivo della suddivisione oraria per ruolo nel primo periodo}
\end{table}

\subsubsubsection{Consuntivo primo periodo: 21/11/2024 - 05/12/2024}
Tutti gli obiettivi predisposti all'inizio del periodo sono stati soddisfatti con successo.\\
In particolare, l'identificazione e lo studio delle tecnologie necessarie per il progetto sono stati completati positivamente sebbene abbia richiesto più tempo del previsto. L'analisi dei \textit{Casi d'uso} è stata portata a termine. La continuazione della documentazione per il documento \textit{Piano di Qualifica} è stata svolta regolarmente. Infine, la prima {\emph{sprint}}\textsubscript{\textit{\textbf{G}}} è stata conclusa con successo, con la successiva redazione del preventivo, consuntivo, prospetto orario, prospetto economico e retrospettiva nel \textit{Piano di Progetto}.

\newpage
\paragraph{Prospetto orario: }
Nel primo periodo si è registrato un consumo orario superiore a quello preventivato, con un eccesso di 11 ore di lavoro complessive per il gruppo. Questo incremento è stato determinato principalmente dalle ore assegnate ad Amministratore, Analista e Verificatore, mentre si è osservata una riduzione delle ore destinate a Progettista e Programmatore.
\begin{table}[h!]
    \centering
    \renewcommand{\arraystretch}{1.5}
    \begin{tabularx}{\textwidth}{|c|X|X|X|X|X|X|c|}\hline
    \rowcolor[HTML]{FFD700} 
    \textbf{Nominativi dei membri} & \textbf{Re} & \textbf{Am} & \textbf{An} & \textbf{Pg} & \textbf{Pr} & \textbf{Ve} & \textbf{Ore per membro} \\ \hline
    Federica Bolognini & 1 & 1 & 2 & 0 & 0 & 2 & 6  \\ \hline
    Michael Fantinato  & 0 & 3 & 3 & 0 & 0 & 2 & 8  \\ \hline
    Giacomo Loat       & 0 & 7 & 2 & 0 & 0 & 2 & 11 \\ \hline
    Filippo Righetto   & 0 & 2 & 4 & 0 & 0 & 2 & 8  \\ \hline
    Riccardo Stefani   & 7 & 3 & 2 & 0 & 0 & 2 & 14 \\ \hline
    Davide Verzotto    & 0 & 1 & 4 & 0 & 0 & 2 & 7  \\ \hline
    \rowcolor[HTML]{FFD700} 
    \textbf{Ore totali per ruolo} & 8 & 17 & 17 & 0 & 0 & 12 & \textbf{Ore totali del gruppo: 54} \\ \hline
    \end{tabularx}
    \caption{Suddivisione oraria per ruolo nel primo periodo}
\end{table}

\paragraph{Prospetto economico primo periodo: 21/11/2024 - 05/12/2024: }
Il prospetto economico relativo al primo periodo evidenzia i costi sostenuti per ciascun membro del team, suddivisi per ruolo, e il saldo complessivo a fine periodo.\\
L'analisi dei costi si è concentrata sulle ore di lavoro effettivamente registrate, che hanno comportato un lieve incremento rispetto al preventivo iniziale, dovuto principalmente a un maggiore impegno nelle attività di analisi e amministrazione.\\
In dettaglio, il costo orario per ogni membro del team è stato applicato in base al ruolo svolto, con i costi totali che sono stati calcolati sommando le ore lavorate.\\
Il totale delle spese sostenute per il primo periodo ammonta a 1185€, con un saldo finale che riflette l'andamento positivo del progetto, nonostante gli aumenti orari in alcune aree.\\
Questo prospetto offre una visione chiara dell'impatto economico del periodo e consente di monitorare il progresso rispetto al budget complessivo del progetto.
\begin{table}[!h]
    \centering
    \renewcommand{\arraystretch}{1.5}
    \begin{tabularx}{\textwidth}{|c|X|X|X|X|X|X|c|}\hline
    \rowcolor[HTML]{FFD700} 
    \textbf{Costo} & \textbf{Re} & \textbf{Am} & \textbf{An} & \textbf{Pg} & \textbf{Pr} & \textbf{Ve} & \textbf{Totale} \\ \hline
    Costo orario & 30€ & 20€ & 25€ & 25€ & 15€ & 15€ & /  \\ \hline
    Costo totale & 240€ & 340€ & 425€ & 0€ & 0€ & 180€ & 1185 \\ \hline
    \rowcolor[HTML]{FFD700} 
    \textbf{Saldo a fine periodo}  & 1740€ & 740€ & 1675€ & 3300€ & 1620€ & 1260€ & 10335€ \\ \hline
    \end{tabularx}
    \caption{Costi sostenuti durante il primo periodo e saldo rimanente}
\end{table}


\paragraph{Rischi occorsi primo periodo: 21/11/2024 - 05/12/2024: }
I rischi occorsi durante il periodo sono stati:
\begin{itemize}
    \item \S\bulref{sec:Complessità delle nuove tecnologie}{: Complessità delle nuove tecnologie};
    \item \S\bulref{sec:Rischi di comunicazione interna}{: Rischi di comunicazione interna};
    \item \S\bulref{sec:Rischi legati alla continuità del progetto}{: Rischi legati alla continuità del progetto}.
\end{itemize}Durante il progetto sono emerse criticità significative legate a diversi rischi. I rischi di comunicazione interna si sono manifestati principalmente nella mancata condivisione delle informazioni relative ai problemi riscontrati nello studio delle tecnologie: alcuni membri del team non hanno comunicato tempestivamente le difficoltà o le soluzioni adottate, causando ritardi e inefficienze.
I rischi legati alla continuità del progetto, invece, si sono concretizzati in assenze improvvise e nella discontinuità di alcune risorse, creando ulteriori ostacoli al rispetto alla gestione operativa del lavoro di gruppo.
Infine, la complessità delle nuove tecnologie ha rappresentato una sfida particolarmente impegnativa, richiedendo uno sforzo significativo per adattare e integrare strumenti e risorse nel processo di sviluppo. La combinazione di questi fattori ha sottolineato l'importanza di una gestione più strutturata e preventiva per garantire il successo del progetto.

\newpage
%Secondo Periodo - CON FINE SECONDO SPRINT
\subsubsection{Secondo periodo: 06/12/2024 - 19/12/2024}
\label{sec:prev_cons_secondo_periodo}
Gli obiettivi fissati per il secondo periodo sono:
    \begin{itemize}
        \item Proseguire con l’\emph{Analisi dei casi d’uso}, identificando e documentando i punti chiave attraverso il confronto con il \emph{proponente} e il dialogo con il professore Cardin.
        \item Trascrivere i \emph{requisiti}, trasformando i \emph{casi d’uso} in specifiche tecniche e funzionali.
        \item Per quanto riguarda le tecnologie:
        \begin{itemize}
            \item Validare le tecnologie scelte e integrarle progressivamente nel \emph{PoC}.
            \item Realizzare un \emph{PoC} che risponda a domande riguardanti dati presenti su \emph{GitHub}, \emph{Confluence} e \emph{Jira}.
            \item Organizzare il \emph{PoC} in una struttura modulare e orientata alle classi.
            \item Implementare header che forniscano istruzioni al browser, incrementandoli in base alle esigenze.
            \item Affrontare le problematiche legate alla ricerca di similarità attraverso una migliore strutturazione dei dati.
        \end{itemize}
        \item Compilare una lista di documenti di riferimento chiave e organizzarli in \emph{Confluence}, \emph{Jira} e \emph{GitHub}, per garantire un contesto chiaro e accessibile al team.
    \end{itemize}
    

\subsubsubsection{Preventivo secondo periodo: 06/12/2024 - 19/12/2024}
\begin{table}[h!]
    \centering
    \renewcommand{\arraystretch}{1.5}
    \begin{tabularx}{\textwidth}{|c|X|X|X|X|X|X|c|}\hline
    \rowcolor[HTML]{FFD700} 
    \textbf{Nominativi dei membri} & \textbf{Re} & \textbf{Am} & \textbf{An} & \textbf{Pg} & \textbf{Pr} & \textbf{Ve} & \textbf{Ore per membro} \\ \hline
    Federica Bolognini  & 2 & 0 & 2 & 0 & 0 & 2 & 6 \\ \hline
    Michael Fantinato   & 0 & 4 & 0 & 0 & 0 & 2 & 6 \\ \hline
    Giacomo Loat        & 0 & 0 & 2 & 0 & 3 & 1 & 6 \\ \hline
    Filippo Righetto    & 0 & 4 & 1 & 0 & 1 & 1 & 7 \\ \hline
    Riccardo Stefani    & 3 & 0 & 4 & 0 & 0 & 1 & 8 \\ \hline
    Davide Verzotto     & 3 & 3 & 0 & 0 & 0 & 1 & 7 \\ \hline
    \rowcolor[HTML]{FFD700} 
    \textbf{Ore totali per ruolo} & 8 & 11 & 9 & 0 & 4 & 8 & \textbf{Ore totali del gruppo: 40} \\ \hline
    \end{tabularx}
    \caption{Preventivo della suddivisione oraria per ruolo nel secondo periodo}
\end{table}

\subsubsubsection{Consuntivo secondo periodo: 06/12/2024 - 19/12/2024}
Tutti gli obiettivi predisposti per il secondo periodo, che coincide con la seconda \emph{sprint}, sono stati soddisfatti con successo. La trascrizione dei \emph{requisiti}, trasformando i \emph{casi d’uso} in specifiche tecniche e funzionali, è stata completata positivamente.\\
Lo studio e la validazione delle tecnologie necessarie per il \emph{PoC} sono stati conclusi, e la loro integrazione iniziale è stata avviata con risultati soddisfacenti. Infine, è stato sviluppato il \emph{PoC} con una struttura modulare e organizzata in classi, rispondendo a domande riguardanti dati presenti su \emph{GitHub}, \emph{Confluence} e \emph{Jira}.\\
La documentazione di riferimento chiave è stata completata e organizzata in \emph{Confluence}, \emph{Jira} e \emph{GitHub}, garantendo maggiore chiarezza per il team.\\
La continuazione del \emph{Piano di Qualifica} è stata portata avanti regolarmente, integrando i progressi fatti durante il periodo. Anche per questa seconda fase, il preventivo, il consuntivo, il prospetto orario, il prospetto economico e la retrospettiva sono stati redatti nel \textit{Piano di Progetto}, chiudendo con successo il periodo di lavoro.


\paragraph{Prospetto orario: }
Nel secondo periodo si è registrato un consumo orario superiore a quello preventivato, con un eccesso di 10 ore di lavoro complessive per il gruppo. Questo incremento è stato determinato principalmente dalle ore assegnate ad Analista e Progettista, mentre si è osservata una riduzione delle ore destinate ad Amministratore.
\begin{table}[h!]
    \centering
    \renewcommand{\arraystretch}{1.5}
    \begin{tabularx}{\textwidth}{|c|X|X|X|X|X|X|c|}\hline
    \rowcolor[HTML]{FFD700} 
    \textbf{Nominativi dei membri} & \textbf{Re} & \textbf{Am} & \textbf{An} & \textbf{Pg} & \textbf{Pr} & \textbf{Ve} & \textbf{Ore per membro} \\ \hline
    Federica Bolognini & 0 & 0 & 6 & 0 & 0 & 1 & 7  \\ \hline
    Michael Fantinato  & 0 & 0 & 2 & 5 & 0 & 0 & 7  \\ \hline
    Giacomo Loat       & 1 & 3 & 3 & 0 & 2 & 1 & 10 \\ \hline
    Filippo Righetto   & 0 & 0 & 6 & 0 & 0 & 1 & 7  \\ \hline
    Riccardo Stefani   & 2 & 1 & 3 & 0 & 5 & 1 & 12 \\ \hline
    Davide Verzotto    & 0 & 0 & 5 & 0 & 0 & 2 & 7  \\ \hline
    \rowcolor[HTML]{FFD700} 
    \textbf{Ore totali per ruolo} & 3 & 4 & 25 & 5 & 7 & 6 & \textbf{Ore totali del gruppo: 50} \\ \hline
    \end{tabularx}
    \caption{Suddivisione oraria per ruolo nel secondo periodo}
\end{table}

\paragraph{Prospetto economico secondo periodo: 06/12/2024 - 19/12/2024: }
Il prospetto economico relativo al primo periodo evidenzia i costi sostenuti per ciascun membro del team, suddivisi per ruolo, e il saldo complessivo a fine periodo.\\
L'analisi dei costi si è concentrata sulle ore di lavoro effettivamente registrate, che hanno comportato un lieve incremento rispetto al preventivo iniziale, dovuto principalmente a un maggiore impegno nelle attività di analisi e progettazione. \\
In dettaglio, il costo orario per ogni membro del team è stato applicato in base al ruolo svolto, con i costi totali che sono stati calcolati sommando le ore lavorate. \\
Il totale delle spese sostenute per il secondo periodo ammonta a 1115€, con un saldo finale che riflette l'andamento positivo del progetto, nonostante gli aumenti orari in alcune aree.\\
Questo prospetto offre una visione chiara dell'impatto economico del periodo e consente di monitorare il progresso rispetto al budget complessivo del progetto.
\begin{table}[!h]
    \centering
    \renewcommand{\arraystretch}{1.5}
    \begin{tabularx}{\textwidth}{|c|X|X|X|X|X|X|c|}\hline
    \rowcolor[HTML]{FFD700} 
    \textbf{Costo} & \textbf{Re} & \textbf{Am} & \textbf{An} & \textbf{Pg} & \textbf{Pr} & \textbf{Ve} & \textbf{Totale} \\ \hline
    Costo orario & 30€ & 20€ & 25€ & 25€ & 15€ & 15€ & /  \\ \hline
    Costo totale & 90€ & 80€ & 625€ & 125€ & 105€ & 90€ & 1115 \\ \hline
    \rowcolor[HTML]{FFD700} 
    \textbf{Saldo a fine periodo}  & 1650€ & 660€  & 1050 & 3175€ & 1515€ & 1170€ & 9220€ \\ \hline
    \end{tabularx}
    \caption{Costi sostenuti durante il secondo periodo e saldo rimanente}
\end{table}


\paragraph{Rischi occorsi secondo periodo: 06/12/2024 - 19/12/2024: }
Il rischio occorso durante il periodo è stato:
\begin{itemize}
    \item \S\bulref{sec:Mancanza di risorse e documentazione}{: Mancanza di risorse e documentazione}.
\end{itemize}
Per mancanza di risorse e documentazioneo si intende principalmente una carenza di conoscenze specifiche necessarie per condurre l’analisi dei \emph{Casi d’Uso}.\\
Questo ha richiesto uno studio approfondito e un maggiore impegno per colmare i dubbi e garantire un’analisi corretta e completa. 

