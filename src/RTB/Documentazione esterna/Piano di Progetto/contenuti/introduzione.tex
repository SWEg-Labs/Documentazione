% Intestazione
\fancyhead[L]{1 \hspace{0.2cm} Introduzione} % Testo a sinistra

\pagenumbering{arabic}  % Numerazione araba per il contenuto

%Esempio che usa una
%\emph{parola da glossario}\textsubscript{\textit{\textbf{G}}}
\section{Introduzione}
\label{sec:introduzione}

\subsection{Scopo del documento}
Il Piano di Progetto è un elemento di fondamentale importanza che offre una visione chiara degli obiettivi del progetto, 
consentendo alle parti interessate di allineare il proprio lavoro verso un obiettivo comune.\\
Esso definisce anche l'ambito del progetto, specificando ciò che è incluso e ciò che ne è escluso, per evitare espansioni 
non controllate e garantire il rispetto dei traguardi stabiliti, fornendo informazioni precise su costi e ripartizioni orarie.\\
In particolare, il Piano di Progetto affronta i seguenti temi:
\begin{itemize}
    \item Analisi dei rischi di progetto
    \item Secondo elemento
    \item Terzo elemento
    \item Ecc...
\end{itemize}

\subsection{Scopo del prodotto}
Nel corso dell’ultimo anno si è verificato un repentino e significativo mutamento nel panorama dello sviluppo e nell’implementazione dell’\emph{Intelligenza Artificiale}\textsubscript{\textit{\textbf{G}}}. Questa trasformazione ha attraversato varie sfaccettature della tecnologia, segnando una transizione dall’uso dell’Intelligenza Artificiale principalmente per l’elaborazione e la raccomandazione di contenuti, a un’era in cui tali sistemi sono capaci di generare contenuti originali.
Il \emph{Capitolato}\textsubscript{\textit{\textbf{G}}} C9, ”BuddyBot”, ha come obiettivo la realizzazione di un assistente virtuale (chatbot) capace di raccogliere rapidamente informazioni dalle fonti indicate e di fornirle in risposta a domande poste in linguaggio naturale tramite chat.
Tale assistente virtuale sarà fruibile attraverso una piattaforma web, dove l’utente potrà interagire con l'\emph{IA}\textsubscript{\textit{\textbf{G}}} per ottenere le risposte desiderate.

\subsection{Glossario}
\label{sec:glossario}
Al fine di evitare eventuali equivoci o incomprensioni riguardo la terminologia utilizzata all’interno di questo documento, 
abbiamo valutato di adottare un Glossario, con file apposito, in cui
vengono riportate tutte le definizioni rigogliose delle parole ambigue utilizzate in ambito di
questo progetto.\\
Nel documento appena descritto verranno riportati tutti i termini definiti nel
loro ambiente di utilizzo con annessa descrizione del loro significato.\\
La presenza di un termine all'interno del Glossario sarà indicata con una "G" posizionata al pedice della parola.

\subsection{Maturità e miglioramenti}
Questo documento è stato redatto seguendo un approccio incrementale, con l'obiettivo di facilitare l'adattamento alle esigenze mutevoli, stabilite di comune accordo tra i membri del gruppo di progetto e l'azienda proponente.\\
Pertanto, il documento non può essere considerato definitivo o esaustivo, ma piuttosto un punto di partenza per un continuo aggiornamento e affinamento.

\subsection{Riferimenti}
\subsubsection{Riferimenti normativi}
\begin{itemize}
    \item \bulhref{https://sweg-labs.github.io/Documentazione/output/RTB/Documentazione\%20interna/norme_progetto_v1.0.0.pdf}{Norme di Progetto v.1.0.0}; \\
    \item \bulhref{https://www.math.unipd.it/~tullio/IS-1/2024/Progetto/C9.pdf}{Capitolato C9: BuddyBot}; \\
    \item \bulhref{https://www.math.unipd.it/~tullio/IS-1/2024/Dispense/PD1.pdf}{Regolamento progetto didattico}; \\
    \item \bulhref{https://www.iso.org/standard/65694.html}{Standard ISO/IEC 31000:2018}.
\end{itemize}

\subsubsection{Riferimenti informativi}
\begin{itemize}
    \item \bulhref{https://sweg-labs.github.io/Documentazione/output/RTB/Documentazione\%20esterna/glossario_v1.0.0.pdf}{Glossario}; \\
    \item \bulhref{https://www.math.unipd.it/~tullio/IS-1/2024/Dispense/T02.pdf}{T2: Ciclo di vita del software}; \\
    \item \bulhref{https://www.math.unipd.it/~tullio/IS-1/2024/Dispense/T04.pdf}{T4: Gestione di Progetto}. \\
\end{itemize}
