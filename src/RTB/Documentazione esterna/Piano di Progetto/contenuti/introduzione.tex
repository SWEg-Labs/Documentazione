% Intestazione
\fancyhead[L]{1 \hspace{0.2cm} Introduzione} % Testo a sinistra

\pagenumbering{arabic} % Numerazione araba per il contenuto

%Esempio che usa una
%\emph{parola da glossario}\textsubscript{\textit{\textbf{G}}}
\section{Introduzione}
\subsection{Scopo del documento}
Il documento riguardante il piano di progetto è un elemento di fondamentale importanza per
i progetti di sviluppo software che voglio rispettare i massimi standard di qualità definiti dall’insegnamento dell’ingegneria del software.
Il seguente documento ha lo scopo di descrivere tutte le pratiche e metodi riguardati il processo
organizzativo e di pianificazione, specificandone l’applicazione.
Oltre a dare modo ad esterni di capire e partecipare all’evoluzione del progetto fornisce dati
precisi su costi e ripartizioni orarie.
Il documento sara’ utile a chi si occupa della creazione del prodotto, dando modo al team di
fare retrospettiva più agilmente, e a chi lo valuterà.
Lo scopo è quindi quello di fornire una descrizione dettagliata e il piu’ precisa possibile sulle
metodolgie e applicazioni delle stesse riguardanti la pianificazione, e quindi la suddivisione oraria e dei costi.
Nel dettaglio, il Piano di Progetto affronta i seguenti temi:
\begin{itemize}
    \item Analisi dei rischi di progetto;
    \item Secondo elemento
    \item Terzo elemento
    \item Ecc...
\end{itemize}

\subsection{Scopo del prodotto}
Il progetto ha lo scopo di realizzare un sistema di raccomandazione con relativa interfaccia web
che guidi le attività dell’azienda, utilizzatrice del prodotto finale, suggerendo a quali clienti
rivolgere le singole attività di marketing e commerciali, cercando i migliori clienti target a cui
indirizzare determinati prodotti.
Dall’interfaccia utente del sistema software sarà possibile selezionare uno specifico cliente e visualizzare i prodotti da lui acquistati e quelli che il sistema ha individuato come raccomandati.
Inoltre selezionato un articolo o un insieme di articoli il sistema suggerisce a quali clienti proporli, selezionandoli in base a quanto probabile siano interessati per i prodotti analizzati. I vari
prodotti possono essere filtrati per categoria così da facilitare ricerche e restringere il campo di
soluzione.
Ogni risultato restituito dal sistema di raccomandazione è classificabile tramite un feedback
così da poter eventualmente correggere il tiro dell’algoritmo che ha fornito l’esito della suggerimento.
L’utente amministratore ha la possibilità di creare ulteriori account per eventuali operatori che
necessitano di utilizzare l’applicativo.

\subsection{Glossario}
Al fine di evitare eventuali equivoci o incomprensioni riguardo la terminologia utilizzata all’interno di questo documento, si è deciso di adottare un Glossario, con file apposito, in cui
vengono riportate tutte le definizioni rigogliose delle parole ambigue utilizzate in ambito di
questo progetto. Nel documento appena descritto verranno riportati tutti i termini definiti nel
loro ambiente di utilizzo con annessa descrizione del loro significato.
La presenza di un termine all’interno del Glossario è evidenziata dal colore blu.

\subsection{Maturità e miglioramenti}
Questo documento è stato realizzato utilizzando un approccio incrementale, con lo scopo di
semplificare i cambiamenti nel tempo in base alle reciproche esigenze decise da entrambi le
parti, ovvero membri del gruppo di progetto e azienda proponente. Pertanto questo documento
non può essere considerato esaustivo e completo.

\subsection{Riferimenti}
\subsubsection{Riferimenti normativi}
Norme di Progetto v.1.0.0;
Capitolato C2: Sistemi di raccomandazione
https://www.math.unipd.it/~tullio/IS-1/2023/Progetto/C2.pdf;
Regolamento progetto ditattico
https://www.math.unipd.it/~tullio/IS-1/2023/Dispense/PD2.pdf.

\subsubsection{Riferimenti informativi}
T3 - Ciclo di vita del software (slide del corso di Ingegneria del Software)
https://www.math.unipd.it/~tullio/IS-1/2023/Dispense/T3.pdf;
T4 - Gestione di progetto (slide del corso di Ingegneria del Software)
https://www.math.unipd.it/~tullio/IS-1/2023/Dispense/T4.pdf.
