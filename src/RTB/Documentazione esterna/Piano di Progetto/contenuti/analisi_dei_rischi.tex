% Intestazione
\fancyhead[L]{2 \hspace{0.2cm} Analisi dei rischi} % Testo a sinistra

%template creazione tabella rischi
\begin{comment}
\begin{table}[h!]
\centering
\renewcommand{\arraystretch}{1.5} % Per aumentare l'altezza delle righe
\begin{tabularx}{\textwidth}{|X|X|}\hline
\rowcolor[HTML]{FFD700} 
\multicolumn{2}{|c|}{\textbf{Tipologia del rischio}} \\ \hline
\textbf{Descrizione} & Descrizione del rischio. \\ \hline
\textbf{Probabilità di occorrenza} & Probabilità di occorrenza (Alta/Media/Bassa) \\ \hline
\textbf{Pericolosità} & Pericolosità(Alta/Media/Bassa) \\ \hline
\textbf{Conseguenze} & Conseguenze del rischio \\ \hline
\textbf{Mitigazioni possibili} & Conseguenze prese \\ \hline
\end{tabularx}
\caption{\textbf{Tipologia del rischio con il numero identificativo}: Descrizione rischio}
\end{table}
\end{comment}
%fine template

\section{Analisi dei rischi}
\label{sec:analisi_rischi}
Durante l'esecuzione di un progetto, è comune incontrare diverse difficoltà. È fondamentale, in tali situazioni, mitigare gli impatti tramite un'analisi attenta dei rischi.\\
Questa sezione del Piano di Progetto è stata redatta per gestire efficacemente le problematiche che possono emergere. Dopo aver identificato i potenziali rischi, il team ha definito una serie di azioni da seguire in caso di manifestazione di tali rischi. Queste azioni rappresentano le soluzioni per superare tempestivamente gli ostacoli, evitando ritardi nello sviluppo del lavoro.\\
In conformità con lo standard \emph{ISO/IEC 31000:2018}\textsubscript{\textit{\textbf{G}}}.\\
Il processo di gestione dei rischi si articola in cinque fasi:
\begin{itemize}
    \item Identificazione dei Rischi: consiste nell'individuare le fonti di rischio, le aree di impatto, gli eventi e le cause potenziali. Per creare un elenco completo dei rischi, si effettua un'analisi delle attività e degli eventi che potrebbero influenzare il raggiungimento degli obiettivi del progetto.
    \item Analisi dei Rischi: questa fase è cruciale per valutare i rischi e determinare le azioni di trattamento più appropriate. L'obiettivo è fornire una base solida per decisioni informate sulle strategie di mitigazione e gestione degli impatti negativi.
    \item Valutazione dei Rischi: consiste nel determinare quali rischi meritano priorità e stabilire l'ordine di attuazione delle misure di mitigazione. Aiuta anche a identificare le aree critiche che necessitano di particolare attenzione, ottimizzando l'uso delle risorse e concentrandosi sulle minacce più rilevanti per il successo del progetto.
    \item Gestione dei Rischi: dopo la valutazione, è necessario determinare come affrontare i rischi identificati. Questo processo implica l'adozione di misure preventive, il trasferimento del rischio tramite assicurazioni o l'implementazione di azioni di mitigazione. La fase di gestione traduce le analisi precedenti in azioni concrete per proteggere il progetto.
    \item Monitoraggio e Revisione dei Rischi: queste attività devono essere integrate nella pianificazione della gestione dei rischi e richiedono un controllo regolare per adattarsi a nuove sfide e valutare l'efficacia delle soluzioni adottate. Il monitoraggio continuo è essenziale per identificare nuove problematiche e garantire che la gestione dei rischi resti allineata agli obiettivi e alle condizioni mutevoli del progetto.
\end{itemize}

È fondamentale attuare costantemente queste fasi lungo l'intero ciclo di vita del progetto, poiché l'evoluzione delle attività può portare all'emergere di nuove sfide che richiedono soluzioni adeguate. Per facilitare l'identificazione dei rischi, è stata introdotta una convenzione di formato specifica che supporta una gestione dinamica e proattiva.
\begin{center}
\textbf{R[Tipo][Indice]}
\end{center}
Il \textbf{Tipo} rappresenta la categoria di rischio, che può essere:
\begin{itemize}
    \item \textbf{T}: Tecnologico;
    \item \textbf{O}: Organizzativo;
    \item \textbf{P}: Relativo al singolo membro del gruppo.
\end{itemize}

L'\textbf{Indice} è un valore numerico incrementale che identifica univocamente ogni rischio all'interno di un \textbf{Tipo}.

\newpage

\subsection{RT: Rischi legati alle tecnologie}

\subsubsection{RT1: Complessità delle nuove tecnologie}
\label{sec:Complessità delle nuove tecnologie}
\begin{table}[h!]
    \centering
    \renewcommand{\arraystretch}{1.5} % Per aumentare l'altezza delle righe
    \begin{tabularx}{\textwidth}{|X|X|}\hline
    \rowcolor[HTML]{FFD700} 
    \multicolumn{2}{|c|}{\textbf{Complessità delle nuove tecnologie}} \\ \hline
    \textbf{Descrizione} & Il team di sviluppo è chiamato a lavorare con tecnologie che non conosce o con cui ha poca esperienza. \\ \hline
    \textbf{Probabilità di occorrenza} & Alta \\ \hline
    \textbf{Pericolosità} & Alta \\ \hline
    \textbf{Conseguenze} &  L'apprendimento di una nuova tecnologia richiede tempo e il team potrebbe trovarsi a rallentare il lavoro a causa della necessità di comprendere e adattarsi alla tecnologia. 
    Questo può tradursi in ritardi rispetto alla pianificazione iniziale e al raggiungimento degli obiettivi del progetto. \\ \hline
    \textbf{Mitigazioni possibili} & È utile dedicare del tempo a una fase di apprendimento e prototipazione permettendo al team di esplorare la tecnologia. Il team può identificare punti di forza e criticità, mentre crea documentazione interna per raccogliere soluzioni e best practices, 
    accelerando così l'apprendimento e evitando di ripetere gli stessi errori nel lungo periodo. \\ \hline
    \end{tabularx}
    \caption{\textbf{RT1}: Complessità delle nuove tecnologie}
    \end{table}

\newpage

\subsubsection{RT2: Mancanza di risorse e documentazione}
\label{sec:Mancanza di risorse e documentazione}
\begin{table}[h!]
    \centering
    \renewcommand{\arraystretch}{1.5} % Per aumentare l'altezza delle righe
    \begin{tabularx}{\textwidth}{|X|X|}\hline
    \rowcolor[HTML]{FFD700} 
    \multicolumn{2}{|c|}{\textbf{Mancanza di risorse e documentazione}} \\ \hline
    \textbf{Descrizione} & Le nuove tecnologie richiedono già di per sé un significativo investimento di tempo e impegno per essere apprese dal team; 
    se poi queste risultano carenti di risorse o documentazione adeguata, 
    il processo di apprendimento diventa ancora più impegnativo e complesso, soprattutto quando differiscono notevolmente dalle tecnologie già note ai membri del team.\\ \hline
    \textbf{Probabilità di occorrenza} & Alta \\ \hline
    \textbf{Pericolosità} & Alta \\ \hline
    \textbf{Conseguenze} & Senza documentazione, il team impiegherà più tempo per comprendere e utilizzare la tecnologia, 
    causando possibili ritardi rispetto alla pianificazione e generando costi aggiuntivi a causa del prolungamento dei tempi di apprendimento e debug. \\ \hline
    \textbf{Mitigazioni possibili} & Si può prevedere una fase iniziale di formazione e prototipazione, in cui il team esplori la tecnologia e si familiarizzi con essa.
    Durante questa fase, è possibile coinvolgere il proponente per farci fornire supporto diretto, risorse o contatti con esperti che abbiano conoscenze nella tecnologia in questione.
    Inoltre, se la tecnologia non è strettamente indispensabile per il progetto, si può valutare l’adozione di una soluzione alternativa simile, 
    ma con documentazione più completa e supporto maggiore. \\ \hline
    \end{tabularx}
    \caption{\textbf{RT2}: Mancanza di risorse e documentazione}
    \end{table}

\newpage

\subsection{RO: Rischi legati all'organizzazione del gruppo}

\subsubsection{RO1: Rischi di comunicazione interna}
\label{sec:Rischi di comunicazione interna}
\begin{table}[h!]
    \centering
    \renewcommand{\arraystretch}{1.5} % Per aumentare l'altezza delle righe
    \begin{tabularx}{\textwidth}{|X|X|}\hline
    \rowcolor[HTML]{FFD700} 
    \multicolumn{2}{|c|}{\textbf{Rischi di comunicazione interna}} \\ \hline
    \textbf{Descrizione} &  Si verificano quando le informazioni non vengono trasmesse in modo chiaro, 
    tempestivo o completo tra i membri di un team, tra i team stessi, o tra il team e il proponente.\\ \hline
    \textbf{Probabilità di occorrenza} & Media \\ \hline
    \textbf{Pericolosità} & Alta \\ \hline
    \textbf{Conseguenze} & Se non c'è un'adeguata comunicazione interna i membri del team potrebbero non essere consapevoli dei problemi che emergono durante lo sviluppo. 
    La mancanza di un confronto continuo tra le varie parti del team può compromettere la qualità del lavoro finale, con soluzioni che non soddisfano le aspettative.\\ \hline
    \textbf{Mitigazioni possibili} & Bisogna definire canali di comunicazione chiari, stabilendo strumenti formali e informali attraverso cui le informazioni possano fluire. Affiancare a questo la comunicazione regolare organizzando riunioni periodiche o stand-up giornalieri per fare il punto sui progressi e risolvere eventuali problemi. 
    Inoltre, è importante avere una documentazione condivisa e centralizzata in uno spazio facilmente consultabile da tutti i membri del team, dove possano essere registrati aggiornamenti, decisioni e soluzioni ai problemi. Infine, è essenziale gestire le aspettative e gli obiettivi, assicurando che tutti siano allineati sugli obiettivi del progetto, sulle priorità e sulle scadenze, evitando malintesi e disallineamenti. \\ \hline
    \end{tabularx}
    \caption{\textbf{RO1}: Rischi di comunicazione interna}
    \end{table}

\newpage

\subsubsection{RO2: Rischi di confusione sulle responsabilità}
\label{sec:Rischi di confusione sulle responsabilità}
\begin{table}[h!]
    \centering
    \renewcommand{\arraystretch}{1.5} % Per aumentare l'altezza delle righe
    \begin{tabularx}{\textwidth}{|X|X|}\hline
    \rowcolor[HTML]{FFD700} 
    \multicolumn{2}{|c|}{\textbf{Rischi di confusione sulle responsabilità}} \\ \hline
    \textbf{Descrizione} & Non c'è chiarezza sui propri compiti e sui ruoli degli altri, portando a sovrapposizioni, 
    mancanza di coordinamento e inefficienze. \\ \hline
    \textbf{Probabilità di occorrenza} & Media \\ \hline
    \textbf{Pericolosità} & Alta\\ \hline
    \textbf{Conseguenze} & Il rischio di attività non completate, conflitti interni e ritardi nel progetto. \\ \hline
    \textbf{Mitigazioni possibili} & È fondamentale definire in modo chiaro le responsabilità di ciascun membro del team fin dall'inizio, 
    utilizzando strumenti di gestione del progetto per tracciare i compiti assegnati e monitorare i progressi.  \\ \hline
    \end{tabularx}
    \caption{\textbf{RO2}: Rischi di confusione sulle responsabilità}
    \end{table}

\newpage

\subsubsection{RO3: Rischi legati alla gestione del tempo e delle scadenze}
\label{sec:Rischi legati alla gestione del tempo e delle scadenze}
\begin{table}[h!]
    \centering
    \renewcommand{\arraystretch}{1.5} % Per aumentare l'altezza delle righe
    \begin{tabularx}{\textwidth}{|X|X|}\hline
    \rowcolor[HTML]{FFD700} 
    \multicolumn{2}{|c|}{\textbf{Rischi legati alla gestione del tempo e delle scadenze}} \\ \hline
    \textbf{Descrizione} & Riguarda la possibilità che un progetto non venga completato entro i termini stabiliti, con conseguenti ritardi e inefficienze. \\ \hline
    \textbf{Probabilità di occorrenza} & Alta \\ \hline
    \textbf{Pericolosità} & Alta \\ \hline
    \textbf{Conseguenze} & Una pianificazione imprecisa o irrealistica può portare a sottovalutare il tempo necessario per completare determinate attività, causando slittamenti nelle scadenze,
    se le risorse non sono allocate correttamente o se si trascura la priorità di alcune attività, si rischia di dedicare troppo tempo a compiti meno urgenti, lasciando poco spazio per quelli più critici.\\ \hline
    \textbf{Mitigazioni possibili} & E' essenziale prima comprendere chiaramente le priorità del progetto, 
    in modo da evitare di sprecare tempo su attività secondarie. Una pianificazione accurata consente di allocare il tempo in modo efficace, 
    mentre il monitoraggio continuo dei progressi aiuta a garantire il rispetto delle scadenze. \\ \hline
    \end{tabularx}
    \caption{\textbf{RO3}: Rischi legati alla gestione del tempo e delle scadenze}
    \end{table}

\newpage

\subsection{RP: Rischi legati ai singoli membri del gruppo}

\subsubsection{RP1: Rischi legati alla continuità del progetto}
\label{sec:Rischi legati alla continuità del progetto}
\begin{table}[h!]
    \centering
    \renewcommand{\arraystretch}{1.5} % Per aumentare l'altezza delle righe
    \begin{tabularx}{\textwidth}{|X|X|}\hline
    \rowcolor[HTML]{FFD700} 
    \multicolumn{2}{|c|}{\textbf{Rischi legati alla mancata continuità del progetto}} \\ \hline
    \textbf{Descrizione} & I rischi di mancata continuità del progetto derivano da interruzioni nel flusso di lavoro causate da risorse intermittenti, malattie o impegni imprevisti.\\ \hline
    \textbf{Probabilità di occorrenza} & Alta \\ \hline
    \textbf{Pericolosità} & Alta \\ \hline
    \textbf{Conseguenze} & Riduzione della disponibilità di tempo per lavorare sul progetto, la procrastinazione e la mancanza di pianificazione possono portare ad inefficienza nell’utilizzo del tempo. Infine, eventi imprevisti come malattie o emergenze possono causare assenze improvvise, rallentando ulteriormente il lavoro e 
    influendo sulla capacità del team di rispettare i tempi stabiliti. \\ \hline
    \textbf{Mitigazioni possibili} & Pianificare con anticipo e creare un programma di lavoro realistico. Utilizzare strumenti di gestione del tempo come calendari condivisi e pianificazioni settimanali 
    aiuta a monitorare i progressi e ad identificare tempestivamente eventuali slittamenti. È anche utile prevedere margini di tempo extra per imprevisti, come malattie o altri ostacoli, per evitare di compromettere il rispetto delle scadenze.\\ \hline
    \end{tabularx}
    \caption{\textbf{RP1}: Rischi legati alla continuità del progetto}
    \end{table}

\newpage

\subsubsection{RP2: Rischi legati alla non conformità rispetto agli impegni dichiarati}
\label{sec:Rischi legati alla non conformità rispetto agli impegni dichiarati}
\begin{table}[h!]
    \centering
    \renewcommand{\arraystretch}{1.5} % Per aumentare l'altezza delle righe
    \begin{tabularx}{\textwidth}{|X|X|}\hline
    \rowcolor[HTML]{FFD700} 
    \multicolumn{2}{|c|}{\textbf{Rischi legati alla non conformità rispetto agli impegni dichiarati}} \\ \hline
    \textbf{Descrizione} &  Se i membri del team non adempiono agli impegni presi, 
    il progetto potrebbe subire ritardi o compromettere la qualità finale. \\ \hline
    \textbf{Probabilità di occorrenza} & Alta \\ \hline
    \textbf{Pericolosità} & Media \\ \hline
    \textbf{Conseguenze} & Perdita di fiducia da parte del proponente, costi e tempi aggiuntivi per rimediare ai problemi e la diminuzione della produttività e la potenziale perdita di coesione all'interno del gruppo. \\ \hline
    \textbf{Mitigazioni possibili} & L' adozione di una gestione rigorosa del progetto, comunicazione continua con il cliente e monitoraggio costante dei progressi per garantire il rispetto degli impegni dichiarati.\\ \hline
    \end{tabularx}
    \caption{\textbf{RP2}: Rischi legati alla non conformità rispetto agli impegni dichiarati}
    \end{table}
