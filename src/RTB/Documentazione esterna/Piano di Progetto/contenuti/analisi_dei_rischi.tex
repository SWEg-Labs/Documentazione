% Intestazione
\fancyhead[L]{1 \hspace{0.2cm} Analisi dei rischi} % Testo a sinistra

%template creazione tabella rischi
\begin{comment}
\begin{table}[h!]
\centering
\renewcommand{\arraystretch}{1.5} % Per aumentare l'altezza delle righe
\begin{tabularx}{\textwidth}{|X|X|}\hline
\rowcolor[HTML]{FFD700} 
\multicolumn{2}{|c|}{\textbf{Tipologia del rischio}} \\ \hline
\textbf{Descrizione} & Descrizione del rischio. \\ \hline
\textbf{Probabilità di occorrenza} & Probabilità di occorrenza (Alta/Media/Bassa) \\ \hline
\textbf{Pericolosità} & Pericolosità(Alta/Media/Bassa) \\ \hline
\textbf{Conseguenze} & Conseguenze del rischio \\ \hline
\textbf{Mitigazioni possibili} & Conseguenze prese \\ \hline
\end{tabularx}
\caption{\textbf{Tipologia del rischio con il numero identificativo}: Descrizione rischio}
\end{table}
\end{comment}
%fine template

\section{Analisi dei rischi}
Spiegazione+spiegazione abbreviazione usate\\
Tutte le tabelle devono essere ancora completate\\
spiegazione dell'iso 13000?

\subsection{RT: Rischi legati alle tecnologie}

\subsubsection{RT1: Complessità delle nuove tecnologie}
\begin{table}[h!]
    \centering
    \renewcommand{\arraystretch}{1.5} % Per aumentare l'altezza delle righe
    \begin{tabularx}{\textwidth}{|X|X|}\hline
    \rowcolor[HTML]{FFD700} 
    \multicolumn{2}{|c|}{\textbf{Complessità delle nuove tecnologie}} \\ \hline
    \textbf{Descrizione} & Descrizione del rischio. \\ \hline
    \textbf{Probabilità di occorrenza} & Alta \\ \hline
    \textbf{Pericolosità} & Alta \\ \hline
    \textbf{Conseguenze} & Conseguenze del rischio \\ \hline
    \textbf{Mitigazioni possibili} & Conseguenze prese \\ \hline
    \end{tabularx}
    \caption{\textbf{RT1}: Complessità delle nuove tecnologie}
    \end{table}

\subsubsection{RT2: Mancanza di risorse e documentazione}
\begin{table}[h!]
    \centering
    \renewcommand{\arraystretch}{1.5} % Per aumentare l'altezza delle righe
    \begin{tabularx}{\textwidth}{|X|X|}\hline
    \rowcolor[HTML]{FFD700} 
    \multicolumn{2}{|c|}{\textbf{Mancanza di risorse e documentazione}} \\ \hline
    \textbf{Descrizione} & Descrizione del rischio. \\ \hline
    \textbf{Probabilità di occorrenza} & Probabilità di occorrenza (Alta/Media/Bassa) \\ \hline
    \textbf{Pericolosità} & Pericolosità(Alta/Media/Bassa) \\ \hline
    \textbf{Conseguenze} & Conseguenze del rischio \\ \hline
    \textbf{Mitigazioni possibili} & Conseguenze prese \\ \hline
    \end{tabularx}
    \caption{\textbf{RT2}: Mancanza di risorse e documentazione}
    \end{table}

\subsubsection{RT3: Aggiornamenti o modifiche agli strumenti e tecnologie in uso}
\begin{table}[h!]
    \centering
    \renewcommand{\arraystretch}{1.5} % Per aumentare l'altezza delle righe
    \begin{tabularx}{\textwidth}{|X|X|}\hline
    \rowcolor[HTML]{FFD700} 
    \multicolumn{2}{|c|}{\textbf{Aggiornamenti o modifiche agli strumenti e tecnologie in uso}} \\ \hline
    \textbf{Descrizione} & Descrizione del rischio. \\ \hline
    \textbf{Probabilità di occorrenza} & Probabilità di occorrenza (Alta/Media/Bassa) \\ \hline
    \textbf{Pericolosità} & Pericolosità(Alta/Media/Bassa) \\ \hline
    \textbf{Conseguenze} & Conseguenze del rischio \\ \hline
    \textbf{Mitigazioni possibili} & Conseguenze prese \\ \hline
    \end{tabularx}
    \caption{\textbf{RT3}: Aggiornamenti o modifiche agli strumenti e tecnologie in uso}
    \end{table}

\newpage

\subsection{RO: Rischi legati all'organizzazione del gruppo}

\subsubsection{RO1: Rischi di comunicazione interna}
\begin{table}[h!]
    \centering
    \renewcommand{\arraystretch}{1.5} % Per aumentare l'altezza delle righe
    \begin{tabularx}{\textwidth}{|X|X|}\hline
    \rowcolor[HTML]{FFD700} 
    \multicolumn{2}{|c|}{\textbf{Rischi di comunicazione interna}} \\ \hline
    \textbf{Descrizione} & Descrizione del rischio. \\ \hline
    \textbf{Probabilità di occorrenza} & Probabilità di occorrenza (Alta/Media/Bassa) \\ \hline
    \textbf{Pericolosità} & Pericolosità(Alta/Media/Bassa) \\ \hline
    \textbf{Conseguenze} & Conseguenze del rischio \\ \hline
    \textbf{Mitigazioni possibili} & Conseguenze prese \\ \hline
    \end{tabularx}
    \caption{\textbf{RO1}: Rischi di comunicazione interna}
    \end{table}

\subsubsection{RO2: Rischi di confusione sulle responsabilità}
\begin{table}[h!]
    \centering
    \renewcommand{\arraystretch}{1.5} % Per aumentare l'altezza delle righe
    \begin{tabularx}{\textwidth}{|X|X|}\hline
    \rowcolor[HTML]{FFD700} 
    \multicolumn{2}{|c|}{\textbf{Rischi di confusione sulle responsabilità}} \\ \hline
    \textbf{Descrizione} & Descrizione del rischio. \\ \hline
    \textbf{Probabilità di occorrenza} & Probabilità di occorrenza (Alta/Media/Bassa) \\ \hline
    \textbf{Pericolosità} & Pericolosità(Alta/Media/Bassa) \\ \hline
    \textbf{Conseguenze} & Conseguenze del rischio \\ \hline
    \textbf{Mitigazioni possibili} & Conseguenze prese \\ \hline
    \end{tabularx}
    \caption{\textbf{RO2}: Rischi di confusione sulle responsabilità}
    \end{table}

\subsubsection{RO3: Rischi legati alla gestione del tempo e delle scadenze}
\begin{table}[h!]
    \centering
    \renewcommand{\arraystretch}{1.5} % Per aumentare l'altezza delle righe
    \begin{tabularx}{\textwidth}{|X|X|}\hline
    \rowcolor[HTML]{FFD700} 
    \multicolumn{2}{|c|}{\textbf{Rischi legati alla gestione del tempo e delle scadenze}} \\ \hline
    \textbf{Descrizione} & Descrizione del rischio. \\ \hline
    \textbf{Probabilità di occorrenza} & Probabilità di occorrenza (Alta/Media/Bassa) \\ \hline
    \textbf{Pericolosità} & Pericolosità(Alta/Media/Bassa) \\ \hline
    \textbf{Conseguenze} & Conseguenze del rischio \\ \hline
    \textbf{Mitigazioni possibili} & Conseguenze prese \\ \hline
    \end{tabularx}
    \caption{\textbf{RO3}: Rischi legati alla gestione del tempo e delle scadenze}
    \end{table}

\newpage

\subsection{RP: Rischi legati ai singoli membri del gruppo}

\subsubsection{RP1: Rischi legati alla continuità del progetto}
\begin{table}[h!]
    \centering
    \renewcommand{\arraystretch}{1.5} % Per aumentare l'altezza delle righe
    \begin{tabularx}{\textwidth}{|X|X|}\hline
    \rowcolor[HTML]{FFD700} 
    \multicolumn{2}{|c|}{\textbf{Rischi legati alla continuità del progetto}} \\ \hline
    \textbf{Descrizione} & Descrizione del rischio. \\ \hline
    \textbf{Probabilità di occorrenza} & Probabilità di occorrenza (Alta/Media/Bassa) \\ \hline
    \textbf{Pericolosità} & Pericolosità(Alta/Media/Bassa) \\ \hline
    \textbf{Conseguenze} & Conseguenze del rischio \\ \hline
    \textbf{Mitigazioni possibili} & Conseguenze prese \\ \hline
    \end{tabularx}
    \caption{\textbf{RP1}: Rischi legati alla continuità del progetto}
    \end{table}

\subsubsection{RP2: Rischi legati alla non conformità rispetto agli impegni dichiarati}
\begin{table}[h!]
    \centering
    \renewcommand{\arraystretch}{1.5} % Per aumentare l'altezza delle righe
    \begin{tabularx}{\textwidth}{|X|X|}\hline
    \rowcolor[HTML]{FFD700} 
    \multicolumn{2}{|c|}{\textbf{Rischi legati alla non conformità rispetto agli impegni dichiarati}} \\ \hline
    \textbf{Descrizione} & Descrizione del rischio. \\ \hline
    \textbf{Probabilità di occorrenza} & Probabilità di occorrenza (Alta/Media/Bassa) \\ \hline
    \textbf{Pericolosità} & Pericolosità(Alta/Media/Bassa) \\ \hline
    \textbf{Conseguenze} & Conseguenze del rischio \\ \hline
    \textbf{Mitigazioni possibili} & Conseguenze prese \\ \hline
    \end{tabularx}
    \caption{\textbf{RP2}: Rischi legati alla non conformità rispetto agli impegni dichiarati}
    \end{table}
