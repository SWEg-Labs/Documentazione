% Configurazione
\documentclass{article}

\usepackage{titling} % Required for inserting the subtitle
\usepackage{graphicx} % Required for inserting images
\usepackage{tabularx} % Per l'ambiente tabularx (tabelle)
\usepackage{calc} % Sempre per le tabelle
\usepackage[hidelinks]{hyperref} % Per i collegamenti ipertestuali, ad esempio sulla table of contents
\usepackage[italian]{babel} % Per la lingua italiana nelle scritte automatiche
\usepackage{xcolor} % Per colorare il testo
\usepackage{colortbl} % Per colorare le celle delle tabelle
\usepackage{lipsum} % Per generare lorem ipsum
\usepackage[normalem]{ulem} % Per sottolineare il testo
\usepackage{array} % Per la visualizzazione fluttuante di array di domande e risposte
\usepackage{ragged2e} % Pacchetto necessario per \justifying che giustifica il testo di tabelle
\usepackage{tikz} % Per spostare elementi nel documento in modo facile e veloce
\usepackage[a4paper, top=2.5cm, bottom=2.5cm, left=2.5cm, right=2.5cm]{geometry} % Per i margini della pagina
\usepackage{fancyhdr} % Per l'intestazione e il piè di pagina
\usepackage{amsmath} % Per scrivere formule matematiche, in particolare per il pedice G

\newcommand{\ulhref}[2]{\href{#1}{\uline{#2}}} % Nuovo comando per sottolineare i link
\newcommand{\ulref}[1]{\uline{\ref{#1}}} % Nuovo comando per sottolineare i collegamenti a immagini e tabelle
\setlength{\parindent}{0pt} % Rimuove il rientro automatico dei paragrafi
\usetikzlibrary{calc} % Libreria per il calcolo delle coordinate di TikZ
\pagestyle{fancy} % Stile della pagina, per l'intestazione e il piè di pagina
\renewcommand{\footrulewidth}{0.4pt} % Inserimento della linea orizzontale in basso
\setlength{\headsep}{1.4cm} % Spazio tra l'intestazione e il testo
\definecolor{lightgray}{gray}{0.95} % Definizione del colore grigio chiaro

\graphicspath{ {immagini/} {../../../shared/immagini/} }



% Struttura
\begin{document}

\input{contenuti/intestazione_titolo.tex}
\section{Informazioni generali}

\begin{itemize}
    \item \textbf{Tipo di riunione:} interna
    \item \textbf{Luogo:} meeting Discord
    \item \textbf{Data:} 16/10/2024
    \item \textbf{Ora inizio:} 14:00
    \item \textbf{Ora fine:} 16:00
    \item \textbf{Responsabile:} Stefani Riccardo
    \item \textbf{Scriba:} Stefani Riccardo
    \item \textbf{Partecipanti:}
    \begin{itemize}
        \renewcommand{\labelitemii}{--}
        \item Bolognini Federica
        \item Fantinato Michael
        \item Loat Giacomo
        \item Righetto Filippo
        \item Stefani Riccardo
        \item Verzotto Davide
    \end{itemize}
\end{itemize}


\newpage
% Intestazione
\fancyhead[L]{Registro delle modifiche} % Testo a sinistra
\fancyhead[R]{\includegraphics[width=0.16\textwidth]{sweg_logo_sito_inverted.png}} % Immagine a destra

% Piè di pagina
\fancyfoot[L]{Analisi dei Requisiti}       % Testo a sinistra
\fancyfoot[C]{\thepage}                % Numero di pagina al centro
\fancyfoot[R]{Versione 1.0.0}          % Testo a destra

\pagenumbering{roman} % Numerazione romana per l'indice


\section*{Registro delle modifiche}

\begin{table}[h]
    \centering
    \rowcolors{2}{lightgray}{white}
    \begin{tabular}{|c|c|p{5cm}|p{3cm}|p{3cm}|}
        \hline
        \rowcolor[gray]{0.75}
        \textbf{Versione} & \textbf{Data} & \multicolumn{1}{|c|}{\textbf{Descrizione}} & 
        \multicolumn{1}{|c|}{\textbf{Autore}} & \multicolumn{1}{|c|}{\textbf{Verifica}}\\
        \hline
        1.0.0 & ... & Approvazione del documento & Filippo Righetto & Filippo Righetto\\
        \hline
        0.2.9 & 13-12-24 & Scrittura del caso d'uso \bulhyperlink{UC9}{UC9} e \bulhyperlink{UC17}{UC17} & Filippo Righetto & Michael Fantinato \\
        \hline
        0.2.8 & 12-12-24 & Scrittura dei casi d'uso \bulhyperlink{UC8}{UC8} e \bulhyperlink{UC13}{UC13} & Federica Bologninini &  Davide Verzotto \\
        \hline
        0.2.7 & 12-12-24 & Scrittura del caso d'uso \bulhyperlink{UC7}{UC7} & Filippo Righetto & Federica Bologninini \\
        \hline
        0.2.6 & 10-12-24 & Sistemazione del caso d'uso \bulhyperlink{UC3}{UC3} e scrittura del caso d'uso \bulhyperlink{UC4}{UC4} & Michael Fantinato & Riccardo Stefani \\
        \hline
        0.2.5 & 11-12-24 & Scrittura del caso d'uso \bulhyperlink{UC3}{UC3} & Davide Verzotto & Giacomo Loat \\
        \hline
        0.2.5 & 10-12-24 & Scrittura dei casi d'uso \bulhyperlink{UC12}{UC12}, \bulhyperlink{UC14}{UC14} e \bulhyperlink{UC15}{UC15}  & Giacomo Loat & Federica Bolognini \\
        \hline
        0.2.4 & 10-12-24 & Scrittura del caso d'uso \bulhyperlink{UC2}{UC2} & Giacomo Loat & Michael Fantinato \\
        \hline
        0.2.3 & 09-12-24 & Scrittura del caso d'uso \bulhyperlink{UC1}{UC1} & Federica Bolognini & Riccardo Stefani \\
        \hline
        0.2.2 & 09-12-24 & Creazione del template per la trascrizione dei casi d'uso in \S\bulref{sec:casi_uso} & Riccardo Stefani & Federica Bolognini\\
        \hline
        0.2.1 & 08-12-24 & Scrittura dei casi d'uso \bulhyperlink{UC5}{UC5}, \bulhyperlink{UC6}{UC6}, \bulhyperlink{UC11}{UC11} e 
        \bulhyperlink{UC16}{UC16} & Riccardo Stefani & Giacomo Loat\\
        \hline
        0.2.0 & 06-12-24 & Verifica del documento allo stato attuale & Riccardo Stefani & Riccardo Stefani\\
        \hline
        0.1.2 & 18-11-24 & Inizio scrittura sezione \S\bulref{sec:Requisiti} & Filippo Righetto & Davide Verzotto\\
        \hline
        0.1.1 & 10-11-24 & Scrittura della sezione \S\bulref{sec:introduzione} di introduzione e della sezione \S\bulref{sec:descrizione_generale} 
        riguardante la descrizione generale & Filippo Righetto & Riccardo Stefani\\
        \hline
        0.1.0 & 05-11-24 & Creazione del documento & Riccardo Stefani & Giacomo Loat\\
        \hline
    \end{tabular}
    \caption{Registro delle modifiche}
\end{table}

\newpage
% Intestazione
\fancyhead[L]{Indice} % Testo a sinistra

\tableofcontents
\section{Introduzione}
\subsection{Scopo del documento}
\subsection{Scopo del prodotto}
\subsection{Glossario}
\subsection{Maturità e miglioramenti}
\subsection{Riferimenti}
\subsubsection{Riferimenti normativi}
\subsubsection{Riferimenti informativi}

\section{Analisi dei rischi}
\subsection{RT: Rischi legati alle tecnologie}
\subsubsection{Complessità delle nuove tecnologie}
\subsubsection{Mancanza di risorse e documentazione}
\subsubsection{Aggiornamenti o modifiche agli strumenti e tecnologie in uso}
\subsection{RO: Rischi legati all'organizzazione del gruppo}
\subsubsection{Rischi di comunicazione interna}
\subsubsection{Rischi di confusione sulle responsabilità}
\subsubsection{Rischi legati alla gestione del tempo e delle scadenze}
\subsection{RP: Rischi legati ai singoli membri del gruppo}
\subsubsection{Rischi legati alla continuità del progetto}
\subsubsection{Rischi legati alla non conformità rispetto agli impegni dichiarati}
\newpage
\input{contenuti/elenchi.tex}
\newpage

% Intestazione
\fancyhead[L]{1 \hspace{0.2cm} Introduzione} % Testo a sinistra

\pagenumbering{arabic} % Numerazione araba per il contenuto

%Esempio che usa una
%\emph{parola da glossario}\textsubscript{\textit{\textbf{G}}}
\section{Introduzione}
\subsection{Scopo del documento}
Il documento riguardante il piano di progetto è un elemento di fondamentale importanza per
i progetti di sviluppo software che voglio rispettare i massimi standard di qualità definiti dall’insegnamento dell’ingegneria del software.
Il seguente documento ha lo scopo di descrivere tutte le pratiche e metodi riguardati il processo
organizzativo e di pianificazione, specificandone l’applicazione.
Oltre a dare modo ad esterni di capire e partecipare all’evoluzione del progetto fornisce dati
precisi su costi e ripartizioni orarie.
Il documento sara’ utile a chi si occupa della creazione del prodotto, dando modo al team di
fare retrospettiva più agilmente, e a chi lo valuterà.
Lo scopo è quindi quello di fornire una descrizione dettagliata e il piu’ precisa possibile sulle
metodolgie e applicazioni delle stesse riguardanti la pianificazione, e quindi la suddivisione oraria e dei costi.
Nel dettaglio, il Piano di Progetto affronta i seguenti temi:
\begin{itemize}
    \item Analisi dei rischi di progetto;
    \item Secondo elemento
    \item Terzo elemento
    \item Ecc...
\end{itemize}

\subsection{Scopo del prodotto}
Il progetto ha lo scopo di realizzare un sistema di raccomandazione con relativa interfaccia web
che guidi le attività dell’azienda, utilizzatrice del prodotto finale, suggerendo a quali clienti
rivolgere le singole attività di marketing e commerciali, cercando i migliori clienti target a cui
indirizzare determinati prodotti.
Dall’interfaccia utente del sistema software sarà possibile selezionare uno specifico cliente e visualizzare i prodotti da lui acquistati e quelli che il sistema ha individuato come raccomandati.
Inoltre selezionato un articolo o un insieme di articoli il sistema suggerisce a quali clienti proporli, selezionandoli in base a quanto probabile siano interessati per i prodotti analizzati. I vari
prodotti possono essere filtrati per categoria così da facilitare ricerche e restringere il campo di
soluzione.
Ogni risultato restituito dal sistema di raccomandazione è classificabile tramite un feedback
così da poter eventualmente correggere il tiro dell’algoritmo che ha fornito l’esito della suggerimento.
L’utente amministratore ha la possibilità di creare ulteriori account per eventuali operatori che
necessitano di utilizzare l’applicativo.

\subsection{Glossario}
Al fine di evitare eventuali equivoci o incomprensioni riguardo la terminologia utilizzata all’interno di questo documento, si è deciso di adottare un Glossario, con file apposito, in cui
vengono riportate tutte le definizioni rigogliose delle parole ambigue utilizzate in ambito di
questo progetto. Nel documento appena descritto verranno riportati tutti i termini definiti nel
loro ambiente di utilizzo con annessa descrizione del loro significato.
La presenza di un termine all’interno del Glossario è evidenziata dal colore blu.

\subsection{Maturità e miglioramenti}
Questo documento è stato realizzato utilizzando un approccio incrementale, con lo scopo di
semplificare i cambiamenti nel tempo in base alle reciproche esigenze decise da entrambi le
parti, ovvero membri del gruppo di progetto e azienda proponente. Pertanto questo documento
non può essere considerato esaustivo e completo.

\subsection{Riferimenti}
\subsubsection{Riferimenti normativi}
Norme di Progetto v.1.0.0;
Capitolato C2: Sistemi di raccomandazione
https://www.math.unipd.it/~tullio/IS-1/2023/Progetto/C2.pdf;
Regolamento progetto ditattico
https://www.math.unipd.it/~tullio/IS-1/2023/Dispense/PD2.pdf.

\subsubsection{Riferimenti informativi}
T3 - Ciclo di vita del software (slide del corso di Ingegneria del Software)
https://www.math.unipd.it/~tullio/IS-1/2023/Dispense/T3.pdf;
T4 - Gestione di progetto (slide del corso di Ingegneria del Software)
https://www.math.unipd.it/~tullio/IS-1/2023/Dispense/T4.pdf.


\end{document}
