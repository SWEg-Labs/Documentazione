% Intestazione
\fancyhead[L]{1 \hspace{0.2cm} Introduzione} % Testo a sinistra

\pagenumbering{arabic} % Numerazione araba per il contenuto


\section{Introduzione}
\label{sec:introduzione}

\subsection{Scopo del documento}
Il presente documento ha lo scopo di definire il \emph{way of working}\textsubscript{\textit{\textbf{G}}} che il gruppo \emph{SWEg Labs} dovrà rispettare al fine di completare il progetto BuddyBot secondo i principi di \emph{efficacia}\textsubscript{\textit{\textbf{G}}} ed \emph{efficienza}\textsubscript{\textit{\textbf{G}}}.
Vengono inoltre specificati gli strumenti utilizzati, insieme al loro scopo e alla spiegazione dietro la loro scelta.
\subsection{Scopo del prodotto}
Al giorno d'oggi, per effetto della digitalizzazione è sempre più importante l'accesso a un numero crescente di informazioni eterogenee per mantenere la produttività. I dati richiesti provengono spesso da  \emph{fonti}\textsubscript{\textit{\textbf{G}}} differenti, risultando talvolta difficili da identificare e  \emph{rintracciare}\textsubscript{\textit{\textbf{G}}} con precisione. Inoltre la digitalizzazione, pur semplificando l'\emph{accessibilità}\textsubscript{\textit{\textbf{G}}} delle informazioni rispetto ai formati analogici, ha reso necessaria un'organizzazione più rigorosa di questi ultimi, aumentando l'importanza di una ricerca efficiente e mirata.

Il capitolato si propone quindi di migliorare il processo di ricerca e recupero delle informazioni tramite l'intelligenza artificiale, riducendo quindi i tempi di ricerca e incrementando la produttività.

Per raggiungere questi obiettivi, il gruppo \emph{SWEg Labs} realizzerà un software basato sull'intelligenza artificiale. Il quale, integrato con \emph{API}\textsubscript{\textit{\textbf{G}}} di terze parti, riceverà le richieste in linguaggio naturale e ne fornirà le risposte adattate alle esigenze dell'utente.
\subsection{Glossario}
Al fine di evitare incomprensioni dovute all'utilizzo di termini ambigui, sarà redatto un glossario ordinato, contenente i termini utilizzati nella documentazione che potrebbero trarre in inganno. Ogni termine incluso nel glossario sarà evidenziato in corsivo e contrassegnato con il simbolo "G" in pedice alla sua prima occorrenza.

