% Intestazione
\fancyhead[L]{2 \hspace{0.2cm} Processi primari} % Testo a sinistra


\section{Processi primari}
\label{sec:processi_primari}
\subsection{Fornitura}
\subsubsection{Descrizione}
Questa sezione riporta tutte le norme, gli strumenti e i metodi che ogni membro del gruppo \emph{SWEg Labs} si impegna a rispettare al fine di preservare al meglio i rapporti con il proponente \emph{AzzurroDigitale}\textsubscript{\textit{\textbf{G}}}.
\subsubsection{Scopo}
Il \emph{processo}\textsubscript{\textit{\textbf{G}}} di fornitura intende occuparsi della gestione dei rapporti con il proponente \emph{AzzurroDigitale} con l’obbiettivo di evitare qualsiasi tipo di ostacolo alla comunicazione ed avere un a buona qualità nella stessa.
\subsubsection{Aspettative}
Durante il rapporto il nostro gruppo desidera mantenere una comunicazione disponibile e con \emph{AzzurroDigitale}, in particolare con i referenti Martina Daniele, Camilla Picello, Nicola Boscaro e Mattia Gottardello, così da poter:
\begin{itemize}
    \item Discutere \emph{requisiti}\textsubscript{\textit{\textbf{G}}} chiave necessari da soddisfare nel prodotto finale;
    \item Stabilire tempistiche di lavoro;
    \item Ricevere \emph{feedback}\textsubscript{\textit{\textbf{G}}} sul lavoro in corso;
    \item Ottenere chiarimenti relativi a dubbi e incomprensioni;
    \item Stabilire i \emph{vincoli}\textsubscript{\textit{\textbf{G}}} riguardanti i processi intermedi.
\end{itemize}
\subsubsection{Rapporti col proponente}
Il proponente mette a disposizione un canale Discord al fine di rispondere a domande/dubbi occasionali, e una mail di riferimento per comunicazioni più formali.

Inoltre il proponente si rende disponibile a degli incontri da remoto, la cui frequenza è di una volta ogni 2 settimane.
E in aggiunta due incontri in presenza, di cui il primo ad inizio progetto e il secondo nella parte finale.
Ad ognuno di questi incontri la discussione verterà su 2 punti:
\begin{itemize}
    \item \emph{Revisione}\textsubscript{\textit{\textbf{G}}} sul lavoro svolto nel precedente sprint;
    \item Raccolta delle nuove \emph{specifiche}\textsubscript{\textit{\textbf{G}}} e richieste da soddisfare per lo sprint successivo.
\end{itemize}
Durante tali incontri, l’azienda non richiede della specifica documentazione, ma gradisce strumenti per visualizzare, anche graficamente, lo stato di avanzamento del lavoro appena svolto.
Per ciascun incontro verrà compilato dal nostro gruppo un verbale esterno, contenente gli argomenti di discussione e le decisioni prese, e sarà firmato dalla rappresentanza del proponente.
Tutti i verbali saranno visibili nella \emph{repository}\textsubscript{\textit{\textbf{G}}} dedicata alla documentazione.

\subsubsection{Strumenti}

Di seguito sono riportati gli strumenti utilizzati per realizzare il processo di fornitura:
\begin{itemize}
    \item \textbf{\emph{Git}}\textsubscript{\textit{\textbf{G}}}: software per il \emph{controllo di versione}\textsubscript{\textit{\textbf{G}}};
    \item \textbf{\emph{GitHub}}\textsubscript{\textit{\textbf{G}}}: servizio di \emph{hosting}\textsubscript{\textit{\textbf{G}}} per progetti software;
    \item \textbf{\emph{Jira}}\textsubscript{\textit{\textbf{G}}}: è un sistema software utilizzato per la gesitone delle attività, l’assegnazione delle
    risorse, la verifica dei tempi del progetto e l’analisi del lavoro svolto e da svolgere.
    Questo strumento è utile anche per generare i \emph{diagrammi di Gantt}\textsubscript{\textit{\textbf{G}}} presenti nel Piano
    di Progetto;
    \item \textbf{\emph{Discord}}\textsubscript{\textit{\textbf{G}}}: \emph{piattaforma}\textsubscript{\textit{\textbf{G}}} che mette a disposizione dei canali vocali con la possibilità di condivisione dello schermo;
    Utilizzata non solo dai membri del gruppo per comunicazioni interne ma anche per le comunicazioni rapide con il proponente;
    \item \textbf{\emph{Google Meet}}\textsubscript{\textit{\textbf{G}}}: piattaforma che permette di effettuare videoconferenze online. Utilizzata per organizzare incontri con il proponente;
    \item \textbf{\emph{\LaTeX}}\textsubscript{\textit{\textbf{G}}}: linguaggio di \emph{markup}\textsubscript{\textit{\textbf{G}}} scelto dal gruppo per la produzione della documentazione.
\end{itemize}

\subsection{Sviluppo}
\subsubsection{Scopo}
La fase di sviluppo si occupa di definire le attività che il team compie per soddisfare i requisiti delineati con il proponente.

\subsubsection{Descrizione}
Il processo di sviluppo consiste nello strutturare, suddividere ai membri del team e completare le attività relative alla \emph{codifica}\textsubscript{\textit{\textbf{G}}}. L’obbiettivo è che il software soddisfi le \emph{aspettative}\textsubscript{\textit{\textbf{G}}} del proponente.
Nel processo di sviluppo saranno effettuate le seguenti attività.
\begin{itemize}
    \item Analisi dei requisiti;
    \item Progettazione;
    \item Codifica.
\end{itemize}

\subsubsection{Aspettative}
Il gruppo \emph{SWEg Labs} intende ottenere tramite il processo di sviluppo un prodotto software in grado di superare i test e soddisfare i requisiti del proponente \emph{AzzurroDigitale}.

\subsubsection{Analisi dei requisiti}

\subsubsubsection{Descrizione}
L’analisi dei requisiti è un’attività svolta dall’analista, e produce il documento denominato “Analisi dei requisiti”. 
Tale documento descrive:
\begin{itemize}
    \item Lo scopo del prodotto;
    \item Le sue \emph{funzionalità}\textsubscript{\textit{\textbf{G}}};
    \item Gli \emph{attori}\textsubscript{\textit{\textbf{G}}} e le loro caratteristiche;
    \item I \emph{casi d'uso}\textsubscript{\textit{\textbf{G}}};
    \item I requisiti;
    \item La stima del lavoro necessario.
\end{itemize}

\subsubsubsection{Scopo}
L’analisi dei requisiti intende raccogliere, chiarire e precisare tutti i requisiti necessari da completare per soddisfare il cliente. Per poter fare ciò è necessario aver letto e compreso al meglio le specifiche del progetto, e poter comunicare al meglio con il proponente.

\subsubsubsection{Casi d'uso}
Un \emph{caso d'uso}\textsubscript{\textit{\textbf{G}}} è un insieme di \emph{scenari}\textsubscript{\textit{\textbf{G}}} che hanno in comune uno scopo finale per un \emph{attore}\textsubscript{\textit{\textbf{G}}}.\\
Gli elementi che lo compongono sono i seguenti:
\begin{itemize}
    \item l'attore;
    \item il sistema;
    \item precondizioni;
    \item postcondizioni;
    \item \emph{Scenario principale}\textsubscript{\textit{\textbf{G}}};
    \item \emph{Scenario alternativo}\textsubscript{\textit{\textbf{G}}};
    \item inclusione/i;
    \item estensione/i;
    \item specializzazione/i;
    \item commento/i;
    \item descrizione.  
\end{itemize}
I casi d’uso sono identificati nel seguente modo:\\
\begin{center}
    \textbf{UC[Numero]+(UC[Numero sottocaso])-[Titolo]}
\end{center}
dove:
\begin{itemize}
    \item UC: acronimo di "use case";
    \item Numero: numero identificativo del caso d’uso;
    \item Numero sottocaso: numero identificativo del sottocaso (se presente);
    \item Titolo: titolo assegnato al caso d’uso.
\end{itemize}

\subsubsubsection{Struttura dei requisiti}
I requisiti sono identificati da un codice univoco strutturato nel seguente modo:\\
\begin{center}
    \textbf{R[Importanza][Tipologia] [Codice](+[Codice figlio])}
\end{center}
dove:
\begin{itemize}
    \item R: acronimo di "Requisito";
    \item Importanza: indica l’importanza del requisito e può assumere i seguenti valori:
    \begin{itemize}
        \item O: requisito obbligatorio, cioè deve essere soddisfatto necessariamente per garantire la realizzazione del prodotto corrispondente agli accordi col \emph{proponente}\textsubscript{\textit{\textbf{G}}};
        \item D: requisito desiderabile, cioè porterebbe al prodotto ulteriori funzionalità e completezza qualora fosse soddisfatto;
        \item Z: requisito opzionale, cioè che potrebbe essere implementato solo se ci sono risorse, tempo e budget sufficienti, senza che la sua mancanza impatti negativamente il prodotto.
    \end{itemize}
    \item Tipologia:
        \begin{itemize} 
            \item F: requisito funzionale che delinea gli obiettivi e le azioni chiave che l’utente deve essere in grado di compiere;
            \item Q: requisito qualitativo che delinea le specifiche qualitative che devono essere rispettate per garantire la qualità del sistema;
            \item V: requisito di vincolo che rappresenta le restrizioni e le condizioni che devono essere soddisfatte durante lo sviluppo e l’implementazione del sistema;
            \item I: requisito implementativo che delinea le specifiche tecniche e operative che indicano come un sistema o una funzionalità deve essere sviluppato per soddisfare i requisiti del progetto;  
            \item P: requisito prestazionale che definisce le metriche di performance che il sistema deve soddisfare. 
        \end{itemize}
    \item Codice: identificatore numerico univoco per quella tipologia di vincolo;
    \item Codice figlio: numero identificativo del sottorequisito (se presente).
\end{itemize}

\subsubsection{Progettazione}

\subsubsubsection{Scopo}
L'attività di progettazione ha l'obiettivo di definire l'architettura del prodotto in modo da soddisfare le esigenze di tutti gli \emph{stakeholder}\textsubscript{\textit{\textbf{G}}}, identificate dall’analisi dei requisiti. Identificando e documentando le soluzioni che rispondono ai requisiti evidenziati si garantisce al contempo una chiara suddivisione delle responsabilità di sviluppo e manutenzione. Questa attività fornisce una documentazione completa e dettagliata sulla struttura del prodotto, incluse le specifiche tecniche e le scelte tecnologiche adottate. 

\subsubsubsection{Descrizione}
La progettazione si articola in più livelli per assicurare una completa copertura delle funzionalità e della struttura del prodotto:
\begin{itemize}
    \item \textbf{Progettazione logica}: Questa fase definisce le tecnologie, i \emph{framework}\textsubscript{\textit{\textbf{G}}} e le librerie scelti per la realizzazione del prodotto, motivando l'adeguatezza delle scelte e dimostrando la fattibilità tecnica attraverso un \emph{Proof of Concept}\textsubscript{\textit{\textbf{G}}}(PoC);
    \item \textbf{Progettazione di dettaglio}: In questa fase si definisce l'\emph{architettura}\textsubscript{\textit{\textbf{G}}} di dettaglio, seguendo quanto stabilito nella progettazione logica e sviluppando una rappresentazione completa delle componenti software.
\end{itemize}

\subsubsection{Codifica}

\subsubsubsection{Scopo}
L’attività di \emph{codifica}\textsubscript{\textit{\textbf{G}}} è mirata allo sviluppo del prodotto software da parte dei programmatori, il quale deve soddisfare le esigenze concordate con il \emph{proponente}.

\subsubsubsection{Descrizione}

Durante questa attività, lo sviluppatore si impegna a soddisfare i requisiti implementando il codice nel linguaggio di programmazione scelto.
Il codice deve rispettare le linee guida definite nella documentazione del progetto. Contestualmente, tutte le nuove unità software sviluppate e le modifiche apportate devono essere adeguatamente documentate.

\subsubsubsection{Stile della codifica}
Per garantire lo sviluppo del codice di qualità useremo i seguenti criteri:

\begin{itemize}
    \item Backend:
    \begin{itemize}
        \item \textbf{Variabili, attributi, funzioni e metodi:} \emph{Snake Case}\textsubscript{\textit{\textbf{G}}}.
        \item \textbf{Classi:} \emph{Pascal Case}\textsubscript{\textit{\textbf{G}}}.
        \item \textbf{Nomi di file:}
        \begin{itemize}
            \item Camel Case se il file contiene una classe;
            \item Snake Case se non contiene una classe.
        \end{itemize}
        \item \textbf{Docstring}: Ogni metodo o funzione deve avere un commento descrittivo sotto alla firma, scritto in inglese.
    \end{itemize}
    
    \item Frontend:
    \begin{itemize}
        \item \textbf{Variabili, attributi, funzioni e metodi:} \emph{Camel Case}\textsubscript{\textit{\textbf{G}}}.
        \item \textbf{Classi:} Pascal Case.
        \item \textbf{Nomi dei file:} \emph{Kebab Case}\textsubscript{\textit{\textbf{G}}}.
    \end{itemize}
    
    \item \textbf{Lunghezza delle righe di codice}:
    \begin{itemize}
        \item La lunghezza massima di una riga di codice non deve superare i 100 caratteri.
    \end{itemize}
    
    \item \textbf{Indentazioni}:
    \begin{itemize}
        \item I blocchi annidati del codice devono seguire un'indentazione con un carattere di tabulazione equivalente a 4 spazi.
    \end{itemize}
\end{itemize}

\subsubsubsection{Tecnologie Utilizzate}
\begin{itemize}
    \item \textbf{\emph{Python}\textsubscript{\textit{\textbf{G}}}}: un linguaggio di programmazione ad alto livello orientato ad oggetti. Si
    utilizza per realizzare la logica dell’applicazione.
    \begin{center}
        \textbf{\url{https://www.python.org/}}
    \end{center}

    \item \textbf{\emph{Angular}\textsubscript{\textit{\textbf{G}}}}: un framework usato per lo sviluppo dell’interfaccia grafica dell’applicazione.
    \begin{center}
        \textbf{\url{https://angular.dev/}}
    \end{center}
    
    \item \textbf{\emph{Chroma}\textsubscript{\textit{\textbf{G}}}}: un database vettoriale che memorizza e ricerca dati basandosi sulla loro somiglianza semantica.
    \begin{center}
        \textbf{\url{https://www.trychroma.com/}}
    \end{center}

    \item \textbf{\emph{GPT-4o}\textsubscript{\textit{\textbf{G}}}}: un \emph{LLM}\textsubscript{\textit{\textbf{G}}}, sviluppato da OpenAI, capace di comprendere e generare testo in modo contestuale.
    \begin{center}
        \textbf{\url{https://openai.com/index/hello-gpt-4o/}}
    \end{center}

    \item \textbf{\emph{Docker}\textsubscript{\textit{\textbf{G}}}}: una piattaforma per eseguire applicazioni in container, ambienti isolati che includono tutto il necessario per funzionare ovunque.
    \begin{center}
        \textbf{\url{https://www.docker.com}}
    \end{center}
    
    \item \textbf{\emph{Python-Crontab}\textsubscript{\textit{\textbf{G}}}}: una libreria Python che permette di leggere, scrivere e gestire {\emph{cron job}\textsubscript{\textit{\textbf{G}}}} direttamente da codice Python.
    \begin{center}
        \textbf{\url{https://pypi.org/project/python-crontab/}}
    \end{center}
   
\end{itemize}


