% Intestazione
\fancyhead[L]{2 \hspace{0.2cm} Processi primari} % Testo a sinistra


\section{Processi primari}
\label{sec:processi_primari}
\subsection{Fornitura}
\subsubsection{Descrizione}
Questa sezione riporta tutte le norme, gli strumenti e i metodi che ogni membro del gruppo \emph{SWEg Labs} si impegna a rispettare al fine di preservare al meglio i rapporti con il proponente \emph{AzzurroDigitale}\textsubscript{\textit{\textbf{G}}}.
\subsubsection{Scopo}
Il \emph{processo}\textsubscript{\textit{\textbf{G}}} di fornitura intende occuparsi della gestione dei rapporti con il proponente \emph{AzzurroDigitale} con l’obbiettivo di evitare qualsiasi tipo di ostacolo alla comunicazione ed avere un a buona qualità nella stessa.
\subsubsection{Aspettative}
Durante il rapporto il nostro gruppo desidera mantenere una comunicazione disponibile e con \emph{AzzurroDigitale}, in particolare con i referenti Martina Daniele, Camilla Picello, Nicola Boscaro e Mattia Gottardello, così da poter:
\begin{itemize}
    \item Discutere \emph{requisiti}\textsubscript{\textit{\textbf{G}}} chiave necessari da soddisfare nel prodotto finale;
    \item Stabilire tempistiche di lavoro;
    \item Ricevere \emph{feedback}\textsubscript{\textit{\textbf{G}}} sul lavoro in corso;
    \item Ottenere chiarimenti relativi a dubbi e incomprensioni;
    \item Stabilire i \emph{vincoli}\textsubscript{\textit{\textbf{G}}} riguardanti i processi intermedi.
\end{itemize}
\subsubsection{Rapporti col proponente}
Il proponente mette a disposizione un canale Discord al fine di rispondere a domande/dubbi occasionali, e una mail di riferimento per comunicazioni più formali.

Inoltre il proponente si rende disponibile a degli incontri da remoto, la cui frequenza è di una volta ogni 2 settimane.
E in aggiunta due incontri in presenza, di cui il primo ad inizio progetto e il secondo nella parte finale.
Ad ognuno di questi incontri la discussione verterà su 2 punti:
\begin{itemize}
    \item \emph{Revisione}\textsubscript{\textit{\textbf{G}}} sul lavoro svolto nel precedente sprint;
    \item Raccolta delle nuove \emph{specifiche}\textsubscript{\textit{\textbf{G}}} e richieste da soddisfare per lo sprint successivo.
\end{itemize}
Durante tali incontri, l’azienda non richiede della specifica documentazione, ma gradisce strumenti per visualizzare, anche graficamente, lo stato di avanzamento del lavoro appena svolto.
Per ciascun incontro verrà compilato dal nostro gruppo un verbale esterno, contenente gli argomenti di discussione e le decisioni prese, e sarà firmato dalla rappresentanza del proponente.
Tutti i verbali saranno visibili nella \emph{repository}\textsubscript{\textit{\textbf{G}}} dedicata alla documentazione.

\subsubsection{Strumenti}

Di seguito sono riportati gli strumenti utilizzati per realizzare il processo di fornitura:
\begin{itemize}
    \item \textbf{\emph{Git}}\textsubscript{\textit{\textbf{G}}}: software per il \emph{controllo di versione}\textsubscript{\textit{\textbf{G}}};
    \item \textbf{\emph{GitHub}}\textsubscript{\textit{\textbf{G}}}: servizio di \emph{hosting}\textsubscript{\textit{\textbf{G}}} per progetti software;
    \item \textbf{\emph{Jira}}\textsubscript{\textit{\textbf{G}}}: è un sistema software utilizzato per la gesitone delle attività, l’assegnazione delle
    risorse, la verifica dei tempi del progetto e l’analisi del lavoro svolto e da svolgere.
    Questo strumento è utile anche per generare i \emph{diagrammi di Gantt}\textsubscript{\textit{\textbf{G}}} presenti nel Piano
    di Progetto;
    \item \textbf{\emph{Discord}}\textsubscript{\textit{\textbf{G}}}: \emph{piattaforma}\textsubscript{\textit{\textbf{G}}} che mette a disposizione dei canali vocali con la possibilità di condivisione dello schermo;
    Utilizzata non solo dai membri del gruppo per comunicazioni interne ma anche per le comunicazioni rapide con il proponente;
    \item \textbf{\emph{Zoom}}\textsubscript{\textit{\textbf{G}}}: piattaforma che permette di effettuare videoconferenze online. Utilizzata per organizzare incontri con il proponente;
    \item \textbf{\emph{\LaTeX}}\textsubscript{\textit{\textbf{G}}}: linguaggio di \emph{markup}\textsubscript{\textit{\textbf{G}}} scelto dal gruppo per la produzione della documentazione.
\end{itemize}

\subsection{Sviluppo}
\subsubsection{Scopo}
La fase di sviluppo si occupa di definire le attività che il team compie per soddisfare i requisiti delineati con il proponente.

\subsubsection{Descrizione}
Il processo di sviluppo consiste nello strutturare, suddividere ai membri del team e completare le attività relative alla \emph{codifica}\textsubscript{\textit{\textbf{G}}}. L’obbiettivo è che il software soddisfi le \emph{aspettative}\textsubscript{\textit{\textbf{G}}} del proponente.
Nel processo di sviluppo saranno effettuate le seguenti attività.
\begin{itemize}
    \item Analisi dei requisiti;
    \item Progettazione;
    \item Codifica.
\end{itemize}

\subsubsection{Aspettative}
Il gruppo \emph{SWEg Labs} intende ottenere tramite il processo di sviluppo un prodotto software in grado di superare i test e soddisfare i requisiti del proponente \emph{AzzurroDigitale}.

\subsubsection{Analisi dei requisiti}

\subsubsubsection{Descrizione}
L’analisi dei requisiti è un’attività svolta dall’analista, e produce il documento denominato “Analisi dei requisiti”. 
Tale documento descrive:
\begin{itemize}
    \item Lo scopo del prodotto;
    \item Le sue \emph{funzionalità}\textsubscript{\textit{\textbf{G}}};
    \item Gli \emph{attori}\textsubscript{\textit{\textbf{G}}} e le loro caratteristiche;
    \item I \emph{casi d'uso}\textsubscript{\textit{\textbf{G}}};
    \item I requisiti;
    \item La stima del lavoro necessario.
\end{itemize}

\subsubsubsection{Scopo}
L’analisi dei requisiti intende raccogliere, chiarire e precisare tutti i requisiti necessari da completare per soddisfare il cliente. Per poter fare ciò è necessario aver letto e compreso al meglio le specifiche del progetto, e poter comunicare al meglio con il proponente.

\subsubsubsection{Casi d'uso}
\subsubsubsection{Struttura dei requisiti}

\subsubsection{Progettazione}

\subsubsubsection{Scopo}
L'attività di progettazione ha l'obiettivo di definire l'architettura del prodotto in modo da soddisfare le esigenze di tutti gli \emph{stakeholder}\textsubscript{\textit{\textbf{G}}}, identificate dall’analisi dei requisiti. Identificando e documentando le soluzioni che rispondono ai requisiti evidenziati si garantisce al contempo una chiara suddivisione delle responsabilità di sviluppo e manutenzione. Questa attività fornisce una documentazione completa e dettagliata sulla struttura del prodotto, incluse le specifiche tecniche e le scelte tecnologiche adottate. 

\subsubsubsection{Descrizione}
La progettazione si articola in più livelli per assicurare una completa copertura delle funzionalità e della struttura del prodotto:
\begin{itemize}
    \item \textbf{Progettazione logica}: Questa fase definisce le tecnologie, i \emph{framework}\textsubscript{\textit{\textbf{G}}} e le librerie scelti per la realizzazione del prodotto, motivando l'adeguatezza delle scelte e dimostrando la fattibilità tecnica attraverso un \emph{Proof of Concept}\textsubscript{\textit{\textbf{G}}}(PoC);
    \item \textbf{Progettazione di dettaglio}: In questa fase si definisce l'\emph{architettura}\textsubscript{\textit{\textbf{G}}} di dettaglio, seguendo quanto stabilito nella progettazione logica e sviluppando una rappresentazione completa delle componenti software.
\end{itemize}

