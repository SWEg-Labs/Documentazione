% Intestazione
\fancyhead[L]{4 \hspace{0.2cm} Decisioni} % Testo a sinistra

\section{Decisioni}

Durante la riunione sono state prese le seguenti decisioni:
 
\vspace{0.5cm}

\begin{table}[htbp]
    \centering
    \rowcolors{2}{lightgray}{white}
    \begin{tabular}{|c|p{0.8\textwidth}|}
        \hline
        \rowcolor[gray]{0.75}
        \textbf{Codice} & \textbf{Descrizione}\\
        \hline
        VI 12.1 & È stato deciso di inserire un ramo develop nella repository, da utilizzare da chi scrive, per poi essere effettuato il merge nel ramo main dopo aver effettuato la verifica.\\
        \hline
        VI 12.2 & È stato deciso di iniziare a inserire task senza assegnatari, valutandone l'efficacia.\\
        \hline
        VI 12.3 & È stato deciso che i costi monetari saranno inseriti nell'analisi dei costi da Giacomo Loat.\\
        \hline
        VI 12.4 & Si è deciso di proporre Next.js all'azienda \emph{AzzurroDigitale} in occasione dell'incontro di Giovedì.\\
        \hline
        VI 12.5 & È stato deciso di utilizzare un sistema per dividere i documenti in sezioni da 500 parole al fine di diminuire il numero di token. Michael Fantinato si occuperà di trovare un modo per applicarlo al nostro caso specificoo.\\
        \hline
        VI 12.6 & È stato deciso che tutta la documentazione dovrà essere terminata prima dell'incontro di Giovedì con il proponente.\\
        \hline
        VI 12.7 & Ogni membro del gruppo è stato invitato a cercare una soluzione per risolvere il problema legato all'apparente impossibilità di caricare file pdf su Confluence. In caso di mancata soluzione si provvederà a farlo presente al proponente nella riunione di Giovedì;\\
        \hline
        VI 12.8 & È stato deciso di effettuare una transizione dal software Github Project a Jira come strumenti utilizzati per la gestione di progetti e il tracking delle attività;\\
        \hline
        VI 12.9 & È stato deciso di non utilizzare Spring Boot come framework per la programmazione back-end;\\
        \hline
        VI 12.10 & È stato deciso di proporre all'azienda di utilizzare GTP4o, chiedendo 100 dollari per pagarne i token necessari all'utilizzo. In caso di riscontro alternativo abbiamo intenzione di proporre GTP4o mini, Grok o Llama come alternative;\\
        \hline
        VI 12.11 & Approvato il piano generale di lavoro e la stima dei costi;\\
        \hline
        VI 12.12 & Deciso di sistemare graficamente il piano generale di alvoro e la stima dei costi. Se ne occuperà Giacomo Loat;\\
        \hline
        VI 12.13 & Stabilito che attendiamo la prima riunione per iniziare a trattare i casi d'uso;\\
        \hline
        VI 12.14 & Fissato prossima riunione interna Mercoledì dalle 16:00 alle 17:00;\\
        \hline
        VI 12.15 & Stabilire Mercoledì le pratiche da svolgere per la prima sprint;\\
        \hline
        VI 12.16 & Pianificata prossima riunione Mercoledì dalle 16:00 alle 17:00;\\
        \hline
        VI 12.17 & Non ci sarà incontro dopo riunione con \emph{AzzurroDigitale} Giovedì dalle 18:00 alle 19:00;\\
        \hline
        VI 12.18 & Deciso che ognuno si farà il proprio account Atlassian, e Riccardo farà la "organization";\\
        \hline
        VI 12.19 & Deciso che Davide farà il verbale di questa riunione;\\
        \hline

    \end{tabular}
\end{table}