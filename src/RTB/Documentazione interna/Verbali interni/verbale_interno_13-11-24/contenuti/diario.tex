% Intestazione
\fancyhead[L]{3 \hspace{0.2cm} Diario della riunione} % Testo a sinistra

\section{Diario della riunione}

\begin{itemize}
    \item Il gruppo ha discusso le varie proposte avanzate da \emph{AzzurroDigitale}\textsubscript{\textit{\textbf{G}}} nell'incontro del 14 novembre. Nello specifico si è trattato di:
    \begin{itemize}
        \renewcommand{\labelitemii}{--}
        \item \textbf{Ruoli e Sprint:} È stato concordato di non stabilire ruoli specifici fino all'incontro di giovedì prossimo. I ruoli saranno definiti al momento dell'avvio dello sprint, previsto per giovedì prossimo.
        \item \textbf{Verificatori Documentazione:} È stata scelta una squadra di verificatori incaricati di controllare le parti di testo scritte nei vari documenti durante giorni precedenti.
        \item \textbf{Documentazione:} È stata programmata un'ulteriore sessione dedicata alla scrittura della documentazione.
        \item \textbf{Pianificazione degli Incontri:} È stato programmato un incontro per la stesura del diario di bordo di lunedì 18 novembre.
    \end{itemize}
    
    \item \textbf{Sistemazione del Piano Generale di Lavoro:} Michael provvederà a strutturare il documento per evitare divagazioni, mentre Giacomo si occuperà dell'abbellimento grafico.
    
    \item \textbf{Scelta del Modello:}
    \begin{itemize}
        \renewcommand{\labelitemii}{--}
        \item \textbf{Michael} ricercerà informazioni riguardanti modelli open source.
        \item \textbf{Giacomo} supporterà la scelta tra \emph{OpenAI}\textsubscript{\textit{\textbf{G}}} e Bert.
        \item \textbf{Davide} proporrà Llama e Grok.
        \item \textbf{Riccardo} suggerirà Claude, Phi e soluzioni open source.
        \item \textbf{Filippo} presenterà Mistral e Llama.
        \item \textbf{Federica} supporterà \emph{OpenAI} e soluzioni open source.
    \end{itemize}
    
    \item \textbf{Stima dei Token Utilizzati in Input:}
    \begin{itemize}
        \renewcommand{\labelitemii}{--}
        \item È necessario decidere tra allenamento preliminare e ricerca sul momento.
        \item \textbf{Michael} si occuperà di effettuare una stima.
    \end{itemize}
\end{itemize}