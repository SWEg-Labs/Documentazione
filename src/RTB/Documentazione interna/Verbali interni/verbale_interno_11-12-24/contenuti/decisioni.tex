% Intestazione
\fancyhead[L]{4 \hspace{0.2cm} Decisioni} % Testo a sinistra

\section{Decisioni}

Durante la riunione sono state prese le seguenti decisioni:

\vspace{0.5cm}

\begin{table}[htbp]
    \centering
    \rowcolors{2}{lightgray}{white}
    \begin{tabular}{|c|p{0.8\textwidth}|}
        \hline
        \rowcolor[gray]{0.75}
        \textbf{Codice} & \textbf{Descrizione}\\
        \hline
        VI 18.1 & Politica di comportamento per il verificatore: modifiche piccole fatte subito, modifiche grosse rimandate all'autore.\\
        \hline
        VI 18.2 & I casi d'uso \emph{back-end}\textsubscript{\textit{\textbf{G}}}devono essere scritti in relazione all'utente.\\
        \hline
        VI 18.3 & Gli attori saranno descritti nel documento \emph{Analisi dei Requisiti}.\\
        \hline
        VI 18.4 & Richiedere un incontro online con il professore Cardin. \\
        \hline
        VI 18.5 & Suddivisione dei compiti: Federica, Filippo e Davide per traduzione dei \emph{casi d'uso} in \emph{requisiti} mentre Riccardo, Giacomo e Michael per amministrazione e programmazione.\\
        \hline
        VI 18.6 & Scelta del database locale per l'aggiornamento automatico. \\
        \hline
        VI 18.7 & Continuare lo studio dei 5 temi tecnologici.\\
        \hline
        VI 18.8 & Pianificazione delle strategie di programmazione.\\
        \hline
        VI 18.9 & Inizio del \emph{Proof of Concept}.\\
        \hline
        VI 18.10 & Esplorazione di \emph{Angular} in parallelo con il \emph{PoC}.\\
        \hline
        VI 18.11 & Non decidere subito l'architettura, iniziare senza un \emph{pattern architetturale}\textsubscript{\textit{\textbf{G}}}.\\
        \hline
        VI 18.12 & Bisogna stabilire un avanzamento minimo per il periodo natalizio.\\
        \hline
    \end{tabular}
\end{table}