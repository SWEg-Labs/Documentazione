% Intestazione
\fancyhead[L]{3 \hspace{0.2cm} Diario della riunione} % Testo a sinistra

\section{Diario della riunione}

\begin{itemize}
    \item Durante la riunione si è discusso della politica di comportamento per il verificatore, delineando due possibili modalità di intervento:
    \begin{itemize}
    \item Se il verificatore individua una modifica di piccola entità, può procedere a farla subito, informando l'autore del lavoro che la modifica è stata già effettuata.
    \item Se il verificatore rileva una modifica più rilevante e impegnativa, può rinviare la correzione all'autore originale, comunicandogli cosa deve modificare.
    \end{itemize}
    \item Si è deciso di procedere con la creazione dei \emph{casi d'uso} relazionati all'utente, in conformità con le istruzioni apprese durante le lezioni in aula.
    \item Abbiamo stabilito di poter scrivere gli \emph{attori} nel documento \emph{Analisi dei Requisiti}.
    \item Durante la riunione si è discusso dell'aggiornamento del database vettoriale, dopo aver escluso la possibilità di un database remoto sono state considerate due possibili alternative:
    \begin{itemize}
        \item La prima alternativa prevede l'uso di pacchetti gratuiti che creano un tunnel tra un server locale e rendono il servizio disponibile online. In questo modo, sarebbe possibile implementare una \emph{GitHub Action}\textsubscript{\textit{\textbf{G}}}. Tuttavia, per ottenere un \emph{URL}\textsubscript{\textit{\textbf{G}}} univoco è necessario un pagamento.
        \item La seconda alternativa consiste nell'hostare un servizio di coda di richieste. In questo scenario, la \emph{GitHub Action} si collegherebbe a un servizio di terze parti che memorizza le richieste \emph{API}\textsubscript{\textit{\textbf{G}}} ricevute. Il container \emph{Docker}\textsubscript{\textit{\textbf{G}}} si connette a questo servizio, verifica se sono state fatte modifiche nel branch principale (main) e avvia l'aggiornamento automatico. Questa soluzione potrebbe avere dei costi variabili, in base ai diversi piani offerti dal servizio.
    \end{itemize}
    In conclusione, è stato deciso di procedere con la soluzione del database locale, con la possibilità di riconsiderare l'opzione tra qualche mese.
    \item Durante la riunione si è discusso dei 5 temi tecnologici, assegnati come task nella precedente riunione.  
    Successivamente, è stata fatta una retrospettiva sullo studio individuale e asincrono di questi temi, con il seguente riscontro:
    \begin{itemize}
    \item Lo studio non è stato valutabile in modo completo, poiché il team è stato fortemente impegnato con l'\emph{Analisi dei Requisiti}. Abbiamo deciso che lo studio proseguirà.
    \end{itemize} 
    \item Si è parlato di iniziare il \emph{Proof of Concept}. Per avviare il progetto, si può creare la \emph{repository}\textsubscript{\textit{\textbf{G}}}.  
    \item Si è discusso di un potenziale problema nella consegna dei dati al \emph{bot}, in particolare per far sì che il \emph{bot} li comprenda correttamente. Un esempio è stato l'invio di una \emph{issue}\textsubscript{\textit{\textbf{G}}} di emph{Jira}\textsubscript{\textit{\textbf{G}}}, dove il \emph{bot} non ha compreso che l'assegnatario era sotto la voce "assignee", nonostante avesse accesso ai dati inviati. Si è ipotizzato che potrebbe essere una limitazione di GPT-4o-mini, mentre una versione più avanzata come GPT-4o potrebbe risolvere il problema. 
    Tuttavia, sarebbe utile fare un test per confermare questa ipotesi.  
    Inoltre, si è proposta l'idea di provare con un header nel emph{prompt}\textsubscript{\textit{\textbf{G}}}, specificando di considerare solo i dati sotto determinate voci.
    \item Durante la riunione si è discusso della scelta dell'\emph{architettura} per il progetto. Si è considerato che, per avere un' "architettura accogliente" come suggerito nel diario di bordo, potrebbe avere senso adottare il pattern \emph{MVC}\textsubscript{\textit{\textbf{G}}}, poiché supporta aggiornamenti in tempo reale.  
    Tuttavia, si è notato che \emph{Angular}\textsubscript{\textit{\textbf{G}}} utilizza il pattern \emph{MVVM}\textsubscript{\textit{\textbf{G}}}, che offre anch'esso la possibilità di gestire aggiornamenti in tempo reale.  
    Al momento, il team ha deciso di non concentrarsi sul design e di procedere senza un pattern architetturale specifico. La questione sarà affrontata in seguito durante il \emph{PB}\textsubscript{\textit{\textbf{G}}}.
    \item Abbiamo affrontato il tema di un avanzamento minimo garantito per il progetto durante le vacanze natalizie.
    Il team ha deciso di definire l'avanzamento minimo in seguito, in base a come evolverà il \emph{PoC}. 
    
    
    
\end{itemize}
