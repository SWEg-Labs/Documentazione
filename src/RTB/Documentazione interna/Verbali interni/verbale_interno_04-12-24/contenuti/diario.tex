% Intestazione
\fancyhead[L]{3 \hspace{0.2cm} Diario della riunione} % Testo a sinistra

\section{Diario della riunione}

\begin{itemize}
    \item È stato presentato un tentativo fallito di inserimento dell'orario nella parte grafica di un messaggio utilizzando la libreria 
    \emph{Streamlit}\textsubscript{\textit{\textbf{G}}}, dimostrando la poca flessibilità della libreria stessa. È stato dunque chiarito quanto
    tale libreria sia limitante in confronto al framework \emph{Angular}\textsubscript{\textit{\textbf{G}}}.
    \item Si è parlato di database vettoriali, in particolare delle due possibilità di hosting, o in locale (localhost) o in remoto (cloud).\\\\
    In locale:
    \begin{itemize}
        \item Caratteristiche positive:
        \begin{itemize}
            \item Non è necessario un account.
        \end{itemize}
        \item Caratteristiche negative:
        \begin{itemize}
            \item In caso di spegnimento del server localhost, i dati vengono persi.
        \end{itemize}
    \end{itemize}
    In remoto:
    \begin{itemize}
        \item Caratteristiche positive:
        \begin{itemize}
            \item I dati sono sempre disponibili.
        \end{itemize}
        \item Caratteristiche negative:
        \begin{itemize}
            \item È necessario un account.
            \item È necessario quantizzare quanto spazio ci serve, con conseguente pagamento se viene superata una determinata soglia.
        \end{itemize}
    \end{itemize}
    In particolare, per il database vettoriale locale si è parlato di \emph{Chroma}\textsubscript{\textit{\textbf{G}}} come provider, mentre invece per il database
    vettoriale remoto si è parlato dei provider \emph{Supabase}\textsubscript{\textit{\textbf{G}}} e \emph{Pinecone}\textsubscript{\textit{\textbf{G}}},
    analizzando alcune prove pratiche da noi svolte durante il precedente periodo asincrono.
    \item  Sono state valutate le modifiche da apportare ai diagrammi dei casi d'uso per attenersi allo standard UML.
    \item È stata preparata una presentazione per l'incontro esterno con \emph{AzzurroDigitale}\textsubscript{\textit{\textbf{G}}} del giorno 5 dicembre 2024,
    contenente la descrizione di quanto è stato fatto nella prima sprint e alcune domande su come procedere a riguardo dei casi d'uso e delle tecnologie.
    \item Abbiamo riflettuto su alcune piccole operazioni che bisogna compiere per mantenere aggiornato il nostro progetto \emph{Jira}\textsubscript{\textit{\textbf{G}}}
    e i dati su \emph{Fogli Google}\textsubscript{\textit{\textbf{G}}} per il \emph{Piano di Progetto}\textsubscript{\textit{\textbf{G}}} e per il 
    \emph{Piano di Qualifica}\textsubscript{\textit{\textbf{G}}}.
\end{itemize}
