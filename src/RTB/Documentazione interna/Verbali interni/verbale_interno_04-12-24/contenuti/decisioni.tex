% Intestazione
\fancyhead[L]{4 \hspace{0.2cm} Decisioni} % Testo a sinistra

\section{Decisioni}

Durante la riunione sono state prese le seguenti decisioni:

\vspace{0.5cm}

\begin{table}[htbp]
    \centering
    \rowcolors{2}{lightgray}{white}
    \begin{tabular}{|c|p{0.8\textwidth}|}
        \hline
        \rowcolor[gray]{0.75}
        \textbf{Codice} & \textbf{Descrizione}\\
        \hline
        VI 16.1 & È stato deciso di sviluppare il \emph{front-end} in \emph{Angular}\textsubscript{\textit{\textbf{G}}}. \\
        \hline
        VI 16.2 & È stato deciso di utilizzare il provider di database vettoriali \emph{ChromaDB}. \\
        \hline
        VI 16.3 & È stato deciso, per i diagrammi dei casi d'uso, di unificare i casi di errore al loro caso d'uso primario attraverso una freccia extend,
        pur rimanendo casi d'uso separati, e invece di mantenere staccati dal flusso precedente i diagrammi dei casi d'uso di visualizzazione a schermo. \\
        \hline
        VI 16.4 & È stato deciso che i casi d'uso finora pensati sono a sufficienza validi da poter già preparare il loro inserimento nell'\emph{Analisi dei Requisiti}\textsubscript{\textit{\textbf{G}}} \\
        \hline
        VI 16.5 & Sono state stabilite le domande che porremo all'incontro esterno con \emph{AzzurroDigitale} del giorno 5 dicembre 2024. \\
        \hline
        VI 16.6 & È stato scelto Riccardo come oratore della parte sui casi d'uso della presentazione dell'incontro esterno con
        \emph{AzzurroDigitale}, e Giacomo come oratore della parte sulle tecnologie. \\
        \hline
        VI 16.7 & È stato stabilito che ogni membro del gruppo si impegnerà a mantenere aggiornato il nostro progetto \emph{Jira} e i dati su 
        \emph{Fogli Google} per il \emph{Piano di Progetto} e per il \emph{Piano di Qualifica}, mano a mano che svolgerà dei compiti legati al progetto. \\
        \hline
    \end{tabular}
\end{table}