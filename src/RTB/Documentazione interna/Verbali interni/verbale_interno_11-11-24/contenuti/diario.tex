% Intestazione
\fancyhead[L]{3 \hspace{0.2cm} Diario della riunione} % Testo a sinistra


\section{Diario della riunione}

\begin{itemize}
    \item Il gruppo ha discusso e steso collettivamente i contenuti del \emph{Diario di Bordo} del 12 novembre.
    \item Sono stati rivisti i temi discussi nei precedenti incontri con \emph{AzzurroDigitale} e
    sono stati scelti alcuni argomenti di discussione da proporre nell'incontro con l'azienda del 12 novembre:
    \begin{itemize}
        \renewcommand{\labelitemii}{--}
        \item Stabilire intermittenza e modalità dei rapporti con il \emph{proponente}\textsubscript{\textit{\textbf{G}}}.
        \item Stabilire il modello di sviluppo da adottare (\emph{agile}\textsubscript{\textit{\textbf{G}}} o emph{incrementale}\textsubscript{\textit{\textbf{G}}}).
    \end{itemize}
    Il gruppo inoltre si aspetta di trattare in modo approfondito l'\emph{Analisi dei Requisiti}\textsubscript{\textit{\textbf{G}}} e i \emph{Casi d'Uso}\textsubscript{\textit{\textbf{G}}}; 
    si è deciso di lasciare ad \emph{AzzurroDigitale} la gestione di questa parte della riunione senza preparare domande o richieste in anticipo.
    \item Il gruppo ha discusso riguardo i grafici da inserire nel \emph{Piano di Progetto} e nel \emph{Piano di Qualifica}.
    Riccardo Stefani ha mostrato al gruppo la configurazione del \emph{Diagramma di Gantt}\textsubscript{\textit{\textbf{G}}} su \emph{Github Projects}\textsubscript{\textit{\textbf{G}}}.
    Inoltre si è deciso di rinviare la discussione riguardo la creazione di un \emph{Diagramma di Burndown}\textsubscript{\textit{\textbf{G}}} e dei grafici per \emph{Preventivo}\textsubscript{\textit{\textbf{G}}} e \emph{Consuntivo}\textsubscript{\textit{\textbf{G}}}.
    Infine il gruppo ha constatato la necessità di predisporre da subito un \emph{Foglio Google}\textsubscript{\textit{\textbf{G}}} per la raccolta dei dati da inserire nei grafici.
    \item Si è discussa la possibilità di non assegnare immediatamente le nuove \emph{issue} inserite sul \emph{repository Github}\textsubscript{\textit{\textbf{G}}} e piuttosto
    lasciare che siano i singoli membri, una volta finito un compito, ad auto assegnarsi una delle nuove attività in attesa.
    Tuttavia si è deciso di mantenere il procedimento attuale che consiste nell'assegnare tutte le \emph{issue} al momento della creazione e,
    per avere un certo grado di flessibilità, permettere ai membri di prendere in carico task precedentemente assegnate ad altri,
    previa notifica nei canali di comunicazione interni del gruppo.
\end{itemize}