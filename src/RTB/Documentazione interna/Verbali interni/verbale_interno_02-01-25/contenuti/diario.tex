% Intestazione
\fancyhead[L]{3 \hspace{0.2cm} Diario della riunione} % Testo a sinistra

\section{Diario della riunione}

\begin{itemize}
    \item Discussione sulle modifiche apportate ai diagrammi dei \emph{casi d'uso}\textsubscript{\textit{\textbf{G}}} e alle numerazioni assegnate per verificare che fossero approvate da tutti. Tutti i \emph{casi d'uso} dovranno essere ripresi in mano per cambiare l'immagine dello screenshot;
    \item Assegnazione dei compiti per la trascrizione dei \emph{casi d'uso} nell'analisi dei requisiti;
    \item Abbiamo discusso e deciso di sospendere tutte le altre attività per concentrarci sulla trascrizione dei \emph{casi d'uso} dato che attualmente è il compito più importante;
    \item Abbiamo discusso su quando organizzare il prossimo incontro con il professor Cardin e quali domande trattare;
    \item Abbiamo fatto un resoconto su come stiamo procedendo con la scrittura dell'analisi dei requisiti e delle norme di progetto;
    \item Abbiamo discusso del Piano di Qualifica e concordato che è il momento opportuno per iniziare la compilazione della sezione "Strategie di test".
    \item Durante la discussione sul PoC, si è deciso di creare due rami separati: uno dedicato all'integrazione di Angular, affidato a Michael, e un altro dedicato allo sviluppo della funzionalità dell'aggiornamento automatico con l'integrazione di \emph{Docker}\textsubscript{\textit{\textbf{G}}}, affidato a Giacomo. Inoltre si ha discusso e deciso di caricare le proprie modifiche in remoto ad ogni cambiamento rilevante;
    \item Durante la scelta delle tecnologie, abbiamo discusso se adottare Flask o FastAPI; 
    \item Pianificazione dell'incontro del 03/01/25 con \emph{Azzurro Digitale}\textsubscript{\textit{\textbf{G}}};
    \item Discussione sulla creazione di un altro ramo nel nostro gruppo Telegram dedicato alla Programmazione, per migliorare l'organizzazione e rendere le informazioni più facilmente accessibili;
 \end{itemize}
