% Intestazione
\fancyhead[L]{4 \hspace{0.2cm} Decisioni} % Testo a sinistra

\section{Decisioni}

Durante la riunione sono state prese le seguenti decisioni:

\vspace{0.5cm}

\begin{table}[htbp]
    \centering
    \rowcolors{2}{lightgray}{white}
    \begin{tabular}{|c|p{0.8\textwidth}|}
        \hline
        \rowcolor[gray]{0.75}
        \textbf{Codice} & \textbf{Descrizione}\\
        \hline
        VI 15.1 & È stato deciso di sviluppare il \emph{back-end} in \emph{Python}\textsubscript{\textit{\textbf{G}}}. \\
        \hline
        VI 15.2 & È stato deciso di approfondire ulteriormente le tecnologie di \emph{front-end} da usare e quindi di posticipare la scelta alla riunione successiva. \\
        \hline
        VI 15.3 & È stato stabilito il significato condiviso dal gruppo di cosa si deve fare quando ci si organizza per uno studio tecnologico individuale:
        alla riunione successiva, ciascuno deve arrivare avendo una scelta di tecnologia in mente, conoscendo i pro e i contro delle tecnologie prese in considerazione,
        e, possibilmente, avendo svolto una o più prove pratiche nel repository \emph{Playground}. \\
        \hline
        VI 15.4 & È stato stabilito che ogni membro dovrà approfondire individualmente, con test pratici e studio della documentazione disponibile online,
        le tecnologie per la gestione del \emph{database vettoriale}, in modo da poter prendere una decisione definitiva entro la fine della prossima settimana. \\
        \hline
        VI 15.5 & È stato stabilito che ogni nuovo programma \emph{Python} che viene pubblicato sul \emph{repository Playground} deve avere un file \emph{dependencies.txt}
        che elenchi le dipendenze del codice, in modo da consentire l'utilizzo agli altri membri. \\
        \hline
        VI 15.6 & È stato deciso di proseguire con la stesura dell'\emph{Analisi dei Requisiti} e del \emph{Piano di Qualifica}, e di trascrivere il preventivo orario della prima sprint nel \emph{Piano di Progetto}. \\
        \hline
        VI 15.7 & È stato deciso che per scegliere il redattore di ogni verbale si seguiranno gli stessi turni stabiliti per il relatore del \emph{Diario di Bordo}. \\
        \hline
        VI 15.8 & È stato deciso che d'ora in poi le riunioni interne avranno luogo ogni lunedì, e anche giovedì pomeriggio se prima c'è stata una riunione esterna, mantenendo la possibilità di organizzare riunioni straordinarie al bisogno. \\
        \hline
    \end{tabular}
\end{table}