% Intestazione
\fancyhead[L]{4 \hspace{0.2cm} Decisioni} % Testo a sinistra

\section{Decisioni}

Durante la riunione sono state prese le seguenti decisioni:

\vspace{0.5cm}

\begin{table}[htbp]
    \centering
    \rowcolors{2}{lightgray}{white}
    \begin{tabular}{|c|p{0.8\textwidth}|}
        \hline
        \rowcolor[gray]{0.75}
        \textbf{Codice} & \textbf{Descrizione}\\
        \hline
        VI 15.1 & È stato deciso di sviluppare il \emph{back-end} in \emph{Python}\textsubscript{\textit{\textbf{G}}}. \\
        \hline
        VI 15.2 & È stato deciso di approfondire ulteriormente le tecnologie di \emph{front-end} da usare e posticipare la decisione alla riunione successiva. \\
        \hline
        VI 15.3 & È stato stabilito che ogni membro dovrà approfondire individualmente, con test pratici e studio della documentazione disponibile online,
        le tecnologie per la gestione del \emph{database vettoriale}, in modo da poter prendere una decisione definitiva entro la fine della prossima settimana. \\
        \hline
        VI 15.4 & È stato stabilito che ogni nuovo programma \emph{Python} che viene pubblicato sul \emph{repository Playground} deve avere un file \emph{dependencies.txt}
        che elenchi le dipendenze del codice, in modo da semplificarne l'utilizzo per gli altri membri. \\
        \hline
        VI 15.5 & È stato deciso di proseguire con la stesura dell'\emph{Analisi dei Requisiti} e del \emph{Piano di Qualifica}. Inoltre è stata approvata una modifica del preventivo orario complessivo dell'intero progetto che deve essere trascritto nel \emph{Piano di Progetto}. \\
        \hline
        VI 15.6 & È stato deciso che per scegliere il redattore di ogni verbale si seguiranno gli stessi turni stabiliti per il relatore del \emph{Diario di Bordo}. \\
        \hline
        VI 15.7 & È stato deciso che d'ora in poi le riunioni interne avranno luogo ogni lunedì e giovedì pomeriggio, mantenendo la possibilità di organizzare riunioni straordinarie al bisogno. \\
        \hline
    \end{tabular}
\end{table}