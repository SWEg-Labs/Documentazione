% Intestazione
\fancyhead[L]{3 \hspace{0.2cm} Diario della riunione} % Testo a sinistra

\section{Diario della riunione}

\begin{itemize}
    \item Abbiamo discusso la cifra massima da comunicare ad \emph{AzzurroDigitale} riguardo ai costi previsti per le API di OpenAI e deciso chi incaricare per l'invio della mail.

    \item Abbiamo discusso la creazione di un branch \emph{develop} per evitare di lavorare direttamente sul branch \emph{main} e abbiamo stabilito a chi affidare questo compito. Inoltre, nella repository \emph{playground}, ciascuno dovrà creare una cartella con il proprio nome.

    \item  Abbiamo deciso che ciascuno dovrà definire i propri casi d'uso, così da poterne discutere insieme e selezionare quelli definitivi.
    
    \item Abbiamo discusso del progetto di allenamento realizzato da Giacomo e concordato che tutti dobbiamo allinearci e cercare di comprendere quanto fatto finora.
    
    \item Abbiamo discusso la questione dello studio delle tecnologie, individuando due strategie di lavoro differenti:
    \begin{itemize}
        \item La prima strategia prevede la suddivisione in gruppi di due persone, ciascuno dei quali si occupa di una ricerca specifica. Successivamente, i gruppi cambiano argomento, riprendendo il lavoro dal punto lasciato dal gruppo precedente, in modo che tutti possano acquisire una conoscenza generale delle varie tecnologie.
        \item La seconda strategia consiste nel far sì che ogni persona studi gli stessi argomenti, esplorandoli in modo generico e autonomo.
    \end{itemize}

    \item Abbiamo discusso quali tecnologie utilizzare tra TypeScript e Python, decidendo che ciascuno analizzerà autonomamente i pro e i contro di entrambe per confrontarci e scegliere quella più adatta. La stessa strategia verrà applicata anche nella scelta tra Node.js puro e un framework nel caso optassimo per TypeScript, e tra NestJS e Next.js nel caso si decidesse di utilizzare un framework TypeScript.
    
    \item Abbiamo discusso delle API di GitHub, Jira e Confluence, decidendo che ciascuno dovrà testarle e studiarle autonomamente. Lo stesso approccio sarà adottato per i database vettoriali, valutando Pinecone, Qdrant e Supabase.
\end{itemize}
