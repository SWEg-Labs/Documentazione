% Intestazione
\fancyhead[L]{5 \hspace{0.2cm} Todo} % Testo a sinistra

\section{Todo}

Durante la riunione sono emersi i seguenti task da svolgere:

\vspace{0.5cm}

\begin{table}[htbp]
\centering
\rowcolors{2}{lightgray}{white}
\begin{tabular}{|c|c|p{0.25\textwidth}|p{0.4\textwidth}|}
    \hline
    \rowcolor[gray]{0.75}
    \textbf{Codice} & \textbf{Dalla decisione} & \textbf{Assegnatario} & \textbf{Task Todo} \\
    \hline
    BUD-26 & VI 14.2 & Riccardo Stefani & Creare il ramo develop della documentazione. \\
    \hline
    BUD-27 & VI 14.3 & Riccardo Stefani & Mettere le label per le issues Jira e informarsi su come fare determinati grafici su Jira. \\
    \hline
    BUD-28 & VI 14.4 & Tutto il gruppo & Ognuno pensa ai propri casi d'uso che farebbe. \\
    \hline
    BUD-29 & VI 14.5 & Tutto il gruppo & Studiare e cercare di allinearsi il più possibile su quanto fatto da Giacomo sul suo progettino per allenarsi\\
    \hline
    BUD-30 & VI 14.6 & Giacomo Loat & Fornire una guida all'installazione del progettino\\
    \hline
    BUD-31 & VI 14.7 & Tutto il gruppo & Studio individuale tra TypeScript e Python per poter scegliere quale tecnologia utilizzare.\\
    \hline
    BUD-32 & VI 14.7 & Tutto il gruppo & Studio di Node.js puro e framework per poter scegliere quale utilizzare nel caso scegliessimo TypeScript come tecnologia.\\
    \hline
    BUD-34 & VI 14.7 & Tutto il gruppo & Studio di NestJS e Next.js per poter scegliere quale utilizzare nel caso scegliessimo un framework TypeScript.\\
    \hline
    BUD-35 & VI 14.8 & Tutto il gruppo & Effettuare uno studio riguardante API di GitHub, API Jira e API di Confluence e uno studio riguardante i database vettoriali tra cui: Pinecone, Qdrant, Supabase  \\
    \hline
    BUD-36 & VI 14.2 & Tutto il gruppo & Ogni persona deve creare una cartella con il suo nome nel repo playground. \\
    \hline
    BUD-37 & VI 14.10 & Davide Verzotto & Scrivere verbale esterno 21/11/2024. \\
    \hline
    BUD-38 & VI 14.11 & Filippo Righetto & Scrivere verbale interno 21/11/2024. \\
    \hline
\end{tabular}
\end{table}