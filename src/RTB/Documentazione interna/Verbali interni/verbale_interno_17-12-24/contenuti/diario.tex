% Intestazione
\fancyhead[L]{3 \hspace{0.2cm} Diario della riunione} % Testo a sinistra

\section{Diario della riunione}

\begin{itemize}
    \item Durante la riunione si è discusso dello stato di avanzamento dei lavori e della pianificazione per le prossime attività.  

    \item \textbf{Consuntivo}:  
    È stato evidenziato che l'utilizzo delle ore si sta allineando meglio rispetto alla \emph{RTB}\textsubscript{\textit{\textbf{G}}} e alla \emph{PB}\textsubscript{\textit{\textbf{G}}}, in linea con le indicazioni di Tullio. Si punta a ridurre ulteriormente le ore su \emph{RTB} a favore della \emph{PB}.  

    \item \textbf{Aggiornamento attività svolte}:  
    \begin{itemize}
        \item \textbf{Analisi dei requisiti}:  

        \item \textbf{Programmazione}:  
        \begin{itemize}
            \item È stato deciso di adottare il \emph{Git flow}\textsubscript{\textit{\textbf{G}}} per la gestione delle versioni (RTB e PB);
            \item Struttura delle classi (Repository, Service, Controller) completata;
            \item Inserite le \emph{docstrings}\textsubscript{\textit{\textbf{G}}} per documentazione interna;
            \item Risolto un problema con la catena \emph{LangChain}\textsubscript{\textit{\textbf{G}}} che ignorava i metadati, ora integrati direttamente nel contenuto;
            \item Inserire \emph{docstrings}\textsubscript{\textit{\textbf{G}}} e rispettare la visibilità dei metodi/attributi in \emph{Python}\textsubscript{\textit{\textbf{G}}} utilizzando:  
            \begin{itemize}
                \item \texttt{\_} per metodi protetti;
                \item \texttt{\_\_} per metodi privati.  
            \end{itemize}
        \end{itemize}

        \item \textbf{Database e implementazione}:  
        \begin{itemize}
            \item Studio di \emph{Postgres}\textsubscript{\textit{\textbf{G}}} in corso;
            \item Problema nel \emph{PoC}\textsubscript{\textit{\textbf{G}}} con gestione timestamp di sorgenti diverse.  
        \end{itemize}
    \end{itemize}

    \item \textbf{Preparazione incontro con Cardin}:  
    \begin{itemize}
        \item È stato assegnato a Riccardo il compito di presentare il PDF relativo all’\emph{analisi dei requisiti}\textsubscript{\textit{\textbf{G}}};
        \item Si è deciso di focalizzare la discussione principalmente sui seguenti casi d’uso:
        \begin{itemize}
            \item UC11: Visualizzazione con componenti di sistema come attori;
            \item UC8 e UC10: Passaggio da "visualizzazione" a "recupero";
            \item UC9: Revisione della specificità con possibile include per "visualizzazione del singolo messaggio";
            \item UC13: Confronto tra lista file e singolo file nei collegamenti a \emph{Jira}\textsubscript{\textit{\textbf{G}}}, \emph{Github}\textsubscript{\textit{\textbf{G}}} e \emph{Confluence}\textsubscript{\textit{\textbf{G}}}.  
        \end{itemize}
        \item È stato concordato che le modifiche ai casi d’uso saranno gestite dagli autori originali.  
    \end{itemize}

    \item \textbf{Preparazione incontro con \emph{AzzurroDigitale}}:  
    \begin{itemize}
        \item Si è deciso di mostrare l'attuale \emph{PoC}\textsubscript{\textit{\textbf{G}}} e chiedere un confronto sulle funzionalità prioritarie, tra cui:  
        \begin{itemize}
            \item Aggiornamento del database vettoriale;
            \item Visualizzazione dello storico delle sessioni;
            \item Gestione degli errori nella generazione delle risposte;
            \item API per integrazione con \emph{Telegram}\textsubscript{\textit{\textbf{G}}} e \emph{Slack}\textsubscript{\textit{\textbf{G}}}.  
        \end{itemize}
    \end{itemize}
    \item \textbf{Prossimi passi}:  
    \begin{itemize}
        \item Briefing post-incontro con Cardin per aggiornare se necessario il diario di bordo e le slide per \emph{AzzurroDigitale};
        \item Incontro con \emph{AzzurroDigitale} e successiva riunione interna di follow-up;
        \item Possibilità di un incontro intermedio durante le vacanze per monitorare gli avanzamenti.  
    \end{itemize}
\end{itemize}
