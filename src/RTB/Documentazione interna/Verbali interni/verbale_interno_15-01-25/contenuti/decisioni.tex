% Intestazione
\fancyhead[L]{4 \hspace{0.2cm} Decisioni} % Testo a sinistra

\section{Decisioni}

Durante la riunione sono state prese le seguenti decisioni:

\vspace{0.5cm}

\begin{table}[htbp]
    \centering
    \rowcolors{2}{lightgray}{white}
    \begin{tabular}{|c|p{0.8\textwidth}|}
        \hline
        \rowcolor[HTML]{C0C0C0}
        \textbf{Codice} & \textbf{Descrizione} \\
        \hline
        VI 24.1 & È stato deciso che bisogna completare la scrittura per il \emph{Piano di Progetto} e la parte del \emph{Piano di Qualifica}. \\
        \hline
        VI 24.2 & È stato deciso che la scadenza per la verifica dei documenti è stata fissata per giovedì 23 gennaio. \\
        \hline
        VI 24.3 & È stato deciso che bisogna iniziare a scrivere la lettera di presentazione della \emph{RTB}. \\
        \hline
        VI 24.4 & È stato deciso che bisogna redigere il verbale della riunione odierna. \\
        \hline
        VI 24.5 & È stato deciso che bisogna preparare i \emph{Fogli Google}\textsubscript{\textit{\textbf{G}}} per il consuntivo. \\
        \hline
        VI 24.6 & È stato deciso che bisogna preparare bene la demo per la riunione con \emph{AzzurroDigitale}. \\
        \hline
        VI 24.7 & È stato deciso che bisogna preparare i discorsi per la scelta delle tecnologie. \\
        \hline
        VI 24.8 & È stato deciso che, per essere pronti per la RTB, dobbiamo raggiungere i seguenti obiettivi:
        \begin{itemize}
        \item I documenti devono essere corretti, sia grammaticalmente sia semanticamente, e devono riferirsi correttamente tra di loro;
        \item Il PoC deve essere funzionante e il suo codice deve rispettare i vincoli di analisi statica. \\
        \end{itemize}
        \hline
    \end{tabular}
\end{table}
