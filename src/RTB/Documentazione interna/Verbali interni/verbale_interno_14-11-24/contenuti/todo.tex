% Intestazione
\fancyhead[L]{5 \hspace{0.2cm} Todo} % Testo a sinistra

\section{Todo}

Durante la riunione sono emersi i seguenti task da svolgere:

\vspace{0.5cm}

\begin{table}[htbp]
\centering
\rowcolors{2}{lightgray}{white}
\begin{tabular}{|c|c|p{0.25\textwidth}|p{0.4\textwidth}|}
    \hline
    \rowcolor[gray]{0.75}
    \textbf{Codice} & \textbf{Dalla decisione} & \textbf{Assegnatario} & \textbf{Task Todo} \\
    \hline
    \#23 & VI 10.2 & Verificatori & Verificare le parti di testo scritte nei giorni precedenti. \\
    \hline
    \#24 & VI 10.3 & Tutti i membri & Continuazione stesura documenti. \\
    \hline
    \#25 & VI 10.4 & Tutti i membri & Partecipare all'incontro per la stesura del diario di bordo di lunedì 18 novembre. \\
    \hline
    \#26 & VI 10.5 & Michael & Strutturare il Piano Generale di Lavoro per evitare divagazioni. \\
    \hline
    \#27 & VI 10.5 & Giacomo & Abbellire graficamente il Piano Generale di Lavoro. \\
    \hline
    \#28 & VI 10.6 & Michael & Ricercare informazioni riguardo all'uso di soluzioni open source. \\
    \hline
    \#29 & VI 10.7 & Michael & Effettuare una stima per minimizzare il numero di token di input tramite allenamento preliminare. \\
    \hline
    \#30 & VI 10.8 & Giacomo & Supportare la scelta tra OpenAI e Bert. \\
    \hline
    \#31 & VI 10.8 & Davide & Supportare Llama e Grok. \\
    \hline
    \#32 & VI 10.8 & Riccardo & Supportare Claude, Phi e soluzioni open source. \\
    \hline
    \#33 & VI 10.8 & Filippo & Supportare Mistral e Llama. \\
    \hline
    \#34 & VI 10.8 & Federica & Supportare OpenAI e soluzioni open source. \\
    \hline
\end{tabular}
\end{table}